\subsection{Language}
\index{Language}
\begin{verbatim}
    ALMA          : Joint deconvolution of ALMA and ACA dirty images
    CLEAN         : Deconvolve a dirty image using the current METHOD
    CLARK         : Deconvolve a dirty image using the CLARK clean algorithm
    DUMP          : Dump the control parameters of the deconvolution algorithms
    FIT           : Fit the dirty beam
    HOGBOM        : Deconvolve a dirty image using the basic clean algorithm
    MAP_RESAMPLE  :  Resample a Map on a different Velocity/Frequency scale
    MAP_INTEGRATE :  Compute a Moment 0 Map
    MAP_COMPRESS  :  Average a Map by several channels
    MOSAIC        : Toggle the mosaic mode
    MULTI         : Deconvolve a dirty image using the MULTI-SCALE Clean
    MRC           : Deconvolve a dirty image using the MULTI-RESOLUTION Clean
    MX            : Iteratively image and deconvolve a dirty image
    PRIMARY       : Apply primary beam correction
    READ          : Read the input files in internal buffers
    SDI           : Deconvolve a dirty image using the Steer-Dewdney-Ito Clean
    SHOW          : Display (in a plot) some internal buffer result
    SPECIFY       : Change the Frequency / Velocity scale or the Telescope
    STATISTIC     : Compute statistics on image
    STOKES        : Extract one polarization state from a polarized UV table
    SUPPORT       : Define the support used to search clean components
    UV_BASELINE   : Subtract a continuum baseline from a Line UV data
    UV_CHECK      : Check UV data for null visibilities or per channel flags.
    UV_COMPRESS   : Compress a Line UV data into another Line UV data
    UV_CONTINUUM  : Compress a Line UV data into a Continuum UV data
    UV_FILTER     : Filter out (line) channels
    UV_FLAG       : Interactively flag UV data
    UV_MAP        : Build the dirty image and beam from a UV table
    UV_RESAMPLE   : Resample (in Velocity) the UV data
    UV_RESIDUAL   : Subtract Clean Component from the UV Data
    UV_RESTORE    : Restore a Clean image from UV data and Clean Components
    UV_SHIFT      : Shift a UV table to common phase center
    UV_SORT       : Sort and Transpose UV data for plotting
    UV_STAT       : Gives beam sizes and noise properties as a function of
                     tapering or robust weighting parameter
    UV_TIME       : Time average the current UV data
    UV_TRUNCATE   : Truncate the baseline range of the UV data
    VIEW          : Show (in a GO VIEW-like plot) some internal buffer result
    WRITE         : Save internal buffers or Image variables into output files

\end{verbatim}
\subsection{ALMA}
\index{ALMA}
\begin{verbatim}
        [CLEAN\]ALMA  [FirstPlane  [LastPlane]]  [/PLOT Clean|Residu] [/FLUX
    Fmin Fmax] [/QUERY] [/NOISE] [/METHOD]

    Joint deconvolution methods specific to ALMA+ACA observations.

    The ALMA simulator can be accessed either by  typing  "@  alma"  at  the
    prompt or from the MAPPING main menu.


\end{verbatim}
\subsection{CLARK}
\index{CLARK}
\begin{verbatim}
        [CLEAN\]CLARK  [FirstPlane  [LastPlane]] [/PLOT Clean|Residu] [/FLUX
    Fmin Fmax] [/QUERY]

    A Major-Minor cycles CLEAN method, originally developed by  B.Clark,  in
    which  clean components are selected using a limited beam patch, and de-
    convolved through Fourier transform at each major cycle. In mosaic  mode
    (See  command MOSAIC), a mosaic clean is performed. Rings and/or stripes
    may appear on extended sources.  Faster than the Hogbom method for  sin-
    gle  fields  but  maybe  slower  for mosaics. The strategy to search for
    CLEAN components in CLARK does not work properly when the secondary side
    lobes  are  too  large  (e.g. larger than 0.3), or in case of high phase
    noise.

    Clean the specified plane interval (default:  planes  between  variables
    FIRST and LAST). If only FirstPlane is specified, Clean only that plane.

    If option /PLOT is given, a display of the CLEAN or RESIDUAL map will be
    shown  at  each major cycle, depending on the argument (default: Residu-
    al). The user will be prompted for continuation when the  /QUERY  option
    is  present.  The  cumulative, already cleaned flux is displayed in real
    time in an additional window while cleaning goes on when the  /FLUX  op-
    tion  is  present.  Parameters of the /FLUX option are then used to give
    the flux limits for this display.

    The user can control the algorithm through SIC variables. New values can
    be given using "LET VARIABLE value". For ease of use, and whenever it is
    possible, a sensible value of each parameter will automatically be  com-
    puted from the context if the value of the corresponding variable is set
    to its default value, i.e. zero value and empty string. A few  variables
    are initialized to "reasonable" values.

        [CLEAN\]CLEAN ?
    Will  list  all main CLEAN_* variables controlling the CLEAN parameters.
    HELP CLEAN Variables will give a more complete list.

\end{verbatim}
\subsubsection{CLARK Variables:}
\index{CLARK!Variables:}
\begin{verbatim}

    Basic parameters
    CLEAN_GAIN       [       ] Loop gain
    CLEAN_NITER      [       ] Maximum number of clean components
    CLEAN_FRES       [      %] Maximum value of residual (Fraction of peak)
    CLEAN_ARES       [Jy/Beam] Maximum value of residual (Absolute)
    CLEAN_POSITIVE   [       ] Minimum number of positive components at start
    CLEAN_NKEEP      [       ] Min number of components before convergence

    Old names like in MAPPING
    BLC              [  pixel] Bottom left corner of cleaning box
    TRC              [  pixel] Top right corner of cleaning box
    MAJOR            [ arcsec] Clean beam major axis
    MINOR            [ arcsec] Clean beam minor axis
    ANGLE            [ degree] Position angle of clean beam
    BEAM_PATCH       [  pixel] Size of cleaning beam ** not clear **

    Method dependent parameters
    CLEAN_INFLATE    [      ] Maximum Inflation factor for UV_RESTORE (MuLTISCAL
    CLEAN_NCYCLE     [      ] Max number of Major Cycles (SDI & CLARK methods)
    CLEAN_NGOAL      [      ] Max number of comp. in Cycles (ALMA method)
    CLEAN_RESTORE    [      ] Threshold for restoring a Mosaic (def 0.2)
    CLEAN_SEARCH     [      ] Threshold to search Clean Comp. in a Mosaic (def 0
    CLEAN_SIDELOBE   [      ] Min threshold to fit the synthesized beam
    CLEAN_SMOOTH     [      ] Smoothing ratio (MRC and MULTISCALE)
    CLEAN_SPEEDY     [      ] Speed-up factor (CLARK)
    CLEAN_WORRY      [      ] Worry factor (MULTISCALE)
\end{verbatim}
\subsubsection{CLARK CLEAN\_ARES}
\index{CLARK!CLEAN\_ARES}
\begin{verbatim}

    This is the minimal flux in the dirty map that the program will consider
    as  significant.   Alternatively,  the  threshold  can be specified as a
    fraction of the peak flux using CLEAN_FRES.  Once this  level  has  been
    reached the program stops subtracting, and starts the restoration phase.
    The unit for this parameter is the map unit  (typically  Jy/Beam).   The
    parameter  should  usually  be of the order of magnitude of the expected
    noise in the clean map.

    If 0, CLEAN_FRES will be used instead. If all of CLEAN_NITER, CLEAN_ARES
    and CLEAN_FRES are 0, an absolute residual equal to the noise level will
    be used for CLEAN_ARES.

    Short form is ARES.

\end{verbatim}
\subsubsection{CLARK CLEAN\_FRES}
\index{CLARK!CLEAN\_FRES}
\begin{verbatim}

    This is the minimal fraction of the peak flux in the dirty map that  the
    program  will  consider  as  significant.   Alternatively,  an  absolute
    threshold can be specified using CLEAN_ARES.  Once this level  has  been
    reached the program stops subtracting, and starts the restoration phase.
    This parameter is normalized to 1 (neither in % nor in db).   It  should
    usually  be of the order of magnitude of the inverse of the expected dy-
    namic range of the intensity.

    If 0, CLEAN_ARES will be used instead. If all of CLEAN_NITER, CLEAN_ARES
    and CLEAN_FRES are 0, an absolute residual equal to the noise level will
    be used for CLEAN_ARES.

    Short form is FRES.

\end{verbatim}
\subsubsection{CLARK CLEAN\_GAIN}
\index{CLARK!CLEAN\_GAIN}
\begin{verbatim}

    This is the gain of the subtraction loop.  It should typically be chosen
    in the range 0.05 and 0.3.  Higher values give faster convergence, while
    lower values give a better restitution of the extended structure. A sen-
    sible default is 0.2.

    Short form is GAIN.

\end{verbatim}
\subsubsection{CLARK CLEAN\_NITER}
\index{CLARK!CLEAN\_NITER}
\begin{verbatim}

    This is the maximum number of components the program will accept to sub-
    tract.  Once it has been reached, the  program  starts  the  restoration
    phase.

    If 0, the program will guess a number, based on the image size and maxi-
    mum signal-to-noise  ratio,  and  specified  residual  level  CLEAN_ARES
    and/or CLEAN_FRES.

    Short form is NITER.

\end{verbatim}
\subsubsection{CLARK CLEAN\_NKEEP}
\index{CLARK!CLEAN\_NKEEP}
\begin{verbatim}

    This is an integer specifying the minimum number of Clean components be-
    fore testing if Cleaning has converged. The convergence is criterium  is
    a  comparison  of the cumulative flux evolution separated by CLEAN_NKEEP
    components. If th

    IF CLEAN_NKEEP is 0, CLEAN will ignore this convergence  criterium,  and
    continue  clean until the CLEAN_NITER, CLEAN_ARES or CLEAN_FRES criteria
    indicate to stop.

    With CLEAN_NKEEP > 0, CLEAN will explore  the  stability  of  the  total
    clean  flux over the last CLEAN_NKEEP  iterations. For a positive (resp.
    negative) source, if the Clean flux becomes smaller (resp. larger)  than
    the Clean flux CLEAN_NKEEP iterations earlier, CLEAN will stop.

    Using  CLEAN_NKEEP  about  70 is a reasonable value.  Some special cases
    (faint extended sources) may require larger values of CLEAN_NKEEP.

\end{verbatim}
\subsubsection{CLARK CLEAN\_POSITIVE}
\index{CLARK!CLEAN\_POSITIVE}
\begin{verbatim}

    The minimum number of positive components before negative ones  are  se-
    lected.

\end{verbatim}
\subsubsection{CLARK CLEAN\_RESTORE}
\index{CLARK!CLEAN\_RESTORE}
\begin{verbatim}

      Fraction  of  peak  response of the primary beams coverage under which
    the Sky brightness image is blanked in a Mosaic deconvolution.

    The default is 0.2.

\end{verbatim}
\subsubsection{CLARK CLEAN\_SEARCH}
\index{CLARK!CLEAN\_SEARCH}
\begin{verbatim}


      Fraction of peak response of the primary beams coverage  beyond  which
    no Clean component is searched in a Mosaic deconvolution.

    The default is 0.2.

\end{verbatim}
\subsubsection{CLARK CLEAN\_SIDELOBE}
\index{CLARK!CLEAN\_SIDELOBE}
\begin{verbatim}

    Minimal  relative  intensity to consider for fitting the syntheized beam
    to obtain the Clean beam parameters (MAJOR, MINOR  and  ANGLE)  when  0.
    The default is 0.35.

    In  case  of  poor UV coverage, CLEAN_SIDELOBE should be higher than the
    maximum sidelobe level to perform a good Gaussian fit. Some particularly
    bad UV coverage may not allow any good fit at all, however.

\end{verbatim}
\subsubsection{CLARK CLEAN\_NGOAL}
\index{CLARK!CLEAN\_NGOAL}
\begin{verbatim}

      Number  of clean components to be selected in a Cycle in the ALMA het-
    erogeneous array cleaning method.

\end{verbatim}
\subsubsection{CLARK CLEAN\_NCYCLE}
\index{CLARK!CLEAN\_NCYCLE}
\begin{verbatim}

      Maximum number of Major Cycles for the SDI and CLARK methods.

\end{verbatim}
\subsubsection{CLARK CLEAN\_SMOOTH}
\index{CLARK!CLEAN\_SMOOTH}
\begin{verbatim}

      Smoothing factor between different scales in the  MRC  and  MULTISCALE
    methods.  The default is sqrt(3).

\end{verbatim}
\subsubsection{CLARK CLEAN\_SPEEDY}
\index{CLARK!CLEAN\_SPEEDY}
\begin{verbatim}

      Speed-up factor for the CLARK major cycles. The default is 1.0.  Larg-
    er values may be used, but at the expense of possible  instabilities  of
    the algorithm.

\end{verbatim}
\subsubsection{CLARK CLEAN\_WORRY}
\index{CLARK!CLEAN\_WORRY}
\begin{verbatim}

      Worry  factor  in the MULTISCALE method for convergence. It propagates
    the S/N from one iteration to the other, so that if this  S/N  degrades,
    the  method  stops.  Default  is 0 (no propagation, and hence no test on
    S/N).  The value should be < 1.0 in all cases.

\end{verbatim}
\subsubsection{CLARK CLEAN\_INFLATE}
\index{CLARK!CLEAN\_INFLATE}
\begin{verbatim}

      Maximum Inflation factor for UV_RESTORE (MULTISCALE method).   If  the
    number  of  true (i.e. pixel based) Clean components found by MULTISCALE
    is larger than CLEAN_INFLATE times the number of compressed (i.e.  those
    with the smoothing factor information) components, expansion of the com-
    pressed components will not be possible, and UV_RESTORE will not be use-
    able.

      A  default  of  50  is in general adequate.  Better solutions might be
    found in the future, and this parameter suppressed.  Apart  from  memory
    usage, this number has no consequence on the algorithm.

\end{verbatim}
\subsubsection{CLARK METHOD}
\index{CLARK!METHOD}
\begin{verbatim}

      Method used for the deconvolution. Can be HOGBOM, MULTI, MRC,
      SDI or CLARK.


\end{verbatim}
\subsubsection{CLARK Old\_Names:}
\index{CLARK!Old\_Names:}
\begin{verbatim}

      Some  of the CLEAN parameters have kept their old names: MAJOR, MINOR,
    ANGLE (which are also used by  command  FIT)  BLC,  TRC  and  BEAM_PATCH
    (which are seldom used)

      Others  have equivalent short names: ARES, FRES, GAIN, NITER for which
    the CLEAN_ prefix may be omitted.

\end{verbatim}
\subsubsection{CLARK BLC}
\index{CLARK!BLC}
\begin{verbatim}

    These are the (pixel) coordinates of  the  Bottom  Left  Corner  of  the
    cleaning  box.   The actual cleaning support will be the intersection of
    the specified window with the inner quarter of the map and with any user
    defined polygon.

\end{verbatim}
\subsubsection{CLARK TRC}
\index{CLARK!TRC}
\begin{verbatim}

    These  are the (pixel) coordinates of the Top Right Corner of the clean-
    ing box.  The actual cleaning window will be  the  intersection  of  the
    specified window with the inner quarter of the map and with any user de-
    fined polygon.


\end{verbatim}
\subsubsection{CLARK MAJOR}
\index{CLARK!MAJOR}
\begin{verbatim}

    This is the major axis  (FWHP)  in  user  coordinates  of  the  Gaussian
    restoring beam. If 0, the program will fit a Gaussian to the dirty beam.
    We strongly discourage to change the default value of 0.

\end{verbatim}
\subsubsection{CLARK MINOR}
\index{CLARK!MINOR}
\begin{verbatim}

    This is the minor axis  (FWHP)  in  user  coordinates  of  the  Gaussian
    restoring beam. If 0, the program will fit a Gaussian to the dirty beam.
    We strongly discourage to change the default value of 0.

\end{verbatim}
\subsubsection{CLARK ANGLE}
\index{CLARK!ANGLE}
\begin{verbatim}

    This is the position angle (from North towards East, i.e. anticlockwise)
    of  the  major  axis of the Gaussian restoring beam (in degrees).  If 0,
    the program will fit a Gaussian to the dirty beam. We strongly  discour-
    age to change the default value of 0.

\end{verbatim}
\subsubsection{CLARK BEAM\_PATCH}
\index{CLARK!BEAM\_PATCH}
\begin{verbatim}

    The  dirty  beam  patch to be used for the minor cycles in CLARK and MRC
    method.  It should be large enough to avoid doing too many major cycles,
    but  has  practically  no  influence on the result.  This size should be
    specified in pixel units.  Reasonable values are between  N/8  and  N/4,
    where N is the number of map pixels in the same dimension.  If set to N,
    the CLARK algorithm becomes identical to the HOGBOM algorithm.



\end{verbatim}
\subsection{CLEAN}
\index{CLEAN}
\begin{verbatim}
        [CLEAN\]CLEAN [FirstPlane [LastPlane]] [/PLOT Clean|Residu]
      [/FLUX Fmin Fmax] [/QUERY] [/NITER NiterList] [/ARES AresList]

    Deconvolve a Mosaic or Single-field using the  current  METHOD  (in  SIC
    variable  METHOD).  See INPUT CLEAN for the other SIC variables control-
    ling the deconvolution process.

    Clean the specified plane interval (default:  planes  between  variables
    FIRST and LAST). If only FirstPlane is specified, Clean only that plane.

    Supported methods are CLARK, HOGBOM, MRC, MULTISCALE and SDI.  See  help
    of  each  of  these commands for further details on each algorithm. This
    command allows  a  per-plane  definition  of  the  convergence  criteria
    CLEAN_NITER and CLEAN_ARES.

    The user can control the algorithm through SIC variables. New values can
    be given using "LET VARIABLE value". For ease of use, and whenever it is
    possible,  a sensible value of each parameter will automatically be com-
    puted from the context if the value of the corresponding variable is set
    to  its default value, i.e. zero value and empty string. A few variables
    are initialized to "reasonable" values.

        [CLEAN\]CLEAN ?
    Will list all main CLEAN_* variables controlling the  CLEAN  parameters.
    HELP CLEAN Variables will give a more complete list.

\end{verbatim}
\subsubsection{CLEAN /FLUX}
\index{CLEAN!/FLUX}
\begin{verbatim}
        [CLEAN\]CLEAN   [FirstPlane[LastPlane]]   /FLUX   Fmin  Fmax  [/PLOT
    Clean|Residu] [/QUERY] [/NITER NiterList] [/ARES AresList]

    Display the cumulative Clean flux as Clean progresses.  This  option  is
    inactive in Parallel mode.

\end{verbatim}
\subsubsection{CLEAN /PLOT}
\index{CLEAN!/PLOT}
\begin{verbatim}
        [CLEAN\]CLEAN  [FirstPlane  [LastPlane]]  /PLOT  Clean|Residu [/FLUX
    Fmin Fmax] [/QUERY] [/NITER NiterList] [/ARES AresList]

    Display the iterated Clean or Residual image for Cleaning methods  which
    have  major  cycles  (CLARK or SDI). This option is inactive in Parallel
    mode.

\end{verbatim}
\subsubsection{CLEAN /QUERY}
\index{CLEAN!/QUERY}
\begin{verbatim}
        [CLEAN\]CLEAN [FirstPlane  [LastPlane]]  /PLOT  Clean|Residu  /QUERY
    [/FLUX Fmin Fmax] [/NITER NiterList] [/ARES AresList]

    *** Obsolescent ***

    Prompt  for  continuation when a Major cycle is complete. This option is
    inactive in Parallel mode.

\end{verbatim}
\subsubsection{CLEAN /NITER}
\index{CLEAN!/NITER}
\begin{verbatim}

        [CLEAN\]CLEAN  [FirstPlane  [LastPlane]]  /NITER  NiterList   [/PLOT
    Clean|Residu] [/FLUX Fmin Fmax] [/QUERY] [/ARES AresList]

    Use a per-plane value for the number of iterations, instead of the glob-
    al NITER variable.  NiterList should be a 1-D integer array of dimension
    the number of channels.

    This option is only available through the CLEAN command, not through the
    specific command of each method.

\end{verbatim}
\subsubsection{CLEAN /ARES}
\index{CLEAN!/ARES}
\begin{verbatim}

        [CLEAN\]CLEAN  [FirstPlane  [LastPlane]]   /ARES   AresList   [/PLOT
    Clean|Residu] [/FLUX Fmin Fmax] [/QUERY] [/NITER NiterList]

    Use  a  per-plane value for the absolute residual used to stop cleaning,
    instead of the global ARES variable.  AresList should be a 1-D real  ar-
    ray of dimension the number of channels.

    This option is only available through the CLEAN command, not through the
    specific command of each method.

\end{verbatim}
\subsubsection{CLEAN Variables:}
\index{CLEAN!Variables:}
\begin{verbatim}

    Basic parameters
    CLEAN_GAIN       [       ] Loop gain
    CLEAN_NITER      [       ] Maximum number of clean components
    CLEAN_FRES       [      %] Maximum value of residual (Fraction of peak)
    CLEAN_ARES       [Jy/Beam] Maximum value of residual (Absolute)
    CLEAN_POSITIVE   [       ] Minimum number of positive components at start
    CLEAN_NKEEP      [       ] Min number of components before convergence

    Old names like in MAPPING
    BLC              [  pixel] Bottom left corner of cleaning box
    TRC              [  pixel] Top right corner of cleaning box
    MAJOR            [ arcsec] Clean beam major axis
    MINOR            [ arcsec] Clean beam minor axis
    ANGLE            [ degree] Position angle of clean beam
    BEAM_PATCH       [  pixel] Size of cleaning beam ** not clear **

    Method dependent parameters
    CLEAN_INFLATE    [      ] Maximum Inflation factor for UV_RESTORE (MuLTISCAL
    CLEAN_NCYCLE     [      ] Max number of Major Cycles (SDI & CLARK methods)
    CLEAN_NGOAL      [      ] Max number of comp. in Cycles (ALMA method)
    CLEAN_RESTORE    [      ] Threshold for restoring a Mosaic (def 0.2)
    CLEAN_SEARCH     [      ] Threshold to search Clean Comp. in a Mosaic (def 0
    CLEAN_SIDELOBE   [      ] Min threshold to fit the synthesized beam
    CLEAN_SMOOTH     [      ] Smoothing ratio (MRC and MULTISCALE)
    CLEAN_SPEEDY     [      ] Speed-up factor (CLARK)
    CLEAN_WORRY      [      ] Worry factor (MULTISCALE)

\end{verbatim}
\subsubsection{CLEAN CLEAN\_ARES}
\index{CLEAN!CLEAN\_ARES}
\begin{verbatim}

    This is the minimal flux in the dirty map that the program will consider
    as  significant.   Alternatively,  the  threshold  can be specified as a
    fraction of the peak flux using CLEAN_FRES.  Once this  level  has  been
    reached the program stops subtracting, and starts the restoration phase.
    The unit for this parameter is the map unit  (typically  Jy/Beam).   The
    parameter  should  usually  be of the order of magnitude of the expected
    noise in the clean map.

    If 0, CLEAN_FRES will be used instead. If all of CLEAN_NITER, CLEAN_ARES
    and CLEAN_FRES are 0, an absolute residual equal to the noise level will
    be used for CLEAN_ARES.

    Short form is ARES.

\end{verbatim}
\subsubsection{CLEAN CLEAN\_FRES}
\index{CLEAN!CLEAN\_FRES}
\begin{verbatim}

    This is the minimal fraction of the peak flux in the dirty map that  the
    program  will  consider  as  significant.   Alternatively,  an  absolute
    threshold can be specified using CLEAN_ARES.  Once this level  has  been
    reached the program stops subtracting, and starts the restoration phase.
    This parameter is normalized to 1 (neither in % nor in db).   It  should
    usually  be of the order of magnitude of the inverse of the expected dy-
    namic range of the intensity.

    If 0, CLEAN_ARES will be used instead. If all of CLEAN_NITER, CLEAN_ARES
    and CLEAN_FRES are 0, an absolute residual equal to the noise level will
    be used for CLEAN_ARES.

    Short form is FRES.

\end{verbatim}
\subsubsection{CLEAN CLEAN\_GAIN}
\index{CLEAN!CLEAN\_GAIN}
\begin{verbatim}

    This is the gain of the subtraction loop.  It should typically be chosen
    in the range 0.05 and 0.3.  Higher values give faster convergence, while
    lower values give a better restitution of the extended structure. A sen-
    sible default is 0.2.

    Short form is GAIN.

\end{verbatim}
\subsubsection{CLEAN CLEAN\_NITER}
\index{CLEAN!CLEAN\_NITER}
\begin{verbatim}

    This is the maximum number of components the program will accept to sub-
    tract.  Once it has been reached, the  program  starts  the  restoration
    phase.

    If 0, the program will guess a number, based on the image size and maxi-
    mum signal-to-noise  ratio,  and  specified  residual  level  CLEAN_ARES
    and/or CLEAN_FRES.

    Short form is NITER.

\end{verbatim}
\subsubsection{CLEAN CLEAN\_NKEEP}
\index{CLEAN!CLEAN\_NKEEP}
\begin{verbatim}

    This is an integer specifying the minimum number of Clean components be-
    fore testing if Cleaning has converged. The convergence is criterium  is
    a  comparison  of the cumulative flux evolution separated by CLEAN_NKEEP
    components. If th

    IF CLEAN_NKEEP is 0, CLEAN will ignore this convergence  criterium,  and
    continue  clean until the CLEAN_NITER, CLEAN_ARES or CLEAN_FRES criteria
    indicate to stop.

    With CLEAN_NKEEP > 0, CLEAN will explore  the  stability  of  the  total
    clean  flux over the last CLEAN_NKEEP  iterations. For a positive (resp.
    negative) source, if the Clean flux becomes smaller (resp. larger)  than
    the Clean flux CLEAN_NKEEP iterations earlier, CLEAN will stop.

    Using  CLEAN_NKEEP  about  70 is a reasonable value.  Some special cases
    (faint extended sources) may require larger values of CLEAN_NKEEP.

\end{verbatim}
\subsubsection{CLEAN CLEAN\_POSITIVE}
\index{CLEAN!CLEAN\_POSITIVE}
\begin{verbatim}

    The minimum number of positive components before negative ones  are  se-
    lected.

\end{verbatim}
\subsubsection{CLEAN CLEAN\_RESTORE}
\index{CLEAN!CLEAN\_RESTORE}
\begin{verbatim}

      Fraction  of  peak  response of the primary beams coverage under which
    the Sky brightness image is blanked in a Mosaic deconvolution.

    The default is 0.2.

\end{verbatim}
\subsubsection{CLEAN CLEAN\_SEARCH}
\index{CLEAN!CLEAN\_SEARCH}
\begin{verbatim}


      Fraction of peak response of the primary beams coverage  beyond  which
    no Clean component is searched in a Mosaic deconvolution.

    The default is 0.2.

\end{verbatim}
\subsubsection{CLEAN CLEAN\_SIDELOBE}
\index{CLEAN!CLEAN\_SIDELOBE}
\begin{verbatim}

    Minimal  relative  intensity to consider for fitting the syntheized beam
    to obtain the Clean beam parameters (MAJOR, MINOR  and  ANGLE)  when  0.
    The default is 0.35.

    In  case  of  poor UV coverage, CLEAN_SIDELOBE should be higher than the
    maximum sidelobe level to perform a good Gaussian fit. Some particularly
    bad UV coverage may not allow any good fit at all, however.

\end{verbatim}
\subsubsection{CLEAN CLEAN\_NGOAL}
\index{CLEAN!CLEAN\_NGOAL}
\begin{verbatim}

      Number  of clean components to be selected in a Cycle in the ALMA het-
    erogeneous array cleaning method.

\end{verbatim}
\subsubsection{CLEAN CLEAN\_NCYCLE}
\index{CLEAN!CLEAN\_NCYCLE}
\begin{verbatim}

      Maximum number of Major Cycles for the SDI and CLARK methods.

\end{verbatim}
\subsubsection{CLEAN CLEAN\_SMOOTH}
\index{CLEAN!CLEAN\_SMOOTH}
\begin{verbatim}

      Smoothing factor between different scales in the  MRC  and  MULTISCALE
    methods.  The default is sqrt(3).

\end{verbatim}
\subsubsection{CLEAN CLEAN\_SPEEDY}
\index{CLEAN!CLEAN\_SPEEDY}
\begin{verbatim}

      Speed-up factor for the CLARK major cycles. The default is 1.0.  Larg-
    er values may be used, but at the expense of possible  instabilities  of
    the algorithm.

\end{verbatim}
\subsubsection{CLEAN CLEAN\_WORRY}
\index{CLEAN!CLEAN\_WORRY}
\begin{verbatim}

      Worry  factor  in the MULTISCALE method for convergence. It propagates
    the S/N from one iteration to the other, so that if this  S/N  degrades,
    the  method  stops.  Default  is 0 (no propagation, and hence no test on
    S/N).  The value should be < 1.0 in all cases.

\end{verbatim}
\subsubsection{CLEAN CLEAN\_INFLATE}
\index{CLEAN!CLEAN\_INFLATE}
\begin{verbatim}

      Maximum Inflation factor for UV_RESTORE (MULTISCALE method).   If  the
    number  of  true (i.e. pixel based) Clean components found by MULTISCALE
    is larger than CLEAN_INFLATE times the number of compressed (i.e.  those
    with the smoothing factor information) components, expansion of the com-
    pressed components will not be possible, and UV_RESTORE will not be use-
    able.

      A  default  of  50  is in general adequate.  Better solutions might be
    found in the future, and this parameter suppressed.  Apart  from  memory
    usage, this number has no consequence on the algorithm.

\end{verbatim}
\subsubsection{CLEAN METHOD}
\index{CLEAN!METHOD}
\begin{verbatim}

      Method used for the deconvolution. Can be HOGBOM, MULTI, MRC,
      SDI or CLARK.


\end{verbatim}
\subsubsection{CLEAN Old\_Names:}
\index{CLEAN!Old\_Names:}
\begin{verbatim}

      Some  of the CLEAN parameters have kept their old names: MAJOR, MINOR,
    ANGLE (which are also used by  command  FIT)  BLC,  TRC  and  BEAM_PATCH
    (which are seldom used)

      Others  have equivalent short names: ARES, FRES, GAIN, NITER for which
    the CLEAN_ prefix may be omitted.

\end{verbatim}
\subsubsection{CLEAN BLC}
\index{CLEAN!BLC}
\begin{verbatim}

    These are the (pixel) coordinates of  the  Bottom  Left  Corner  of  the
    cleaning  box.   The actual cleaning support will be the intersection of
    the specified window with the inner quarter of the map and with any user
    defined polygon.

\end{verbatim}
\subsubsection{CLEAN TRC}
\index{CLEAN!TRC}
\begin{verbatim}

    These  are the (pixel) coordinates of the Top Right Corner of the clean-
    ing box.  The actual cleaning window will be  the  intersection  of  the
    specified window with the inner quarter of the map and with any user de-
    fined polygon.


\end{verbatim}
\subsubsection{CLEAN MAJOR}
\index{CLEAN!MAJOR}
\begin{verbatim}

    This is the major axis  (FWHP)  in  user  coordinates  of  the  Gaussian
    restoring beam. If 0, the program will fit a Gaussian to the dirty beam.
    We strongly discourage to change the default value of 0.

\end{verbatim}
\subsubsection{CLEAN MINOR}
\index{CLEAN!MINOR}
\begin{verbatim}

    This is the minor axis  (FWHP)  in  user  coordinates  of  the  Gaussian
    restoring beam. If 0, the program will fit a Gaussian to the dirty beam.
    We strongly discourage to change the default value of 0.

\end{verbatim}
\subsubsection{CLEAN ANGLE}
\index{CLEAN!ANGLE}
\begin{verbatim}

    This is the position angle (from North towards East, i.e. anticlockwise)
    of  the  major  axis of the Gaussian restoring beam (in degrees).  If 0,
    the program will fit a Gaussian to the dirty beam. We strongly  discour-
    age to change the default value of 0.

\end{verbatim}
\subsubsection{CLEAN BEAM\_PATCH}
\index{CLEAN!BEAM\_PATCH}
\begin{verbatim}

    The  dirty  beam  patch to be used for the minor cycles in CLARK and MRC
    method.  It should be large enough to avoid doing too many major cycles,
    but  has  practically  no  influence on the result.  This size should be
    specified in pixel units.  Reasonable values are between  N/8  and  N/4,
    where N is the number of map pixels in the same dimension.  If set to N,
    the CLARK algorithm becomes identical to the HOGBOM algorithm.



\end{verbatim}
\subsection{DUMP}
\index{DUMP}
\begin{verbatim}
        [CLEAN\]DUMP [U]

    Dump on screen the control parameters of the different CLEAN  deconvolu-
    tion algorithms, mainly for debugging purpose. "DUMP U" dumps the param-
    eters as input by the user while "DUMP" dumps the parameters really used
    and/or modified by the deconvolution algorithm.

\end{verbatim}
\subsection{FIT}
\index{FIT}
\begin{verbatim}
        [CLEAN\]FIT [Field]

    Fit  the  dirty  beam to obtain the clean beam parameters.  This is done
    automatically by all CLEAN algorithm when needed.  In mosaic  mode,  you
    can specify on which field the fit has to be made, using the first argu-
    ment.

    Fixed values: Clean beam parameters can also be specified by the  users,
    by  setting  the  variables  MAJOR, MINOR and ANGLE to values other than
    their default values (0,0,0). In this case, no  automatic  fit  will  be
    performed.  We strongly discourage the use of those variables, which may
    result in a very improper flux scaling if the beam size  is  inappropri-
    ate. This mode should be reserved to special cases where the beam cannot
    be properly fitted, or to have a circular beam by taking  as  beam  size
    the geometrical mean of the fitted major and minor sizes.

\end{verbatim}
\subsubsection{FIT CLEAN\_SIDELOBE}
\index{FIT!CLEAN\_SIDELOBE}
\begin{verbatim}

    Minimal  relative  intensity to consider for fitting the syntheized beam
    to obtain the Clean beam parameters (MAJOR, MINOR  and  ANGLE)  when  0.
    The default is 0.30.

    In  case  of  poor UV coverage, CLEAN_SIDELOBE should be higher than the
    maximum sidelobe level to perform a good Gaussian fit. Some particularly
    bad UV coverage may not allow any good fit at all, however.

\end{verbatim}
\subsection{HOGBOM}
\index{HOGBOM}
\begin{verbatim}
        [CLEAN\]HOGBOM [FirstPlane [LastPlane]] [/FLUX Fmin Fmax]

    See  HELP  CLEAN  for  the  SIC variables controlling the deeconvolution
    process.

    The simplest CLEAN algorithm, originally developed by Hogbom. In  mosaic
    mode  (See  command  MOSAIC), a mosaic clean is performed.  Rings and or
    strips may appear on extended sources. It is slower  than  CLARK  for  a
    single  field  but  maybe  faster  for a mosaic. It is extremely robust.
    Cleaning can be interrupted by pressing C at any time.

    Clean the specified plane interval (default:  planes  between  variables
    FIRST and LAST). If only FirstPlane is specified, Clean only that plane.

    The cumulative, already cleaned flux is displayed in real time in an ad-
    ditional window while cleaning goes on when the /FLUX option is present.
    Parameters of the /FLUX option are then used to give the flux limits for
    this display.

    The user can control the algorithm through SIC variables. New values can
    be given using "LET VARIABLE value". For ease of use, and whenever it is
    possible,  a sensible value of each parameter will automatically be com-
    puted from the context if the value of the corresponding variable is set
    to  its default value, i.e. zero value and empty string. A few variables
    are initialized to "reasonable" values.

        [CLEAN\]CLEAN ?
    Will list all main CLEAN_* variables controlling the  CLEAN  parameters.
    HELP CLEAN Variables will give a more complete list.

\end{verbatim}
\subsubsection{HOGBOM Variables:}
\index{HOGBOM!Variables:}
\begin{verbatim}

    Basic parameters
    CLEAN_GAIN       [       ] Loop gain
    CLEAN_NITER      [       ] Maximum number of clean components
    CLEAN_FRES       [      %] Maximum value of residual (Fraction of peak)
    CLEAN_ARES       [Jy/Beam] Maximum value of residual (Absolute)
    CLEAN_POSITIVE   [       ] Minimum number of positive components at start
    CLEAN_NKEEP      [       ] Min number of components before convergence

    Old names like in MAPPING
    BLC              [  pixel] Bottom left corner of cleaning box
    TRC              [  pixel] Top right corner of cleaning box
    MAJOR            [ arcsec] Clean beam major axis
    MINOR            [ arcsec] Clean beam minor axis
    ANGLE            [ degree] Position angle of clean beam
    BEAM_PATCH       [  pixel] Size of cleaning beam ** not clear **

    Method dependent parameters
    CLEAN_INFLATE    [      ] Maximum Inflation factor for UV_RESTORE (MuLTISCAL
    CLEAN_NCYCLE     [      ] Max number of Major Cycles (SDI & CLARK methods)
    CLEAN_NGOAL      [      ] Max number of comp. in Cycles (ALMA method)
    CLEAN_RESTORE    [      ] Threshold for restoring a Mosaic (def 0.2)
    CLEAN_SEARCH     [      ] Threshold to search Clean Comp. in a Mosaic (def 0
    CLEAN_SIDELOBE   [      ] Min threshold to fit the synthesized beam
    CLEAN_SMOOTH     [      ] Smoothing ratio (MRC and MULTISCALE)
    CLEAN_SPEEDY     [      ] Speed-up factor (CLARK)
    CLEAN_WORRY      [      ] Worry factor (MULTISCALE)
\end{verbatim}
\subsubsection{HOGBOM CLEAN\_ARES}
\index{HOGBOM!CLEAN\_ARES}
\begin{verbatim}

    This is the minimal flux in the dirty map that the program will consider
    as significant.  Alternatively, the threshold  can  be  specified  as  a
    fraction  of  the  peak flux using CLEAN_FRES.  Once this level has been
    reached the program stops subtracting, and starts the restoration phase.
    The  unit  for  this parameter is the map unit (typically Jy/Beam).  The
    parameter should usually be of the order of magnitude  of  the  expected
    noise in the clean map.

    If 0, CLEAN_FRES will be used instead. If all of CLEAN_NITER, CLEAN_ARES
    and CLEAN_FRES are 0, an absolute residual equal to the noise level will
    be used for CLEAN_ARES.

    Short form is ARES.

\end{verbatim}
\subsubsection{HOGBOM CLEAN\_FRES}
\index{HOGBOM!CLEAN\_FRES}
\begin{verbatim}

    This  is the minimal fraction of the peak flux in the dirty map that the
    program  will  consider  as  significant.   Alternatively,  an  absolute
    threshold  can  be specified using CLEAN_ARES.  Once this level has been
    reached the program stops subtracting, and starts the restoration phase.
    This  parameter  is normalized to 1 (neither in % nor in db).  It should
    usually be of the order of magnitude of the inverse of the expected  dy-
    namic range of the intensity.

    If 0, CLEAN_ARES will be used instead. If all of CLEAN_NITER, CLEAN_ARES
    and CLEAN_FRES are 0, an absolute residual equal to the noise level will
    be used for CLEAN_ARES.

    Short form is FRES.

\end{verbatim}
\subsubsection{HOGBOM CLEAN\_GAIN}
\index{HOGBOM!CLEAN\_GAIN}
\begin{verbatim}

    This is the gain of the subtraction loop.  It should typically be chosen
    in the range 0.05 and 0.3.  Higher values give faster convergence, while
    lower values give a better restitution of the extended structure. A sen-
    sible default is 0.2.

    Short form is GAIN.

\end{verbatim}
\subsubsection{HOGBOM CLEAN\_NITER}
\index{HOGBOM!CLEAN\_NITER}
\begin{verbatim}

    This is the maximum number of components the program will accept to sub-
    tract.   Once  it  has  been reached, the program starts the restoration
    phase.

    If 0, the program will guess a number, based on the image size and maxi-
    mum  signal-to-noise  ratio,  and  specified  residual  level CLEAN_ARES
    and/or CLEAN_FRES.

    Short form is NITER.

\end{verbatim}
\subsubsection{HOGBOM CLEAN\_NKEEP}
\index{HOGBOM!CLEAN\_NKEEP}
\begin{verbatim}

    This is an integer specifying the minimum number of Clean components be-
    fore  testing if Cleaning has converged. The convergence is criterium is
    a comparison of the cumulative flux evolution separated  by  CLEAN_NKEEP
    components. If th

    IF  CLEAN_NKEEP  is 0, CLEAN will ignore this convergence criterium, and
    continue clean until the CLEAN_NITER, CLEAN_ARES or CLEAN_FRES  criteria
    indicate to stop.

    With  CLEAN_NKEEP  >  0,  CLEAN  will explore the stability of the total
    clean flux over the last CLEAN_NKEEP  iterations. For a positive  (resp.
    negative)  source, if the Clean flux becomes smaller (resp. larger) than
    the Clean flux CLEAN_NKEEP iterations earlier, CLEAN will stop.

    Using CLEAN_NKEEP about 70 is a reasonable value.   Some  special  cases
    (faint extended sources) may require larger values of CLEAN_NKEEP.

\end{verbatim}
\subsubsection{HOGBOM CLEAN\_POSITIVE}
\index{HOGBOM!CLEAN\_POSITIVE}
\begin{verbatim}

    The  minimum  number of positive components before negative ones are se-
    lected.

\end{verbatim}
\subsubsection{HOGBOM CLEAN\_RESTORE}
\index{HOGBOM!CLEAN\_RESTORE}
\begin{verbatim}

      Fraction of peak response of the primary beams  coverage  under  which
    the Sky brightness image is blanked in a Mosaic deconvolution.

    The default is 0.2.

\end{verbatim}
\subsubsection{HOGBOM CLEAN\_SEARCH}
\index{HOGBOM!CLEAN\_SEARCH}
\begin{verbatim}


      Fraction  of  peak response of the primary beams coverage beyond which
    no Clean component is searched in a Mosaic deconvolution.

    The default is 0.2.

\end{verbatim}
\subsubsection{HOGBOM CLEAN\_SIDELOBE}
\index{HOGBOM!CLEAN\_SIDELOBE}
\begin{verbatim}

    Minimal relative intensity to consider for fitting the  syntheized  beam
    to  obtain  the  Clean  beam parameters (MAJOR, MINOR and ANGLE) when 0.
    The default is 0.35.

    In case of poor UV coverage, CLEAN_SIDELOBE should be  higher  than  the
    maximum sidelobe level to perform a good Gaussian fit. Some particularly
    bad UV coverage may not allow any good fit at all, however.

\end{verbatim}
\subsubsection{HOGBOM CLEAN\_NGOAL}
\index{HOGBOM!CLEAN\_NGOAL}
\begin{verbatim}

      Number of clean components to be selected in a Cycle in the ALMA  het-
    erogeneous array cleaning method.

\end{verbatim}
\subsubsection{HOGBOM CLEAN\_NCYCLE}
\index{HOGBOM!CLEAN\_NCYCLE}
\begin{verbatim}

      Maximum number of Major Cycles for the SDI and CLARK methods.

\end{verbatim}
\subsubsection{HOGBOM CLEAN\_SMOOTH}
\index{HOGBOM!CLEAN\_SMOOTH}
\begin{verbatim}

      Smoothing  factor  between  different scales in the MRC and MULTISCALE
    methods.  The default is sqrt(3).

\end{verbatim}
\subsubsection{HOGBOM CLEAN\_SPEEDY}
\index{HOGBOM!CLEAN\_SPEEDY}
\begin{verbatim}

      Speed-up factor for the CLARK major cycles. The default is 1.0.  Larg-
    er  values  may be used, but at the expense of possible instabilities of
    the algorithm.

\end{verbatim}
\subsubsection{HOGBOM CLEAN\_WORRY}
\index{HOGBOM!CLEAN\_WORRY}
\begin{verbatim}

      Worry factor in the MULTISCALE method for convergence.  It  propagates
    the  S/N  from one iteration to the other, so that if this S/N degrades,
    the method stops. Default is 0 (no propagation, and  hence  no  test  on
    S/N).  The value should be < 1.0 in all cases.

\end{verbatim}
\subsubsection{HOGBOM CLEAN\_INFLATE}
\index{HOGBOM!CLEAN\_INFLATE}
\begin{verbatim}

      Maximum  Inflation  factor for UV_RESTORE (MULTISCALE method).  If the
    number of true (i.e. pixel based) Clean components found  by  MULTISCALE
    is  larger than CLEAN_INFLATE times the number of compressed (i.e. those
    with the smoothing factor information) components, expansion of the com-
    pressed components will not be possible, and UV_RESTORE will not be use-
    able.

      A default of 50 is in general adequate.   Better  solutions  might  be
    found  in  the  future, and this parameter suppressed. Apart from memory
    usage, this number has no consequence on the algorithm.

\end{verbatim}
\subsubsection{HOGBOM METHOD}
\index{HOGBOM!METHOD}
\begin{verbatim}

      Method used for the deconvolution. Can be HOGBOM, MULTI, MRC,
      SDI or CLARK.


\end{verbatim}
\subsubsection{HOGBOM Old\_Names:}
\index{HOGBOM!Old\_Names:}
\begin{verbatim}

      Some of the CLEAN parameters have kept their old names: MAJOR,  MINOR,
    ANGLE  (which  are  also  used  by  command FIT) BLC, TRC and BEAM_PATCH
    (which are seldom used)

      Others have equivalent short names: ARES, FRES, GAIN, NITER for  which
    the CLEAN_ prefix may be omitted.

\end{verbatim}
\subsubsection{HOGBOM BLC}
\index{HOGBOM!BLC}
\begin{verbatim}

    These  are  the  (pixel)  coordinates  of  the Bottom Left Corner of the
    cleaning box.  The actual cleaning support will be the  intersection  of
    the specified window with the inner quarter of the map and with any user
    defined polygon.

\end{verbatim}
\subsubsection{HOGBOM TRC}
\index{HOGBOM!TRC}
\begin{verbatim}

    These are the (pixel) coordinates of the Top Right Corner of the  clean-
    ing  box.   The  actual  cleaning window will be the intersection of the
    specified window with the inner quarter of the map and with any user de-
    fined polygon.


\end{verbatim}
\subsubsection{HOGBOM MAJOR}
\index{HOGBOM!MAJOR}
\begin{verbatim}

    This  is  the  major  axis  (FWHP)  in  user coordinates of the Gaussian
    restoring beam. If 0, the program will fit a Gaussian to the dirty beam.
    We strongly discourage to change the default value of 0.

\end{verbatim}
\subsubsection{HOGBOM MINOR}
\index{HOGBOM!MINOR}
\begin{verbatim}

    This  is  the  minor  axis  (FWHP)  in  user coordinates of the Gaussian
    restoring beam. If 0, the program will fit a Gaussian to the dirty beam.
    We strongly discourage to change the default value of 0.

\end{verbatim}
\subsubsection{HOGBOM ANGLE}
\index{HOGBOM!ANGLE}
\begin{verbatim}

    This is the position angle (from North towards East, i.e. anticlockwise)
    of the major axis of the Gaussian restoring beam (in  degrees).   If  0,
    the  program will fit a Gaussian to the dirty beam. We strongly discour-
    age to change the default value of 0.

\end{verbatim}
\subsubsection{HOGBOM BEAM\_PATCH}
\index{HOGBOM!BEAM\_PATCH}
\begin{verbatim}

    The dirty beam patch to be used for the minor cycles in  CLARK  and  MRC
    method.  It should be large enough to avoid doing too many major cycles,
    but has practically no influence on the result.   This  size  should  be
    specified  in  pixel  units.  Reasonable values are between N/8 and N/4,
    where N is the number of map pixels in the same dimension.  If set to N,
    the CLARK algorithm becomes identical to the HOGBOM algorithm.



\end{verbatim}
\subsection{MAP\_COMPRESS}
\index{MAP\_COMPRESS}
\begin{verbatim}
        [CLEAN\]MAP_COMPRESS WhichOne Nc

    Resample  (in  frequency/velocity)  the  images  (computed  by UV_MAP or
    CLEAN, or loaded by READ WhichOne) by averaging NC adjacent channels.

    WhichOne indicates which image must be compressed (DIRTY,  CLEAN,  SKY).
    WHichOne = * can be used to treat all the possible images.

\end{verbatim}
\subsection{MAP\_INTEGRATE}
\index{MAP\_INTEGRATE}
\begin{verbatim}
        [CLEAN\]MAP_INTEGRATE WhichOne Min Max Type

    Compute  the  integrated  intensity map(s) over the specified range from
    the current image()s (computed by UV_MAP or CLEAN,  or  loaded  by  READ
    WhichOne). Type can be VELOCITY FREQUENCY or CHANNELS.

    WhichOne  indicates  which image must be compressed (DIRTY, CLEAN, SKY).
    WhichOne = * can be used to treat all the possible images.

\end{verbatim}
\subsection{MAP\_RESAMPLE}
\index{MAP\_RESAMPLE}
\begin{verbatim}
        [CLEAN\]MAP_RESAMPLE WhichOne Nc Ref Val Inc

    Resample the images (computed by UV_MAP or CLEAN, or loaded by READ  Wi-
    chOne) on a different velocity scale.
         Nc   new number of channels
         Ref  New reference pixel
         Val  New velocity at reference pixel
         Inc  Velocity increment
    WhichOne  indicates  which  image must be compressed (DIRTY, CLEAN, SKY)
    WHichOne = * can be used to treat all the possible images.

\end{verbatim}
\subsection{SPECIFY}
\index{SPECIFY}
\begin{verbatim}
        [CLEAN\]SPECIFY FREQUENCY|VELOCITY|TELESCOPE Value

        SPECIFY FREQUENCY Value
    Modify the rest frequency and recompute the velocity scale  accordingly.
    Value is the new rest frequency in MHz

        SPECIFY VELOCITY Value
    Modify  the  source  velocity and recompute the rest frequency scale ac-
    cordingly Value is the new velocity in km/s.

        SPECIFY TELESCOPE Name

    Add or Replace the telescope section with the  parameters  (name,  size,
    position)  for the specified telescope name, essentially to get the most
    appropriate beam parameter.

    A Telescope section is required for MOSAIC. The beamsize will depend  on
    telescope diameter and frequency, with a telescope dependent factor. The
    default beam size is 1.13 Lambda/D.

\end{verbatim}
\subsection{MOSAIC}
\index{MOSAIC}
\begin{verbatim}
        [CLEAN\]MOSAIC On|Off

    Turn on or off the mosaic mode for deconvolution. Note that a READ  PRI-
    MARY command, or a UV_MAP command working on a Mosaic UV Table, automat-
    ically switches on the mosaic mode. The program prompt changes to inform
    the user of the current operating mode for deconvolution.

\end{verbatim}
\subsection{MRC}
\index{MRC}
\begin{verbatim}
        [CLEAN\]MRC  [FirstPlane  [LastPlane]]  [/PLOT  Clean|Residu] [/FLUX
    Fmin Fmax] [/QUERY]

    Perform a Multi-Resolution CLEAN on the current dirty image.   MRC  does
    not support mosaics for theoretical reasons.

    Clean  the  specified  plane interval (default: planes between variables
    FIRST and LAST). If only FirstPlane is specified, Clean only that plane.

    If option /PLOT is given, a display of the CLEAN or RESIDUAL map will be
    shown at each major cycle, depending on the argument  (default:  Residu-
    al).  The  user will be prompted for continuation when the /QUERY option
    is present. The cumulative, already cleaned flux is  displayed  in  real
    time  in  an additional window while cleaning goes on when the /FLUX op-
    tion is present. Parameters of the /FLUX option are then  used  to  give
    the  flux  limits  for this display. A summary plot with the Difference,
    Smooth, and total CLEANed maps is also displayed.

    The user can control the algorithm through SIC variables. New values can
    be given using "LET VARIABLE value". For ease of use, and whenever it is
    possible, a sensible value of each parameter will automatically be  com-
    puted from the context if the value of the corresponding variable is set
    to its default value, i.e. zero value and empty string. A few  variables
    are initialized to "reasonable" values.

        [CLEAN\]CLEAN ?
    Will  list  all main CLEAN_* variables controlling the CLEAN parameters.
    HELP CLEAN Variables will give a more complete list.

\end{verbatim}
\subsubsection{MRC Variables:}
\index{MRC!Variables:}
\begin{verbatim}

    Basic parameters
    CLEAN_GAIN       [       ] Loop gain
    CLEAN_NITER      [       ] Maximum number of clean components
    CLEAN_FRES       [      %] Maximum value of residual (Fraction of peak)
    CLEAN_ARES       [Jy/Beam] Maximum value of residual (Absolute)
    CLEAN_POSITIVE   [       ] Minimum number of positive components at start
    CLEAN_NKEEP      [       ] Min number of components before convergence

    Old names like in MAPPING
    BLC              [  pixel] Bottom left corner of cleaning box
    TRC              [  pixel] Top right corner of cleaning box
    MAJOR            [ arcsec] Clean beam major axis
    MINOR            [ arcsec] Clean beam minor axis
    ANGLE            [ degree] Position angle of clean beam
    BEAM_PATCH       [  pixel] Size of cleaning beam ** not clear **

    Method dependent parameters
    CLEAN_INFLATE    [      ] Maximum Inflation factor for UV_RESTORE (MuLTISCAL
    CLEAN_NCYCLE     [      ] Max number of Major Cycles (SDI & CLARK methods)
    CLEAN_NGOAL      [      ] Max number of comp. in Cycles (ALMA method)
    CLEAN_RESTORE    [      ] Threshold for restoring a Mosaic (def 0.2)
    CLEAN_SEARCH     [      ] Threshold to search Clean Comp. in a Mosaic (def 0
    CLEAN_SIDELOBE   [      ] Min threshold to fit the synthesized beam
    CLEAN_SMOOTH     [      ] Smoothing ratio (MRC and MULTISCALE)
    CLEAN_SPEEDY     [      ] Speed-up factor (CLARK)
    CLEAN_WORRY      [      ] Worry factor (MULTISCALE)
    RATIO      [       ] Smoothing factor (default 0: guess, otherwise must be 2
\end{verbatim}
\subsubsection{MRC CLEAN\_ARES}
\index{MRC!CLEAN\_ARES}
\begin{verbatim}

    This is the minimal flux in the dirty map that the program will consider
    as  significant.   Alternatively,  the  threshold  can be specified as a
    fraction of the peak flux using CLEAN_FRES.  Once this  level  has  been
    reached the program stops subtracting, and starts the restoration phase.
    The unit for this parameter is the map unit  (typically  Jy/Beam).   The
    parameter  should  usually  be of the order of magnitude of the expected
    noise in the clean map.

    If 0, CLEAN_FRES will be used instead. If all of CLEAN_NITER, CLEAN_ARES
    and CLEAN_FRES are 0, an absolute residual equal to the noise level will
    be used for CLEAN_ARES.

    Short form is ARES.

\end{verbatim}
\subsubsection{MRC CLEAN\_FRES}
\index{MRC!CLEAN\_FRES}
\begin{verbatim}

    This is the minimal fraction of the peak flux in the dirty map that  the
    program  will  consider  as  significant.   Alternatively,  an  absolute
    threshold can be specified using CLEAN_ARES.  Once this level  has  been
    reached the program stops subtracting, and starts the restoration phase.
    This parameter is normalized to 1 (neither in % nor in db).   It  should
    usually  be of the order of magnitude of the inverse of the expected dy-
    namic range of the intensity.

    If 0, CLEAN_ARES will be used instead. If all of CLEAN_NITER, CLEAN_ARES
    and CLEAN_FRES are 0, an absolute residual equal to the noise level will
    be used for CLEAN_ARES.

    Short form is FRES.

\end{verbatim}
\subsubsection{MRC CLEAN\_GAIN}
\index{MRC!CLEAN\_GAIN}
\begin{verbatim}

    This is the gain of the subtraction loop.  It should typically be chosen
    in the range 0.05 and 0.3.  Higher values give faster convergence, while
    lower values give a better restitution of the extended structure. A sen-
    sible default is 0.2.

    Short form is GAIN.

\end{verbatim}
\subsubsection{MRC CLEAN\_NITER}
\index{MRC!CLEAN\_NITER}
\begin{verbatim}

    This is the maximum number of components the program will accept to sub-
    tract.  Once it has been reached, the  program  starts  the  restoration
    phase.

    If 0, the program will guess a number, based on the image size and maxi-
    mum signal-to-noise  ratio,  and  specified  residual  level  CLEAN_ARES
    and/or CLEAN_FRES.

    Short form is NITER.

\end{verbatim}
\subsubsection{MRC CLEAN\_NKEEP}
\index{MRC!CLEAN\_NKEEP}
\begin{verbatim}

    This is an integer specifying the minimum number of Clean components be-
    fore testing if Cleaning has converged. The convergence is criterium  is
    a  comparison  of the cumulative flux evolution separated by CLEAN_NKEEP
    components. If th

    IF CLEAN_NKEEP is 0, CLEAN will ignore this convergence  criterium,  and
    continue  clean until the CLEAN_NITER, CLEAN_ARES or CLEAN_FRES criteria
    indicate to stop.

    With CLEAN_NKEEP > 0, CLEAN will explore  the  stability  of  the  total
    clean  flux over the last CLEAN_NKEEP  iterations. For a positive (resp.
    negative) source, if the Clean flux becomes smaller (resp. larger)  than
    the Clean flux CLEAN_NKEEP iterations earlier, CLEAN will stop.

    Using  CLEAN_NKEEP  about  70 is a reasonable value.  Some special cases
    (faint extended sources) may require larger values of CLEAN_NKEEP.

\end{verbatim}
\subsubsection{MRC CLEAN\_POSITIVE}
\index{MRC!CLEAN\_POSITIVE}
\begin{verbatim}

    The minimum number of positive components before negative ones  are  se-
    lected.

\end{verbatim}
\subsubsection{MRC CLEAN\_RESTORE}
\index{MRC!CLEAN\_RESTORE}
\begin{verbatim}

      Fraction  of  peak  response of the primary beams coverage under which
    the Sky brightness image is blanked in a Mosaic deconvolution.

    The default is 0.2.

\end{verbatim}
\subsubsection{MRC CLEAN\_SEARCH}
\index{MRC!CLEAN\_SEARCH}
\begin{verbatim}


      Fraction of peak response of the primary beams coverage  beyond  which
    no Clean component is searched in a Mosaic deconvolution.

    The default is 0.2.

\end{verbatim}
\subsubsection{MRC CLEAN\_SIDELOBE}
\index{MRC!CLEAN\_SIDELOBE}
\begin{verbatim}

    Minimal  relative  intensity to consider for fitting the syntheized beam
    to obtain the Clean beam parameters (MAJOR, MINOR  and  ANGLE)  when  0.
    The default is 0.35.

    In  case  of  poor UV coverage, CLEAN_SIDELOBE should be higher than the
    maximum sidelobe level to perform a good Gaussian fit. Some particularly
    bad UV coverage may not allow any good fit at all, however.

\end{verbatim}
\subsubsection{MRC CLEAN\_NGOAL}
\index{MRC!CLEAN\_NGOAL}
\begin{verbatim}

      Number  of clean components to be selected in a Cycle in the ALMA het-
    erogeneous array cleaning method.

\end{verbatim}
\subsubsection{MRC CLEAN\_NCYCLE}
\index{MRC!CLEAN\_NCYCLE}
\begin{verbatim}

      Maximum number of Major Cycles for the SDI and CLARK methods.

\end{verbatim}
\subsubsection{MRC CLEAN\_SMOOTH}
\index{MRC!CLEAN\_SMOOTH}
\begin{verbatim}

      Smoothing factor between different scales in the  MRC  and  MULTISCALE
    methods.  The default is sqrt(3).

\end{verbatim}
\subsubsection{MRC CLEAN\_SPEEDY}
\index{MRC!CLEAN\_SPEEDY}
\begin{verbatim}

      Speed-up factor for the CLARK major cycles. The default is 1.0.  Larg-
    er values may be used, but at the expense of possible  instabilities  of
    the algorithm.

\end{verbatim}
\subsubsection{MRC CLEAN\_WORRY}
\index{MRC!CLEAN\_WORRY}
\begin{verbatim}

      Worry  factor  in the MULTISCALE method for convergence. It propagates
    the S/N from one iteration to the other, so that if this  S/N  degrades,
    the  method  stops.  Default  is 0 (no propagation, and hence no test on
    S/N).  The value should be < 1.0 in all cases.

\end{verbatim}
\subsubsection{MRC CLEAN\_INFLATE}
\index{MRC!CLEAN\_INFLATE}
\begin{verbatim}

      Maximum Inflation factor for UV_RESTORE (MULTISCALE method).   If  the
    number  of  true (i.e. pixel based) Clean components found by MULTISCALE
    is larger than CLEAN_INFLATE times the number of compressed (i.e.  those
    with the smoothing factor information) components, expansion of the com-
    pressed components will not be possible, and UV_RESTORE will not be use-
    able.

      A  default  of  50  is in general adequate.  Better solutions might be
    found in the future, and this parameter suppressed.  Apart  from  memory
    usage, this number has no consequence on the algorithm.

\end{verbatim}
\subsubsection{MRC METHOD}
\index{MRC!METHOD}
\begin{verbatim}

      Method used for the deconvolution. Can be HOGBOM, MULTI, MRC,
      SDI or CLARK.


\end{verbatim}
\subsubsection{MRC Old\_Names:}
\index{MRC!Old\_Names:}
\begin{verbatim}

      Some  of the CLEAN parameters have kept their old names: MAJOR, MINOR,
    ANGLE (which are also used by  command  FIT)  BLC,  TRC  and  BEAM_PATCH
    (which are seldom used)

      Others  have equivalent short names: ARES, FRES, GAIN, NITER for which
    the CLEAN_ prefix may be omitted.

\end{verbatim}
\subsubsection{MRC BLC}
\index{MRC!BLC}
\begin{verbatim}

    These are the (pixel) coordinates of  the  Bottom  Left  Corner  of  the
    cleaning  box.   The actual cleaning support will be the intersection of
    the specified window with the inner quarter of the map and with any user
    defined polygon.

\end{verbatim}
\subsubsection{MRC TRC}
\index{MRC!TRC}
\begin{verbatim}

    These  are the (pixel) coordinates of the Top Right Corner of the clean-
    ing box.  The actual cleaning window will be  the  intersection  of  the
    specified window with the inner quarter of the map and with any user de-
    fined polygon.


\end{verbatim}
\subsubsection{MRC MAJOR}
\index{MRC!MAJOR}
\begin{verbatim}

    This is the major axis  (FWHP)  in  user  coordinates  of  the  Gaussian
    restoring beam. If 0, the program will fit a Gaussian to the dirty beam.
    We strongly discourage to change the default value of 0.

\end{verbatim}
\subsubsection{MRC MINOR}
\index{MRC!MINOR}
\begin{verbatim}

    This is the minor axis  (FWHP)  in  user  coordinates  of  the  Gaussian
    restoring beam. If 0, the program will fit a Gaussian to the dirty beam.
    We strongly discourage to change the default value of 0.

\end{verbatim}
\subsubsection{MRC ANGLE}
\index{MRC!ANGLE}
\begin{verbatim}

    This is the position angle (from North towards East, i.e. anticlockwise)
    of  the  major  axis of the Gaussian restoring beam (in degrees).  If 0,
    the program will fit a Gaussian to the dirty beam. We strongly  discour-
    age to change the default value of 0.

\end{verbatim}
\subsubsection{MRC BEAM\_PATCH}
\index{MRC!BEAM\_PATCH}
\begin{verbatim}

    The  dirty  beam  patch to be used for the minor cycles in CLARK and MRC
    method.  It should be large enough to avoid doing too many major cycles,
    but  has  practically  no  influence on the result.  This size should be
    specified in pixel units.  Reasonable values are between  N/8  and  N/4,
    where N is the number of map pixels in the same dimension.  If set to N,
    the CLARK algorithm becomes identical to the HOGBOM algorithm.



\end{verbatim}
\subsubsection{MRC RATIO}
\index{MRC!RATIO}
\begin{verbatim}

    Used smoothing factor, which must be a power of 2. The default is 0,  to
    indicate that the actual value must be estimated from the image size.

\end{verbatim}
\subsection{MULTI}
\index{MULTI}
\begin{verbatim}
        [CLEAN\]MULTI [FirstPlane [LastPlane]] [/FLUX Fmin Fmax]

    Multiscale CLEAN algorithm.

    Clean  the  specified  plane interval (default: Planes between variables
    FIRST and LAST). The cumulative, already cleaned flux  is  displayed  in
    real  time in an additional window while cleaning goes on when the /FLUX
    option is present. Parameters of the /FLUX option are then used to  give
    the flux limits for this display.

    MULTI does not yet work on mosaics.

    The user can control the algorithm through SIC variables. New values can
    be given using "LET VARIABLE value". For ease of use, and whenever it is
    possible,  a sensible value of each parameter will automatically be com-
    puted from the context if the value of the corresponding variable is set
    to  its default value, i.e. zero value and empty string. A few variables
    are initialized to "reasonable" values.

        [CLEAN\]CLEAN ?
    Will list all main CLEAN_* variables controlling the  CLEAN  parameters.
    HELP CLEAN Variables will give a more complete list.
\end{verbatim}
\subsubsection{MULTI Variables:}
\index{MULTI!Variables:}
\begin{verbatim}

    Basic parameters
    CLEAN_GAIN       [       ] Loop gain
    CLEAN_NITER      [       ] Maximum number of clean components
    CLEAN_FRES       [      %] Maximum value of residual (Fraction of peak)
    CLEAN_ARES       [Jy/Beam] Maximum value of residual (Absolute)
    CLEAN_POSITIVE   [       ] Minimum number of positive components at start
    CLEAN_NKEEP      [       ] Min number of components before convergence

    Old names like in MAPPING
    BLC              [  pixel] Bottom left corner of cleaning box
    TRC              [  pixel] Top right corner of cleaning box
    MAJOR            [ arcsec] Clean beam major axis
    MINOR            [ arcsec] Clean beam minor axis
    ANGLE            [ degree] Position angle of clean beam
    BEAM_PATCH       [  pixel] Size of cleaning beam ** not clear **

    Method dependent parameters
    CLEAN_INFLATE    [      ] Maximum Inflation factor for UV_RESTORE (MuLTISCAL
    CLEAN_NCYCLE     [      ] Max number of Major Cycles (SDI & CLARK methods)
    CLEAN_NGOAL      [      ] Max number of comp. in Cycles (ALMA method)
    CLEAN_RESTORE    [      ] Threshold for restoring a Mosaic (def 0.2)
    CLEAN_SEARCH     [      ] Threshold to search Clean Comp. in a Mosaic (def 0
    CLEAN_SIDELOBE   [      ] Min threshold to fit the synthesized beam
    CLEAN_SMOOTH     [      ] Smoothing ratio (MRC and MULTISCALE)
    CLEAN_SPEEDY     [      ] Speed-up factor (CLARK)
    CLEAN_WORRY      [      ] Worry factor (MULTISCALE)
    CLEAN_SMOOTH     [       ] Smoothing factor (default sqrt(3))
\end{verbatim}
\subsubsection{MULTI CLEAN\_ARES}
\index{MULTI!CLEAN\_ARES}
\begin{verbatim}

    This is the minimal flux in the dirty map that the program will consider
    as significant.  Alternatively, the threshold  can  be  specified  as  a
    fraction  of  the  peak flux using CLEAN_FRES.  Once this level has been
    reached the program stops subtracting, and starts the restoration phase.
    The  unit  for  this parameter is the map unit (typically Jy/Beam).  The
    parameter should usually be of the order of magnitude  of  the  expected
    noise in the clean map.

    If 0, CLEAN_FRES will be used instead. If all of CLEAN_NITER, CLEAN_ARES
    and CLEAN_FRES are 0, an absolute residual equal to the noise level will
    be used for CLEAN_ARES.

    Short form is ARES.

\end{verbatim}
\subsubsection{MULTI CLEAN\_FRES}
\index{MULTI!CLEAN\_FRES}
\begin{verbatim}

    This  is the minimal fraction of the peak flux in the dirty map that the
    program  will  consider  as  significant.   Alternatively,  an  absolute
    threshold  can  be specified using CLEAN_ARES.  Once this level has been
    reached the program stops subtracting, and starts the restoration phase.
    This  parameter  is normalized to 1 (neither in % nor in db).  It should
    usually be of the order of magnitude of the inverse of the expected  dy-
    namic range of the intensity.

    If 0, CLEAN_ARES will be used instead. If all of CLEAN_NITER, CLEAN_ARES
    and CLEAN_FRES are 0, an absolute residual equal to the noise level will
    be used for CLEAN_ARES.

    Short form is FRES.

\end{verbatim}
\subsubsection{MULTI CLEAN\_GAIN}
\index{MULTI!CLEAN\_GAIN}
\begin{verbatim}

    This is the gain of the subtraction loop.  It should typically be chosen
    in the range 0.05 and 0.3.  Higher values give faster convergence, while
    lower values give a better restitution of the extended structure. A sen-
    sible default is 0.2.

    Short form is GAIN.

\end{verbatim}
\subsubsection{MULTI CLEAN\_NITER}
\index{MULTI!CLEAN\_NITER}
\begin{verbatim}

    This is the maximum number of components the program will accept to sub-
    tract.   Once  it  has  been reached, the program starts the restoration
    phase.

    If 0, the program will guess a number, based on the image size and maxi-
    mum  signal-to-noise  ratio,  and  specified  residual  level CLEAN_ARES
    and/or CLEAN_FRES.

    Short form is NITER.

\end{verbatim}
\subsubsection{MULTI CLEAN\_NKEEP}
\index{MULTI!CLEAN\_NKEEP}
\begin{verbatim}

    This is an integer specifying the minimum number of Clean components be-
    fore  testing if Cleaning has converged. The convergence is criterium is
    a comparison of the cumulative flux evolution separated  by  CLEAN_NKEEP
    components. If th

    IF  CLEAN_NKEEP  is 0, CLEAN will ignore this convergence criterium, and
    continue clean until the CLEAN_NITER, CLEAN_ARES or CLEAN_FRES  criteria
    indicate to stop.

    With  CLEAN_NKEEP  >  0,  CLEAN  will explore the stability of the total
    clean flux over the last CLEAN_NKEEP  iterations. For a positive  (resp.
    negative)  source, if the Clean flux becomes smaller (resp. larger) than
    the Clean flux CLEAN_NKEEP iterations earlier, CLEAN will stop.

    Using CLEAN_NKEEP about 70 is a reasonable value.   Some  special  cases
    (faint extended sources) may require larger values of CLEAN_NKEEP.

\end{verbatim}
\subsubsection{MULTI CLEAN\_POSITIVE}
\index{MULTI!CLEAN\_POSITIVE}
\begin{verbatim}

    The  minimum  number of positive components before negative ones are se-
    lected.

\end{verbatim}
\subsubsection{MULTI CLEAN\_RESTORE}
\index{MULTI!CLEAN\_RESTORE}
\begin{verbatim}

      Fraction of peak response of the primary beams  coverage  under  which
    the Sky brightness image is blanked in a Mosaic deconvolution.

    The default is 0.2.

\end{verbatim}
\subsubsection{MULTI CLEAN\_SEARCH}
\index{MULTI!CLEAN\_SEARCH}
\begin{verbatim}


      Fraction  of  peak response of the primary beams coverage beyond which
    no Clean component is searched in a Mosaic deconvolution.

    The default is 0.2.

\end{verbatim}
\subsubsection{MULTI CLEAN\_SIDELOBE}
\index{MULTI!CLEAN\_SIDELOBE}
\begin{verbatim}

    Minimal relative intensity to consider for fitting the  syntheized  beam
    to  obtain  the  Clean  beam parameters (MAJOR, MINOR and ANGLE) when 0.
    The default is 0.35.

    In case of poor UV coverage, CLEAN_SIDELOBE should be  higher  than  the
    maximum sidelobe level to perform a good Gaussian fit. Some particularly
    bad UV coverage may not allow any good fit at all, however.

\end{verbatim}
\subsubsection{MULTI CLEAN\_NGOAL}
\index{MULTI!CLEAN\_NGOAL}
\begin{verbatim}

      Number of clean components to be selected in a Cycle in the ALMA  het-
    erogeneous array cleaning method.

\end{verbatim}
\subsubsection{MULTI CLEAN\_NCYCLE}
\index{MULTI!CLEAN\_NCYCLE}
\begin{verbatim}

      Maximum number of Major Cycles for the SDI and CLARK methods.

\end{verbatim}
\subsubsection{MULTI CLEAN\_SMOOTH}
\index{MULTI!CLEAN\_SMOOTH}
\begin{verbatim}

      Smoothing  factor  between  different scales in the MRC and MULTISCALE
    methods.  The default is sqrt(3).

\end{verbatim}
\subsubsection{MULTI CLEAN\_SPEEDY}
\index{MULTI!CLEAN\_SPEEDY}
\begin{verbatim}

      Speed-up factor for the CLARK major cycles. The default is 1.0.  Larg-
    er  values  may be used, but at the expense of possible instabilities of
    the algorithm.

\end{verbatim}
\subsubsection{MULTI CLEAN\_WORRY}
\index{MULTI!CLEAN\_WORRY}
\begin{verbatim}

      Worry factor in the MULTISCALE method for convergence.  It  propagates
    the  S/N  from one iteration to the other, so that if this S/N degrades,
    the method stops. Default is 0 (no propagation, and  hence  no  test  on
    S/N).  The value should be < 1.0 in all cases.

\end{verbatim}
\subsubsection{MULTI CLEAN\_INFLATE}
\index{MULTI!CLEAN\_INFLATE}
\begin{verbatim}

      Maximum  Inflation  factor for UV_RESTORE (MULTISCALE method).  If the
    number of true (i.e. pixel based) Clean components found  by  MULTISCALE
    is  larger than CLEAN_INFLATE times the number of compressed (i.e. those
    with the smoothing factor information) components, expansion of the com-
    pressed components will not be possible, and UV_RESTORE will not be use-
    able.

      A default of 50 is in general adequate.   Better  solutions  might  be
    found  in  the  future, and this parameter suppressed. Apart from memory
    usage, this number has no consequence on the algorithm.

\end{verbatim}
\subsubsection{MULTI METHOD}
\index{MULTI!METHOD}
\begin{verbatim}

      Method used for the deconvolution. Can be HOGBOM, MULTI, MRC,
      SDI or CLARK.


\end{verbatim}
\subsubsection{MULTI Old\_Names:}
\index{MULTI!Old\_Names:}
\begin{verbatim}

      Some of the CLEAN parameters have kept their old names: MAJOR,  MINOR,
    ANGLE  (which  are  also  used  by  command FIT) BLC, TRC and BEAM_PATCH
    (which are seldom used)

      Others have equivalent short names: ARES, FRES, GAIN, NITER for  which
    the CLEAN_ prefix may be omitted.

\end{verbatim}
\subsubsection{MULTI BLC}
\index{MULTI!BLC}
\begin{verbatim}

    These  are  the  (pixel)  coordinates  of  the Bottom Left Corner of the
    cleaning box.  The actual cleaning support will be the  intersection  of
    the specified window with the inner quarter of the map and with any user
    defined polygon.

\end{verbatim}
\subsubsection{MULTI TRC}
\index{MULTI!TRC}
\begin{verbatim}

    These are the (pixel) coordinates of the Top Right Corner of the  clean-
    ing  box.   The  actual  cleaning window will be the intersection of the
    specified window with the inner quarter of the map and with any user de-
    fined polygon.


\end{verbatim}
\subsubsection{MULTI MAJOR}
\index{MULTI!MAJOR}
\begin{verbatim}

    This  is  the  major  axis  (FWHP)  in  user coordinates of the Gaussian
    restoring beam. If 0, the program will fit a Gaussian to the dirty beam.
    We strongly discourage to change the default value of 0.

\end{verbatim}
\subsubsection{MULTI MINOR}
\index{MULTI!MINOR}
\begin{verbatim}

    This  is  the  minor  axis  (FWHP)  in  user coordinates of the Gaussian
    restoring beam. If 0, the program will fit a Gaussian to the dirty beam.
    We strongly discourage to change the default value of 0.

\end{verbatim}
\subsubsection{MULTI ANGLE}
\index{MULTI!ANGLE}
\begin{verbatim}

    This is the position angle (from North towards East, i.e. anticlockwise)
    of the major axis of the Gaussian restoring beam (in  degrees).   If  0,
    the  program will fit a Gaussian to the dirty beam. We strongly discour-
    age to change the default value of 0.

\end{verbatim}
\subsubsection{MULTI BEAM\_PATCH}
\index{MULTI!BEAM\_PATCH}
\begin{verbatim}

    The dirty beam patch to be used for the minor cycles in  CLARK  and  MRC
    method.  It should be large enough to avoid doing too many major cycles,
    but has practically no influence on the result.   This  size  should  be
    specified  in  pixel  units.  Reasonable values are between N/8 and N/4,
    where N is the number of map pixels in the same dimension.  If set to N,
    the CLARK algorithm becomes identical to the HOGBOM algorithm.



\end{verbatim}
\subsubsection{MULTI SMOOTH}
\index{MULTI!SMOOTH}
\begin{verbatim}

    Smoothing  ratio  between  the different scales. The default is sqrt(3),
    but larger values should be used for large images with wide spatial  dy-
    namic range.

\end{verbatim}
\subsection{MX}
\index{MX}
\begin{verbatim}
        [CLEAN\]MX [FirstPlane [LastPlane]] [/PLOT Clean|Residu] [/FLUX Fmin
    Fmax] [/QUERY]

    Make and deconvolve maps starting from a UV table.  It  combines  UV_MAP
    and CLEAN in a single step.

    The mapping process is identical to UV_MAP.  It makes a map from UV data
    by griding the UV data using a convolving function, and then Fast Fouri-
    er  Transforming the individual channels.  However, MX always produces a
    single beam for all channels, thus neglecting frequency  change  between
    channels.   MX  enables to shift the map center and rotate the image, by
    shifting the phase tracking center and rotating the  UV  coordinates  of
    the input UV table.

    The  CLEAN  algorithm is similar to the CLARK method, but with major cy-
    cles operating directly on the ungridded UV table rather than in the im-
    age  plane. Accordingly, aliasing affects only of the residuals, not the
    clean components. It is thus more accurate but also slower than CLARK as
    it  asks  for the gridding step at each major cycle.  MX also shares the
    same limitation as CLARK on large sidelobes.

    The user can control the algorithm through SIC variables. New values can
    be given using "LET VARIABLE value". For ease of use, and whenever it is
    possible, a sensible value of each parameter will automatically be  com-
    puted from the context if the value of the corresponding variable is set
    to its default value, i.e. zero value and empty string. A few  variables
    are initialized to "reasonable" values.

        [CLEAN\]CLEAN ?
    Will  list  all main CLEAN_* variables controlling the CLEAN parameters.
    HELP CLEAN Variables will give a more complete list.

                [CLEAN\]UV_MAP ?
        Will list all MAP_* variables controlling the UV_MAP parameters.

    The list of control variables is (by  alphabetic  order,  with  the  old
    names used by Mapping on the right)
    New names       [   unit]       -- Description --    % Old Name
    MAP_BEAM_STEP   [       ]  Number of channels per single dirty beam
    MAP_CELL        [ arcsec]  Image pixel size
    MAP_CENTER      [ string]  RA, Dec of map center, and Position Angle
    MAP_CONVOLUTION [       ]  Convolution function    % CONVOLUTION
    MAP_FIELD       [ arcsec]  Map field of view
    MAP_POWER       [       ]  Maximum exponent of 3 and 5 allowed in MAP_SIZE
    MAP_PRECIS      [       ]  Fraction of pixel tolerance on beam matching
    MAP_ROBUST      [       ]  Robustness factor        % UV_CELL[2]
    MAP_ROUNDING    [       ]  Precision of MAP_SIZE
    MAP_SIZE        [       ]  Number of pixels
    MAP_TAPEREXPO   [       ]  Taper exponent           % TAPER_EXPO
    MAP_TRUNCATE    [      %]  Mosaic truncation level
    MAP_UVCELL      [      m]  UV cell size             % UV_CELL[1]
    MAP_UVTAPER     [m,m,deg]  Gaussian taper           % UV_TAPER
    MAP_VERSION     [       ]  Code version (0 new, -1 old)

    NAME is no longer used, and WEIGHT_MODE is obsolete.
    MAP_RA          [  hours]  RA of map center
    MAP_DEC         [    deg]  Dec of map center
    MAP_ANGLE       [    deg]  Map position angle
    MAP_SHIFT       [Yes/No ]  Shift phase center
    are  obsolescent,  superseded  by MAP_CENTER. They are provided only for
    compatibility with older scripts.
\end{verbatim}
\subsubsection{MX Variables:}
\index{MX!Variables:}
\begin{verbatim}

    Basic parameters
    CLEAN_GAIN       [       ] Loop gain
    CLEAN_NITER      [       ] Maximum number of clean components
    CLEAN_FRES       [      %] Maximum value of residual (Fraction of peak)
    CLEAN_ARES       [Jy/Beam] Maximum value of residual (Absolute)
    CLEAN_POSITIVE   [       ] Minimum number of positive components at start
    CLEAN_NKEEP      [       ] Min number of components before convergence

    Old names like in MAPPING
    BLC              [  pixel] Bottom left corner of cleaning box
    TRC              [  pixel] Top right corner of cleaning box
    MAJOR            [ arcsec] Clean beam major axis
    MINOR            [ arcsec] Clean beam minor axis
    ANGLE            [ degree] Position angle of clean beam
    BEAM_PATCH       [  pixel] Size of cleaning beam ** not clear **

    Method dependent parameters
    CLEAN_INFLATE    [      ] Maximum Inflation factor for UV_RESTORE (MuLTISCAL
    CLEAN_NCYCLE     [      ] Max number of Major Cycles (SDI & CLARK methods)
    CLEAN_NGOAL      [      ] Max number of comp. in Cycles (ALMA method)
    CLEAN_RESTORE    [      ] Threshold for restoring a Mosaic (def 0.2)
    CLEAN_SEARCH     [      ] Threshold to search Clean Comp. in a Mosaic (def 0
    CLEAN_SIDELOBE   [      ] Min threshold to fit the synthesized beam
    CLEAN_SMOOTH     [      ] Smoothing ratio (MRC and MULTISCALE)
    CLEAN_SPEEDY     [      ] Speed-up factor (CLARK)
    CLEAN_WORRY      [      ] Worry factor (MULTISCALE)
\end{verbatim}
\subsubsection{MX MAP\_BEAM\_STEP}
\index{MX!MAP\_BEAM\_STEP}
\begin{verbatim}

      MAP_BEAM_STEP   Integer

    Number of channels per synthesized beam plane.

    Default is 0, meaning only 1 beam plane for all channels.  N (>0)  indi-
    cates N consecutive channels will share the same dirty beam.

    A  value  of  -1  can be used to compute the number of channels per beam
    plane to ensure the angular scale does not deviate more than a  fraction
    of the map cell at the map edge. This fraction is controlled by variable
    MAP_PRECIS (default 0.1)

\end{verbatim}
\subsubsection{MX MAP\_CELL}
\index{MX!MAP\_CELL}
\begin{verbatim}

          MAP_CELL[2]    Real

    The map pixel size [arcsec]. It is recommended to use  identical  values
    in  X  and Y.  A sampling of at least 3 pixel per beam is recommended to
    ease the deconvolution. Enter 0,0 to let the task find the best  values.

\end{verbatim}
\subsubsection{MX MAP\_CENTER}
\index{MX!MAP\_CENTER}
\begin{verbatim}

          MAP_CENTER     Character String

    Specify  the Map center and orientation in the same way as the arguments
    of UV_MAP.

\end{verbatim}
\subsubsection{MX MAP\_CONVOLUTION}
\index{MX!MAP\_CONVOLUTION}
\begin{verbatim}

        MAP_CONVOLUTION    Integer

    Select the desired convolution function for gridding  in  the  UV  plane
    Choices are
            0    Default (currently 5)
            1    Boxcar
            2    Gaussian
            3    Sin(x)/x
            4    Gaussian * Sin(x)/x
            5    Spheroidal
    Spheroidal  functions  is  the optimal choice. So we strongly discourage
    use of any other convolution function, which are here for tests only.

\end{verbatim}
\subsubsection{MX MAP\_FIELD}
\index{MX!MAP\_FIELD}
\begin{verbatim}

      MAP_FIELD[2]     Real

    Field of view in X and Y in arcsec.  The field  of  view  MAP_FIELD  has
    precedence  over  the number of pixels MAP_SIZE to define the actual map
    size when both are non-zero.

\end{verbatim}
\subsubsection{MX MAP\_POWER}
\index{MX!MAP\_POWER}
\begin{verbatim}

          MAP_POWER[2]     Integer

    Maximum exponent of 3 and 5 allowed  in  automatic  guess  of  MAP_SIZE.
    MAP_SIZE is decomposed in 2^k 3^p 5^q, and p and q must be less or equal
    to MAP_POWER.

    Default is 0: MAP_SIZE is just a power of 2. A value of 1 allows approx-
    imation of any map size to 20 %, while a value of 2 allows 10 % approxi-
    mation. Fast Fourier Transform are slightly slower with powers of 3  and
    5,  but  limiting  the  map  size can gain a lot in the Cleaning process
    (which can scale as MAP_SIZE^4).


\end{verbatim}
\subsubsection{MX MAP\_PRECIS}
\index{MX!MAP\_PRECIS}
\begin{verbatim}

      MAP_PRECIS    Real

    Maximum mismatch in pixel at map edge between the true synthesized  beam
    (which  would  have been computed using the exact channel frequency) and
    the computed synthesized beam with  the mean frequency of  the  channels
    sharing  the same beam. This is used (with the actual image size) to de-
    rive the actual number of channels which can share the same  beam,  i.e.
    the effective value of MAP_BEAM_STEP when MAP_BEAM_STEP is -1.

    Default is 0.1

\end{verbatim}
\subsubsection{MX MAP\_ROBUST}
\index{MX!MAP\_ROBUST}
\begin{verbatim}

      MAP_ROBUST     Real

    Robust weighting factor. A number between 0 and +infty.

    Robust  weighting  gives  the  natural  weight to UV cells whose natural
    weight is lower than a given threshold.  In  contrast,  if  the  natural
    weight  of  the UV cell is larger than this threshold, the weight is set
    to this (uniform) threshold. The UV cell size is defined  by  MAP_UVCELL
    and the threshold value is in MAP_ROBUST.

    0  means  natural weighting, which is optimal for point sources. The Ro-
    bust weighting factor controls the resolution: better resolution is  ob-
    tained  for small values (at the expense of noise), resolution approach-
    ing the natural weighting scheme for large values.  Larger UV cell  size
    give higher angular resolution (but again more noise).

    MAP_ROBUST  around  .5  to 1 is a good compromise between noise increase
    and angular resolution.

\end{verbatim}
\subsubsection{MX MAP\_ROUNDING}
\index{MX!MAP\_ROUNDING}
\begin{verbatim}

      MAP_ROUNDING     Real

    Maximum error between optimal size (MAP_FIELD /  MAP_CELL)  and  rounded
    (as  a  power  of 2^k 3^p 5^q) MAP_SIZE to round by floor (thus limiting
    the field of view), instead of ceiling (which guarantees a larger  field
    of view, but leads to bigger images).

    Default is 0.05.


\end{verbatim}
\subsubsection{MX MAP\_SHIFT}
\index{MX!MAP\_SHIFT}
\begin{verbatim}

      MAP_SHIFT        Logical

    Obsolescent, superseded by MAP_CENTER, or the UV_MAP arguments.

    Logical variable indicating whether map center (i.e. phase tracking cen-
    ter) or orientation should be changed.

\end{verbatim}
\subsubsection{MX MAP\_SIZE}
\index{MX!MAP\_SIZE}
\begin{verbatim}

      MAP_SIZE[2]      Integer

    Number of pixels in X and Y. It should preferentially be a power of two,
    (although   this   is  not  strictly  required)  to  speed-up  the  FFT.
    MAP_SIZE*MAP_CELL should be at least twice the size of the field-of-view
    (primary  beam  size  for  a single field). Enter 0,0 to let the command
    find a sensible map size.

    MAP_SIZE is not used if MAP_FIELD is non zero.

    Odd values are forbidden.

    Default is 0,0, i.e. UV_MAP will guess the most appropriate values which
    depend on MAP_ROUNDING and MAP_POWER.


\end{verbatim}
\subsubsection{MX MAP\_TAPEREXPO}
\index{MX!MAP\_TAPEREXPO}
\begin{verbatim}

      MAP_TAPEREXPO    Real

    Taper exponent. The default is 2 (indicating a Gaussian) but smoother or
    sharper functions can be used. 1 would give an Exponential, 4  would  be
    getting close to square profile...

\end{verbatim}
\subsubsection{MX MAP\_TRUNCATE}
\index{MX!MAP\_TRUNCATE}
\begin{verbatim}

      MAP_TRUNCATE    Real

    Mosaic truncation level in PerCent.  Default value is 0.2. Current value
    can be overriden by option /TRUNCATE in commands UV_MAP or PRIMARY.

\end{verbatim}
\subsubsection{MX MAP\_UVTAPER}
\index{MX!MAP\_UVTAPER}
\begin{verbatim}

      MAP_UVTAPER[3]  Real

    Parameters of the tapering function (Gaussian if MAP_TAPEREXPO = 2): ma-
    jor axis at 1/e level [m], minor axis at 1/e level [m], and position an-
    gle [deg].

\end{verbatim}
\subsubsection{MX MAP\_UVCELL}
\index{MX!MAP\_UVCELL}
\begin{verbatim}

      MAP_UVCELL   Real

    UV cell size for robust weighting [m].  Should be of the order  of  half
    the  dish diameter (7.5 m for PdBI), or smaller or even larger.  It con-
    trols the beam shape in Robust weighting.

\end{verbatim}
\subsubsection{MX MAP\_VERSION}
\index{MX!MAP\_VERSION}
\begin{verbatim}

      MAP_VERSION  Integer

    [EXPERT Only] Code indicating which version of the UV_MAP and UV_RESTORE
    algorithm  should  be  used.  0  is  optimal.  -1  is  the  "historical"
    (pre-2016) version. 1 is an intermediate version used during  multi-fre-
    quency beams development.

\end{verbatim}
\subsubsection{MX MCOL}
\index{MX!MCOL}
\begin{verbatim}

      MCOL[2]   Integer

    First  and  Last  channel  to  image.  Values  of 0 mean imaging all the
    planes.

\end{verbatim}
\subsubsection{MX WCOL}
\index{MX!WCOL}
\begin{verbatim}

      WCOL      Integer

    [Obsolescent] The channel from which the weight should be  taken.   WCOL
    set  to 0 means using a default channel. WCOL has no real meaning in all
    cases where more than one beam is computed for all channels.


\end{verbatim}
\subsubsection{MX Old\_Names:}
\index{MX!Old\_Names:}
\begin{verbatim}
     NAME        [       ]  Label of the dirty image and beam plots
     UV_TAPER    [m,m,deg]  UV-apodization by convolution with a Gaussian
     WEIGHT_MODE [       ]  Weighting mode (NA|UN)
     UV_CELL     [m, ??  ]  UV cell size and threshold for Robust weighting
     MAP_FIELD   [ arcsec]  Map field of view
     MAP_CELL    [ arcsec]  Map cell size
     MAP_SIZE    [ pixels]  Map size in pixels (if MAP_FIELD is zero)
     MCOL        [       ]  First and Last channel to map
     WCOL        [       ]  Channel from which the weights are taken
     CONVOLUTION [       ]  Convolution function (5)
     UV_SHIFT    [       ]  Change the map phase center or map orientation?
     MAP_RA      [       ]  RA of map phase center
     MAP_DEC     [       ]  Dec of map phase center
     MAP_ANGLE   [    deg]  Map position angle
     MAP_BEAM_STEP [     ]  Number of channels per synthesized beam plane

\end{verbatim}
\subsubsection{MX convolution}
\index{MX!convolution}
\begin{verbatim}

      Older variable name for MAP_CONVOLUTION

\end{verbatim}
\subsubsection{MX map\_angle}
\index{MX!map\_angle}
\begin{verbatim}

      MAP_ANGLE      Real

    Position Angle of the direction which will become the apparent North  in
    the map. Used only if UV_SHIFT is YES.

    Superseded by MAP_CENTER.

\end{verbatim}
\subsubsection{MX map\_dec}
\index{MX!map\_dec}
\begin{verbatim}

      MAP_DEC     Real

    Dec of map center. Used only if UV_SHIFT is YES.

    Superseded by MAP_CENTER.

\end{verbatim}
\subsubsection{MX map\_ra}
\index{MX!map\_ra}
\begin{verbatim}

      MAP_RA      Real

    RA of map center. Used only if UV_SHIFT is YES.

    Superseded by MAP_CENTER.

\end{verbatim}
\subsubsection{MX uv\_cell}
\index{MX!uv\_cell}
\begin{verbatim}

      Older variables for MAP_UVCELL (uv
\end{verbatim}
\subsubsection{MX uv\_shift}
\index{MX!uv\_shift}
\begin{verbatim}

      Older variable name of MAP_SHIFT (this one is also obsolescent)

\end{verbatim}
\subsubsection{MX uv\_taper}
\index{MX!uv\_taper}
\begin{verbatim}

      Older variable name of MAP_UVTAPER

\end{verbatim}
\subsubsection{MX taper\_expo}
\index{MX!taper\_expo}
\begin{verbatim}

      Older variable name for MAP_TAPEREXPO

\end{verbatim}
\subsubsection{MX weight\_mode}
\index{MX!weight\_mode}
\begin{verbatim}

      weightde      Character

    Weighting  mode:  Natural  (optimum  in  terms of sensitivity) or robust
    (usually lower sidelobes and higher spatial resolution) weighting.  This
    was  needed  in  Mapping to toggle between Natural and Robust weighting,
    while IMAGER does that based on MAP_ROBUST value.






\end{verbatim}
\subsubsection{MX CLEAN\_ARES}
\index{MX!CLEAN\_ARES}
\begin{verbatim}

    This is the minimal flux in the dirty map that the program will consider
    as  significant.   Alternatively,  the  threshold  can be specified as a
    fraction of the peak flux using CLEAN_FRES.  Once this  level  has  been
    reached the program stops subtracting, and starts the restoration phase.
    The unit for this parameter is the map unit  (typically  Jy/Beam).   The
    parameter  should  usually  be of the order of magnitude of the expected
    noise in the clean map.

    If 0, CLEAN_FRES will be used instead. If all of CLEAN_NITER, CLEAN_ARES
    and CLEAN_FRES are 0, an absolute residual equal to the noise level will
    be used for CLEAN_ARES.

    Short form is ARES.

\end{verbatim}
\subsubsection{MX CLEAN\_FRES}
\index{MX!CLEAN\_FRES}
\begin{verbatim}

    This is the minimal fraction of the peak flux in the dirty map that  the
    program  will  consider  as  significant.   Alternatively,  an  absolute
    threshold can be specified using CLEAN_ARES.  Once this level  has  been
    reached the program stops subtracting, and starts the restoration phase.
    This parameter is normalized to 1 (neither in % nor in db).   It  should
    usually  be of the order of magnitude of the inverse of the expected dy-
    namic range of the intensity.

    If 0, CLEAN_ARES will be used instead. If all of CLEAN_NITER, CLEAN_ARES
    and CLEAN_FRES are 0, an absolute residual equal to the noise level will
    be used for CLEAN_ARES.

    Short form is FRES.

\end{verbatim}
\subsubsection{MX CLEAN\_GAIN}
\index{MX!CLEAN\_GAIN}
\begin{verbatim}

    This is the gain of the subtraction loop.  It should typically be chosen
    in the range 0.05 and 0.3.  Higher values give faster convergence, while
    lower values give a better restitution of the extended structure. A sen-
    sible default is 0.2.

    Short form is GAIN.

\end{verbatim}
\subsubsection{MX CLEAN\_NITER}
\index{MX!CLEAN\_NITER}
\begin{verbatim}

    This is the maximum number of components the program will accept to sub-
    tract.  Once it has been reached, the  program  starts  the  restoration
    phase.

    If 0, the program will guess a number, based on the image size and maxi-
    mum signal-to-noise  ratio,  and  specified  residual  level  CLEAN_ARES
    and/or CLEAN_FRES.

    Short form is NITER.

\end{verbatim}
\subsubsection{MX CLEAN\_NKEEP}
\index{MX!CLEAN\_NKEEP}
\begin{verbatim}

    This is an integer specifying the minimum number of Clean components be-
    fore testing if Cleaning has converged. The convergence is criterium  is
    a  comparison  of the cumulative flux evolution separated by CLEAN_NKEEP
    components. If th

    IF CLEAN_NKEEP is 0, CLEAN will ignore this convergence  criterium,  and
    continue  clean until the CLEAN_NITER, CLEAN_ARES or CLEAN_FRES criteria
    indicate to stop.

    With CLEAN_NKEEP > 0, CLEAN will explore  the  stability  of  the  total
    clean  flux over the last CLEAN_NKEEP  iterations. For a positive (resp.
    negative) source, if the Clean flux becomes smaller (resp. larger)  than
    the Clean flux CLEAN_NKEEP iterations earlier, CLEAN will stop.

    Using  CLEAN_NKEEP  about  70 is a reasonable value.  Some special cases
    (faint extended sources) may require larger values of CLEAN_NKEEP.

\end{verbatim}
\subsubsection{MX CLEAN\_POSITIVE}
\index{MX!CLEAN\_POSITIVE}
\begin{verbatim}

    The minimum number of positive components before negative ones  are  se-
    lected.

\end{verbatim}
\subsubsection{MX CLEAN\_RESTORE}
\index{MX!CLEAN\_RESTORE}
\begin{verbatim}

      Fraction  of  peak  response of the primary beams coverage under which
    the Sky brightness image is blanked in a Mosaic deconvolution.

    The default is 0.2.

\end{verbatim}
\subsubsection{MX CLEAN\_SEARCH}
\index{MX!CLEAN\_SEARCH}
\begin{verbatim}


      Fraction of peak response of the primary beams coverage  beyond  which
    no Clean component is searched in a Mosaic deconvolution.

    The default is 0.2.

\end{verbatim}
\subsubsection{MX CLEAN\_SIDELOBE}
\index{MX!CLEAN\_SIDELOBE}
\begin{verbatim}

    Minimal  relative  intensity to consider for fitting the syntheized beam
    to obtain the Clean beam parameters (MAJOR, MINOR  and  ANGLE)  when  0.
    The default is 0.35.

    In  case  of  poor UV coverage, CLEAN_SIDELOBE should be higher than the
    maximum sidelobe level to perform a good Gaussian fit. Some particularly
    bad UV coverage may not allow any good fit at all, however.

\end{verbatim}
\subsubsection{MX CLEAN\_NGOAL}
\index{MX!CLEAN\_NGOAL}
\begin{verbatim}

      Number  of clean components to be selected in a Cycle in the ALMA het-
    erogeneous array cleaning method.

\end{verbatim}
\subsubsection{MX CLEAN\_NCYCLE}
\index{MX!CLEAN\_NCYCLE}
\begin{verbatim}

      Maximum number of Major Cycles for the SDI and CLARK methods.

\end{verbatim}
\subsubsection{MX CLEAN\_SMOOTH}
\index{MX!CLEAN\_SMOOTH}
\begin{verbatim}

      Smoothing factor between different scales in the  MRC  and  MULTISCALE
    methods.  The default is sqrt(3).

\end{verbatim}
\subsubsection{MX CLEAN\_SPEEDY}
\index{MX!CLEAN\_SPEEDY}
\begin{verbatim}

      Speed-up factor for the CLARK major cycles. The default is 1.0.  Larg-
    er values may be used, but at the expense of possible  instabilities  of
    the algorithm.

\end{verbatim}
\subsubsection{MX CLEAN\_WORRY}
\index{MX!CLEAN\_WORRY}
\begin{verbatim}

      Worry  factor  in the MULTISCALE method for convergence. It propagates
    the S/N from one iteration to the other, so that if this  S/N  degrades,
    the  method  stops.  Default  is 0 (no propagation, and hence no test on
    S/N).  The value should be < 1.0 in all cases.

\end{verbatim}
\subsubsection{MX CLEAN\_INFLATE}
\index{MX!CLEAN\_INFLATE}
\begin{verbatim}

      Maximum Inflation factor for UV_RESTORE (MULTISCALE method).   If  the
    number  of  true (i.e. pixel based) Clean components found by MULTISCALE
    is larger than CLEAN_INFLATE times the number of compressed (i.e.  those
    with the smoothing factor information) components, expansion of the com-
    pressed components will not be possible, and UV_RESTORE will not be use-
    able.

      A  default  of  50  is in general adequate.  Better solutions might be
    found in the future, and this parameter suppressed.  Apart  from  memory
    usage, this number has no consequence on the algorithm.

\end{verbatim}
\subsubsection{MX METHOD}
\index{MX!METHOD}
\begin{verbatim}

      Method used for the deconvolution. Can be HOGBOM, MULTI, MRC,
      SDI or CLARK.


\end{verbatim}
\subsubsection{MX Old\_Names:}
\index{MX!Old\_Names:}
\begin{verbatim}

      Some  of the CLEAN parameters have kept their old names: MAJOR, MINOR,
    ANGLE (which are also used by  command  FIT)  BLC,  TRC  and  BEAM_PATCH
    (which are seldom used)

      Others  have equivalent short names: ARES, FRES, GAIN, NITER for which
    the CLEAN_ prefix may be omitted.

\end{verbatim}
\subsubsection{MX BLC}
\index{MX!BLC}
\begin{verbatim}

    These are the (pixel) coordinates of  the  Bottom  Left  Corner  of  the
    cleaning  box.   The actual cleaning support will be the intersection of
    the specified window with the inner quarter of the map and with any user
    defined polygon.

\end{verbatim}
\subsubsection{MX TRC}
\index{MX!TRC}
\begin{verbatim}

    These  are the (pixel) coordinates of the Top Right Corner of the clean-
    ing box.  The actual cleaning window will be  the  intersection  of  the
    specified window with the inner quarter of the map and with any user de-
    fined polygon.


\end{verbatim}
\subsubsection{MX MAJOR}
\index{MX!MAJOR}
\begin{verbatim}

    This is the major axis  (FWHP)  in  user  coordinates  of  the  Gaussian
    restoring beam. If 0, the program will fit a Gaussian to the dirty beam.
    We strongly discourage to change the default value of 0.

\end{verbatim}
\subsubsection{MX MINOR}
\index{MX!MINOR}
\begin{verbatim}

    This is the minor axis  (FWHP)  in  user  coordinates  of  the  Gaussian
    restoring beam. If 0, the program will fit a Gaussian to the dirty beam.
    We strongly discourage to change the default value of 0.

\end{verbatim}
\subsubsection{MX ANGLE}
\index{MX!ANGLE}
\begin{verbatim}

    This is the position angle (from North towards East, i.e. anticlockwise)
    of  the  major  axis of the Gaussian restoring beam (in degrees).  If 0,
    the program will fit a Gaussian to the dirty beam. We strongly  discour-
    age to change the default value of 0.

\end{verbatim}
\subsubsection{MX BEAM\_PATCH}
\index{MX!BEAM\_PATCH}
\begin{verbatim}

    The  dirty  beam  patch to be used for the minor cycles in CLARK and MRC
    method.  It should be large enough to avoid doing too many major cycles,
    but  has  practically  no  influence on the result.  This size should be
    specified in pixel units.  Reasonable values are between  N/8  and  N/4,
    where N is the number of map pixels in the same dimension.  If set to N,
    the CLARK algorithm becomes identical to the HOGBOM algorithm.



\end{verbatim}
\subsection{PRIMARY}
\index{PRIMARY}
\begin{verbatim}
        [CLEAN\]PRIMARY [BeamSize] [/TRUNCATE Percent]

    Apply approximate primary beam correction to a single field  deconvolved
    (CLEAN)  image, in order to create the sky brightness image (named SKY).
    The primary beam model is a simple Gaussian.

    If BeamSize (in radian) is specified, uses the corresponding  half-power
    beam size to determine the Gaussian beam.

    If not, the parameters are taken from the telescope parameters, as found
    in the telescope section of the CLEAN image and the observing  frequency
    from the CLEAN image (e.g. for ALMA it uses 1.13 Lambda/D).

    The SKY image is written with extension .lmv-sky by command WRITE.


\end{verbatim}
\subsubsection{PRIMARY /TRUNCATE}
\index{PRIMARY!/TRUNCATE}
\begin{verbatim}
        [CLEAN\]PRIMARY [BeamSize] /TRUNCATE Percent

    Specify  the  truncation  level. Default (in % of peak beam response) is
    given by the MAP_TRUNCATE variable.  Values are blanked beyond this.

\end{verbatim}
\subsection{READ}
\index{READ}
\begin{verbatim}
        [CLEAN\]READ Buffer File [/COMPACT] |/FREQUENCY RestFreq]
      [/RANGE Min Max Type] [/NOTRAIL]

    Read the specified internal buffer (BEAM,  CCT,  CGAINS,  CLEAN,  DIRTY,
    FIELDS,  MASK,  MODEL,  PRIMARY, RESIDUAL, SKY, SUPPORT, SINGLEDISH, UV)
    from input File.  Default extension is .uvt for UV,  CGAINS  and  MODEL,
    and  for  the  others,  in  order,  .beam, .cct, .lmv-clean, .lmv, .msk,
    .lobe, .lmv-res, be indicated through the /RANGE option, even for UV ta-
    bles.

    The  corresponding  buffer  is available as a SIC image-like variable of
    the same name, so that one can use e.g. HEADER DIRTY command.

    READ * Name   will attempt to read all existing files of same Name  with
    the standard file types corresponding to the respective buffers.

    The  MODEL UV table is for use in conjunction with commands in the CALI-
    BRATE\ language.

    The /COMPACT option is used to load  the  ACA-specific  internal  buffer
    used in the ALMA joint deconvolution method.

    The  /NOTRAIL allows to ignore trailing columns (normally not recommend-
    ed. Used for debug only).

\end{verbatim}
\subsubsection{READ Optimisation}
\index{READ!Optimisation}
\begin{verbatim}

        Reading can be lengthy, especially for ALMA data.  Mapping  attempts
    to  minimize  read  operations by checking if anything has changed since
    the last command.  This capability is enable if the  SIC  variable  MAP-
    PING_OPTIMIZE is non zero, disabled otherwise.
      Currently,  the READ command always display a message about what read-
    ing may have been skipped, and whether this  possible  optimization  has
    been overridden by user choice.

\end{verbatim}
\subsubsection{READ /COMPACT}
\index{READ!/COMPACT}
\begin{verbatim}
        [CLEAN\]READ Buffer File /COMPACT [/RANGE Min Max Type]

    Read  the  specified  internal  buffer (UV, MODEL, BEAM, PRIMARY, DIRTY,
    CLEAN, MASK, CCT) from input File to the "compact array" data area.

\end{verbatim}
\subsubsection{READ /FREQUENCY}
\index{READ!/FREQUENCY}
\begin{verbatim}
        [CLEAN\]READ Buffer File /FREQUENCY RestFreq [/RANGE Min Max Type]

    Read the specified internal buffer and reset the velocity scale  to  the
    corresponding  rest  frequencies. Velocities specified in the /RANGE Min
    Max VELOCITY option would then refer to this new frequency.

\end{verbatim}
\subsubsection{READ /NOTRAIL}
\index{READ!/NOTRAIL}
\begin{verbatim}
        [CLEAN\]READ Buffer File /NOTRAIL [/FREQUENCY Freq] [/RANGE Min  Max
    Type]

    When  reading  UV  Tables, ignores any trailing column. Trailing columns
    normally appear for mosaics. However, ALMA sometimes uses  known  proper
    motions  to shift (by small amounts) phase centers between two observing
    periods, yielding pseudo-mosaics with tiny  (in  general  insignificant)
    displacements.  The /NOTRAIL option allows to ignore these details.

\end{verbatim}
\subsubsection{READ /PLANES}
\index{READ!/PLANES}
\begin{verbatim}
        [CLEAN\]READ Buffer File /PLANES First Last

      ** OBSOLESCENT ** Use /RANGE for a simpler interface.

    Read  only  "channels"  between First and Last.  For UV tables with more
    than 1 Stokes parameter, which are **NOT** fully  supported  by  IMAGER,
    the meaning of "channel" is ambiguous.

\end{verbatim}
\subsubsection{READ /RANGE}
\index{READ!/RANGE}
\begin{verbatim}
        [CLEAN\]READ Buffer File /RANGE Min Max Type

    Load only the channels between the First and Last defined by Min Max and
    Type.  Type can be CHANNEL, VELOCITY or FREQUENCY.

    For type CHANNEL, Min and Max indicate offsets from Channel 1 and  Chan-
    nel Nchan (the number of channels in the data set). Thus Max can be neg-
    ative: it then indicates Last = Nchan-Max. Also Min=0 and Max=0  implies
    loading all the channels.

\end{verbatim}
\subsubsection{READ SINGLE}
\index{READ!SINGLE}
\begin{verbatim}
        [CLEAN\]READ SINGLE File[.ext] [/RANGE Min Max Type]

    Read  the  "Single  Dish" data set. It can be a Class table (.tab), or a
    3-D data cube (.lmv).

    The /RANGE option only works for a .lmv data cube so far.

\end{verbatim}
\subsection{SDI}
\index{SDI}
\begin{verbatim}
        [CLEAN\]SDI [FirstPlane  [LastPlane]]  [/PLOT  Clean|Residu]  [/FLUX
    Fmin Fmax] [/QUERY]

    Perform a Steer-Dewdney-Ito CLEAN. This clean method selects an ensemble
    of clean components and remove them at once using FFTs.  It  works  best
    for  extended  sources  and  UV coverages with short spacings. In such a
    case, it may avoid the "ringing" features which appear using  the  CLARK
    or  HOGBOM  techniques.  In  mosaic  mode (see command MOSAIC), a mosaic
    clean is performed.

    Clean the specified plane interval (default:  planes  between  variables
    FIRST and LAST). If only FirstPlane is specified, Clean only that plane.

    If option /PLOT is given, a display of the CLEAN or RESIDUAL map will be
    shown  at  each major cycle, depending on the argument (default: Residu-
    al). The user will be prompted for continuation when the  /QUERY  option
    is  present.  The  cumulative, already cleaned flux is displayed in real
    time in an additional window while cleaning goes on when the  /FLUX  op-
    tion  is  present.  Parameters of the /FLUX option are then used to give
    the flux limits for this display.

    The user can control the algorithm through SIC variables. New values can
    be given using "LET VARIABLE value". For ease of use, and whenever it is
    possible, a sensible value of each parameter will automatically be  com-
    puted from the context if the value of the corresponding variable is set
    to its default value, i.e. zero value and empty string. A few  variables
    are initialized to "reasonable" values.

        [CLEAN\]CLEAN ?
    Will  list  all main CLEAN_* variables controlling the CLEAN parameters.
    HELP CLEAN Variables will give a more complete list.
\end{verbatim}
\subsubsection{SDI Variables:}
\index{SDI!Variables:}
\begin{verbatim}

    Basic parameters
    CLEAN_GAIN       [       ] Loop gain
    CLEAN_NITER      [       ] Maximum number of clean components
    CLEAN_FRES       [      %] Maximum value of residual (Fraction of peak)
    CLEAN_ARES       [Jy/Beam] Maximum value of residual (Absolute)
    CLEAN_POSITIVE   [       ] Minimum number of positive components at start
    CLEAN_NKEEP      [       ] Min number of components before convergence

    Old names like in MAPPING
    BLC              [  pixel] Bottom left corner of cleaning box
    TRC              [  pixel] Top right corner of cleaning box
    MAJOR            [ arcsec] Clean beam major axis
    MINOR            [ arcsec] Clean beam minor axis
    ANGLE            [ degree] Position angle of clean beam
    BEAM_PATCH       [  pixel] Size of cleaning beam ** not clear **

    Method dependent parameters
    CLEAN_INFLATE    [      ] Maximum Inflation factor for UV_RESTORE (MuLTISCAL
    CLEAN_NCYCLE     [      ] Max number of Major Cycles (SDI & CLARK methods)
    CLEAN_NGOAL      [      ] Max number of comp. in Cycles (ALMA method)
    CLEAN_RESTORE    [      ] Threshold for restoring a Mosaic (def 0.2)
    CLEAN_SEARCH     [      ] Threshold to search Clean Comp. in a Mosaic (def 0
    CLEAN_SIDELOBE   [      ] Min threshold to fit the synthesized beam
    CLEAN_SMOOTH     [      ] Smoothing ratio (MRC and MULTISCALE)
    CLEAN_SPEEDY     [      ] Speed-up factor (CLARK)
    CLEAN_WORRY      [      ] Worry factor (MULTISCALE)
\end{verbatim}
\subsubsection{SDI CLEAN\_ARES}
\index{SDI!CLEAN\_ARES}
\begin{verbatim}

    This is the minimal flux in the dirty map that the program will consider
    as  significant.   Alternatively,  the  threshold  can be specified as a
    fraction of the peak flux using CLEAN_FRES.  Once this  level  has  been
    reached the program stops subtracting, and starts the restoration phase.
    The unit for this parameter is the map unit  (typically  Jy/Beam).   The
    parameter  should  usually  be of the order of magnitude of the expected
    noise in the clean map.

    If 0, CLEAN_FRES will be used instead. If all of CLEAN_NITER, CLEAN_ARES
    and CLEAN_FRES are 0, an absolute residual equal to the noise level will
    be used for CLEAN_ARES.

    Short form is ARES.

\end{verbatim}
\subsubsection{SDI CLEAN\_FRES}
\index{SDI!CLEAN\_FRES}
\begin{verbatim}

    This is the minimal fraction of the peak flux in the dirty map that  the
    program  will  consider  as  significant.   Alternatively,  an  absolute
    threshold can be specified using CLEAN_ARES.  Once this level  has  been
    reached the program stops subtracting, and starts the restoration phase.
    This parameter is normalized to 1 (neither in % nor in db).   It  should
    usually  be of the order of magnitude of the inverse of the expected dy-
    namic range of the intensity.

    If 0, CLEAN_ARES will be used instead. If all of CLEAN_NITER, CLEAN_ARES
    and CLEAN_FRES are 0, an absolute residual equal to the noise level will
    be used for CLEAN_ARES.

    Short form is FRES.

\end{verbatim}
\subsubsection{SDI CLEAN\_GAIN}
\index{SDI!CLEAN\_GAIN}
\begin{verbatim}

    This is the gain of the subtraction loop.  It should typically be chosen
    in the range 0.05 and 0.3.  Higher values give faster convergence, while
    lower values give a better restitution of the extended structure. A sen-
    sible default is 0.2.

    Short form is GAIN.

\end{verbatim}
\subsubsection{SDI CLEAN\_NITER}
\index{SDI!CLEAN\_NITER}
\begin{verbatim}

    This is the maximum number of components the program will accept to sub-
    tract.  Once it has been reached, the  program  starts  the  restoration
    phase.

    If 0, the program will guess a number, based on the image size and maxi-
    mum signal-to-noise  ratio,  and  specified  residual  level  CLEAN_ARES
    and/or CLEAN_FRES.

    Short form is NITER.

\end{verbatim}
\subsubsection{SDI CLEAN\_NKEEP}
\index{SDI!CLEAN\_NKEEP}
\begin{verbatim}

    This is an integer specifying the minimum number of Clean components be-
    fore testing if Cleaning has converged. The convergence is criterium  is
    a  comparison  of the cumulative flux evolution separated by CLEAN_NKEEP
    components. If th

    IF CLEAN_NKEEP is 0, CLEAN will ignore this convergence  criterium,  and
    continue  clean until the CLEAN_NITER, CLEAN_ARES or CLEAN_FRES criteria
    indicate to stop.

    With CLEAN_NKEEP > 0, CLEAN will explore  the  stability  of  the  total
    clean  flux over the last CLEAN_NKEEP  iterations. For a positive (resp.
    negative) source, if the Clean flux becomes smaller (resp. larger)  than
    the Clean flux CLEAN_NKEEP iterations earlier, CLEAN will stop.

    Using  CLEAN_NKEEP  about  70 is a reasonable value.  Some special cases
    (faint extended sources) may require larger values of CLEAN_NKEEP.

\end{verbatim}
\subsubsection{SDI CLEAN\_POSITIVE}
\index{SDI!CLEAN\_POSITIVE}
\begin{verbatim}

    The minimum number of positive components before negative ones  are  se-
    lected.

\end{verbatim}
\subsubsection{SDI CLEAN\_RESTORE}
\index{SDI!CLEAN\_RESTORE}
\begin{verbatim}

      Fraction  of  peak  response of the primary beams coverage under which
    the Sky brightness image is blanked in a Mosaic deconvolution.

    The default is 0.2.

\end{verbatim}
\subsubsection{SDI CLEAN\_SEARCH}
\index{SDI!CLEAN\_SEARCH}
\begin{verbatim}


      Fraction of peak response of the primary beams coverage  beyond  which
    no Clean component is searched in a Mosaic deconvolution.

    The default is 0.2.

\end{verbatim}
\subsubsection{SDI CLEAN\_SIDELOBE}
\index{SDI!CLEAN\_SIDELOBE}
\begin{verbatim}

    Minimal  relative  intensity to consider for fitting the syntheized beam
    to obtain the Clean beam parameters (MAJOR, MINOR  and  ANGLE)  when  0.
    The default is 0.35.

    In  case  of  poor UV coverage, CLEAN_SIDELOBE should be higher than the
    maximum sidelobe level to perform a good Gaussian fit. Some particularly
    bad UV coverage may not allow any good fit at all, however.

\end{verbatim}
\subsubsection{SDI CLEAN\_NGOAL}
\index{SDI!CLEAN\_NGOAL}
\begin{verbatim}

      Number  of clean components to be selected in a Cycle in the ALMA het-
    erogeneous array cleaning method.

\end{verbatim}
\subsubsection{SDI CLEAN\_NCYCLE}
\index{SDI!CLEAN\_NCYCLE}
\begin{verbatim}

      Maximum number of Major Cycles for the SDI and CLARK methods.

\end{verbatim}
\subsubsection{SDI CLEAN\_SMOOTH}
\index{SDI!CLEAN\_SMOOTH}
\begin{verbatim}

      Smoothing factor between different scales in the  MRC  and  MULTISCALE
    methods.  The default is sqrt(3).

\end{verbatim}
\subsubsection{SDI CLEAN\_SPEEDY}
\index{SDI!CLEAN\_SPEEDY}
\begin{verbatim}

      Speed-up factor for the CLARK major cycles. The default is 1.0.  Larg-
    er values may be used, but at the expense of possible  instabilities  of
    the algorithm.

\end{verbatim}
\subsubsection{SDI CLEAN\_WORRY}
\index{SDI!CLEAN\_WORRY}
\begin{verbatim}

      Worry  factor  in the MULTISCALE method for convergence. It propagates
    the S/N from one iteration to the other, so that if this  S/N  degrades,
    the  method  stops.  Default  is 0 (no propagation, and hence no test on
    S/N).  The value should be < 1.0 in all cases.

\end{verbatim}
\subsubsection{SDI CLEAN\_INFLATE}
\index{SDI!CLEAN\_INFLATE}
\begin{verbatim}

      Maximum Inflation factor for UV_RESTORE (MULTISCALE method).   If  the
    number  of  true (i.e. pixel based) Clean components found by MULTISCALE
    is larger than CLEAN_INFLATE times the number of compressed (i.e.  those
    with the smoothing factor information) components, expansion of the com-
    pressed components will not be possible, and UV_RESTORE will not be use-
    able.

      A  default  of  50  is in general adequate.  Better solutions might be
    found in the future, and this parameter suppressed.  Apart  from  memory
    usage, this number has no consequence on the algorithm.

\end{verbatim}
\subsubsection{SDI METHOD}
\index{SDI!METHOD}
\begin{verbatim}

      Method used for the deconvolution. Can be HOGBOM, MULTI, MRC,
      SDI or CLARK.


\end{verbatim}
\subsubsection{SDI Old\_Names:}
\index{SDI!Old\_Names:}
\begin{verbatim}

      Some  of the CLEAN parameters have kept their old names: MAJOR, MINOR,
    ANGLE (which are also used by  command  FIT)  BLC,  TRC  and  BEAM_PATCH
    (which are seldom used)

      Others  have equivalent short names: ARES, FRES, GAIN, NITER for which
    the CLEAN_ prefix may be omitted.

\end{verbatim}
\subsubsection{SDI BLC}
\index{SDI!BLC}
\begin{verbatim}

    These are the (pixel) coordinates of  the  Bottom  Left  Corner  of  the
    cleaning  box.   The actual cleaning support will be the intersection of
    the specified window with the inner quarter of the map and with any user
    defined polygon.

\end{verbatim}
\subsubsection{SDI TRC}
\index{SDI!TRC}
\begin{verbatim}

    These  are the (pixel) coordinates of the Top Right Corner of the clean-
    ing box.  The actual cleaning window will be  the  intersection  of  the
    specified window with the inner quarter of the map and with any user de-
    fined polygon.


\end{verbatim}
\subsubsection{SDI MAJOR}
\index{SDI!MAJOR}
\begin{verbatim}

    This is the major axis  (FWHP)  in  user  coordinates  of  the  Gaussian
    restoring beam. If 0, the program will fit a Gaussian to the dirty beam.
    We strongly discourage to change the default value of 0.

\end{verbatim}
\subsubsection{SDI MINOR}
\index{SDI!MINOR}
\begin{verbatim}

    This is the minor axis  (FWHP)  in  user  coordinates  of  the  Gaussian
    restoring beam. If 0, the program will fit a Gaussian to the dirty beam.
    We strongly discourage to change the default value of 0.

\end{verbatim}
\subsubsection{SDI ANGLE}
\index{SDI!ANGLE}
\begin{verbatim}

    This is the position angle (from North towards East, i.e. anticlockwise)
    of  the  major  axis of the Gaussian restoring beam (in degrees).  If 0,
    the program will fit a Gaussian to the dirty beam. We strongly  discour-
    age to change the default value of 0.

\end{verbatim}
\subsubsection{SDI BEAM\_PATCH}
\index{SDI!BEAM\_PATCH}
\begin{verbatim}

    The  dirty  beam  patch to be used for the minor cycles in CLARK and MRC
    method.  It should be large enough to avoid doing too many major cycles,
    but  has  practically  no  influence on the result.  This size should be
    specified in pixel units.  Reasonable values are between  N/8  and  N/4,
    where N is the number of map pixels in the same dimension.  If set to N,
    the CLARK algorithm becomes identical to the HOGBOM algorithm.



\end{verbatim}
\subsection{SHOW}
\index{SHOW}
\begin{verbatim}
        [CLEAN\]SHOW Variable [Arg1 [Arg2]]

    where
    Variable
        is an internal buffer to be plotted (BEAM, CCT, CLEAN, DIRTY,  etc..
    ) or a SIC image variable.

        [CLEAN\]SHOW ?

    will  list  the  names of recognized keywords. If Variable is not one of
    the recognized keywords, but an existing Image variable, this image will
    be displayed.

        Except for UV data where they have a different meaning,
    Arg1 and Arg2
        are optional arguments to restrict the range of channels to be plot-
    ted. They default to  FIRST and LAST variable values  respectively,  and
    Arg2 defaults to Arg1 if only Arg1 is specified.

    The buffer is displayed using the appropriate procedure
      p_show_map for data cubes
      p_show_cct for Clean Component Tables
      p_uvshow_sub for UV data
    Command VIEW offers a different style of display for cubes.

    The  UV data in the internal buffer is the one loaded by command READ UV
    File and optionally resampled by UV_RESAMPLE, UV_COMPRESS or transformed
    by UV_CONT. Flagged UV data  will appear in a different color.

    Note and Caution:
        GO  PLOT  (or its variants GO BIT, GO NICE and GO MAP) and GO UVSHOW
    offer similar features, but take data from files or SIC image variables,
    depending on variables NAME and TYPE.

\end{verbatim}
\subsubsection{SHOW COVERAGE}
\index{SHOW!COVERAGE}
\begin{verbatim}
        SHOW COVERAGE [Ant [Date]]

    Displays  the UV coverage.  Ant and Date are optional arguments indicat-
    ing which Antenna is to be highlighted, and for which date.  Date  is  a
    sequential number from 1 to the number of observing dates.

    For  spectral line UV data, SHOW COVERAGE will only show one UV coverage
    if FIRST and LAST are set to zero. It will show one per  channel  other-
    wise.

\end{verbatim}
\subsubsection{SHOW UV}
\index{SHOW!UV}
\begin{verbatim}
        SHOW UV [Ant [Date]]

    Displays the UV data.  Items to be displayed are controlled by XTYPE and
    YTYPE variables.  Ant and Date are optional arguments  indicating  which
    Antenna  is  to be highlighted, and for which date. Date is a sequential
    number from 1 to the number of observing dates.

    Structure uvshow% controls some additional options, such as coloring  of
    various dates, of data flagging, etc...

\end{verbatim}
\subsubsection{SHOW GO\_PLOT}
\index{SHOW!GO\_PLOT}
\begin{verbatim}
        GO  PLOT provides a somewhat different plotting mechanism, using the
    disk files specified by NAME and TYPE (or the SIC variable specified  by
    NAME if TYPE is empty)
      GO PLOT Variable [First [Last]]
    which  will plot channels of the specified data set.  First (default: 0)
    and Last (default: First) are optional arguments  indicating  the  first
    and  last  channel to be displayed. If not specified the variables FIRST
    and LAST are used.

    Most aspects of the display are controlled through the  same  SIC  vari-
    ables as the ones used by the "GO BIT|MAP|LMV" commands. Detailed infor-
    mation about those variables is found through the "INPUT LMV" command.

\end{verbatim}
\subsubsection{SHOW GO\_UVSHOW}
\index{SHOW!GO\_UVSHOW}
\begin{verbatim}
        GO UVSHOW

    Display UV data from the 'name'.uvt file. Control parameters are  avail-
    able as SIC variables. Use command INPUT UVSHOW to check them.

\end{verbatim}
\subsection{STATISTIC}
\index{STATISTIC}
\begin{verbatim}
        [CLEAN\]STATISTIC [Clean|Dirty|Residu] [Plane] [/WHOLE]

    Compute some basic statistics (min, max, mean, rms,...) on the specified
    buffer (default: Clean) and plane (default: variable FIRST). The current
    polygon  (see  command  SUPPORT) is used to define the area on which the
    pixels are to be taken  into  account,  except  when  option  /WHOLE  is
    present: The whole image is then considered.

\end{verbatim}
\subsection{STOKES}
\index{STOKES}
\begin{verbatim}
        [CLEAN\]STOKES Key UVin  [UVout]

    Derive  a  single  polarization  UV  table (UVout) with the polarization
    state specified by Key from a multi-polarization UV  table  (Uvin).   If
    UVout  is not specified, UVin is overwritten.  Key is any of the follow-
    ing: NONE, I, Q, U, V, LL, RR, HH, VV.

    A typical use is after command FITS on CASA data:
      FITS Fits.uvfits TO UVin.uvt
      STOKES NONE UVin

\end{verbatim}
\subsection{SUPPORT}
\index{SUPPORT}
\begin{verbatim}
        [CLEAN\]SUPPORT [Polygon] [/CURSOR] [/MASK] [/PLOT] [/RESET]
      [/VARIABLE] [/THRESHOLD [Raw Smooth Length Guard]]

    Define and/or plot the support inside which to search for  CLEAN  compo-
    nents. The support can be defined through a mask or a polygon, depending
    on the selected options.  selected.

    A polygon stored in a file, or in a Sic variable (/VARIABLE option), can
    be loaded as support. The /PLOT option can then be used to plot it.


\end{verbatim}
\subsubsection{SUPPORT /CURSOR}
\index{SUPPORT!/CURSOR}
\begin{verbatim}
        [CLEAN\]SUPPORT /CURSOR

    With option /CURSOR, it calls the interactive cursor to define the poly-
    gon summits.  Type any key to go to next summit, D to correct  the  last
    one and type E to end the polygon definition. The last polygon side will
    then appear. The polygon definition may be  aborted  by  typing  Q.  For
    graphical displays, you may use the mouse buttons for the commands.  The
    left mouse button draws a vertex, the middle mouse  button  deletes  the
    last vertex, and the right mouse button ends the polygon definition.

    The resulting support is available in the Sic structure SUPPORT:
      - SUPPORT%NXY  [Integer]  Number of summits
      - SUPPORT%X    [Double]   X coordinates
      - SUPPORT%Y    [Double]   Y coordinates

\end{verbatim}
\subsubsection{SUPPORT /RESET}
\index{SUPPORT!/RESET}
\begin{verbatim}
        [CLEAN\]SUPPORT /RESET

    Reset  any support to default. This deletes the current polygon support.
    This does not unload any mask defined  by READ MASK: it  can  be  re-in-
    stated by SUPPORT /MASK.

\end{verbatim}
\subsubsection{SUPPORT /MASK}
\index{SUPPORT!/MASK}
\begin{verbatim}
        [CLEAN\]SUPPORT /MASK

    Use  the  mask  defined by READ MASK as the clean support. This does not
    suppress the current polygon: it can be re-instated  by  SUPPORT  /PLOT.
    The  Mask can be a 3-D array: CLEAN will find out match which mask plane
    must be used for each spectral channel.

    Caution: READ MASK does not perform an implicit SUPPORT /MASK command.

\end{verbatim}
\subsubsection{SUPPORT /PLOT}
\index{SUPPORT!/PLOT}
\begin{verbatim}
        [CLEAN\]SUPPORT [Name] /PLOT

    Plot the current (or specified) polygon, or plot the current mask (if it
    is 2-D only).

\end{verbatim}
\subsubsection{SUPPORT /THRESHOLD}
\index{SUPPORT!/THRESHOLD}
\begin{verbatim}
        [CLEAN\]SUPPORT /THRESHOLD [Raw Smooth [Length [Guard]]

    Define  a  Mask from thresholding the CLEAN image. If the CLEAN image is
    3-D, the Mask will be 3-D.   `

    The method involves a first thresholding, followed by a smoothing to ex-
    tend the support and a second thresholding.

    The algorithm to define the mask is controlled by 4 parameters.
        Raw
    Initial  threshold  (in units of CLEAN image noise) under which the mask
    is set to 0.  Default is 6 sigma. The noise is taken from  the  computed
    Clean  noise  (clean%gil%rms)  if defined, or from the theoretical noise
    (dirty%gil%noise) if not.
        Smooth
    Threshold under which the Mask is set to 0 after smoothing. The  default
    is 2 sigma.
        Length
    FWHM  of  the smoothing gaussian to derive the smooth mask from the ini-
    tial mask.  The default is the Clean beam major axis.
        Guard
    Size of the guard band at edges where the mask is set to zero, in  units
    of  image  size. The default is 0.18, i.e. the mask can extends a little
    more than the inner quarter.  This is to avoid the edges where sidelobes
    aliasing occurs.

    The  computed mask can be displayed by SHOW MASK and saved by WRITE MASK
    for further use.

\end{verbatim}
\subsubsection{SUPPORT /VARIABLE}
\index{SUPPORT!/VARIABLE}
\begin{verbatim}
        [CLEAN\]SUPPORT VarName /VARIABLE

    Load the support from the Sic variable VarName. VarName can be an  array
    of the form:
             VarName[NXY,2]  Real or Double

    or a structure of the form (i.e. same as output):
             VarName%NXY             Integer or Long
             VarName%X[VarName%NXY]  Real or Double
             VarName%Y[VarName%NXY]  Real or Double


\end{verbatim}
\subsection{UV\_BASELINE}
\index{UV\_BASELINE}
\begin{verbatim}
        [CLEAN\]UV_BASELINE [Degree] [/CHANNELS Channel_List]
      [/FREQUENCY List Of Frequencies] [/RANGE Min Max [TYPE]]
      [/VELOCITY List of Velocities] [/WIDTH Width [TYPE]]

    Subtract a continuum from a line UV data set, by fitting  a baseline for
    each visibility.  The channels to be ignored in this process  (i.e.  the
    ones  including  the line emission) can be specified either by the /FRE-
    QUENCY or /VELOCITY options in combination with the /WIDTH option, or by
    a  channel  list  with /CHANNELS or a range (of channels, frequencies or
    velocities) with /RANGE.

\end{verbatim}
\subsubsection{UV\_BASELINE /CHANNELS}
\index{UV\_BASELINE!/CHANNELS}
\begin{verbatim}
        [CLEAN\]UV_BASELINE /CHANNELS Channel_List

    Channel_List must be a 1-D SIC variable containing the list of  channels
    to filter out.

\end{verbatim}
\subsubsection{UV\_BASELINE /FREQUENCY}
\index{UV\_BASELINE!/FREQUENCY}
\begin{verbatim}
        [CLEAN\]UV_BASELINE /FREQUENCY F1 [... [Fn]] [/WIDTH Width [TYPE]]

    Specify  around  which frequencies the line emission should be filtered.
    Frequencies F1 to Fn must be in MHz. The full  width  of  the  filtering
    window around every frequency can be set by option /WIDTH.  The optional
    argument TYPE indicates the type of width: FREQUENCY (in MHz),  VELOCITY
    (in  km/s)  or  CHANNEL (no unit), the default being FREQUENCY.  The de-
    fault width is the current channel width.

    Tip: it can be convenient to have a list of SIC variables containing the
    frequencies of the most intense spectral lines, e.g.
            HCO10 = 89188.52

\end{verbatim}
\subsubsection{UV\_BASELINE /RANGE}
\index{UV\_BASELINE!/RANGE}
\begin{verbatim}
        [CLEAN\]UV_BASELINE /FREQUENCY F1 [... [Fn]] /RANGE Min Max [TYPE]

    Indicate that channels between the First and Last defined by Min Max and
    Type contain line emission and should be ignored in  the  baseline  fit-
    ting.  Type can be CHANNEL, VELOCITY or FREQUENCY.

    For  type CHANNEL, Min and Max indicate offsets from Channel 1 and Chan-
    nel Nchan (the number of channels in the data set). Thus Max can be neg-
    ative:  it then indicates Last = Nchan-Max. Also Min=0 and Max=0 implies
    loading all the channels.

\end{verbatim}
\subsubsection{UV\_BASELINE /VELOCITY}
\index{UV\_BASELINE!/VELOCITY}
\begin{verbatim}
        [CLEAN\]UV_BASELINE /VELOCITY V1 [... [Vn]] [/WIDTH Width [TYPE]]

    Specify around which velocities the line emission  should  be  filtered.
    Velocities  V1  to  Vn  must be in km/s. The full width of the filtering
    window around every frequency can be set by option /WIDTH.  The optional
    argument  TYPE indicates the type of width: FREQUENCY (in MHz), VELOCITY
    (in km/s) or CHANNEL (no unit), the default being  FREQUENCY.   The  de-
    fault width is the current channel width.

\end{verbatim}
\subsubsection{UV\_BASELINE /WIDTH}
\index{UV\_BASELINE!/WIDTH}
\begin{verbatim}
        [CLEAN\]UV_BASELINE /FREQUENCY F1 [... [Fn]] /WIDTH Width [TYPE]

    Specify the full width of the window around every frequency given in the
    /FREQUENCY option.  The optional argument TYPE  indicates  the  type  of
    width:  FREQUENCY (in MHz), VELOCITY (in km/s) or CHANNEL (no unit), the
    default being FREQUENCY.  The  default  width  is  the  current  channel
    width.  The default width is the current channel width.

\end{verbatim}
\subsection{UV\_CHECK}
\index{UV\_CHECK}
\begin{verbatim}
        [CLEAN\]UV_CHECK  Beams|Nulls

      Check UV data for weight consistency.

      BEAMS
      List  which  channel  range can be processed with the same synthesized
    beam.

      NULLS
      Check if there are null visibilities with non-zero weights, and flag
      them if found.


\end{verbatim}
\subsection{UV\_COMPRESS}
\index{UV\_COMPRESS}
\begin{verbatim}
        [CLEAN\]UV_COMPRESS Nc

    Resample the UV data loaded by READ UV by averaging  NC  adjacent  chan-
    nels.  All  other UV commands except UV_RESAMPLE work on the "Resampled"
    UV table.

    The "Resampled" UV table is a simple copy of the original  one  after  a
    READ  UV command, or after a UV_RESAMPLE or UV_COMPRESS commands without
    arguments.



\end{verbatim}
\subsection{UV\_CONTINUUM}
\index{UV\_CONTINUUM}
\begin{verbatim}
        [CLEAN\]UV_CONTINUUM Naver [First Last]

    Transform the (presumably spectral line) UV data set loaded by  READ  UV
    into a "continuum" data set.

    The  transformation  selects  line  channels from First to Last, average
    them by groups of Naver contiguous channels, and concatenate the result-
    ing visibilities into a "continuum" UV table.

    For each of the (First-Last+1)/Naver channels, and for all visibilities,
    the U and V coordinates are rescaled to the  mean  observing  frequency,
    and  the  resulting  (single-channel) visibilities are concatenated into
    the "continuum" UV data set. The continuum dataset becomes  the  current
    UV data. Flagged channels are ignored: this allows to mask channels con-
    taining spectral lines (see UV_FILTER).

    Default for First and Last is 0 0, meaning all  channels  are  selected.
    Negative  value  for  Last indicate an offset from end channel (i.e. -20
    means ignore the last 20 channels).

\end{verbatim}
\subsection{UV\_FILTER}
\index{UV\_FILTER}
\begin{verbatim}
        [CLEAN\]UV_FILTER [/ZERO] [/CHANNELS Channel_List]
      [/FREQUENCY List Of Frequencies] [/RANGE Min Max [TYPE]]
      [/VELOCITY List of Velocities] [/WIDTH Width [TYPE]]

    "Filter" line emission, by  flagging  the  corresponding  channels.  The
    channels  can be specified either by the /FREQUENCY or /VELOCITY options
    in combination with the /WIDTH option, or by a channel list with  /CHAN-
    NELS or a range (of channels, frequencies or velocities) with /RANGE.

    By  default, channels to be filtered are flagged (weights becoming nega-
    tive).  Option /ZERO can be used to erase them: weights and visibilities
    are set to zero, see HELP UV_FILTER /ZERO.

\end{verbatim}
\subsubsection{UV\_FILTER /CHANNELS}
\index{UV\_FILTER!/CHANNELS}
\begin{verbatim}
        [CLEAN\]UV_FILTER [/ZERO] /CHANNELS Channel_List

    Channel_List  must be a 1-D SIC variable containing the list of channels
    to filter out.

\end{verbatim}
\subsubsection{UV\_FILTER /FREQUENCY}
\index{UV\_FILTER!/FREQUENCY}
\begin{verbatim}
        [CLEAN\]UV_FILTER [/ZERO] /FREQUENCY F1 [... [Fn]] [/WIDTH Width]

    Specify around which frequencies the line emission should  be  filtered.
    Frequencies  F1  to  Fn  must be in MHz. The full width of the filtering
    window around every frequency can be set by option /WIDTH.  The optional
    argument  TYPE indicates the type of width: FREQUENCY (in MHz), VELOCITY
    (in km/s) or CHANNEL (no unit), the default being  FREQUENCY.   The  de-
    fault width is the current channel width.

    Tip: it can be convenient to have a list of SIC variables containing the
    frequencies of the most intense spectral lines, e.g.
            HCO10 = 89188.52

\end{verbatim}
\subsubsection{UV\_FILTER /RANGE}
\index{UV\_FILTER!/RANGE}
\begin{verbatim}
        [CLEAN\]UV_FILTER [/ZERO] /RANGE Min Max [TYPE]

    Indicate that channels between the First and Last defined by Min Max and
    Type  contain  line  emission  and  should be filtered out.  Type can be
    CHANNEL, VELOCITY or FREQUENCY.

    For type CHANNEL, Min and Max indicate offsets from Channel 1 and  Chan-
    nel Nchan (the number of channels in the data set). Thus Max can be neg-
    ative: it then indicates Last = Nchan-Max. Also Min=0 and Max=0  implies
    loading all the channels.

\end{verbatim}
\subsubsection{UV\_FILTER /VELOCITY}
\index{UV\_FILTER!/VELOCITY}
\begin{verbatim}
        [CLEAN\]UV_FILTER /VELOCITY V1 [... [Vn]] [/WIDTH Width [TYPE]]

    Specify  around  which  velocities the line emission should be filtered.
    Velocities V1 to Vn must be in km/s. The full  width  of  the  filtering
    window around every frequency can be set by option /WIDTH.  The optional
    argument TYPE indicates the type of width: FREQUENCY (in MHz),  VELOCITY
    (in  km/s)  or  CHANNEL (no unit), the default being FREQUENCY.  The de-
    fault width is the current channel width.

\end{verbatim}
\subsubsection{UV\_FILTER /WIDTH}
\index{UV\_FILTER!/WIDTH}
\begin{verbatim}
        [CLEAN\]UV_FILTER [/ZERO] /FREQUENCY  F1  [...  [Fn]]  /WIDTH  Width
    [TYPE]

        [CLEAN\]UV_FILTER  [/ZERO]  /VELOCITY  V1  [...  [Vn]]  /WIDTH Width
    [TYPE]

    Specify the full width of the filtering window  around  every  frequency
    given  in  the  /FREQUENCY option or any velocity given in the /VELOCITY
    option.  The optional argument TYPE indicates the type  of  width:  FRE-
    QUENCY  (in  MHz),  VELOCITY (in km/s) or CHANNEL (no unit), the default
    being FREQUENCY.  The default width is the current channel width.

\end{verbatim}
\subsubsection{UV\_FILTER /ZERO}
\index{UV\_FILTER!/ZERO}
\begin{verbatim}
        [CLEAN\]UV_FILTER /ZERO [/CHANNELS Channel_List]
      [/FREQUENCY F1 [... [Fn]] [/VELOCITY V1 [... [Vn]]
      [/RANGE Min Max [TYPE]] [/WIDTH Width [TYPE]]

    Erase filtered channels (set weight and  visibilities  to  zero)  rather
    than  simply  flagging  them (weight set to negative value). This can be
    more convenient for further display, but is not reversible.

    By default, channels to be filtered are flagged (weights becoming  nega-
    tive,  and  data can be unflagged by UV_FLAG).  Flagged channels are ig-
    nored in the averaging process, such as UV_RESAMPLE or UV_CONT.   Howev-
    er,  as  UV_MAP (in general) uses only one channel to define the weights
    for all others, these flagged channels will nevertheless appear  in  the
    imaged data cube.

    See UV_CHECK for information about handling flagged channels and differ-
    ent beams in a single UV table.


\end{verbatim}
\subsection{UV\_FLAG}
\index{UV\_FLAG}
\begin{verbatim}
        [CLEAN\]UV_FLAG [/RESET]

    Display UV data and calls the cursor to interactively  select  a  region
    where  UV  data will be flagged (sign of weight is reversed). The /RESET
    option is used to unflag the data. UV data flagged using command UV_FLAG
    can  be  saved  on file with command WRITE UV File. Detailed information
    about the display control of the UV data may be  found  in  the  UV_SHOW
    help.

    The  user  can  control the algorithm through variables. Values of those
    variables can be checked using "EXAMINE VARIABLE".  New  values  can  be
    given using "LET VARIABLE value".

    DATE_START  [    ] The date of the first data to be flagged
    UT_START    [    ] The UT time of the first data to be flagged
    DATE_END    [    ] The date of the last data to be flagged
    UT_END      [    ] The time of the last data to be flagged
    BASELINE    [    ] Baseline to flagged
    CHANNEL     [    ] Channel range to be flagged/unflagged
    FLAG        [    ] Flag or unflag the specified time range

    Subsequent  mapping  with  UV_MAP will ignore flagged data. However, for
    multi-channel imaging, the weight column is (by default) taken from  one
    channel,  so  that  channel-based flagging will not be recognized in the
    default mode, but only in the "One Beam per Channel" mode.

\end{verbatim}
\subsubsection{UV\_FLAG DATE\_START}
\index{UV\_FLAG!DATE\_START}
\begin{verbatim}

    The date (DD-MMM-YYYY) of the first data to be flagged.

\end{verbatim}
\subsubsection{UV\_FLAG UT\_START}
\index{UV\_FLAG!UT\_START}
\begin{verbatim}

    The UT time (hh:mm:ss.ss) of the first data to be flagged.

\end{verbatim}
\subsubsection{UV\_FLAG DATE\_END}
\index{UV\_FLAG!DATE\_END}
\begin{verbatim}

    The date (DD-MMM-YYYY) of the last data to be flagged.

\end{verbatim}
\subsubsection{UV\_FLAG UT\_END}
\index{UV\_FLAG!UT\_END}
\begin{verbatim}

    The time (hh:mm:ss.ss) of the last data to be flagged.

\end{verbatim}
\subsubsection{UV\_FLAG BASELINE}
\index{UV\_FLAG!BASELINE}
\begin{verbatim}

    Baseline (ALL, 12, 13, 23, ...) to flagged.

\end{verbatim}
\subsubsection{UV\_FLAG FLAG}
\index{UV\_FLAG!FLAG}
\begin{verbatim}

    Flag (T) or unflag (F) the specified time range.

\end{verbatim}
\subsubsection{UV\_FLAG CHANNEL}
\index{UV\_FLAG!CHANNEL}
\begin{verbatim}

    Channel range to be flagged/unflagged.

\end{verbatim}
\subsection{UV\_MAP}
\index{UV\_MAP}
\begin{verbatim}
        [CLEAN\]UV_MAP [CenterX CenterY UNIT  [Angle]]  [/FIELDS  FieldList]
    [/TRUNCATE Percent]

    Compute  a  dirty  map  and  beam from a UV data. UV data must have been
    loaded from a UV table by command "READ UV File". UV_MAP processes  sin-
    gle fields as well as Mosaics.

    The user can control the algorithm through SIC variables. New values can
    be given using "LET VARIABLE value". For ease of use, and whenever it is
    possible,  a sensible value of each parameter will automatically be com-
    puted from the context if the value of the corresponding variable is set
    to  its default value, i.e. zero value and empty string. A few variables
    are initialized to "reasonable" values.

        [CLEAN\]UV_MAP ?
    Will list all MAP_* variables controlling the UV_MAP parameters

\end{verbatim}
\subsubsection{UV\_MAP Mosaics}
\index{UV\_MAP!Mosaics}
\begin{verbatim}
        [CLEAN\]UV_MAP [CenterX  CenterY  UNIT  [Angle]]  /FIELDS  FieldList
    [/TRUNCATE Percent]

    The  UV  data  can be a Mosaic UV table. In this case, UV_MAP will image
    the Mosaic, using appropriate primary beam size and truncation level.

    By default, the primary beam size is taken from  the  telescope  parame-
    ters,  either  as found in the telescope section of the UV table and the
    observing frequency (e.g. for ALMA it uses 1.13  Lambda/D).  If  absent,
    the  telescope section information can be added by command SPECIFY TELE-
    SCOPE.

    The truncation level is taken from variable  MAP_TRUNCATE  or  from  the
    /TRUNCATE option argument.


\end{verbatim}
\subsubsection{UV\_MAP /FIELDS}
\index{UV\_MAP!/FIELDS}
\begin{verbatim}
        [CLEAN\]UV_MAP  [CenterX  CenterY  UNIT  [Angle]]  /FIELDS FieldList
    [/TRUNCATE Percent]

    For a Mosaic, only image the fields specified in a 1-D Integer  variable
    FieldList.   SHOW  FIELDS  will highlight the list of fields selected by
    this command.

\end{verbatim}
\subsubsection{UV\_MAP /TRUNCATE}
\index{UV\_MAP!/TRUNCATE}
\begin{verbatim}
        [CLEAN\]UV_MAP [CenterX  CenterY  UNIT  [Angle]]  /FIELDS  FieldList
    /TRUNCATE Percent

    For  a Mosaic, truncate the primary beam to the specified level (in per-
    cent). SHOW FIELDS will use this level to show the beams.   The  default
    is to use MAP_TRUNCATE.

\end{verbatim}
\subsubsection{UV\_MAP Variables:}
\index{UV\_MAP!Variables:}
\begin{verbatim}
                [CLEAN\]UV_MAP ?
        Will list all MAP_* variables controlling the UV_MAP parameters.

    The  list  of  control  variables  is (by alphabetic order, with the old
    names used by Mapping on the right)
    New names       [   unit]       -- Description --    % Old Name
    MAP_BEAM_STEP   [       ]  Number of channels per single dirty beam
    MAP_CELL        [ arcsec]  Image pixel size
    MAP_CENTER      [ string]  RA, Dec of map center, and Position Angle
    MAP_CONVOLUTION [       ]  Convolution function    % CONVOLUTION
    MAP_FIELD       [ arcsec]  Map field of view
    MAP_POWER       [       ]  Maximum exponent of 3 and 5 allowed in MAP_SIZE
    MAP_PRECIS      [       ]  Fraction of pixel tolerance on beam matching
    MAP_ROBUST      [       ]  Robustness factor        % UV_CELL[2]
    MAP_ROUNDING    [       ]  Precision of MAP_SIZE
    MAP_SIZE        [       ]  Number of pixels
    MAP_TAPEREXPO   [       ]  Taper exponent           % TAPER_EXPO
    MAP_TRUNCATE    [      %]  Mosaic truncation level
    MAP_UVCELL      [      m]  UV cell size             % UV_CELL[1]
    MAP_UVTAPER     [m,m,deg]  Gaussian taper           % UV_TAPER
    MAP_VERSION     [       ]  Code version (0 new, -1 old)

    NAME is no longer used, and WEIGHT_MODE is obsolete.
    MAP_RA          [  hours]  RA of map center
    MAP_DEC         [    deg]  Dec of map center
    MAP_ANGLE       [    deg]  Map position angle
    MAP_SHIFT       [Yes/No ]  Shift phase center
    are obsolescent, superseded by MAP_CENTER. They are  provided  only  for
    compatibility with older scripts.

\end{verbatim}
\subsubsection{UV\_MAP MAP\_BEAM\_STEP}
\index{UV\_MAP!MAP\_BEAM\_STEP}
\begin{verbatim}

      MAP_BEAM_STEP   Integer

    Number of channels per synthesized beam plane.

    Default  is 0, meaning only 1 beam plane for all channels.  N (>0) indi-
    cates N consecutive channels will share the same dirty beam.

    A value of -1 can be used to compute the number  of  channels  per  beam
    plane  to ensure the angular scale does not deviate more than a fraction
    of the map cell at the map edge. This fraction is controlled by variable
    MAP_PRECIS (default 0.1)

\end{verbatim}
\subsubsection{UV\_MAP MAP\_CELL}
\index{UV\_MAP!MAP\_CELL}
\begin{verbatim}

          MAP_CELL[2]    Real

    The  map  pixel size [arcsec]. It is recommended to use identical values
    in X and Y.  A sampling of at least 3 pixel per beam is  recommended  to
    ease  the deconvolution. Enter 0,0 to let the task find the best values.

\end{verbatim}
\subsubsection{UV\_MAP MAP\_CENTER}
\index{UV\_MAP!MAP\_CENTER}
\begin{verbatim}

          MAP_CENTER     Character String

    Specify the Map center and orientation in the same way as the  arguments
    of UV_MAP.

\end{verbatim}
\subsubsection{UV\_MAP MAP\_CONVOLUTION}
\index{UV\_MAP!MAP\_CONVOLUTION}
\begin{verbatim}

        MAP_CONVOLUTION    Integer

    Select the desired convolution function for gridding  in  the  UV  plane
    Choices are
            0    Default (currently 5)
            1    Boxcar
            2    Gaussian
            3    Sin(x)/x
            4    Gaussian * Sin(x)/x
            5    Spheroidal
    Spheroidal functions is the optimal choice. So  we  strongly  discourage
    use of any other convolution function, which are here for tests only.

\end{verbatim}
\subsubsection{UV\_MAP MAP\_FIELD}
\index{UV\_MAP!MAP\_FIELD}
\begin{verbatim}

      MAP_FIELD[2]     Real

    Field  of  view  in  X and Y in arcsec.  The field of view MAP_FIELD has
    precedence over the number of pixels MAP_SIZE to define the  actual  map
    size when both are non-zero.

\end{verbatim}
\subsubsection{UV\_MAP MAP\_POWER}
\index{UV\_MAP!MAP\_POWER}
\begin{verbatim}

          MAP_POWER[2]     Integer

    Maximum  exponent  of  3  and  5 allowed in automatic guess of MAP_SIZE.
    MAP_SIZE is decomposed in 2^k 3^p 5^q, and p and q must be less or equal
    to MAP_POWER.

    Default is 0: MAP_SIZE is just a power of 2. A value of 1 allows approx-
    imation of any map size to 20 %, while a value of 2 allows 10 % approxi-
    mation.  Fast Fourier Transform are slightly slower with powers of 3 and
    5, but limiting the map size can gain a  lot  in  the  Cleaning  process
    (which can scale as MAP_SIZE^4).


\end{verbatim}
\subsubsection{UV\_MAP MAP\_PRECIS}
\index{UV\_MAP!MAP\_PRECIS}
\begin{verbatim}

      MAP_PRECIS    Real

    Maximum  mismatch in pixel at map edge between the true synthesized beam
    (which would have been computed using the exact channel  frequency)  and
    the  computed  synthesized beam with  the mean frequency of the channels
    sharing the same beam. This is used (with the actual image size) to  de-
    rive  the  actual number of channels which can share the same beam, i.e.
    the effective value of MAP_BEAM_STEP when MAP_BEAM_STEP is -1.

    Default is 0.1

\end{verbatim}
\subsubsection{UV\_MAP MAP\_ROBUST}
\index{UV\_MAP!MAP\_ROBUST}
\begin{verbatim}

      MAP_ROBUST     Real

    Robust weighting factor. A number between 0 and +infty.

    Robust weighting gives the natural weight  to  UV  cells  whose  natural
    weight  is  lower  than  a  given threshold. In contrast, if the natural
    weight of the UV cell is larger than this threshold, the weight  is  set
    to  this  (uniform) threshold. The UV cell size is defined by MAP_UVCELL
    and the threshold value is in MAP_ROBUST.

    0 means natural weighting, which is optimal for point sources.  The  Ro-
    bust  weighting factor controls the resolution: better resolution is ob-
    tained for small values (at the expense of noise), resolution  approach-
    ing  the natural weighting scheme for large values.  Larger UV cell size
    give higher angular resolution (but again more noise).

    MAP_ROBUST around .5 to 1 is a good compromise  between  noise  increase
    and angular resolution.

\end{verbatim}
\subsubsection{UV\_MAP MAP\_ROUNDING}
\index{UV\_MAP!MAP\_ROUNDING}
\begin{verbatim}

      MAP_ROUNDING     Real

    Maximum  error  between  optimal size (MAP_FIELD / MAP_CELL) and rounded
    (as a power of 2^k 3^p 5^q) MAP_SIZE to round by  floor  (thus  limiting
    the  field of view), instead of ceiling (which guarantees a larger field
    of view, but leads to bigger images).

    Default is 0.05.


\end{verbatim}
\subsubsection{UV\_MAP MAP\_SHIFT}
\index{UV\_MAP!MAP\_SHIFT}
\begin{verbatim}

      MAP_SHIFT        Logical

    Obsolescent, superseded by MAP_CENTER, or the UV_MAP arguments.

    Logical variable indicating whether map center (i.e. phase tracking cen-
    ter) or orientation should be changed.

\end{verbatim}
\subsubsection{UV\_MAP MAP\_SIZE}
\index{UV\_MAP!MAP\_SIZE}
\begin{verbatim}

      MAP_SIZE[2]      Integer

    Number of pixels in X and Y. It should preferentially be a power of two,
    (although  this  is  not  strictly  required)  to  speed-up   the   FFT.
    MAP_SIZE*MAP_CELL should be at least twice the size of the field-of-view
    (primary beam size for a single field). Enter 0,0  to  let  the  command
    find a sensible map size.

    MAP_SIZE is not used if MAP_FIELD is non zero.

    Odd values are forbidden.

    Default is 0,0, i.e. UV_MAP will guess the most appropriate values which
    depend on MAP_ROUNDING and MAP_POWER.


\end{verbatim}
\subsubsection{UV\_MAP MAP\_TAPEREXPO}
\index{UV\_MAP!MAP\_TAPEREXPO}
\begin{verbatim}

      MAP_TAPEREXPO    Real

    Taper exponent. The default is 2 (indicating a Gaussian) but smoother or
    sharper  functions  can be used. 1 would give an Exponential, 4 would be
    getting close to square profile...

\end{verbatim}
\subsubsection{UV\_MAP MAP\_TRUNCATE}
\index{UV\_MAP!MAP\_TRUNCATE}
\begin{verbatim}

      MAP_TRUNCATE    Real

    Mosaic truncation level in PerCent.  Default value is 0.2. Current value
    can be overriden by option /TRUNCATE in commands UV_MAP or PRIMARY.

\end{verbatim}
\subsubsection{UV\_MAP MAP\_UVTAPER}
\index{UV\_MAP!MAP\_UVTAPER}
\begin{verbatim}

      MAP_UVTAPER[3]  Real

    Parameters of the tapering function (Gaussian if MAP_TAPEREXPO = 2): ma-
    jor axis at 1/e level [m], minor axis at 1/e level [m], and position an-
    gle [deg].

\end{verbatim}
\subsubsection{UV\_MAP MAP\_UVCELL}
\index{UV\_MAP!MAP\_UVCELL}
\begin{verbatim}

      MAP_UVCELL   Real

    UV  cell  size for robust weighting [m].  Should be of the order of half
    the dish diameter (7.5 m for PdBI), or smaller or even larger.  It  con-
    trols the beam shape in Robust weighting.

\end{verbatim}
\subsubsection{UV\_MAP MAP\_VERSION}
\index{UV\_MAP!MAP\_VERSION}
\begin{verbatim}

      MAP_VERSION  Integer

    [EXPERT Only] Code indicating which version of the UV_MAP and UV_RESTORE
    algorithm  should  be  used.  0  is  optimal.  -1  is  the  "historical"
    (pre-2016)  version. 1 is an intermediate version used during multi-fre-
    quency beams development.

\end{verbatim}
\subsubsection{UV\_MAP MCOL}
\index{UV\_MAP!MCOL}
\begin{verbatim}

      MCOL[2]   Integer

    First and Last channel to image.  Values  of  0  mean  imaging  all  the
    planes.

\end{verbatim}
\subsubsection{UV\_MAP WCOL}
\index{UV\_MAP!WCOL}
\begin{verbatim}

      WCOL      Integer

    [Obsolescent]  The  channel from which the weight should be taken.  WCOL
    set to 0 means using a default channel. WCOL has no real meaning in  all
    cases where more than one beam is computed for all channels.


\end{verbatim}
\subsubsection{UV\_MAP Old\_Names:}
\index{UV\_MAP!Old\_Names:}
\begin{verbatim}
     NAME        [       ]  Label of the dirty image and beam plots
     UV_TAPER    [m,m,deg]  UV-apodization by convolution with a Gaussian
     WEIGHT_MODE [       ]  Weighting mode (NA|UN)
     UV_CELL     [m, ??  ]  UV cell size and threshold for Robust weighting
     MAP_FIELD   [ arcsec]  Map field of view
     MAP_CELL    [ arcsec]  Map cell size
     MAP_SIZE    [ pixels]  Map size in pixels (if MAP_FIELD is zero)
     MCOL        [       ]  First and Last channel to map
     WCOL        [       ]  Channel from which the weights are taken
     CONVOLUTION [       ]  Convolution function (5)
     UV_SHIFT    [       ]  Change the map phase center or map orientation?
     MAP_RA      [       ]  RA of map phase center
     MAP_DEC     [       ]  Dec of map phase center
     MAP_ANGLE   [    deg]  Map position angle
     MAP_BEAM_STEP [     ]  Number of channels per synthesized beam plane

\end{verbatim}
\subsubsection{UV\_MAP convolution}
\index{UV\_MAP!convolution}
\begin{verbatim}

      Older variable name for MAP_CONVOLUTION

\end{verbatim}
\subsubsection{UV\_MAP map\_angle}
\index{UV\_MAP!map\_angle}
\begin{verbatim}

      MAP_ANGLE      Real

    Position  Angle of the direction which will become the apparent North in
    the map. Used only if UV_SHIFT is YES.

    Superseded by MAP_CENTER.

\end{verbatim}
\subsubsection{UV\_MAP map\_dec}
\index{UV\_MAP!map\_dec}
\begin{verbatim}

      MAP_DEC     Real

    Dec of map center. Used only if UV_SHIFT is YES.

    Superseded by MAP_CENTER.

\end{verbatim}
\subsubsection{UV\_MAP map\_ra}
\index{UV\_MAP!map\_ra}
\begin{verbatim}

      MAP_RA      Real

    RA of map center. Used only if UV_SHIFT is YES.

    Superseded by MAP_CENTER.

\end{verbatim}
\subsubsection{UV\_MAP uv\_cell}
\index{UV\_MAP!uv\_cell}
\begin{verbatim}

      Older variables for MAP_UVCELL (uv
\end{verbatim}
\subsubsection{UV\_MAP uv\_shift}
\index{UV\_MAP!uv\_shift}
\begin{verbatim}

      Older variable name of MAP_SHIFT (this one is also obsolescent)

\end{verbatim}
\subsubsection{UV\_MAP uv\_taper}
\index{UV\_MAP!uv\_taper}
\begin{verbatim}

      Older variable name of MAP_UVTAPER

\end{verbatim}
\subsubsection{UV\_MAP taper\_expo}
\index{UV\_MAP!taper\_expo}
\begin{verbatim}

      Older variable name for MAP_TAPEREXPO

\end{verbatim}
\subsubsection{UV\_MAP weight\_mode}
\index{UV\_MAP!weight\_mode}
\begin{verbatim}

      weightde      Character

    Weighting mode: Natural (optimum in  terms  of  sensitivity)  or  robust
    (usually lower sidelobes and higher spatial resolution) weighting.  This
    was needed in Mapping to toggle between Natural  and  Robust  weighting,
    while IMAGER does that based on MAP_ROBUST value.







\end{verbatim}
\subsection{UV\_RESAMPLE}
\index{UV\_RESAMPLE}
\begin{verbatim}
        [CLEAN\]UV_RESAMPLE [Nc Ref Val Inc]

    Resample  the  UV  data loaded by READ UV on a different velocity scale.
    All other UV commands except UV_COMPRESS work on the "Resampled" UV  ta-
    ble.
         Nc   new number of channels
         Ref  New reference pixel
         Val  New velocity at reference pixel
         Inc  Velocity increment
    Any argument can be set to * for an automatic determination based on the
    values of the other arguments. The automatic determination  of  NC  pre-
    serves the velocity coverage.

    The  "Resampled"  UV  table is a simple copy of the original one after a
    READ UV command, or after a UV_RESAMPLE or UV_COMPRESS  command  without
    arguments.

\end{verbatim}
\subsection{UV\_RESIDUAL}
\index{UV\_RESIDUAL}
\begin{verbatim}
        [CLEAN\]UV_RESIDUAL [Niter]

      Subtract  the  Niter  first (default all) Clean Components from the UV
    data.  The residual UV data can be written by WRITE UV,  and  imaged  by
    UV_MAP.

\end{verbatim}
\subsection{UV\_RESTORE}
\index{UV\_RESTORE}
\begin{verbatim}
        [CLEAN\]UV_RESTORE

    Create  a  Clean image from the current UV data set and the Clean Compo-
    nent list.  The Clean Components are subtracted from the  UV  data  set,
    and  these  residuals are gridded and Fourier transformed to compute the
    Residual image. This Residual image is added to the Gaussian  beam  con-
    volved  image of the sum of Clean components. The results are similar to
    those of MX, since only the residual are aliased.

    This command can be used after HOGBOM, CLARK, MULTI, SDI or MX (although
    it is pointless after MX), but not MRC which has no notion of Clean Com-
    ponents.

\end{verbatim}
\subsection{UV\_REWEIGHT}
\index{UV\_REWEIGHT}
\begin{verbatim}
        [CLEAN\]UV_REWEIGHT Scale

    Scale the weights of the current UV data with the defined Scale  factor.
    Can  be  used to patch e.g. JVLA data files which may have only relative
    weights, not absolute values indicating the noise.

\end{verbatim}
\subsection{UV\_SHIFT}
\index{UV\_SHIFT}
\begin{verbatim}
        [CLEAN\]UV_SHIFT  [CenterX CenterY UNIT [Angle]]

    Shift the current UV table (single field or mosaic) to  a  common  phase
    center.  If no argument is specified, the string contained in MAP_CENTER
    is used instead.

    Although UV_MAP also provides a direct possibility to re-center the  im-
    age  on a specified projection (phase) center, the modified visibilities
    cannot be saved on file. It is sometimes required to do this for specif-
    ic use. UV_SHIFT provides this possibility, and the shifted UV table can
    be written using command WRITE UV.

    UV_DEPROJECT includes the UV_SHIFT capabilities, but also has additional
    functions.

\end{verbatim}
\subsection{UV\_SORT}
\index{UV\_SORT}
\begin{verbatim}
        [CLEAN\]UV_SORT TIME|BASE

    Sort and transpose the UV data set, loaded by command READ UV File.  Or-
    der is either TIME for Time-Baseline ordering,  BASE  for  Baseline-Time
    ordering.  The sorted UV data is then available in variable UVS for fur-
    ther plotting.

    NOTE: This is only done in an internal buffer.  WRITE  UV  will  **NOT**
    write this sorted, transposed, buffer.

\end{verbatim}
\subsection{UV\_STAT}
\index{UV\_STAT}
\begin{verbatim}
        [CLEAN\]UV_STAT CELL|HEADER|SETUP|TAPER|WEIGHT  [Step Start]

    UV_STAT  allows  the astronomer to select the best weighting and imaging
    parameters according to its personal trade off between  angular  resolu-
    tion, sensitivity and field of view.

      Default is HEADER+SETUP

\end{verbatim}
\subsubsection{UV\_STAT CELL}
\index{UV\_STAT!CELL}
\begin{verbatim}
        [CLEAN\]UV_STAT CELL [Step Start]

    Predict the synthesized beam, expected noise level, and recommended pix-
    el size for different values of  the uv cell  size  for  current  robust
    weighting parameters.

\end{verbatim}
\subsubsection{UV\_STAT HEADER}
\index{UV\_STAT!HEADER}
\begin{verbatim}
        [CLEAN\]UV_STAT HEADER

    Display  a  brief  summary of the UV data: number of antennas, observing
    dates, baseline ranges,  spectroscopic information.

\end{verbatim}
\subsubsection{UV\_STAT SETUP}
\index{UV\_STAT!SETUP}
\begin{verbatim}
        [CLEAN\]UV_STAT SETUP

    Display recommended values for the  imaging:  image  size,  pixel  size,
    field of view, largest angular scale, etc...

\end{verbatim}
\subsubsection{UV\_STAT TAPER}
\index{UV\_STAT!TAPER}
\begin{verbatim}
        [CLEAN\]UV_STAT TAPER [Step Start]

    For  TAPER,  beam sizes and noise level (in flux and brightness) will be
    computed for 9 different tapers (from Start to  Start*Step^9).   Default
    value  for  Step  is sqrt(2), Default value for Start is 50 m. Weighting
    mode, UV cell size  and  "robust"  parameter  are  taken  from  variable
    UV_CELL (i.e. one can combine Robust weighting and Tapering).

\end{verbatim}
\subsubsection{UV\_STAT WEIGHT}
\index{UV\_STAT!WEIGHT}
\begin{verbatim}
        [CLEAN\]UV_STAT WEIGHT [Step Start]

    For  WEIGHT, beam sizes and noise level (in flux and brightness) will be
    computed for 9 different "robust" weighting parameters  (from  Start  to
    Start*Step^9). Default value for Step is sqrt(10), and default value for
    Start is derived to center the "robust" parameter values around  1.   UV
    cell  size  is  taken  from variable UV_CELL[1], and Taper is taken from
    variable UV_TAPER.

\end{verbatim}
\subsection{UV\_TIME}
\index{UV\_TIME}
\begin{verbatim}
        [CLEAN]UV_TIME [Time] [/Weight Wcol]

    Average in time the current UV data set to reduce the number of visibil-
    ities.  Time must be in seconds. If not specified, an automatic guess is
    performed based on antenna size and baseline lengths.

\end{verbatim}
\subsubsection{UV\_TIME /WEIGHT}
\index{UV\_TIME!/WEIGHT}
\begin{verbatim}
        [CLEAN]UV_TIME Time /WEIGHT Wcol

    Select the weight column. Default is 0.

\end{verbatim}
\subsection{UV\_TRUNCATE}
\index{UV\_TRUNCATE}
\begin{verbatim}
        [CLEAN]UV_TRUNCATE Max [Min]

    Truncate the UV data, by removing baselines out of the  specified  range
    Min (default 0) and Max (in meter).

\end{verbatim}
\subsection{VIEW}
\index{VIEW}
\begin{verbatim}
        [CLEAN\]VIEW Variable [FirstPlane [LastPlane]]

    where
        is  the  internal  buffer  to  be  plotted (BEAM, CCT, CLEAN, DIRTY,
    FIELDS, MASK, etc.. ), or any Sic Image variable.


        [CLEAN\]VIEW ?

    will list the names of recognized keywords. If Variable is  not  one  of
    the recognized keywords, but an existing Image variable, this image will
    be displayed.

    FirstPlane and LastPlane
        are optional arguments to restrict the range of channels to be plot-
    ted  (default:  planes between variables FIRST and LAST). If only First-
    Plane is specified, SHOW only that plane.

    The plot is done by procedure p_view_cct for Clean  Components  CCT  and
    p_view_map  for data cubes. For data cubes, the behaviour is the same as
    GO VIEW with NAME set to the appropriate buffer name.

    VIEW is also controlled by a set of variables, available  in  the  View%
    global structure

\end{verbatim}
\subsubsection{VIEW Variables}
\index{VIEW!Variables}
\begin{verbatim}

    User control variables
    view%expand       0.8 Character and Tick expansion factor
    view%contour      NO  Contour the current channel map
    view%movie        0   Elapsed time in seconds for a movie (0 means guess)
    view%side         YES Display values in side Window
    view%status%rima  NO  Relative coordinates in Images
    view%status%rspe  NO  Relative coordinates in Spectra

    Returned  values  are  found  in view%current structure. The ZOOM action
    produces a sub-cube named EXTRACTED.  The SLICE action  produces  a  2-D
    data  set  named  SLICE.  As any other SIC Image variable, EXTRACTED and
    SLICE can be written on disk using command WRITE.

\end{verbatim}
\subsubsection{VIEW Keys}
\index{VIEW!Keys}
\begin{verbatim}

    Position-independent actions:
      Press E key: EXIT loop
      Press H key: HELP display
      Press M key: MOVIE
      Press P key: PRINT plot in "ha" subdirectory
      Press Q key: QUIT loop
      Press X key: EXTRACT on disk current zoomed region
      Press N key: Toggle  Narrow - Wide mode

    Cursor on images:
      Left  clic: Display spectrum at pointed position
      Right clic: Define a polygon
      Press C key: COORDINATES toggled from absolute to relative and back
      Press S key: SLICE definition (velocity-position)
      Press K key: KILL   pointed pixel
      Press U key: UNKILL pointed pixel
      Press Z key: ZOOM defined spatial region
      Press B key: BACK to full field of view
      Press V key: Display map coordinates at current position (the associated b
      Press W key: WRITE Integrated Area image
    Cursor on spectra:
      Left clic in Selected   Spectrum: Display selected velocity channel
      Left clic in Integrated Spectrum: Define velocity range
      Press C key: COORDINATES toggled from freq/velo to channels and back
      Press Z key: ZOOM defined velocity region
      Press B key: BACK to full velocity range
      Press W key: WRITE Integrated (and current) Spectrum
    Cursor outside plots:
      Press B key: BACK to full velocity range AND full field of view

    ZOOM action produces a sub-cube named EXTRACTED.  SLICE action  produces
    a  2-D  data set named SLICE. As any other SIC Image variable, EXTRACTED
    and SLICE can be written on disk using command WRITE.


\end{verbatim}
\subsection{WRITE}
\index{WRITE}
\begin{verbatim}
        [CLEAN\]WRITE Name File [/APPEND] [/RANGE Start End Kind] [/REPLACE]
    [/TRIM]

    WRITE  the  buffer  specified  by  Name  (UV, CGAINS, and BEAM, PRIMARY,
    DIRTY, CLEAN, RESIDUAL, SUPPORT, CCT, SKY) onto a File.  Default  exten-
    sions  are  .uvt  for  UV and CGAINS and .beam, .lmv, .lobe, .lmv-clean,
    .lmv-res, .pol, .cct, and lmv-sky respectively for  the  next  ones.  If
    Name does not refer to a known buffer, but to a SIC Image variable, this
    variable is written. The default extension is then .gdf.

    For UV data, the flagged data are written by default. They may be  omit-
    ted using the /TRIM option.

    WRITE  *  File can be used to write all modified image-like buffers (not
    the UV tables) under a common File name. This typically  include  .beam,
    .lmv  and  .lmv-clean, but also the .lmv-sky file if the PRIMARY command
    has been used after deconvolution.

\end{verbatim}
\subsubsection{WRITE /RANGE}
\index{WRITE!/RANGE}
\begin{verbatim}
        [CLEAN\]WRITE Buffer File /RANGE Start End Kind [/REPLACE]

    A range of image planes can be specified through /RANGE option. Kind  is
    the unit of the Start-End values: CHANNEL, VELOCITY, or FREQUENCY.

    Restrictions:
    - The option is still experimental
    - The Buffer name must be specified (* is not valid here)
    - This does not apply to UV data.

\end{verbatim}
\subsubsection{WRITE /APPEND}
\index{WRITE!/APPEND}
\begin{verbatim}
        [CLEAN\]WRITE Buffer File /APPEND [/RANGE Start End Kind]

    EXPERIMENTAL: The selected channels are appended to an existing file.

\end{verbatim}
\subsubsection{WRITE /REPLACE}
\index{WRITE!/REPLACE}
\begin{verbatim}
        [CLEAN\]WRITE Buffer File /REPLACE [/RANGE Start End Kind]

    EXPERIMENTAL: The selected channels are replaced in an existing file.

\end{verbatim}
\subsubsection{WRITE /TRIM}
\index{WRITE!/TRIM}
\begin{verbatim}
        [CLEAN\]WRITE UV File /TRIM

    Remove the flagged visibilities while writing.



\end{verbatim}
