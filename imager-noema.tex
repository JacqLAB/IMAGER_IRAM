\documentclass[11pt]{article}

\usepackage{graphicx}
% \usepackage{pdfpages}
\usepackage{fancyhdr}

% GILDAS specific definitions
\include{gildas-def}

\makeindex{}

\makeatletter
\textheight 625pt
\textwidth 460pt
\oddsidemargin 0pt
\evensidemargin 0pt
\marginparwidth 50pt
\topmargin 0pt
\brokenpenalty=10000

% Makes fancy headings and footers.
\pagestyle{fancy}

\fancyhf{} % Clears all fields.

\lhead{\scshape Guilloteau, 2016}
\rhead{\scshape \nouppercase{\rightmark}}   %
\fancyfoot[C]{\bfseries \thepage{}}

\renewcommand{\headrulewidth}{0pt}    %
\renewcommand{\footrulewidth}{0pt}    %

\renewcommand{\sectionmark}[1]{%
  \markright{\thesection. \MakeLowercase{#1}}}
\renewcommand{\subsectionmark}[1]{}

\fancypagestyle{firststyle}
{
   \fancyhf{}
   \fancyhead[C]{Original version at \tt http://iram-institute.org/medias/uploads/imaging-noema.pdf}
   \fancyfoot[C]{\bfseries \thepage{}}
}

%%%%%%%%%%%%%%%%%%%%%%%%%%%%%%%%%%%%%%%%%%%%%%%%%%%%%%%%%%%%%%%%%%%%%%%%%%%%%%%%

\title{IRAM Memo 2016-?\\[3\bigskipamount]
  IMAGER: Concepts and usage}
\author{S. Guilloteau$^{1}$\\
 \vspace{0.3cm}
  1. LAB (Bordeaux)\\
 \vspace{0.5cm}
{\begin{tabular}{lll}
  14-Jul-2016  -- &  version 1.0 & \\
  09-Sep-2016  --  & version 1.1  & -- Add "On Going Work" section \\
  17-Nov-2016  --  & version 1.2  & -- Add the UV\_SHORT task \\
  17-Oct-2017  --  & Version 1.3  & -- Clean up some variables - Zero spacings \\
  23-Apr-2018  --  & Version 1.4  & -- Mention SELFCAL and UV\_SHORT command
 \end{tabular}}
 }
\makeindex{}

\begin{document}

\maketitle

%\begin{rawhtml}
%  Note: this is the on-line version of the ``GREG 2012 UV table format''
%  document for Gildas users and programmers.
%  A <A HREF="../../pdf/greg-uvt-v2.pdf"> PDF version</A>
%  is also available.
%
%  Related information is available in:
%  <UL>
%  <LI> <A HREF="../greg-html/greg.html"> GREG:</A> Grenoble Graphics
%  <LI> <A HREF="../sic-html/sic.html"> SIC:</A> Simple Interpretor of Commands
%  </UL>
%\end{rawhtml}

\begin{abstract}
  With the advent of ALMA and NOEMA, the interferometers deliver much
  larger data sets than initially anticipated for GILDAS software.
  This document describes a new imaging program called IMAGER
  1) Wide bandwidth data sets, 2) Mosaics,  3) Short spacings,
  and 4) Self-calibration, in a (much) simpler and faster way than MAPPING.

  Related documents: \emph{Imager documentation, Mapping documentation}
\end{abstract}

\newpage
\tableofcontents{}

\section{Objectives}

The main goals of the IMAGER are
\begin{enumerate}
\item to offer a proper implementation of imaging in case of
wide relative bandwidth, where the natural angular resolution
changes with frequency.
\item to implement a simpler (and incidentally faster) scheme to process Mosaics,
including short spacings from single dish data
\item to minimize image sizes 
\item to minimize processing time by using parallel
programming as much as possible and reducing Input/Output to the strict minimum.
\item to simplify user interfaces, by providing sensible defaults.
\item to take advantage of improved capabilities of NOEMA, by offering new
tools like self-calibration or wide bandwidth analysis.
\end{enumerate}

\section{Principle}

\subsection{User viewpoint}
For the user, IMAGER reduces the number of actions to the strict minimum.
The imaging sequence is always the same
\begin{verbatim}
! Read data
READ UV MyData.uvt /RANGE Min Max Type
! here, optionally use UV_TIME, UV_COMPRESS, UV_BASELINE
! or UV_FILTER, UV_CONT to filter lines or remove continuum
UV_MAP          ! Image
CLEAN           ! Deconvolve
! here, optionally use UV_RESTORE
VIEW CLEAN      ! or SHOW CLEAN     Have a look at the result
WRITE * MyData  ! Write result if OK
\end{verbatim}
IMAGER recognizes whether the UV table is for a single field or a 
mosaic. The only difference between the single field and mosaic cases 
is that IMAGER yields a Sky brightness image for Mosaics, but does not 
automatically correct for primary beam attenuation for Single fields.

For UV\_MAP, imaging parameters are controlled by the MAP\_* variables. 
Deconvolution parameters are controlled by CLEAN\_* variables.
Suitable defaults are provided, so that only specific cases should
require customization by the user.  Parameters can be list by
commands UV\_MAP ?  and CLEAN ?.



\section{Mosaics and Short Spacings}

\subsection{Multi-Fields UV table}
One major step to allow this simplification has been the possibility offered by
the GDF UV data format version 2 to allow ``extra'' columns, i.e. additional data for
each visbility. We use it here to store several possible quantities
\begin{itemize}
\item  column of type \texttt{code\_uvt\_id} contains a real (actually, only
integer values are to be found) representing a numerical identification number. This
typically handles the field number, for example, but the column type is generic.
\item  columns of types \texttt{code\_uvt\_loff} and \texttt{code\_uvt\_moff}
contain the phase offsets (from the phase center)
\item  columns of types \texttt{code\_uvt\_xoff} and \texttt{code\_uvt\_yoff}
contain the pointing offsets (from the pointing center).
\end{itemize}
If the phase offset columns are present, but the pointing offset columns
are not present, the pointing centers are assumed to be identical to the 
phase centers. This naturally happens for data taken by all interferometers 
(NOEMA, ALMA, JVLA, ...).

\subsubsection{Importing from UVFITS}

To import UVFITS files, some trick is needed because the UVFITS format 
provided by other packages (mostly CASA) has some specific structure 
and implicit assumptions that do not map directly to the GILDAS UV data 
format. Specifically, UVFITS provides for each visibility an ID number 
corresponding to the source coordinates given in a  FITS  binary table 
named "SU".

Thus, importing cannot be done through a simple command, and
is done only through the \texttt{fits\_to\_uvt.map} script
The UV table is temporarily created with two additional columns, one of which is holding
the source ID number, and after the SU Table is read, these two columns are filled with
the phase offset centers, and labelled of type \texttt{code\_uvt\_loff} and
\texttt{code\_uvt\_moff} respectively.
As UVFITS provides absolute coordinates, and not offset, the first field center is arbitrarily
taken as the reference for offsets. This may change for a more convenient value
(e.g. the centroid) later. If only one field is found, the columns are simply labelled
as type \texttt{code\_uvt\_id} so that they remain ignored in any further processing.

\textbf{NOTE: }\textit{the UVFITS format allows for cases where the SU tables also
provide pointing centers. So far, this has not been implemented in any known FITS writer,
and CASA does not support this possibility. However, once GILDAS UV tables have been
converted to a common phase center (Mosaic UV tables with only pointing offsets, see
below), this possibility may be required to export simply the data into UVFITS format for
archival. The current code does not allow this yet.}


\subsubsection{Importing from CLIC}

CLIC can produce multi-field UV tables. This is done using command
\begin{verbatim}
  TABLE [MyTable [Old|New]] /MOSAIC
\end{verbatim}
which adds the phase offset columns and automatically avoids checking
pointing and phase offsets in the selected observations.

Note: the /MOSAIC option replaces the experimental /POSITION option.
The same result can also be obtained for spectral line
data only (SET SELECTION LINE) through the  command
\begin{verbatim}
  SG_TABLE [MyTable [Old|New]] /ADD L_PHASE_OFF M_PHASE_OFF /NOCHECK POINT PHASE
\end{verbatim}

\subsubsection{Importing from separate fields.}

It is also possible to merge UV tables corresponding to separate fields into a
multi-field UV table by using task UV\_MOSAIC. The same task can also do the splitting
per field (for test purpose).

\subsection{Multi-Field UV\_MAP processing}

When UV\_MAP encounters more than 1 field in a UV Table, it first verifies whether
phase (\texttt{code\_uvt\_loff \& code\_uvt\_moff}) or pointing 
(\texttt{code\_uvt\_xoff \& code\_uvt\_yoff})
offsets are present.  In case phase offsets are present, UV\_MAP first automatically process
the UV table to recenter it onto a common phase center, and converts the offsets
to pointing offsets. Here, the centroid of all fields is used by default as phase
center (this action is obtained by implicitely calling the UV\_SHIFT command). 
When pointing offsets are present, this step is no longer required and
imaging can proceed immediately. 

Dirty and primary beams are frequency sliced in "chunks" as for single fields.

SHOW FIELDS (or VIEW FIELDS) can display the observed pointing centers.


\textit{The display of frequency dependent beams is still to be implemented for
Mosaics, as SHOW or VIEW only handles 3-D data cubes.}

With such data, UV\_MAP automatically activates the MOSAIC mode for further image processing.

\textit{\textbf{Caution:} The MOSAIC mode is not yet automatically turned off
if a single-field UV table is processed.}

\subsection{Multi-Field CLEANing}

Apart from the proper selection of possibly frequency dependent beams, there has been no
change here. The new UV\_MAP command produces the same results as the old system using imaging
of separate fields and mosaicing through the MAKE\_MOSAIC task.

\subsection{Short Spacings}

Short Spacings in IMAGER is now also handled by a command instead 
of a set of tasks in MAPPING.  The processing sequence is the following:
\begin{itemize}\itemsep 0pt
\item READ UV  FileWithoutShortSpacings\\
Read the original UV data set
\item READ SINGLE  SingleDishDataFile \\
Read the Single Dish data file. It can be a CLASS Table (extension .tab)
or a 3-D data cube in LMV order (Velocity as 3rd axis). 
\item UV\_SHORT\\
Process the Single Dish data, extract short spacing visibilities,
merge with curren UV data set to produce a combined UV data set.
\end{itemize}
The output of command UV\_SHORT becomes the current UV data, and can
be handled as usual with UV\_MAP, WRITE UV, etc...

The UV\_SHORT command as the following features:
\begin{itemize}\itemsep 0pt
\item Minimal interface: mosaic characteristics are recovered
from the mosaic-like UV table, and telescope characteristics from the telescope section
of the UV table and Single-dish data if present.
\item Automatic recognition of offset types (phase or pointing)
\item Automatic rescaling of the short spacings weights, but under user control.
\item Automatic fall-back on Zero spacings if the single-dish diameter
is identical to the diameter of the interferometer antennas.
\end{itemize}
The command checks the spectral axis consistency, but does not automatically
re-interpolate the Single Dish data spectra onto the UV spectral axis.

\section{Imaging}

\subsection{Control parameters}
The UV\_MAP command is controlled by a set of SIC variables of names starting by MAP\_
\begin{tabular}{lll}
%\hspace{0.2cm} \= \hspace{4.2cm} \= \kill
% \> MAP\_ANGLE \>  Position angle of map axis when MAP\_SHIFT is set \\
& MAP\_BEAM\_STEP &  Number of channels per common dirty beam, if $> 0$. If 0, only one \\
                & & beam is produced in total. If $-1$, an automatic guess is performed \\
                & & from the map size and requested precision (MAP\_PRECIS). \\
& MAP\_CONVOLUTION  &  Convolution mode (default 5) (old name CONVOLUTION) \\
& MAP\_CELL    & Pixel size in arcsecond \\
% & MAP\_DEC     & Declination of new map center  \\
& MAP\_FIELD   & Image size in arcsecond \\
& MAP\_PRECIS  & Position precision at edge of image, in fraction of pixel size. Default\\
                & & is 0.1. \\
& MAP\_POWER   & Rounding scheme for default image size, to numbers like 2\^n 3\^p 5\^q. \\
  & & p and q are less or equal to MAP\_POWER. \\
% & MAP\_RA      &   Right ascension of new map center \\
& MAP\_ROBUST  &   Robust weighting factor, in range 0 - infty.  Default 1.0. (Old name \\
      &  & UV\_CELL[1]) \\
      &  &  0 means  Natural weighting (as infty, actually) \\
& MAP\_ROUNDING  & Tolerance to round image size by floor instead of ceiling. \\
%& MAP\_SHIFT    &  Logical indicating whether the phase center must be shifted and/or \\
%                & & the image rotated (old name UV\_SHIFT). \\
& MAP\_SIZE     &   Image size in pixels \\
& MAP\_TAPEREXPO &  the taper exponent. Default 2 (Old name TAPER\_EXPO). \\
& MAP\_UVCELL   &  UV Cell size for Robust weighting. Default is 0, meaning that the \\
              & & cell size is derived from the antenna diameter. (Old name \\
              & & UV\_CELL[2]) \\
& MAP\_UVTAPER &   Array of 3 values giving  the UV taper in m (first two values), its \\
                & & position angle (third value).  Default (0,0,0). (Old name \\
                & & UV\_TAPER[3]). \\
& MAP\_VERSION &  Version of code to be used. This is a temporary variable to allow \\
                & & comparison between the new and old codes without quitting \\
                & & MAPPING. \\
%\end{tabbing}
\end{tabular}

In addition, WCOL indicates the weight channel and MCOL the channel range to be imaged.
However, WCOL should in general be set to zero to allow the beam steps to be set.

\subsection{Defining the Projection Center of the image}

The scheme for shifting the phase center of images has been modified, because of
improvements required for large scale Mosaics.  Command UV\_MAP now handles phase 
shifting through its arguments, or through the string variable MAP\_CENTER.

The scheme for shifting the phase center of images has been modified, because of
improvements required for large scale Mosaics. 
Command UV\_SHIFT was introduced to phase shift Mosaics to a
common phase center in Mosaic. It also works for single fields if needed,
but this command is deprecated as the mechanism should be modified
for large field mosaics.

The old code can still be executed by setting MAP\_VERSION = -1. MAP\_VERSION set to 0
(the default) uses the new code, and MAP\_VERSION = 1 allows access to an intermediate
version. For phase center shifting, the old code  still requires the use of the
variables MAP\_RA, MAP\_DEC, MAP\_ANGLE and UV\_SHIFT.


\section{Implementation issues}

The new implementation of the UV\_MAP command uses most of the older code,
but re-arranged such that ensembles of contiguous channels
(``chunks'') are treated at once and share the same synthesized beam.
Deconvolution with CLEAN then proceeds by using the synthesized beam with the
appropriate frequency for each channel. The user can control the
``chunk'' size, and hence the precision of the process given the
desired field of view.

As a result of the new concept, beams (whether primary or synthesized) can be 4-D
arrays, as they may depend on Frequency and Field.

\section{Deconvolution}
 
\subsection{CLEAN}
Progress has been made on automatic guess for Cleaning parameters, and 
a renaming of the parameter names is proposed. The table below is the 
current renaming scheme, with previous names mentionned in 
parentheses. It is proposed that the equivalent Old names 
(mentionned in Upper case below) will remain as aliases, while those
mentionned in mixed case  will disappear as they were seldom used before. 
 
\begin{tabbing}
\hspace{0.2cm} \= \hspace{4.2cm} \= \kill
\> CLEAN\_ARES  \>  Absolute residual  (ARES) \\
\> CLEAN\_FRES  \>  Fractional residual (FRES) \\
\> CLEAN\_GAIN  \>  Loop gain  (GAIN) \\
\> CLEAN\_INFLATE \> Inflation factor allowed to display MultiScale clean components \\
\> CLEAN\_METHOD \>  Cleaning Method  (METHOD) \\
\> CLEAN\_NCYCLE \>  Maximum number of Major cycles (Nmajor) \\
\> CLEAN\_NITER \>  Maximum number of iterations (NITER) \\
\> CLEAN\_NKEEP \>  Number of iterations used to check convergence (see below)\\
\> CLEAN\_POSITIVE \> Minimum number of positive Clean components \\
\> CLEAN\_SEARCH \>  Minimum primary beam threshold for searching (Search\_W) \\
\> CLEAN\_RESTORE \>  Minimum primary beam threshold for restoring  (Restore\_W) \\
\> CLEAN\_SPEEDY \> Speeding factor for Clark (Spexp) \\
\> CLEAN\_WORRY \> "Worry" factor for Clark (Worry) \\
\> CLEAN\_SMOOTH \> Smoothing factor for Multi Scale Clean (Smooth) \\
\> CLEAN\_RATIO \> Ratio for  Dual Resolution clean (Ratio) \\
\> CLEAN\_NGOAL \> A number of components for ALMA joint deconvolution only (Ngoal)\\
\end{tabbing}

Clean convergence is controlled by the usual ARES, FRES and NITER criteria,
plus CLEAN\_MAJOR for methods with major cycles. However, these criteria
can be set to 0, allowing Mapping to automatically guess when to stop.
In this case, convergence is controlled by CLEAN\_NKEEP, which is 
a number of components. Deconvolution of a given channel stops if
the cumulative flux at iteration number N is smaller (resp. larger) than at iteration
N-CLEAN\_KEEP for positive signals (resp. negative). In essence, CLEAN\_NKEEP
is the number of components when only noise is present. Experimentation
with various types of images have shown that CLEAN\_KEEP = 70 is
a good compromise. 

\section{Self Calibration}

Self calibration (phase and, if needed, amplitude) has been implemented
through the SELFCAL command and CALIBRATE$\backslash$ Languages.

The implementation uses the SELFCAL command, whose parameters
are available in the as SELF\_Names SIC variables. 
SELFCAL also uses the UV\_MAP and CLEAN input parameters. 
SELFCAL actually activates a script using elementary commands UV\_SELF,
SOLVE and APPLY of the CALIBRATE$\backslash$ language.

Self calibration works for Phase and Amplitude. For amplitude, the
scalar gain is de-biased from the noise to preserve flux calibration
(see Fig.\ref{fig:self}), using a correction of -0.5*(Noise/Signal)$^2$.

\begin{figure}[!h]
\centering
    \includegraphics[width=0.8\textwidth]{ampnoise.pdf}
    \caption{Effective de-biasing of the Amplitude gain as
    a function of Signal to Noise. The second order correction
    gives an excellent precision for all sensible values
    of the Signal to Noise ($> 2$) .}
    \label{fig:self}
\end{figure}



\section{Visualisation}

Results are displayed using commands SHOW (channel-like display) or VIEW
(compact comprehensive display)

\subsection{Input parameters}

The implementation has been made in such a way that the name changes
between MAPPING and IMAGER do not break habits (as far as possible...).

The scheme is the following. A Fortran derived type handles all UV\_MAP
associated parameters, and SIC variables are pointing directly towards one instance
of this derived type, handling all default values.  Command UV\_MAP converts
the default values to actual values. Command UV\_STAT SETUP does the same. Both
use the same routines.

A second set of instances of the same Fortran derived type is used as target
for the old SIC variables, so that the code can check whether the user has been
modifying these ones instead of the new ones, and a warning is issued in such cases.
Both instances are made identical after each UV\_MAP command.

The only exception is WEIGHT\_MODE, which has no meaning in the new implementation,
where MAP\_ROBUST encompasses all possibilities.

The  \texttt{define.map} script as been changed to provide an implementation which
is independent of the MAPPING version.

\subsection{Handling Large Files}
With ALMA or NOEMA, files can be large. We are not changing our paradigm
but still keep copies in Virtual memory for processing in Mapping. 

However, it became critical to avoid multiple copies which would lead
to excessive memory usage. 
The main ``memory-hungry'' tools were the visualisation commands, VIEW
and SHOW (and their disk-based equivalent GO VIEW and GO BIT). Code cleaning
to avoid useless copies has been made. It is now possible to use 
the same scripts to display disk-based files, or program built-in 
SIC Image variables. 

Another major tool to minimize memory usage is the ability
to read only selected channels of an input UV table.

\section{Additional Capabilities}

Data size reduction routines:
\begin{itemize}
\item UV\_TIME  can be used to time-average the UV data set, leading
to faster processing
\item UV\_RESAMPLE allow spectral smoothing and resampling
\item UV\_COMPRESS is a simpler version, with only channel averaging
\end{itemize}

Continuum processing commands:
\begin{itemize}
\item UV\_FILTER and UV\_BASELINE allow to filter line emission, or
conversely to remove continuum baseline.
\item UV\_PREVIEW attempts to find automatically the continuum level
and part of the bandwidth with line emissions.
\item UV\_CONTINUUM converts a spectra line UV table into a bandwidth 
synthesis continuum UV table. UV\_CONTINUUM requires some knowledge of 
the image size to evaluate how many channels should be averaged 
together. This is done using the same routine as in UV\_STAT SETUP. 
Optimization of the evaluation of the Min-Max baseline length has also 
been made.
\end{itemize}



\section{On going work}

Although all functionalities are now available, they are not yet fully integrated
in a comprehensive, simple, way. This section describes the required developments
still missing for a simple, fully integrated usage by users. They are ordered
by decreasing priority.


\subsection{Widgets}
Once all the above tools are ready,  it will be useful to refurbish the 
widgets for an efficient use of the new capabilities. Only tests have 
been performed at this stage, but it should be remembered that the new 
method basically does everything with 4 command lines (READ;  UV\_MAP; 
CLEAN; WRITE) except for the short spacings issue. 

\subsection{Mosaic and FITS export}
Mosaic with pointing offsets (and thus a common phase center) cannot be 
correctly exported through UVFITS. They must be re-processed back to 
pointing center offsets, but there is no such tool yet.

In fact, to avoid smearing effects to the projection of the sphere
on a plane for wide field Mosaics, the current scheme to handle
mosaics should be changed. The optimal scheme only involves
the so called "phase offset" representations (where pointing centers
and phase centers are identical) with a reprojection only at the
UV\_MAP stage.

Once this scheme is implemented, the need to re-process back
"pointing offsets" mosaic UV table will disappear. 

\subsection{New Tools}

IMAGER  allows to test new facilities.
These are in a separate language, NEWSTUFF$\backslash$. 

\end{document}
