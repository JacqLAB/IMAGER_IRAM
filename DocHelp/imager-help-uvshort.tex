Compute  the  Short Spacings from the current Single Dish dataset
(read by READ SINGLE) and merge it to the current UV data.
  UV__SHORT takes sensible default guesses for  most  parameters.
UV__SHORT  ?   lists the essential parameter values, UV__SHORT ??
some additional ones, and UV__SHORT ??? even the  debugging  con-
trol variables.
    The  current values can be overriden by the user, who need to
set  (and  if  needed  to   define   first)   the   corresponding
SHORT__whatever   variable.   SHORT__SD__FACTOR   is the main one
which may need to be specified by the user, as  the  Single  Dish
data is rarely in the appropriate unit.
  The resulting UV table becomes the current UV data, and can  be
imaged, written, etc...
\subsubsection{/REMOVE}
\index{�!/REMOVE}
\begin{verbatim}
    UV__SHORT /REMOVE

  Removes any short spacing from the current UV data set.

\end{verbatim}
\subsubsection{Algorithm}
\index{�!Algorithm}
\begin{verbatim}

UV__SHORT task computes pseudo-visibilities  for  short  or  zero
spacings from a single dish table of spectra (Class table) or LMV
data cube.  These pseudo visibilities are  appended to  the  cur-
rent (presumably a Mosaic) UV table.


Short spacings are computed when the Interferometer dish diameter
is smaller than the Single-dish diameter, Zero spacings otherwise
(see HELP UV__SHORT Zero__Spacing for this case)

For short spacings, the command performs two steps
  (1) Creation of a "well behaved" map from the spectra.
  (2) Extraction of UV visibilities from this map.

See  HELP  UV__SHORT  Step__i  for  detailed  explanations of the
method steps.

With recent UV tables and Single Dish CLASS table,  most  parame-
ters are automatically determined. The only parameter to be spec-
ified remains SHORT__SD__FACTOR (although that one  may  also  be
determined  automatically if the input single dish data set is in
main-beam brightness temperature).

A parameter set to 0  value  indicates  the  appropriate  default
should be used.

\end{verbatim}
\subsubsection{Zero\_\_Spacing}
\index{�!Zero\_\_Spacing}
\begin{verbatim}

      Zero spacings are computed when the single dish diameter is
the same as the interferometer dish diameter.  Zero  spacing  ex-
traction  proceeds differently for Class data tables and 3-D data
cubes.

In the data cube case, the nearest pixel matching  the  direction
of  each  field  is  taken as the Zero spacing for this field. If
there is no point close enough, according to the specified  posi-
tion tolerance SHORT__TOLE, an error occurs.

In  the  Class  data table case, all spectra within the specified
position tolerance of a field center  are  averaged  together  to
produce the Zero spacing. If none is found, an error occcurs.


\end{verbatim}
\subsubsection{Step\_\_1}
\index{�!Step\_\_1}
\begin{verbatim}

      Step (1) Creation of a "well behaved" map from the spectra.

Step  (1)  only occurs if the input single-dish data set (read by
READ SINGLE) is a table of spectra.  The  table  format   is  de-
scribed in  the CLASS_GRID command of CLASS.
  The identification of the input single-dish data set as a table
of spectra is currently only based on the specified  file  exten-
sion: if that is .tab, it is assumed to be a table of spectra.

It  is recommended that this input  table is a collection of sin-
gle-dish, Nyquist sampled  spectra  covering  twice   the  inter-
ferometric  field  of  view  of interest. However, UV__SHORT does
*NOT* make any  assumption. It thus tries to compute  a "well be-
haved"  map by linear operations  (convolutions) from the  origi-
nal spectra, in an  optimum way  from signal  to noise  point  of
view. The  map is extrapolated smoothly  towards zero at  the map
edge in order  to  avoid   further   aliasing   in  the   Fourier
transform   operations required in  Step (2).  This extrapolation
has a scale  length of  twice the single-dish beam, in  order  to
avoid spurious Fourier components.

In detail, UV__SHORT performs the following operations:

  - Resampling  (in space) of  the original  spectra on  a regular  grid by
    convolution  with  a small  (typically  1/4  of  the single-dish  beam)
    gaussian convolving kernel. In  this process, the weights of individual
    spectra is carried on a weight map.
  - Extrapolation by  zero outside  the convex hull  of the  mapped region.
  - Convolution  of  the  result  by  a  gaussian  twice  as  wide  as  the
    single-dish beam.   Within the  convex hull of  the mapped  region, the
    smoothed map is replaced by the original map.

\end{verbatim}
\subsubsection{Step\_\_2}
\index{�!Step\_\_2}
\begin{verbatim}
      Step (2) Extraction of UV visibilities from this map.

From  the  given input data cube, or the "well behaved" data cube
created by Step (1), UV__SHORT  computes the visibilities in  the
following way:

  - Fourier transform of the single dish map;
  - Division by  the Fourier  transform of  the single dish  beam, up  to a
    maximum spacing (SHORT__SD__DIAM, in meters);
  - Inverse Fourier transform to the image plane and then for each pointing
    center;
  - Multiplication of the  image by the primary beam  of the interferometer
    elements;
  - Fourier transform back to the UV plane;
  - Creation  of the  visibilities,  with  a given  weight  SHORT__SD__WEIGHT
    and an appropriate calibration factor to Janskys SHORT__SD__FACTOR.

Both  the   single-dish  and the  interferometer antennas are as-
sumed    to   have   gaussian    beams    (SHORT__SD__BEAM    and
SHORT__IP__BEAM, in radians).

\end{verbatim}
\subsubsection{Variables:}
\index{�!Variables:}
\begin{verbatim}

  Control  variables for UV__SHORT are not predefined, except for
the  3  main  ones:   SHORT__SD__FACTOR,  SHORT__SD__WEIGHT   and
SHORT__UV__TRUNC.

  All  others  should  be defined by the user in case the default
value is not appropriate, with their appropriate (Real,  Char  or
Logical) types.

\end{verbatim}
\subsubsection{SHORT\_\_DO\_\_SINGLE}
\index{�!SHORT\_\_DO\_\_SINGLE}
\begin{verbatim}

Logical value, should be YES except for test purposes.

\end{verbatim}
\subsubsection{SHORT\_\_DO\_\_PRIMARY}
\index{�!SHORT\_\_DO\_\_PRIMARY}
\begin{verbatim}

Logical value, should be YES except for test purposes.

\end{verbatim}
\subsubsection{SHORT\_\_IP\_\_BEAM}
\index{�!SHORT\_\_IP\_\_BEAM}
\begin{verbatim}

Half-power  beam  width of the interferometer antennas, in  radi-
ans. The beam is assumed to be gaussian.

  Default value is 0, meaning that the beam  is  taken  from  the
Telescope section if present.

\end{verbatim}
\subsubsection{SHORT\_\_IP\_\_DIAM}
\index{�!SHORT\_\_IP\_\_DIAM}
\begin{verbatim}

Interferometer  diameter  for  which UV__SHORT will compute short
spacing visibilities.

  Default value is 0, meaning that the diameter is taken from the
Telescope section if present.

\end{verbatim}
\subsubsection{SHORT\_\_MCOL}
\index{�!SHORT\_\_MCOL}
\begin{verbatim}

  *** Obsolescent ***

See  READ SINGLE ClassTable.tab /RANGE  command for an equivalent
method of selecting the appropriate channel range.

The first and last column to be mapped. For  tables  produced  by
GRID command of CLASS, SHORT__MCOL[1] should be 4 and SHORT__MCOL
can be set to 0 to process all channels.

  Default  value:  4  0,  appropriate  for  tables  coming   from
CLASS_GRID command.


\end{verbatim}
\subsubsection{SHORT\_\_MIN\_\_WEIGHT}
\index{�!SHORT\_\_MIN\_\_WEIGHT}
\begin{verbatim}

The  minimum  weights under which a given point in the map should
be filled by the smooth map rather than by the gridded (original]
map.

  Default value: 0.01

\end{verbatim}
\subsubsection{SHORT\_\_MODE}
\index{�!SHORT\_\_MODE}
\begin{verbatim}

This  is  an integer code used for backward compatibility with an
older version of the UV__SHORT task, and also for test purpose.

Allowed values are :

-1  indicates to create a single UV table with columns for the Phase
    center offsets only
-2  indicates to create a UV table with columns for the Pointing
    center offsets
-3  indicates to create a UV table with the additional columns
    type being Pointing or Phase, as in the original UV__TABLE$
+1  indicates to append to the initial UV table the short spacings
    with Phase center offsets (which must thus match the initial UV
    table shape)
+2  indicates to append to the initial UV table the short spacings
    with Pointing center offsets (which must thus match the initial
    UV table shape)
+3  indicates to append to the initial UV table the short spacings
    (The extra column type being determined automatically).

The default value is 3, i.e. automatic merging with  the  current
UV table.

\end{verbatim}
\subsubsection{SHORT\_\_SD\_\_BEAM}
\index{�!SHORT\_\_SD\_\_BEAM}
\begin{verbatim}

Half-power  beam  width  of  the single dish antenna, in radians.
The  beam is assumed to be gaussian.

  Default value is 0, meaning that the beam  is  taken  from  the
Telescope section if present.

\end{verbatim}
\subsubsection{SHORT\_\_SD\_\_DIAM}
\index{�!SHORT\_\_SD\_\_DIAM}
\begin{verbatim}

Single  dish  diameter  used to produce the input spectra, in me-
ters.

  Default value is 0, meaning that the diameter is taken from the
Telescope section if present.

\end{verbatim}
\subsubsection{SHORT\_\_SD\_\_FACTOR}
\index{�!SHORT\_\_SD\_\_FACTOR}
\begin{verbatim}

Multiplicative calibration factor; it is used to convert from the
single dish map units (e.g., main-beam brightness temperature) to
janskys.

A  default value of 0 can be used if the original data file is in
unit of Tmb, the main beam  brightness  temperature,  because  in
this  case  the  conversion  factor  can be derived from the beam
size.

\end{verbatim}
\subsubsection{SHORT\_\_SD\_\_WEIGHT}
\index{�!SHORT\_\_SD\_\_WEIGHT}
\begin{verbatim}

Weight scaling factor for the generated visibilities.

It is a relative scaling factor in the weights compared to a sup-
posedly  optimal  weighting to give the best combined synthesized
beam. That optimal weighting essentially gives  the  same  weight
density par unit area in the UV plane than the shortest baselines
measured with the interferometer only.  However, if  the  single-
dish  data  has  not  been observed long enough, or has baselines
problems for example, this weight may add noise  to  the  overall
data set, so could be down-weighted.

  Default: 1.0

\end{verbatim}
\subsubsection{SHORT\_\_TOLE}
\index{�!SHORT\_\_TOLE}
\begin{verbatim}

The  tolerance in position (in radians). The behaviour differ for
Short and Zero spacings and  Table or 3-D  cubes  as  Single-Dish
data.

If  the Single-Dish data is a table of spectra, Spectra differing
by less than this amount will be added together  prior  to  grid-
ding. A recommended value is below 1/10th of the Single Dish pri-
mary beam. This is valid for Short Spacings and Zero Spacing cas-
es.

If the Single-Dish data is 3-D data cube, SHORT__TOLE is used on-
ly for Zero Spacings. If no pixel is within SHORT__TOLE of an In-
terferometer  pointing center, no short spacing is added for this
field and an error occur.

  Default value is 0, meaning using  1/16th of  the  Single  Dish
primary beam.

\end{verbatim}
\subsubsection{SHORT\_\_UV\_\_TRUNC}
\index{�!SHORT\_\_UV\_\_TRUNC}
\begin{verbatim}

No  visibility at spacings higher than SHORT__UV__TRUNC is gener-
ated. Theoretical consideration on the method used in  this  task
implies that SHORT__UV__TRUNC should be at most (SHORT__SD__DIAM-
SHORT__IP__DIAM).  Smaller values may need to be applied if,  for
example,  the  pointing  accuracy  of the Single Dish is insuffi-
cient.

  Default  value  is   0,   meaning   to   use   SHORT__SD__DIAM-
SHORT__IP__DIAM

\end{verbatim}
\subsubsection{SHORT\_\_WCOL}
\index{�!SHORT\_\_WCOL}
\begin{verbatim}


For  tests  only:  The column of the spectra table containing the
weights.

  Default  value:  0=3,  appropriate  for  tables   coming   from
CLASS_GRID command.

\end{verbatim}
\subsubsection{SHORT\_\_WEIGHT\_\_MODE}
\index{�!SHORT\_\_WEIGHT\_\_MODE}
\begin{verbatim}

The  weighting mode (NATURAL, UNIFORM or GRIDDED).  It is advised
to use 'NA' for Natural weighting.


\end{verbatim}
\subsubsection{SHORT\_\_XCOL}
\index{�!SHORT\_\_XCOL}
\begin{verbatim}

    For tests only: The column of the spectra table containing  X
offsets.

  Default   value:   0=1,  appropriate  for  tables  coming  from
CLASS_GRID command.

\end{verbatim}
\subsubsection{SHORT\_\_YCOL}
\index{�!SHORT\_\_YCOL}
\begin{verbatim}

    For tests only: The column of the spectra table containing  Y
offsets.

  Default   value:   0=2,  appropriate  for  tables  coming  from
CLASS_GRID command.

\end{verbatim}
