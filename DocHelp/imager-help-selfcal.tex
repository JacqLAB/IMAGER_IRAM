        SELFCAL  [?|AMPLI|APPLY|PHASE|SUMMARY|SHOW [Last [First]]
[/WIDGET]
  Command to perform Self Calibration (even on spectral line  da-
ta).  The solution is computed, saved and/or applied.
  The arguments control the action.
    SELFCAL ?        Lists the self calibration parameters
    SELFCAL AMPLI    Compute an Amplitude only self calibration
    SELFCAL APPLY    Apply the computed solution
    SELFCAL PHASE    Compute a Phase only self calibration
    SELFCAL SHOW []  Show the computed corrections
    SELFCAL SAVE     Save self calibration parameters in selfcal.last
    SELFCAL SUMMARY  Display the improvements in Noise & Dynamic range
                     and calibration status
\subsubsection{/WIDGET}
\index{�!/WIDGET}
\begin{verbatim}
        SELFCAL /WIDGET

Activates  the widget interface to perform self calibration (even
on spectral line data set). A continuum  data  set  is  extracted
from  the specified channel range, with all selected channels av-
eraged to provide improved sensitivity to find a solution.

  The buttons control the action.
     AMPLI      Perform an Amplitude only self calibration
     PHASE      Perform a Phase only self calibration
     Continue   Continue execution when the script is in Pause
     APPLY      Apply the solution
     INPUT      List parameters (as in SELFCAL ?)
     SAVE       Save the parameters in selfcal.last
     SHOW       Show the computed corrections
     SUMMARY    Display the improvements in Noise & Dynamic range
\end{verbatim}
\subsubsection{Arguments:}
\index{�!Arguments:}
\begin{verbatim}
      The arguments control the action.
    SELFCAL ?        Lists the self calibration parameters
    SELFCAL AMPLI    Perform an Amplitude only self calibration
    SELFCAL APPLY    Apply the solution
    SELFCAL PHASE    Perform a Phase only self calibration
    SELFCAL SAVE     Save the parameters in selfcal.last
    SELFCAL SHOW []  Show the computed corrections
    SELFCAL SUMMARY  Display the improvements in Noise & Dynamic range
\end{verbatim}
\subsubsection{AMPLITUDE}
\index{�!AMPLITUDE}
\begin{verbatim}
        SELFCAL AMPLI

Compute  an  Amplitude  only  self-calibration.  The  integration
times, SELF__TIME, should in general be significantly larger than
for a Phase only self-calibration.

SELFCAL automatically adjusts the gains so  that  their  mean  is
\end{verbatim}
\subsection{.0,}
\index{.0,}
\begin{verbatim}

\end{verbatim}
\subsubsection{.0, APPLY}
\index{.0,!APPLY}
\begin{verbatim}
        SELFCAL APPLY [Type [Gain]]

Apply the self calibration solution. This is done only if the re-
turn status  from  the  previous  computation,  SELF__STATUS,  is
greater than 0.  SELFCAL APPLY automatically saves the parameters
and results in the "selfcal-AMPLI.last"  or  "selfcal-PHASE.last"
file, depending on the Type of solution applied.

  Type  is  the  type  of  solution  to  apply.  The  default  is
'SELF__MODE', i.e.  PHASE or AMPLI depending on the last type  of
self-calibration  computed.   Type can also take the DELAY value,
where the "phase" corrections are interpreted as atmospheric path
changes and scale as Frequency.

  Gain  is an optional gain factor (default is 1.) on the correc-
tion.

\end{verbatim}
\subsubsection{.0, INPUT}
\index{.0,!INPUT}
\begin{verbatim}
        SELFCAL INPUT  or SELFCAL ?

Display SELFCAL control variables

\end{verbatim}
\subsubsection{.0, PHASE}
\index{.0,!PHASE}
\begin{verbatim}
        SELFCAL PHASE

Compute a  Phase-only  self  calibration.  The  integration  time
should  be  short  enough  to correct for atmospheric errors, but
large enough to obtain significant signal to noise for  most  an-
tennas.

\end{verbatim}
\subsubsection{.0, SAVE}
\index{.0,!SAVE}
\begin{verbatim}
        SELFCAL SAVE

    Save the parameters and results in the "selfcal.last" file.

\end{verbatim}
\subsubsection{.0, SHOW}
\index{.0,!SHOW}
\begin{verbatim}
        SELFCAL SHOW [Last [First]]

Shows  the  correction  computed by self calibration. By default,
the difference between the  last  two  iterations  is  displayed:
phase  should be around 0, and amplitude around 1 if the solution
is converged.

If Last and First are present, it shows  the  difference  between
these  two specified iteration. If Last only is present, it shows
the total correction between the original data  and  that  itera-
tion.

The displayed ranges are controlled by SELF__ARANGE[2] (limits of
Amplitude gain), SELF__PRANGE[2] (limits of Phase correction) and
SELF__TRANGE[2] (limits for time axis).

\end{verbatim}
\subsubsection{.0, SUMMARY}
\index{.0,!SUMMARY}
\begin{verbatim}
        SELFCAL SUMMARY

Display  a summary of the Self-calibration process: type of cali-
bration, rms and dynamic range at each iteration, as well as  the
number of flagged or uncalibrated visibilities.

\end{verbatim}
\subsubsection{.0, Results:}
\index{.0,!Results:}
\begin{verbatim}

SELFCAL  returns  results  in several variables, creates a CGAINS
array containing the Gain values, and one  file  to  display  the
computed correction. The file name is specified by SELF__SNAME.

The result variables are:
    SELF__APPLIED
      Indicates whether the solution has been applied
    SELF__STATUS
      Indicates if a solution has been computed
    SELF__DYNAMIC[Self__Loop+1]
      The dynamic range at each iteration
    SELF__RMSCLEAN[Self__Loop+1]
      The rms noise at each iteration
\end{verbatim}
\subsubsection{SELF\_\_APPLIED}
\index{.0,!SELF\_\_APPLIED}
\begin{verbatim}

Variable SELF__APLPLIED indicates whether a solution has been ap-
plied (#0) or not (0). The value indicates the type  and  quality
of solution, as for SELF__STATUS

\end{verbatim}
\subsubsection{SELF\_\_STATUS}
\index{.0,!SELF\_\_STATUS}
\begin{verbatim}

Variable SELF__STATUS indicates if a solution has been computed
      0 no solution
      +/- 1   Phase solution
      +/- 2   Gain solution
    >0 means a good solution, <0 a poor one.
\end{verbatim}
\subsubsection{SELF\_\_DYNAMIC}
\index{.0,!SELF\_\_DYNAMIC}
\begin{verbatim}

SELF__DYNAMIC  is  a  variable length array of size Self__Loop+1,
containing the dynamic range at each iteration.

\end{verbatim}
\subsubsection{SELF\_\_RMSCLEAN}
\index{.0,!SELF\_\_RMSCLEAN}
\begin{verbatim}

SELF__RMSCLEAN is a variable length array of  size  Self__Loop+1,
containing the Clean map rms noise at each iteration.

\end{verbatim}
\subsubsection{Variables:}
\index{.0,!Variables:}
\begin{verbatim}

The  SELFCAL  behaviour can be adjusted through control variables
named SELF__... (see EXA SELF__ for a list), in addition  to  the
control variables of UV__MAP (MAP__...) and CLEAN (CLEAN__...)

The most important variable is SELF__TIMES, a variable length ar-
ray which controls the integration time at each loop. SELF__NITER
and SELF__MINFLUX also control the number of Clean components and
minimum flux retained in each loop. The size of these arrays  de-
fine the number of loops.

See HELP SELFCAL SELF__LOOP to find out how to control the number
of loops.

\end{verbatim}
\subsubsection{SELF\_\_CHANNEL}
\index{.0,!SELF\_\_CHANNEL}
\begin{verbatim}
        SELF__CHANNEL[2]

First and last channel to define the range to compute the UV  ta-
ble  for  the self-calibration solution. For example, if you have
used UV_PLUGC to plug a continuum data  into  the  last  channel,
this  could be set to the number of channels. 0 0 means all chan-
nels are averaged to compute the "continuum" image.

\end{verbatim}
\subsubsection{SELF\_\_COLOR}
\index{.0,!SELF\_\_COLOR}
\begin{verbatim}
        SELF__COLOR

Controls the LUT color range at each  self-calibration  cycle  if
non  Zero.  Since the dynamic range evolves, this can be a useful
way to highlight the gain. If non Zero, SELF__COLOR is passed  as
argument to a COLOR command executed at each cycle after display.

Pratical  values  for SELF__COLOR are -8 ("bright" version, high-
lights the noise level) or +8 ("dark" version, hides the  noise),
but  lower  absolute  values  may  be  needed  for higher dynamic
ranges.

See HELP COLOR for details.

\end{verbatim}
\subsubsection{SELF\_\_LOOP}
\index{.0,!SELF\_\_LOOP}
\begin{verbatim}

Number of self-iteration loops. It is a ReadOnly variable that is
automatically   computed   from  the  size  of  the  SELF__TIMES,
SELF__NITER and SELF__MINFLUX arrays.

  These variable length arrays can be resized using the LET  /RE-
SIZE command. For example
    LET SELF__TIMES 40 20 10 /RESIZE
will lead to an array of 3 elements, SELF__TIMES[3].

SELF__TIMES, SELF__NITER and SELF__MINFLUX must be of equal size.
To simplify the process, constant arrays are assumed of arbitrary
length in this determination.  If all 3 arrays have constant val-
ues, SELF__TIMES determines the number of loops.

Constant arrays are assumed of arbitrary length in this  determi-
nation.  If all 3 arrays have constant values, SELF__TIMES deter-
mines the number of loops.

\end{verbatim}
\subsubsection{SELF\_\_NITER}
\index{.0,!SELF\_\_NITER}
\begin{verbatim}

Variable length array specifying the number of  Clean  components
retained  for each loop of the self-calibration process.  Default
is 0, meaning all Clean components found by CLEAN.

For simple structures and Phase calibration, 10  may  be  enough.
For more complex ones, be sure to include enough Clean components
in the model.  More components can be taken  at  each  step,  al-
though  the  better  knowledge  of  phase errors often allows the
source to be represented with a smaller number of components  af-
ter  self-calibration  than  before.  The default is a reasonably
good compromise, although faster convergence may be obtained with
smaller number of components.

This  variable  length array can be resized using the LET /RESIZE
command. For example
    LET SELF__NITER 10 0 0 /RESIZE
will  lead  to  3  self-calibration  loops  (if  SELF__TIMES  and
SELF__MINFLUX  are  also of size 3), the first one selecting only
\end{verbatim}
\subsection{0}
\index{0}
\begin{verbatim}

\end{verbatim}
\subsubsection{0 SELF\_\_TIMES}
\index{0!SELF\_\_TIMES}
\begin{verbatim}

Variable length array specifying the integration  time  (in  sec-
onds) for each loop of the self-calibration process.

At  NOEMA, 45 sec is the normal minimum integration time. Depend-
ing on Signal to Noise, you may need to have  this  longer,  e.g.
\end{verbatim}
\subsection{20}
\index{20}
\begin{verbatim}
crease only at the end.

At ALMA, the minimum integration time is 6 sec. At VLA, this  may
be as small as 1 sec.

It  is  recommended to use the same integration time for the last
two iterations, to simplify the interpretation of the results and
of the SELFCAL SHOW display.

This  variable  length array can be resized using the LET /RESIZE
command. For example
    LET SELF__TIMES 180 90 45 /RESIZE
will lead 3 self-calibration loops (if SELF__TIMES and SELF__MIN-
FLUX are also of size 3) with decreasing integration times.

\end{verbatim}
\subsubsection{20 SELF\_\_MINFLUX}
\index{20!SELF\_\_MINFLUX}
\begin{verbatim}

Variable  length  array  specifying  the minimum flux density (in
Jy/beam) to be considered in the Clean image for each loop of the
self-calibration  process.  Note that this is the brightness of a
pixel, not the flux of a Clean component.

This variable length array can be resized using the  LET  /RESIZE
command. For example
    LET SELF__MINFLUX 0.001 0 0  /RESIZE
will   lead   3   self-calibration   loops  (if  SELF__TIMES  and
SELF__NITER are also of size 3)  selecting  on  regions  brighter
than  1mJy/beam  in  the first one, and all regions in the last 2
ones.

\end{verbatim}
\subsubsection{20 SELF\_\_REFANT}
\index{20!SELF\_\_REFANT}
\begin{verbatim}

The reference antenna number. If 0, the program will peak the one
with the shortest average baselines.

\end{verbatim}
\subsubsection{20 SELF\_\_FLUX}
\index{20!SELF\_\_FLUX}
\begin{verbatim}

Maximum  value in the FLUX window. If 0, the FlUX window of Clean
is not displayed

\end{verbatim}
\subsubsection{20 SELF\_\_PRECISION}
\index{20!SELF\_\_PRECISION}
\begin{verbatim}

Tolerance to test for self-calibration convergence.  The  default
is  0.01. SELFCAL SUMMARY will write a message when no more self-
cal iteration improves the solution (noise and dynamic range)  at
this precision level.

\end{verbatim}
\subsubsection{20 SELF\_\_RESTORE}
\index{20!SELF\_\_RESTORE}
\begin{verbatim}

Use UV__SELF /RESTORE after Cleaning. This avoids signal aliasing
at image edges, and leads to a better estimate of the noise  lev-
el.

It also allows to use smaller images, in practice,

\end{verbatim}
\subsubsection{20 SELF\_\_DISPLAY}
\index{20!SELF\_\_DISPLAY}
\begin{verbatim}

If  YES,  display  Clean  image before each calibration loop, and
prompt for user input. If YES, Cleaning at each step will use the
number  of  iterations  specified  by  NITER, while if set to NO,
Cleaning will stop at SELF__NITER, saving time.

The dynamic range progress report is accurate only if  SELF__DIS-
PLAY is set.

\end{verbatim}
\subsubsection{20 SELF\_\_FLAG}
\index{20!SELF\_\_FLAG}
\begin{verbatim}

    If  SELF__FLAG  is  YES, SELFCAL will flag data with no solu-
tion. If NO, it will keep the data as it was before.

\end{verbatim}
\subsubsection{20 SELF\_\_SNR}
\index{20!SELF\_\_SNR}
\begin{verbatim}

Minimum Signal to Noise ratio on the antenna gain to  consider  a
solution to be valid for an antenna. 6 is a good value, 3 a lower
limit. Beware that this SNR makes sense only if the a priori  es-
timate  of  the  noise  from  the  UV  weights  is  correct:  see
SELF__SNOISE.

\end{verbatim}
\subsubsection{20 SELF\_\_SNOISE}
\index{20!SELF\_\_SNOISE}
\begin{verbatim}

Noise scaling factor. This should be 1, but some noise  estimates
need corrections. Continuum data from ALMA often requires sqrt(2)
instead, for example.  Command UV__PREVIEW may help you  checking
the noise scale.

\end{verbatim}
\subsubsection{20 CLEAN\_\_ARES}
\index{20!CLEAN\_\_ARES}
\begin{verbatim}

    Maximum absolute residual to stop Cleaning

\end{verbatim}
\subsubsection{20 CLEAN\_\_FRES}
\index{20!CLEAN\_\_FRES}
\begin{verbatim}

    Maximum fractional residual to stop Cleaning

\end{verbatim}
\subsubsection{20 CLEAN\_\_NITER}
\index{20!CLEAN\_\_NITER}
\begin{verbatim}

Maximum  number  of  Clean  components.  If  all  of CLEAN__ARES,
CLEAN__FRES and CLEAN__NITER are 0, Clean stops by  checking  the
stability  of  the cleaned flux over CLEAN__NKEEP iterations. See
HELP CLEAN for details.


























































\end{verbatim}
