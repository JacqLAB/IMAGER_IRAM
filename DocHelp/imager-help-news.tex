\subsection{Language}
\index{Language}
\begin{verbatim}
    CATALOG       : Define the spectral line catalog(s)
    EXTRACT       : Extract a subset from a data cube
    FLUX          : Compute integrated flux from Support or Mask
    MAP_CONTINUUM : Determine the continuum image from a spectral cube
    MASK          : Define the MASK
    MFS           : Multi Frequency Synthesis (under development - not functiona
    MOMENTS       : Compute Moments of the SKY or CLEAN data cubes
    SELFCAL       : Perform a self calibration
    SLICE         : Extract a Slice of the specified data cube
    STOKES        : Extract one Stokes parameter from multi-polarization UV tabl
    UV_ADD        : Add some extra column to the current UV data
    UV_DEPROJECT  : Deproject UV data
    UV_FIT        : Fit UV data with simple functions
    UV_MERGE      : Merge (possibly many) UV tables
    UV_PREVIEW    : Quick look at the spectral aspects of the UV data
    UV_RADIAL     : Deproject and compute radial average of UV data
    UV_SHORT      : Compute and add short spacings to UV data


\end{verbatim}
\subsection{CATALOG}
\index{CATALOG}
\begin{verbatim}
        [ADVANCED\]CATALOG [FileName [.. [FileNameN]]

    Define  or list the current catalog(s) for spectral line identification.
    A catalog is a file in the LINEDB data format. See HELP LINEDB\ for  de-
    tails.  Without argument, or with argument ?, the command will just list
    the names of the current catalog(s).

    At initialization, IMAGER search for $HOME/imager.linedb as default cat-
    alog, then gag_data:imager.linedb if not found. If the specified catalog
    does not exist, no spectral line identification will be done.

    Direct use of the LINEDB\USE command is also possible to define the cat-
    alog(s).

\end{verbatim}
\subsubsection{CATALOG Default}
\index{CATALOG!Default}
\begin{verbatim}

        A default catalog can be created (in the local directory) by execut-
    ing once
      @ gag_data:imager-linedb.sic
    This may take a while, and even get stuck occasionally because  it  per-
    forms  network access to remote databases. You can later copy this cata-
    log to your $HOME. See HELP LINEDB\ to add or remove species  from  such
    catalogs.

\end{verbatim}
\subsection{EXTRACT}
\index{EXTRACT}
\begin{verbatim}
        [ADVANCED\]EXTRACT VarName blc1 blc2 blc3 trc1 trc2 trc3

      Extract the subset VarName[blc1:trc1,blc2:trc2,blc3:trc3] of the image
    variable VarName and put it in the EXTRACTED image variable.

\end{verbatim}
\subsection{FLUX}
\index{FLUX}
\begin{verbatim}
        [ADVANCED\]FLUX Cursor|Mask|Support

      Compute the integrated flux from the CLEAN image, using zones  defined
    either  by calling the Cursor, or by the current Support, or by the cur-
    rent Mask.

      If the CLEAN data is a 3-D cube, the integrated flux will be  a  spec-
    trum.

      For  FLUX  MASK,  if the Mask defines several separate zones, an inte-
    grated flux will be computed for each zone. Zones  are  ordered  by  de-
    creasing number of pixels.

      The  FLUX  results are available in the FLUX SIC structure, and can be
    displayed by SHOW FLUX  and also SHOW COMPOSITE.

\end{verbatim}
\subsubsection{FLUX Limitations}
\index{FLUX!Limitations}
\begin{verbatim}

        FLUX MASK does not yet recognize properly Frequency dependent Masks.

\end{verbatim}
\subsubsection{FLUX Results}
\index{FLUX!Results}
\begin{verbatim}

          The FLUX results are available in the FLUX SIC structure
    FLUX%NC                       Number of channels
    FLUX%FREQUENCIES[Flux%Nc]     Frequencies of the Channels
    FLUX%VELOCITIES[Flux%Nc]      Velocities of the Channels
    FLUX%NF                       Number of Fields (Zones)
    FLUX%VALUES[Flux%Nc,Flux%Nf]  Integrated Flux for each channel and Zone

    For Cursor or Support defined regions, the enclosing polygon is also ac-
    cessible:
    FLUX%NXY                      Number of polygon summits
    FLUX%X                        X coordinates of summits
    FLUX%Y                        Y coordinates of summits

\end{verbatim}
\subsection{MAP\_CONTINUUM}
\index{MAP\_CONTINUUM}
\begin{verbatim}
        [ADVANCED\]MAP_CONTINUUM [DIRTY|CLEAN [Threshold]]

      Compute  a  continuum  image  from  the  Dirty or Clean data set.  The
    derivation of the continuum follows the same algorithm than  in  UV_CON-
    TINUUM, using the integrated spectrum over the image.

\end{verbatim}
\subsection{MASK}
\index{MASK}
\begin{verbatim}
        [ADVANCED\]MASK [Key [Arguments ...]

      Handle the MASK buffer, which can be used to select regions for Clean-
    ing (see SUPPORT /MASK) or to mask any data cube.

      Without argument (or with argument INTERACTIVE), an  interactive  tool
    is used to manipulate the mask. Valid operations are
      ADD         Add a region to the mask
      APPLY       Apply the mask to a buffer
      CHECK       Check Clean and Mask consistency
      INITIALIZE  Initialize the Mask (2D or 3D)
      INTERACTIVE Interactively define mask planes Step by Step
      OVERLAY     Overlay the Mask to the Clean image
      READ        Read the Mask from a file
      REMOVE      Remove a region from the mask
      SHOW        Show the Mask (as SHOW MASK)
      THRESHOLD   Compute an automatic mask
      USE         Use the Mask in Clean (as SUPPORT /MASK)
      WRITE       Write the Mask on file (as WRITE MASK)

\end{verbatim}
\subsubsection{MASK Tricks}
\index{MASK!Tricks}
\begin{verbatim}
          The MASK variable is ReadOnly. Yet, the user may want to modify it
    directly by himself.  The following commands
      DEFINE ALIAS WMASK MASK /GLOBAL
      LET WMASK /STATUS WRITE
    will create an alias to MASK which can be written by the user  at  will.
    This method is actually used in the Interactive MASK tool.

\end{verbatim}
\subsubsection{MASK ADD}
\index{MASK!ADD}
\begin{verbatim}
        [ADVANCED\]MASK ADD Shape [Arguments ...]

      Add the specified Shape to the current mask.  Shape can be
      CIRCLE      Ox Oy Diameter
      ELLIPSE     Ox Oy Major Minor Angle
      RECTANGLE   Ox Oy Major Minor Angle
      POLYGON     File
    where  Ox  Oy  are  the  center of the shapes,  Major and Minor the axes
    length (= side lengths for the Rectangle...) and Angle the Position  An-
    gle of the Major axis.

    For the POLYGON, the cursor is called if no File argument is given; else
    the polygon is read from the specified File.

\end{verbatim}
\subsubsection{MASK APPLY}
\index{MASK!APPLY}
\begin{verbatim}
        [ADVANCED\]MASK APPLY SicVariable

    Apply the mask to the corresponding 3-D variable.

    The space dimensions must coincide (pixel per pixel), but  the  spectral
    axes  can  differ.  The command will select the appropriate Mask channel
    for each plane of the SicVariable.  If the Mask is 2-D only, it will  be
    applied to all planes.

\end{verbatim}
\subsubsection{MASK CHECK}
\index{MASK!CHECK}
\begin{verbatim}
        [ADVANCED\]MASK CHECK [SicVariable]

    Check  the  Mask consistency against the specified SicVariable.  The de-
    fault is against Clean.

\end{verbatim}
\subsubsection{MASK INITIALIZE}
\index{MASK!INITIALIZE}
\begin{verbatim}
        [ADVANCED\]MASK INITIALIZE 2D|3D

      Initialize an empty 2-D or 3-D mask. For a 2-D mask,  the  interactive
    tool will use the mean image as a background for the definition.

\end{verbatim}
\subsubsection{MASK INTERACTIVE}
\index{MASK!INTERACTIVE}
\begin{verbatim}
        [ADVANCED\]MASK INTERACTIVE [Nchan]

      Enter the interactive tool, moving Nchan channels at each
      Next or Previous button. Default is 1.

        [ADVANCED\]MASK
    is equivalent to
        [ADVANCED\]MASK INTERACTIVE 1

\end{verbatim}
\subsubsection{MASK OVERLAY}
\index{MASK!OVERLAY}
\begin{verbatim}
        [ADVANCED\]MASK OVERLAY

    Display a SHOW-like overlay of the Mask on top of the current image (de-
    fault is Clean, normally). All SHOW parameters apply.

\end{verbatim}
\subsubsection{MASK READ}
\index{MASK!READ}
\begin{verbatim}
        [ADVANCED\]MASK READ File.[msk]

      Read the Mask from a Gildas data-cube.

\end{verbatim}
\subsubsection{MASK REMOVE}
\index{MASK!REMOVE}
\begin{verbatim}
        [ADVANCED\]MASK REMOVE Shape [Arguments ...]

      Remove the specified Shape from the current Mask.  Shape can be
      CIRCLE      Ox Oy Diameter
      ELLIPSE     Ox Oy Major Minor Angle
      RECTANGLE   Ox Oy Major Minor Angle
      POLYGON     File
    where Ox Oy are the center of the shapes,   Major  and  Minor  the  axes
    length  (= side lengths for the Rectangle...) and Angle the Position An-
    gle of the Major axis.

    For the POLYGON, the cursor is called if no File argument is given; else
    the polygon is read from the specified File.

\end{verbatim}
\subsubsection{MASK SHOW}
\index{MASK!SHOW}
\begin{verbatim}
        [ADVANCED\]MASK SHOW

      Similar to SHOW MASK command, but useing a spacing equal
      to 0.5 to illustrate the boundaries.

\end{verbatim}
\subsubsection{MASK THRESHOLD}
\index{MASK!THRESHOLD}
\begin{verbatim}
        [ADVANCED\]MASK THRESHOLD [Raw Smooth [Length [Guard]]

    Define  a  Mask from thresholding the CLEAN image. If the CLEAN image is
    3-D, the Mask will be 3-D.   `

    The method involves a first thresholding, followed by a smoothing to ex-
    tend the support and a second thresholding.

    The algorithm to define the mask is controlled by 4 parameters.
        Raw
    Initial  threshold  (in units of CLEAN image noise) under which the mask
    is set to 0.  Default is 6 sigma. The noise is taken from  the  computed
    Clean  noise  (clean%gil%rms)  if defined, or from the theoretical noise
    (dirty%gil%noise) if not.
        Smooth
    Threshold under which the Mask is set to 0 after smoothing. The  default
    is 2 sigma.
        Length
    FWHM  of  the smoothing gaussian to derive the smooth mask from the ini-
    tial mask.  The default is the Clean beam major axis.
        Guard
    Size of the guard band at edges where the mask is set to zero, in  units
    of  image  size. The default is 0.18, i.e. the mask can extends a little
    more than the inner quarter.  This is to avoid the edges where sidelobes
    aliasing occurs.

\end{verbatim}
\subsubsection{MASK USE}
\index{MASK!USE}
\begin{verbatim}
        [ADVANCED\]MASK USE

      Use the MASK in Clean deconvolution.

      Equivalent to SUPPORT /MASK command.

\end{verbatim}
\subsubsection{MASK WRITE}
\index{MASK!WRITE}
\begin{verbatim}
        [ADVANCED\]MASK WRITE File

      Save the Mask on a File. Default extension is .msk.

      Equivalent to WRITE MASK command.


\end{verbatim}
\subsection{MFS}
\index{MFS}
\begin{verbatim}
        [ADVANCED\]MFS

      ** NOT OPERATIONAL -- UNDER DEVELOPMENT **

      Multi-frequency  synthesis,  allowing  to  account  for spectral index
    changes across the observed field of view for very wide bandwidths  con-
    tinuum imaging.

\end{verbatim}
\subsection{MOMENTS}
\index{MOMENTS}
\begin{verbatim}
        [ADVANCED\]MOMENTS  [/METHOD Mean|Peak|Parabolic]
                         [/RANGE Min Max TYPE] [/THRESHOLD Thre]


      Compute  the  4  main "moments" of a data cube, and return them as SIC
    Image variables.  The 4 moments are
    M_AREA   Integrated area over the velocity range
    M_PEAK   Peak brightness value
    M_VELO   Mean or peak velocity
    M_WIDTH  Weighted line width
    The options control the method, the selected part of the  cube  (default
    channels from FIRST to LAST), and the threshold for detection (default 3
    sigma).

    The resulting images can be displayed using  command  SHOW  MOMENTS  and
    saved using WRITE MOMENTS.
\end{verbatim}
\subsubsection{MOMENTS /METHOD}
\index{MOMENTS!/METHOD}
\begin{verbatim}
        [ADVANCED\]MOMENTS  /METHOD Mean|Peak|Parabola
                         [/RANGE Min Max TYPE] [/THRESHOLD Thre]

    Specify which method to be used to compute the velocity and peak values.
    The default method, MEAN, computes an intensity weighted  velocity,  and
    takes  the peak intensity at each pixel.  Method PEAK takes the velocity
    at the peak intensity. Method PARABOLA makes  a  parabolic  fit  over  3
    channels  around  the peak intensity channel to derive the velocity, and
    takes the peak value of this parabola for the peak brightness.

    MOMENTS works on either the CLEAN or the SKY brightness data cubes  (SKY
    has priority if both are defined).

\end{verbatim}
\subsubsection{MOMENTS /RANGE}
\index{MOMENTS!/RANGE}
\begin{verbatim}
        [ADVANCED\]MOMENTS /RANGE Min Max TYPE
          [/METHOD Mean|Peak|Parabola] [/THRESHOLD Thre]

    Specify the range of channels to be selected. TYPE can be VELOCITY, FRE-
    QUENCY or CHANNELS.  If not present, the range is defined by  FIRST  and
    LAST variables.

\end{verbatim}
\subsubsection{MOMENTS /THRESHOLDS}
\index{MOMENTS!/THRESHOLDS}
\begin{verbatim}
        [ADVANCED\]MOMENTS /THRESHOLD Thre
          [/METHOD Mean|Peak|Parabola] [/RANGE Min Max TYPE]

    Specify the minimum intensity of the input data cube to consider a given
    pixel,channel as valid. The default is 3 sigma. The unit is that of  in-
    put data cube.


\end{verbatim}
\subsection{SELFCAL}
\index{SELFCAL}
\begin{verbatim}

            SELFCAL  [?|AMPLI|APPLY|PHASE|SUMMARY|SHOW [Last [First]] [/WID-
    GET]

      Command to perform Self Calibration (even on spectral line data).  The
    solution is computed, saved and/or applied.

      The arguments control the action.

        SELFCAL ?        Lists the self calibration parameters
        SELFCAL AMPLI    Compute an Amplitude only self calibration
        SELFCAL APPLY    Apply the computed solution
        SELFCAL PHASE    Compute a Phase only self calibration
        SELFCAL SHOW []  Show the computed corrections
        SELFCAL SAVE     Save self calibration parameters in selfcal.last
        SELFCAL SUMMARY  Display the improvements in Noise & Dynamic range
                         and calibration status
\end{verbatim}
\subsubsection{SELFCAL /WIDGET}
\index{SELFCAL!/WIDGET}
\begin{verbatim}
            SELFCAL /WIDGET

    Activates  the  widget  interface  to  perform self calibration (even on
    spectral line data set). A continuum data  set  is  extracted  from  the
    specified  channel range, with all selected channels averaged to provide
    improved sensitivity to find a solution.

      The buttons control the action.
         AMPLI      Perform an Amplitude only self calibration
         PHASE      Perform a Phase only self calibration
         Continue   Continue execution when the script is in Pause
         APPLY      Apply the solution
         INPUT      List parameters (as in SELFCAL ?)
         SAVE       Save the parameters in selfcal.last
         SHOW       Show the computed corrections
         SUMMARY    Display the improvements in Noise & Dynamic range
\end{verbatim}
\subsubsection{SELFCAL Arguments:}
\index{SELFCAL!Arguments:}
\begin{verbatim}
          The arguments control the action.
        SELFCAL ?        Lists the self calibration parameters
        SELFCAL AMPLI    Perform an Amplitude only self calibration
        SELFCAL APPLY    Apply the solution
        SELFCAL PHASE    Perform a Phase only self calibration
        SELFCAL SAVE     Save the parameters in selfcal.last
        SELFCAL SHOW []  Show the computed corrections
        SELFCAL SUMMARY  Display the improvements in Noise & Dynamic range
\end{verbatim}
\subsubsection{AMPLITUDE}
\index{SELFCAL!AMPLITUDE}
\begin{verbatim}
            SELFCAL AMPLI

    Compute an  Amplitude  only  self-calibration.  The  integration  times,
    SELF_TIME,  should  in  general be significantly larger than for a Phase
    only self-calibration.

    SELFCAL automatically adjusts the gains so that their mean  is  1.0,  to
    avoid changing the flux scale.

\end{verbatim}
\subsubsection{APPLY}
\index{SELFCAL!APPLY}
\begin{verbatim}
            SELFCAL APPLY [Type [Gain]]

    Apply  the  self  calibration  solution. This is done only if the return
    status from the previous computation, SELF_STATUS, is  greater  than  0.
    SELFCAL  APPLY  automatically  saves  the  parameters and results in the
    "selfcal-AMPLI.last" or "selfcal-PHASE.last" file, depending on the Type
    of solution applied.

      Type  is  the  type  of solution to apply. The default is 'SELF_MODE',
    i.e.  PHASE or AMPLI depending on the last type of self-calibration com-
    puted.   Type  can  also take the DELAY value, where the "phase" correc-
    tions are interpreted as atmospheric path changes and scale as  Frequen-
    cy.

      Gain is an optional gain factor (default is 1.) on the correction.

\end{verbatim}
\subsubsection{INPUT}
\index{SELFCAL!INPUT}
\begin{verbatim}
            SELFCAL INPUT  or SELFCAL ?

    Display SELFCAL control variables

\end{verbatim}
\subsubsection{PHASE}
\index{SELFCAL!PHASE}
\begin{verbatim}
            SELFCAL PHASE

    Compute  a  Phase-only  self calibration. The integration time should be
    short enough to correct for atmospheric errors, but large enough to  ob-
    tain significant signal to noise for most antennas.

\end{verbatim}
\subsubsection{SAVE}
\index{SELFCAL!SAVE}
\begin{verbatim}
            SELFCAL SAVE

        Save the parameters and results in the "selfcal.last" file.

\end{verbatim}
\subsubsection{SHOW}
\index{SELFCAL!SHOW}
\begin{verbatim}
            SELFCAL SHOW [Last [First]]

    Shows  the correction computed by self calibration. By default, the dif-
    ference between the last two iterations is displayed:  phase  should  be
    around 0, and amplitude around 1 if the solution is converged.

    If Last and First are present, it shows the difference between these two
    specified iteration. If Last only is present, it shows the total correc-
    tion between the original data and that iteration.

    The  displayed ranges are controlled by SELF_ARANGE[2] (limits of Ampli-
    tude  gain),   SELF_PRANGE[2]   (limits   of   Phase   correction)   and
    SELF_TRANGE[2] (limits for time axis).

\end{verbatim}
\subsubsection{SUMMARY}
\index{SELFCAL!SUMMARY}
\begin{verbatim}
            SELFCAL SUMMARY

    Display  a summary of the Self-calibration process: type of calibration,
    rms and dynamic range at each  iteration,  as  well  as  the  number  of
    flagged or uncalibrated visibilities.

\end{verbatim}
\subsubsection{Results:}
\index{SELFCAL!Results:}
\begin{verbatim}

    SELFCAL  returns  results  in  several variables, creates a CGAINS array
    containing the Gain values, and one file to display the computed correc-
    tion. The file name is specified by SELF_SNAME.

    The result variables are:
        SELF_APPLIED
          Indicates whether the solution has been applied
        SELF_STATUS
          Indicates if a solution has been computed
        SELF_DYNAMIC[Self_Loop+1]
          The dynamic range at each iteration
        SELF_RMSCLEAN[Self_Loop+1]
          The rms noise at each iteration
\end{verbatim}
\subsubsection{SELF\_APPLIED}
\index{SELFCAL!SELF\_APPLIED}
\begin{verbatim}

    Variable  SELF_APLPLIED  indicates  whether  a solution has been applied
    (#0) or not (0). The value indicates the type and quality  of  solution,
    as for SELF_STATUS

\end{verbatim}
\subsubsection{SELF\_STATUS}
\index{SELFCAL!SELF\_STATUS}
\begin{verbatim}

    Variable SELF_STATUS indicates if a solution has been computed
          0 no solution
          +/- 1   Phase solution
          +/- 2   Gain solution
        >0 means a good solution, <0 a poor one.
\end{verbatim}
\subsubsection{SELF\_DYNAMIC}
\index{SELFCAL!SELF\_DYNAMIC}
\begin{verbatim}

    SELF_DYNAMIC  is a variable length array of size Self_Loop+1, containing
    the dynamic range at each iteration.

\end{verbatim}
\subsubsection{SELF\_RMSCLEAN}
\index{SELFCAL!SELF\_RMSCLEAN}
\begin{verbatim}

    SELF_RMSCLEAN is a variable length array of size Self_Loop+1, containing
    the Clean map rms noise at each iteration.

\end{verbatim}
\subsubsection{Variables:}
\index{SELFCAL!Variables:}
\begin{verbatim}

    The  SELFCAL  behaviour  can be adjusted through control variables named
    SELF_... (see EXA SELF_ for a list), in addition to  the  control  vari-
    ables of UV_MAP (MAP_...) and CLEAN (CLEAN_...)

    The most important variable is SELF_TIMES, a variable length array which
    controls the integration time at each loop. SELF_NITER and  SELF_MINFLUX
    also control the number of Clean components and minimum flux retained in
    each loop. The size of these arrays define the number of loops.

    See HELP SELFCAL SELF_LOOP to find out how  to  control  the  number  of
    loops.

\end{verbatim}
\subsubsection{SELF\_CHANNEL}
\index{SELFCAL!SELF\_CHANNEL}
\begin{verbatim}
            SELF_CHANNEL[2]

    First  and  last channel to define the range to compute the UV table for
    the self-calibration solution. For example, if you have used UV\PLUGC to
    plug  a  continuum  data into the last channel, this could be set to the
    number of channels. 0 0 means all channels are averaged to  compute  the
    "continuum" image.

\end{verbatim}
\subsubsection{SELF\_COLOR}
\index{SELFCAL!SELF\_COLOR}
\begin{verbatim}
            SELF_COLOR

    Controls the LUT color range at each self-calibration cycle if non Zero.
    Since the dynamic range evolves, this can be a useful way  to  highlight
    the  gain. If non Zero, SELF_COLOR is passed as argument to a COLOR com-
    mand executed at each cycle after display.

    Pratical values for SELF_COLOR are -8 ("bright" version, highlights  the
    noise level) or +8 ("dark" version, hides the noise), but lower absolute
    values may be needed for higher dynamic ranges.

    See HELP COLOR for details.

\end{verbatim}
\subsubsection{SELF\_LOOP}
\index{SELFCAL!SELF\_LOOP}
\begin{verbatim}

    Number of self-iteration loops. It is a ReadOnly variable that is  auto-
    matically  computed  from  the  size  of  the SELF_TIMES, SELF_NITER and
    SELF_MINFLUX arrays.

      These variable length arrays can be resized using the LET /RESIZE com-
    mand. For example
        LET SELF_TIMES 40 20 10 /RESIZE
    will lead to an array of 3 elements, SELF_TIMES[3].

    SELF_TIMES,  SELF_NITER  and SELF_MINFLUX must be of equal size. To sim-
    plify the process, constant arrays are assumed of  arbitrary  length  in
    this  determination.   If  all 3 arrays have constant values, SELF_TIMES
    determines the number of loops.

    Constant arrays are assumed of arbitrary length in  this  determination.
    If  all  3 arrays have constant values, SELF_TIMES determines the number
    of loops.

\end{verbatim}
\subsubsection{SELF\_NITER}
\index{SELFCAL!SELF\_NITER}
\begin{verbatim}

    Variable length array specifying the number of Clean components retained
    for  each  loop  of the self-calibration process.  Default is 0, meaning
    all Clean components found by CLEAN.

    For simple structures and Phase calibration, 10 may be enough. For  more
    complex  ones,  be sure to include enough Clean components in the model.
    More components can be taken at each step, although the better knowledge
    of phase errors often allows the source to be represented with a smaller
    number of components after self-calibration than before.  The default is
    a  reasonably  good  compromise,  although faster convergence may be ob-
    tained with smaller number of components.

    This variable length array can be resized using the LET /RESIZE command.
    For example
        LET SELF_NITER 10 0 0 /RESIZE
    will  lead  to  3 self-calibration loops (if SELF_TIMES and SELF_MINFLUX
    are also of size 3), the first one selecting only 10 components, the two
    others all components.

\end{verbatim}
\subsubsection{SELF\_TIMES}
\index{SELFCAL!SELF\_TIMES}
\begin{verbatim}

    Variable  length  array specifying the integration time (in seconds) for
    each loop of the self-calibration process.

    At NOEMA, 45 sec is the normal minimum integration  time.  Depending  on
    Signal  to  Noise,  you  may need to have this longer, e.g. 120 sec. For
    several loops, start with a longer value, and decrease only at the end.

    At ALMA, the minimum integration time is 6 sec. At VLA, this may  be  as
    small as 1 sec.

    It  is recommended to use the same integration time for the last two it-
    erations, to simplify the interpretation of the results and of the SELF-
    CAL SHOW display.

    This variable length array can be resized using the LET /RESIZE command.
    For example
        LET SELF_TIMES 180 90 45 /RESIZE
    will lead 3 self-calibration loops (if SELF_TIMES and  SELF_MINFLUX  are
    also of size 3) with decreasing integration times.

\end{verbatim}
\subsubsection{SELF\_MINFLUX}
\index{SELFCAL!SELF\_MINFLUX}
\begin{verbatim}

    Variable  length  array specifying the minimum flux density (in Jy/beam)
    to be considered in the Clean image for each loop of  the  self-calibra-
    tion  process. Note that this is the brightness of a pixel, not the flux
    of a Clean component.

    This variable length array can be resized using the LET /RESIZE command.
    For example
        LET SELF_MINFLUX 0.001 0 0  /RESIZE
    will lead 3 self-calibration loops (if SELF_TIMES and SELF_NITER are al-
    so of size 3) selecting on regions brighter than 1mJy/beam in the  first
    one, and all regions in the last 2 ones.

\end{verbatim}
\subsubsection{SELF\_REFANT}
\index{SELFCAL!SELF\_REFANT}
\begin{verbatim}

    The  reference  antenna number. If 0, the program will peak the one with
    the shortest average baselines.

\end{verbatim}
\subsubsection{SELF\_FLUX}
\index{SELFCAL!SELF\_FLUX}
\begin{verbatim}

    Maximum value in the FLUX window. If 0, the FlUX window of Clean is  not
    displayed

\end{verbatim}
\subsubsection{SELF\_PRECISION}
\index{SELFCAL!SELF\_PRECISION}
\begin{verbatim}

    Tolerance to test for self-calibration convergence. The default is 0.01.
    SELFCAL SUMMARY will write a message when no more selfcal iteration  im-
    proves the solution (noise and dynamic range) at this precision level.

\end{verbatim}
\subsubsection{SELF\_RESTORE}
\index{SELFCAL!SELF\_RESTORE}
\begin{verbatim}

    Use  UV_SELF /RESTORE after Cleaning. This avoids signal aliasing at im-
    age edges, and leads to a better estimate of the noise level.

    It also allows to use smaller images, in practice,

\end{verbatim}
\subsubsection{SELF\_DISPLAY}
\index{SELFCAL!SELF\_DISPLAY}
\begin{verbatim}

    If YES, display Clean image before each calibration loop, and prompt for
    user  input. If YES, Cleaning at each step will use the number of itera-
    tions specified by NITER, while if set to  NO,  Cleaning  will  stop  at
    SELF_NITER, saving time.

    The  dynamic  range  progress report is accurate only if SELF_DISPLAY is
    set.

\end{verbatim}
\subsubsection{SELF\_FLAG}
\index{SELFCAL!SELF\_FLAG}
\begin{verbatim}

        If SELF_FLAG is YES, SELFCAL will flag data with no solution. If NO,
    it will keep the data as it was before.

\end{verbatim}
\subsubsection{SELF\_SNR}
\index{SELFCAL!SELF\_SNR}
\begin{verbatim}

    Minimum Signal to Noise ratio on the antenna gain to consider a solution
    to be valid for an antenna. 6 is a good value, 3 a lower  limit.  Beware
    that  this  SNR  makes  sense only if the a priori estimate of the noise
    from the UV weights is correct: see SELF_SNOISE.

\end{verbatim}
\subsubsection{SELF\_SNOISE}
\index{SELFCAL!SELF\_SNOISE}
\begin{verbatim}

    Noise scaling factor. This should be 1, but some  noise  estimates  need
    corrections.  Continuum  data  from ALMA often requires sqrt(2) instead,
    for example.  Command UV_PREVIEW may help you checking the noise scale.

\end{verbatim}
\subsubsection{CLEAN\_ARES}
\index{SELFCAL!CLEAN\_ARES}
\begin{verbatim}

        Maximum absolute residual to stop Cleaning

\end{verbatim}
\subsubsection{CLEAN\_FRES}
\index{SELFCAL!CLEAN\_FRES}
\begin{verbatim}

        Maximum fractional residual to stop Cleaning

\end{verbatim}
\subsubsection{CLEAN\_NITER}
\index{SELFCAL!CLEAN\_NITER}
\begin{verbatim}

    Maximum number of Clean components. If all of CLEAN_ARES, CLEAN_FRES and
    CLEAN_NITER  are 0, Clean stops by checking the stability of the cleaned
    flux over CLEAN_NKEEP iterations. See HELP CLEAN for details.



\end{verbatim}
\subsection{SLICE}
\index{SLICE}
\begin{verbatim}
        [ADVANCED\]SLICE VarName Start1 Start2 End1 End2 UNIT

      Cut a slice through the 3-D image variable VarName with the  specified
    edges and put it in the SLICE image variable. UNIT is a keyword indicat-
    ing in which units the edges are specified. It can  be  PIXELS,  DEGREE,
    MINUTE,  SECOND, RADIAN, of ABSOLUTE.  For ABSOLUTE, Start1 and End1 are
    assumed to be in hours and Start2 End2 in degrees, with sexagesimal  no-
    tation allowed.

\end{verbatim}
\subsection{STOKES}
\index{STOKES}
\begin{verbatim}
        [CLEAN\]STOKES Key /FILE UVin UVout

    Derive  a  single  polarization  UV  table (UVout) with the polarization
    state specified by Key from a multi-polarization UV table  (Uvin).   Key
    is any of the following: NONE, I, Q, U, V, RR, LL, HH, VV, XX, YY.

    A typical use is after command FITS on CASA data:
      FITS Fits.uvfits TO UVin.uvt
      STOKES NONE /File UVin Uvout

\end{verbatim}
\subsection{UV\_ADD}
\index{UV\_ADD}
\begin{verbatim}
        [ADVANCED\]UV_ADD ITEM [Mode] [/FILE FileIn FileOut]

      Compute  and  add  some missing information in a UV Table, such as the
    Doppler correction and the Parallactic Angle

    The information is derived from the Observatory  coordinates,  from  the
    Telescope name in the input UV table. This can be supplied by the SPECI-
    FY TELESCOPE command if not available.

    ITEM can take values DOPPLER, PARALLACTIC or * for both.

    Mode is a debug control integer which indicates in which column the  in-
    formation  should be placed. The default is 0, i.e. the command will re-
    use the column of the appropriate type, or add it if  not  present.   It
    may  be  set  to 3, as often column 3 contains the Scan number, which is
    not a relevant information for imaging.  Other values are at the  user's
    peril...


\end{verbatim}
\subsubsection{UV\_ADD /FILE}
\index{UV\_ADD!/FILE}
\begin{verbatim}
        [ADVANCED\]UV_ADD ITEM [Mode] /FILE FileIn FileOut

    Use the UV data in the file FileIn, and write the completed visibilities
    in the file FileOut.

\end{verbatim}
\subsection{UV\_DEPROJECT}
\index{UV\_DEPROJECT}
\begin{verbatim}
        [ADVANCED\]UV_DEPROJECT x0 y0 Rota Incli

    Deproject a UV table for inclination and orientation  (Incli,  Rota,  in
    Degrees)  around  a  specified  center (x0,y0  in Radians).  This can be
    useful for almost planar or cylindrical objects (e.g.  galaxies, or pro-
    toplanetary disks)

    The result becomes the current UV table.

\end{verbatim}
\subsection{UV\_FIT}
\index{UV\_FIT}
\begin{verbatim}
        [ADVANCED\]UV_FIT   [Func1 .. FuncN]  [/QUIET] [/SAVE File] [/WIDGET
    N]

    Fit UV data with a few simple functions. Func1 to FuncN (currently  N  <
    5)  are  the  names of the functions to be fitted. If not specified, the
    last names (or those attributed by the /WIDGET options) are used.

    The models are either simple functions or linear combinations of  simple
    functions.   The results of the fitting process are the position offsets
    in R.A. and Dec (in arc second) of the model source from the phase  ref-
    erence  center,  and  its flux (Jansky).  Depending on the fitting func-
    tions additional fit results are possible. Currently supported distribu-
    tions and additional fit parameters are:
    POINT     Point source              : None
    E_GAUSS   Elliptic Gaussian source  : FWHM Axes (Major and Minor), Pos Ang
    C_GAUSS   Circular Gaussian source  : FWHM Axis
    C_DISK    Circular Disk             : Diameter
    E_DISK    Elliptical (inclined) Disk: Axis (Major and Minor), Pos Ang
    RING      Annulus                   : Diameter (Inner and Outer)
    EXPO      Exponential brightness    : FWHM Axis
    POWER-2   B = 1/r^2                 : FWHM Axis
    POWER-3   B = 1/r^3                 : FWHM Axis
    E_RING    Inclined Annulus          : Inner, Outer, Pos Ang, Ratio

    The  function  parameters  are found in a SIC structure named UVF% under
    names UVF%PARi%PAR[7]  (starting  values),  UVF%PARi%RANGE[7]  (starting
    ranges)  and UVF%PARi%START[7] (number of starts). The /WIDGET option is
    a convenient way to set these variables.

    UV_RESIDUAL will compute the fit residual when used after UV_FIT,  while
    it  computes  the  residual  from the Clean component list if used after
    CLEAN.

    WRITE UV_FIT file[.uvfit] will save the fit results in a  GILDAS  table,
    in the same format than that of the UV_FIT task.

    This  command is similar to the task UV_FIT, but works on the current UV
    buffer.

\end{verbatim}
\subsubsection{UV\_FIT /QUIET}
\index{UV\_FIT!/QUIET}
\begin{verbatim}
        [ADVANCED\]UV_FIT   /QUIET  [/WIDGET N]

    Activate the quiet mode. Only a progress report (25%, 50% and 75%  done)
    is  issued in this case, not a per-channel message.  This mode is recom-
    mended for many channels.

\end{verbatim}
\subsubsection{UV\_FIT /SAVE}
\index{UV\_FIT!/SAVE}
\begin{verbatim}
        [ADVANCED\]UV_FIT   /SAVE OutputFile

    Save the input parameters, into a text output file. This file is then  a
    script which can be re-executed to set the input parameters for UV_FIT.

    If the data has only 1 channel, the fit results (see HELP UV_FIT Result-
    Values) are also written in the same file.

\end{verbatim}
\subsubsection{UV\_FIT ResultValues}
\index{UV\_FIT!ResultValues}
\begin{verbatim}

        For data with only 1 channel, the fit UV_FIT results  are  available
    as  variables   UVF%PARi%RESULTS[7]  (for  the results) and UVF%PARi%ER-
    RORS[7] (for their formal errors) for every  function  number  i.  These
    variables  are written in the output file by the UV_FIT /SAVE OutputFile
    command.

    They are also available in the internal table named UV_FIT, as  for  any
    other number of channels (see HELP UV_FIT ResultTable)

\end{verbatim}
\subsubsection{UV\_FIT /WIDGET}
\index{UV\_FIT!/WIDGET}
\begin{verbatim}
        [ADVANCED\]UV_FIT   /WIDGET N [/QUIET]

    Create  a  Widget to specify the function names and input parameters for
    UV_FIT for N functions.  Once these are defined by  the  user,  clicking
    the GO button will launch the computation.

\end{verbatim}
\subsubsection{UV\_FIT ResultTable}
\index{UV\_FIT!ResultTable}
\begin{verbatim}

    The  UV_FIT  results  are stored in an internal table named UV_FIT. This
    table can later be saved using command WRITE UV_FIT FileName. The  table
    is  organized  as a MxN matrix, where M is the number of channels in the
    UV data and the organization in N is the following:
    N:  P1 P2 P3 Vel A1 A2 A3 Par1 Err1 Par2 Err2 ... A1 A2 A3 Par1 Err1 ...

    where P1 = RMS of the fitting process
          P2 = number of supplied functions  (NF$)
          P3 = total number of parameter
         Vel = velocity of the i-th channel (i lies between 1 and M)
          A1 = 1 (for the first function), = 2 (for the second function)
          A2 = function code (POINT = 1, E_GAUSS = 2, ... , POWER-3 = 8)
          A3 = number of parameters of the function
        Parx = result of the fit parameter (order of appearance in PARAMxx$)
        Errx = error of the fitting process for the parameter Parx

    For example, fitting models with the two  functions  POINT  and  C_GAUSS
    produce files with N=24.

    SHOW UV_FIT will display plots from this table.

\end{verbatim}
\subsection{UV\_MERGE}
\index{UV\_MERGE}
\begin{verbatim}

          [ADVANCED]\UV_MERGE OutFile /FILES In1 In2 ... Inn
        [/MODE [STACK|CONTINUUM [Index [Frequency]]]
        [/SCALES F1 ... Fn] [/WEIGHTS W1 ... Wn ]

      Merge  many  UV data files, with calibration factor and weight factors
    and (for Line data) spectral resampling as in the first  one. OutFile is
    the name of the output UV table.

      For  Line  data,  the  default is to merge lines of the same molecular
    transition (same Rest Frequency). However the STACK mode allows stacking
    UV data from different spectral lines, re-aligned in velocity.  This can
    allow detection of molecules with many transitions.

      For Continuum data (1 channel and/or option /MODE CONTINUUM), a  spec-
    tral  index  and  a reference Frequency can be specified to merge all UV
    tables.

\end{verbatim}
\subsubsection{UV\_MERGE /FILES}
\index{UV\_MERGE!/FILES}
\begin{verbatim}

          [ADVANCED]\UV_MERGE OutFile /FILES In1 In2 ... Inn
        [/MODE [STACK|CONTINUUM [Index [Frequency]]]
        [/SCALES F1 ... Fn] [/WEIGHTS W1 ... Wn ]

      Specify the names of the UV tables to be merged. The first one is used
    as a reference for resampling (line data) or Frequency (continuum data).

\end{verbatim}
\subsubsection{UV\_MERGE /MODE}
\index{UV\_MERGE!/MODE}
\begin{verbatim}

          [ADVANCED]\UV_MERGE OutFile /FILES In1 In2 ... Inn
        /MODE [STACK|CONTINUUM [Index [Frequency]]
        [/SCALES F1 ... Fn] [/WEIGHTS W1 ... Wn ]

      Specify the merging mode.

      For spectral line UV tables (more than 1 channel), the default is that
    the spectral lines must have the same Rest Frequency. Resampling (in ve-
    locity, which is then identical in frequency) is done on the grid of the
    first UV table In1.

      In mode STACK, the Rest Frequencies can  differ.  Resampling  is  then
    done in velocity, and the (u,v) coordinates scaled as the Rest Frequency
    ratios to conserve the angular resolution of the data.  The /SCALES  and
    /WEIGHT factors can be used to incorporate prio knowledge of the expect-
    ed line rations to optimize S/N.

      For continuum UV tables (1 channel or option /MODE CONTINUUM), a spec-
    tral  index  can  be specified, as well as a reference Frequency.  (u,v)
    coordinates are scaled appropriately, as well as  Flux  and  Weights  in
    this  case.  The /SCALES and /WEIGHTS factors are applied on top of this
    automatic spectral index scaling. Input Line UV Tables  are  treated  as
    multi-frequency Continuum ones (as in UV_CONTINUUM command).

\end{verbatim}
\subsubsection{UV\_MERGE /SCALES}
\index{UV\_MERGE!/SCALES}
\begin{verbatim}

          [ADVANCED]\UV_MERGE OutFile /FILES In1 In2 ... Inn
        [/MODE [STACK|CONTINUUM [Index [Frequency]]]
        /SCALES F1 ... Fn [/WEIGHTS W1 ... Wn ]

      Specify the flux scaling factors for each UV table.

      For  /MODE  CONTINUUM,  the  spectral index is applied separately from
    this flux scale factor (by further multiplication).

\end{verbatim}
\subsubsection{UV\_MERGE /WEIGHTS}
\index{UV\_MERGE!/WEIGHTS}
\begin{verbatim}

          [ADVANCED]\UV_MERGE OutFile /FILES In1 In2 ... Inn
        [/MODE [STACK|CONTINUUM [Index [Frequency]]]
        [/SCALES F1 ... Fn] /WEIGHTS W1 ... Wn

      Specify the weight scaling factors for each UV table. The weight scal-
    ing factor is independent of the flux scaling factor. This means that to
    conserve the Signa-to-Noise ration, one should normally use Wi = 1/Fi^2.

      For /MODE CONTINUUM, the spectral index  is  applied  separately  from
    this weight scaling factor (by further multiplication).

\end{verbatim}
\subsection{UV\_PREVIEW}
\index{UV\_PREVIEW}
\begin{verbatim}
        [ADVANCED\]UV_PREVIEW [Ntapers [Threshold [NHist]]] [/FILE FileIn]

    A  fast previewer to figure out if there is signal and what is its spec-
    tral shape.  The command attempts to find out the line free regions  and
    to estimate a continuum region.

    The  output  of  UV_PREVIEW  can be used for further processing commands
    (UV_BASELINE and UV_FILTER, UV_SPLIT, SPECIFY, etc...)

    If a catalog is defined (see HELP CATALOG), it will  also  display  line
    identification (red for detected ones, blue for the others).

      Optional arguments are
      Ntapers:      number of scale sizes  (default 4)
      Threshold:    Truncation level in Sigma (default 3.5)
      Nhist:        Histogram size  (default is variable)

\end{verbatim}
\subsubsection{UV\_PREVIEW /FILE}
\index{UV\_PREVIEW!/FILE}
\begin{verbatim}
        [ADVANCED\]UV_PREVIEW [Ntapers [Threshold [NHist]]] /FILE FileIn

    Without /FILE, UV_PREVIEW works from the current UV data set.

    With  the /FILE option, it will pre-view the UV data set from the speci-
    fied file. Edge channels are automatically dropped in this  process,  as
    many  telescopes  (NOEMA  or  ALMA) do not produce useable data in these
    ones. The default drop is 5% of bandwidth on each side.

\end{verbatim}
\subsubsection{UV\_PREVIEW Algorithm}
\index{UV\_PREVIEW!Algorithm}
\begin{verbatim}

        UV_PREVIEW computes for Ntapers different tapers  the  spectrum  to-
    wards  the phase center. The taper ranges are determined from the avail-
    able baseline lengths and telescope diameter.

    For each spectrum, UV_PREVIEW then attempts to figure out  if  there  is
    line  emission and the line-free channels to define the continuum level.
    This is based on the histogram of  the  intensity  distribution  of  all
    channels. The most likely value and the noise level is derived from this
    histogram. An iterative scheme, blanking out of range  (presumably  line
    emission) channels, is used for this to converge towards a Gaussian his-
    togram, which normally represents the noise distribution around the con-
    tinuum level.

    As  a last step, blanked channels are accumulated in a list of channels,
    which thus contain possible line emission at any of the Ntapers scales.

\end{verbatim}
\subsubsection{UV\_PREVIEW Limitations}
\index{UV\_PREVIEW!Limitations}
\begin{verbatim}

        UV_PREVIEW cannot identify lines if there are two few  channels.  It
    will  only display the spectra in this case. A minimum of 32 channels is
    required, but confused spectra may also prevent a proper  line  recogni-
    tion.

    The  line  detection is based on the current specified velocity. If this
    is incorrect, lines may appear shifted and considered has not  detected.
    PREVIEW%TOLERANCE indicates the matching precision in frequency (default
    is 2 MHz).

    Currently, no account for a Redshift is made.

\end{verbatim}
\subsubsection{UV\_PREVIEW Output}
\index{UV\_PREVIEW!Output}
\begin{verbatim}

          UV_PREVIEW returns the list  of  possible  line  channels  through
    variable  PREVIEW%CHANNELS.  If  no  line emission was identified at any
    scale, the list is empty and the variable does not exist.

    With this list, the user can compute the continuum image, using commands
    UV_FILTER /CHANNELS PREVIEW%CHANNELS (or simply UV_FILTER), then UV_CON-
    TINUUM and the usual UV_MAP and CLEAN. Alternatively, the user can  fil-
    ter out the continuum emission using UV_BASELINE.

    If  a  catalog  is  present UV_PREVIEW also creates two other variables,
    PREVIEW%EDGES and PREVIEW%FREQUENCIES which contains the start  and  end
    channels  (for  %EDGES, resp. frequencies for %FREQUENCIES) of each con-
    tiguous range of channels in PREVIOUS%CHANNELS. These variables are used
    for line identification and imaging in the
      @ image_lines
    script.

    Finally,  UV_PREVIEW  returns in PREVIEW%FMIN PREVIEW%FMAX the frequency
    coverage, and in PREVIEW%FREQ the rest  frequency  of  the  most  likely
    spectral line in the window, and in PREVIEW%LINES its name.

    If  no  spectral line has been identified, PREVIEW%FREQ is just the mean
    of PREVIEW%FMIN and PREVIEW%FMAX, and PREVIEW%LINES does not exist.

\end{verbatim}
\subsection{UV\_RADIAL}
\index{UV\_RADIAL}
\begin{verbatim}
        [ADVANCED\]UV_RADIAL x0 y0 Rota Incli  [/SAMPLING QSTEP [QMIN QMAX]]
    [/U_ONLY] [/ZERO [Flux]]

    Compute  a  UV  table containing the radial distribution of the azimutal
    average of the visibilities after deprojection for inclination and  ori-
    entation  (Incli, Rota, in Degrees) around a specified center (x0,y0  in
    Radians).

    The result becomes the current UV table.

    If not specified, x0 y0 Rota and Incli default to 0.

\end{verbatim}
\subsubsection{UV\_RADIAL /SAMPLING}
\index{UV\_RADIAL!/SAMPLING}
\begin{verbatim}
        [ADVANCED\]UV_RADIAL x0 y0 Rota Incli /SAMPLING  QSTEP  [QMIN  QMAX]
    [/U_ONLY] [/ZERO [Flux]]

    Specify  the  sampling  of the UV distances: Qstep is the step, Qmin and
    Qmax the min and max. Distances are in meter. If not present,  an  auto-
    matic  guess is made from the minimum and maximum baselines and the dish
    diameter.

\end{verbatim}
\subsubsection{UV\_RADIAL /U\_ONLY}
\index{UV\_RADIAL!/U\_ONLY}
\begin{verbatim}
        [ADVANCED\]UV_RADIAL x0 y0 Rota Incli /U_ONLY [/SAMPLING QSTEP [QMIN
    QMAX]] [/ZERO [Flux]]

    Indicate  that  the resulting UV table should have all V values equal to
    zero. This is convenient to display the azimutal average of the visibil-
    ities as a function of UV distance, but cannot be used for further imag-
    ing.

    If not present, the resulting UV table has a (u,v) coverage which is ex-
    tended  by  rotation,  so that it can be used to image the (rotationally
    symmetric) 2-D radial distribution using standard commands  like  UV_MAP
    and  CLEAN.   The radial profile of the brightness distribution can then
    be recovered by using any radial cut through this image.

\end{verbatim}
\subsubsection{UV\_RADIAL /ZERO}
\index{UV\_RADIAL!/ZERO}
\begin{verbatim}
        [ADVANCED\]UV_RADIAL x0 y0 Rota Incli /ZERO [Flux] [/SAMPLING  QSTEP
    [QMIN QMAX]] [/U_ONLY]

    Add  the zero spacing flux to the azimutal average.  If no value is giv-
    en, the zero spacing flux is extrapolated from  the  shortest  baselines
    using a parabolic interpolation centered on (u,v)=(0,0).


\end{verbatim}
\subsection{UV\_SHORT}
\index{UV\_SHORT}
\begin{verbatim}
        [ADVANCED\]UV_SHORT [Arg]


    Compute the Short Spacings from the current Single Dish dataset (read by
    READ SINGLE) and merge it to the current UV data.

      UV_SHORT takes sensible default guesses for most parameters.  UV_SHORT
    ?   lists  the  essential  parameter values, UV_SHORT ?? some additional
    ones, and UV_SHORT ??? even the debugging control variables.

        The current values can be overriden by the user,  who  need  to  set
    (and  if  needed  to  define  first)  the  corresponding  SHORT_whatever
    variable.  SHORT_SD_FACTOR  is the main one which may need to be  speci-
    fied  by  the user, as the Single Dish data is rarely in the appropriate
    unit.

      The resulting UV table becomes the current UV data, and can   be   im-
    aged, written, etc...

\end{verbatim}
\subsubsection{UV\_SHORT /REMOVE}
\index{UV\_SHORT!/REMOVE}
\begin{verbatim}
        UV_SHORT /REMOVE

      Removes any short spacing from the current UV data set.

\end{verbatim}
\subsubsection{UV\_SHORT Algorithm}
\index{UV\_SHORT!Algorithm}
\begin{verbatim}

    UV_SHORT  task  computes  pseudo-visibilities for short or zero spacings
    from a single dish table of spectra (Class  table)  or  LMV  data  cube.
    These pseudo visibilities are  appended to the current (presumably a Mo-
    saic) UV table.


    Short spacings are computed when the  Interferometer  dish  diameter  is
    smaller than the Single-dish diameter, Zero spacings otherwise (see HELP
    UV_SHORT Zero_Spacing for this case)

    For short spacings, the command performs two steps
      (1) Creation of a "well behaved" map from the spectra.
      (2) Extraction of UV visibilities from this map.

    See HELP UV_SHORT Step_i for detailed explanations of the method steps.

    With recent UV tables and Single Dish CLASS table, most  parameters  are
    automatically  determined.  The  only  parameter to be specified remains
    SHORT_SD_FACTOR (although that one may also be determined  automatically
    if  the  input  single dish data set is in main-beam brightness tempera-
    ture).

    A parameter set to 0 value indicates the appropriate default  should  be
    used.

\end{verbatim}
\subsubsection{UV\_SHORT Zero\_Spacing}
\index{UV\_SHORT!Zero\_Spacing}
\begin{verbatim}

          Zero  spacings  are  computed when the single dish diameter is the
    same as the interferometer dish diameter. Zero spacing  extraction  pro-
    ceeds differently for Class data tables and 3-D data cubes.

    In  the data cube case, the nearest pixel matching the direction of each
    field is taken as the Zero spacing for this field. If there is no  point
    close  enough, according to the specified position tolerance SHORT_TOLE,
    an error occurs.

    In the Class data table case, all spectra within the specified  position
    tolerance  of  a  field center are averaged together to produce the Zero
    spacing. If none is found, an error occcurs.


\end{verbatim}
\subsubsection{UV\_SHORT Step\_1}
\index{UV\_SHORT!Step\_1}
\begin{verbatim}

          Step (1) Creation of a "well behaved" map from the spectra.

    Step (1) only occurs if the input single-dish data  set  (read  by  READ
    SINGLE)  is  a  table of spectra. The table format  is described in  the
    CLASS\GRID command of CLASS.
      The identification of the input single-dish data set  as  a  table  of
    spectra is currently only based on the specified file extension: if that
    is .tab, it is assumed to be a table of spectra.

    It is recommended that this input  table is a collection of single-dish,
    Nyquist sampled  spectra  covering  twice  the interferometric field  of
    view  of interest. However, UV_SHORT does *NOT* make any  assumption. It
    thus  tries to compute  a "well behaved" map by linear operations  (con-
    volutions) from the  original spectra, in an  optimum way   from  signal
    to noise  point of view. The  map is extrapolated smoothly  towards zero
    at  the map  edge in order  to avoid  further  aliasing  in the  Fourier
    transform   operations required in  Step (2).  This extrapolation  has a
    scale  length of  twice the single-dish beam, in order to avoid spurious
    Fourier components.

    In detail, UV_SHORT performs the following operations:

      - Resampling  (in space) of  the original  spectra on  a regular  grid by
        convolution  with  a small  (typically  1/4  of  the single-dish  beam)
        gaussian convolving kernel. In  this process, the weights of individual
        spectra is carried on a weight map.
      - Extrapolation by  zero outside  the convex hull  of the  mapped region.
      - Convolution  of  the  result  by  a  gaussian  twice  as  wide  as  the
        single-dish beam.   Within the  convex hull of  the mapped  region, the
        smoothed map is replaced by the original map.

\end{verbatim}
\subsubsection{UV\_SHORT Step\_2}
\index{UV\_SHORT!Step\_2}
\begin{verbatim}
          Step (2) Extraction of UV visibilities from this map.

    From  the given input data cube, or the "well behaved" data cube created
    by Step (1), UV_SHORT  computes the visibilities in the following way:

      - Fourier transform of the single dish map;
      - Division by  the Fourier  transform of  the single dish  beam, up  to a
        maximum spacing (SHORT_SD_DIAM, in meters);
      - Inverse Fourier transform to the image plane and then for each pointing
        center;
      - Multiplication of the  image by the primary beam  of the interferometer
        elements;
      - Fourier transform back to the UV plane;
      - Creation  of the  visibilities,  with  a given  weight  SHORT_SD_WEIGHT
        and an appropriate calibration factor to Janskys SHORT_SD_FACTOR.

    Both the  single-dish and the  interferometer antennas are  assumed   to
    have gaussian beams (SHORT_SD_BEAM and SHORT_IP_BEAM, in radians).

\end{verbatim}
\subsubsection{UV\_SHORT Variables:}
\index{UV\_SHORT!Variables:}
\begin{verbatim}

      Control  variables  for  UV_SHORT are not predefined, except for the 3
    main ones:  SHORT_SD_FACTOR, SHORT_SD_WEIGHT and SHORT_UV_TRUNC.

      All others should be defined by the user in case the default value  is
    not appropriate, with their appropriate (Real, Char or Logical) types.

\end{verbatim}
\subsubsection{SHORT\_DO\_SINGLE}
\index{UV\_SHORT!SHORT\_DO\_SINGLE}
\begin{verbatim}

    Logical value, should be YES except for test purposes.

\end{verbatim}
\subsubsection{SHORT\_DO\_PRIMARY}
\index{UV\_SHORT!SHORT\_DO\_PRIMARY}
\begin{verbatim}

    Logical value, should be YES except for test purposes.

\end{verbatim}
\subsubsection{SHORT\_IP\_BEAM}
\index{UV\_SHORT!SHORT\_IP\_BEAM}
\begin{verbatim}

    Half-power  beam  width of the interferometer antennas, in  radians. The
    beam is assumed to be gaussian.

      Default value is 0, meaning that the beam is taken from the  Telescope
    section if present.

\end{verbatim}
\subsubsection{SHORT\_IP\_DIAM}
\index{UV\_SHORT!SHORT\_IP\_DIAM}
\begin{verbatim}

    Interferometer  diameter  for  which UV_SHORT will compute short spacing
    visibilities.

      Default value is 0, meaning that the diameter is taken from the  Tele-
    scope section if present.

\end{verbatim}
\subsubsection{SHORT\_MCOL}
\index{UV\_SHORT!SHORT\_MCOL}
\begin{verbatim}

      *** Obsolescent ***

    See  READ SINGLE ClassTable.tab /RANGE  command for an equivalent method
    of selecting the appropriate channel range.

    The first and last column to be mapped. For tables produced by GRID com-
    mand  of CLASS, SHORT_MCOL[1] should be 4 and SHORT_MCOL can be set to 0
    to process all channels.

      Default value: 4 0, appropriate for tables coming from CLASS\GRID com-
    mand.


\end{verbatim}
\subsubsection{SHORT\_MIN\_WEIGHT}
\index{UV\_SHORT!SHORT\_MIN\_WEIGHT}
\begin{verbatim}

    The  minimum  weights  under  which  a  given point in the map should be
    filled by the smooth map rather than by the gridded (original] map.

      Default value: 0.01

\end{verbatim}
\subsubsection{SHORT\_MODE}
\index{UV\_SHORT!SHORT\_MODE}
\begin{verbatim}

    This is an integer code used for backward compatibility  with  an  older
    version of the UV_SHORT task, and also for test purpose.

    Allowed values are :

    -1  indicates to create a single UV table with columns for the Phase
        center offsets only
    -2  indicates to create a UV table with columns for the Pointing
        center offsets
    -3  indicates to create a UV table with the additional columns
        type being Pointing or Phase, as in the original UV_TABLE$
    +1  indicates to append to the initial UV table the short spacings
        with Phase center offsets (which must thus match the initial UV
        table shape)
    +2  indicates to append to the initial UV table the short spacings
        with Pointing center offsets (which must thus match the initial
        UV table shape)
    +3  indicates to append to the initial UV table the short spacings
        (The extra column type being determined automatically).

    The  default  value is 3, i.e. automatic merging with the current UV ta-
    ble.

\end{verbatim}
\subsubsection{SHORT\_SD\_BEAM}
\index{UV\_SHORT!SHORT\_SD\_BEAM}
\begin{verbatim}

    Half-power beam width of the single dish antenna, in radians.  The  beam
    is assumed to be gaussian.

      Default  value is 0, meaning that the beam is taken from the Telescope
    section if present.

\end{verbatim}
\subsubsection{SHORT\_SD\_DIAM}
\index{UV\_SHORT!SHORT\_SD\_DIAM}
\begin{verbatim}

    Single dish diameter used to produce the input spectra, in meters.

      Default value is 0, meaning that the diameter is taken from the  Tele-
    scope section if present.

\end{verbatim}
\subsubsection{SHORT\_SD\_FACTOR}
\index{UV\_SHORT!SHORT\_SD\_FACTOR}
\begin{verbatim}

    Multiplicative calibration factor; it is used to convert from the single
    dish map units (e.g., main-beam brightness temperature) to janskys.

    A default value of 0 can be used if the original data file is in unit of
    Tmb, the main beam brightness temperature, because in this case the con-
    version factor can be derived from the beam size.

\end{verbatim}
\subsubsection{SHORT\_SD\_WEIGHT}
\index{UV\_SHORT!SHORT\_SD\_WEIGHT}
\begin{verbatim}

    Weight scaling factor for the generated visibilities.

    It is a relative scaling factor in the weights compared to a  supposedly
    optimal weighting to give the best combined synthesized beam. That opti-
    mal weighting essentially gives the same weight density par unit area in
    the  UV plane than the shortest baselines measured with the interferome-
    ter only.  However, if the single-dish data has not been  observed  long
    enough, or has baselines problems for example, this weight may add noise
    to the overall data set, so could be down-weighted.

      Default: 1.0

\end{verbatim}
\subsubsection{SHORT\_TOLE}
\index{UV\_SHORT!SHORT\_TOLE}
\begin{verbatim}

    The tolerance in position (in radians). The behaviour differ  for  Short
    and Zero spacings and  Table or 3-D cubes as Single-Dish data.

    If the Single-Dish data is a table of spectra, Spectra differing by less
    than this amount will be added together prior to gridding. A recommended
    value is below 1/10th of the Single Dish primary beam. This is valid for
    Short Spacings and Zero Spacing cases.

    If the Single-Dish data is 3-D data cube, SHORT_TOLE is  used  only  for
    Zero  Spacings.  If  no  pixel is within SHORT_TOLE of an Interferometer
    pointing center, no short spacing is added for this field and  an  error
    occur.

      Default  value  is 0, meaning using  1/16th of the Single Dish primary
    beam.

\end{verbatim}
\subsubsection{SHORT\_UV\_TRUNC}
\index{UV\_SHORT!SHORT\_UV\_TRUNC}
\begin{verbatim}

    No visibility at spacings higher than SHORT_UV_TRUNC is generated. Theo-
    retical  consideration  on  the  method  used  in this task implies that
    SHORT_UV_TRUNC should be at most (SHORT_SD_DIAM-SHORT_IP_DIAM).  Smaller
    values  may need to be applied if, for example, the pointing accuracy of
    the Single Dish is insufficient.

      Default value is 0, meaning to use SHORT_SD_DIAM-SHORT_IP_DIAM

\end{verbatim}
\subsubsection{SHORT\_WCOL}
\index{UV\_SHORT!SHORT\_WCOL}
\begin{verbatim}


    For tests only: The column of the spectra table containing the weights.

      Default value: 0=3, appropriate for tables coming from CLASS\GRID com-
    mand.

\end{verbatim}
\subsubsection{SHORT\_WEIGHT\_MODE}
\index{UV\_SHORT!SHORT\_WEIGHT\_MODE}
\begin{verbatim}

    The  weighting mode (NATURAL, UNIFORM or GRIDDED).  It is advised to use
    'NA' for Natural weighting.


\end{verbatim}
\subsubsection{SHORT\_XCOL}
\index{UV\_SHORT!SHORT\_XCOL}
\begin{verbatim}

        For tests only: The column of the spectra table  containing  X  off-
    sets.

      Default value: 0=1, appropriate for tables coming from CLASS\GRID com-
    mand.

\end{verbatim}
\subsubsection{SHORT\_YCOL}
\index{UV\_SHORT!SHORT\_YCOL}
\begin{verbatim}

        For tests only: The column of the spectra table  containing  Y  off-
    sets.

      Default value: 0=2, appropriate for tables coming from CLASS\GRID com-
    mand.

\end{verbatim}
