\subsubsection{CLEAN\_\_ARES}
\index{�!CLEAN\_\_ARES}
\begin{verbatim}

This  is  the minimal flux in the dirty map that the program will
consider as significant.  Alternatively,  the  threshold  can  be
specified as a fraction of the peak flux using CLEAN__FRES.  Once
this level has been reached the program  stops  subtracting,  and
starts the restoration phase.  The unit for this parameter is the
map unit (typically Jy/Beam).  The parameter should usually be of
the order of magnitude of the expected noise in the clean map.

If  0,  CLEAN__FRES will be used instead. If all of CLEAN__NITER,
CLEAN__ARES and CLEAN__FRES are 0, an absolute residual equal  to
the noise level will be used for CLEAN__ARES.

Short form is ARES.

This  may  be overseded by the /ARES option which imposes a limit
per plane through an array of values.

\end{verbatim}
\subsubsection{CLEAN\_\_FRES}
\index{�!CLEAN\_\_FRES}
\begin{verbatim}

This is the minimal fraction of the peak flux in  the  dirty  map
that the program will consider as significant.  Alternatively, an
absolute threshold can be specified using CLEAN__ARES.  Once this
level  has been reached the program stops subtracting, and starts
the restoration phase.  This parameter is normalized to  1  (nei-
ther  in % nor in db).  It should usually be of the order of mag-
nitude of the inverse of the expected dynamic range of the inten-
sity.

If  0,  CLEAN__ARES will be used instead. If all of CLEAN__NITER,
CLEAN__ARES and CLEAN__FRES are 0, an absolute residual equal  to
the noise level will be used for CLEAN__ARES.

Short form is FRES.

\end{verbatim}
\subsubsection{CLEAN\_\_GAIN}
\index{�!CLEAN\_\_GAIN}
\begin{verbatim}

This is the gain of the subtraction loop.  It should typically be
chosen in the range 0.05 and 0.3.  Higher values give faster con-
vergence, while lower values give a better restitution of the ex-
tended structure. A sensible default is 0.2.

Short form is GAIN.

\end{verbatim}
\subsubsection{CLEAN\_\_NITER}
\index{�!CLEAN\_\_NITER}
\begin{verbatim}

This is the maximum number of components the program will  accept
to  subtract.   Once  it has been reached, the program starts the
restoration phase.

If 0, the program will guess a number, based on  the  image  size
and  maximum  signal-to-noise ratio, and specified residual level
CLEAN__ARES and/or CLEAN__FRES.

Short form is NITER.

This may be overseded by the /NITER option which imposes a  limit
per plane through an array of values.

\end{verbatim}
\subsubsection{CLEAN\_\_NKEEP}
\index{�!CLEAN\_\_NKEEP}
\begin{verbatim}

This  is an integer specifying the minimum number of Clean compo-
nents before testing if Cleaning has converged.  The  convergence
is  criterium  is  a  comparison of the cumulative flux evolution
separated by CLEAN__NKEEP components. If th

IF CLEAN__NKEEP is 0, CLEAN will ignore this convergence criteri-
um,  and  continue  clean  until the CLEAN__NITER, CLEAN__ARES or
CLEAN__FRES criteria indicate to stop.

With CLEAN__NKEEP > 0, CLEAN will explore the  stability  of  the
total  clean  flux  over the last CLEAN__NKEEP  iterations. For a
positive (resp.  negative) source,  if  the  Clean  flux  becomes
smaller  (resp.  larger)  than the Clean flux CLEAN__NKEEP itera-
tions earlier, CLEAN will stop.

Using CLEAN__NKEEP about 70 is a reasonable value.  Some  special
cases  (faint  extended  sources)  may  require  larger values of
CLEAN__NKEEP.

\end{verbatim}
\subsubsection{CLEAN\_\_POSITIVE}
\index{�!CLEAN\_\_POSITIVE}
\begin{verbatim}

The minimum number of positive components  before  negative  ones
are selected.

\end{verbatim}
\subsubsection{CLEAN\_\_RESTORE}
\index{�!CLEAN\_\_RESTORE}
\begin{verbatim}

  Fraction  of  peak response of the primary beams coverage under
which the Sky brightness image is blanked in a Mosaic  deconvolu-
tion.

The default is 0.2.

\end{verbatim}
\subsubsection{CLEAN\_\_SEARCH}
\index{�!CLEAN\_\_SEARCH}
\begin{verbatim}


  Fraction  of peak response of the primary beams coverage beyond
which no Clean component is searched in a Mosaic deconvolution.

The default is 0.2.

\end{verbatim}
\subsubsection{CLEAN\_\_SIDELOBE}
\index{�!CLEAN\_\_SIDELOBE}
\begin{verbatim}

Minimal relative intensity to consider for fitting the syntheized
beam to obtain the Clean beam parameters (MAJOR, MINOR and ANGLE)
when 0.  The default is 0.35.

In case of poor UV coverage,  CLEAN__SIDELOBE  should  be  higher
than  the  maximum sidelobe level to perform a good Gaussian fit.
Some particularly bad UV coverage may not allow any good  fit  at
all, however.

\end{verbatim}
\subsubsection{CLEAN\_\_NGOAL}
\index{�!CLEAN\_\_NGOAL}
\begin{verbatim}

  Number of clean components to be selected in a Cycle in the AL-
MA heterogeneous array cleaning method.

\end{verbatim}
\subsubsection{CLEAN\_\_NCYCLE}
\index{�!CLEAN\_\_NCYCLE}
\begin{verbatim}

  Maximum number of Major Cycles for the SDI and CLARK methods.

\end{verbatim}
\subsubsection{CLEAN\_\_SMOOTH}
\index{�!CLEAN\_\_SMOOTH}
\begin{verbatim}

  Smoothing factor between different scales in the MRC and MULTI-
SCALE  methods.  The default is 2 or 4 for MRC depending on image
size. It is sqrt(3) for MULTISCALE and may be set to larger  val-
ues if needed.

\end{verbatim}
\subsubsection{CLEAN\_\_SPEEDY}
\index{�!CLEAN\_\_SPEEDY}
\begin{verbatim}

  Speed-up factor for the CLARK major cycles. The default is 1.0.
Larger values may be used, but at the expense of possible  insta-
bilities of the algorithm.

\end{verbatim}
\subsubsection{CLEAN\_\_WORRY}
\index{�!CLEAN\_\_WORRY}
\begin{verbatim}

  Worry factor in the MULTISCALE method for convergence. It prop-
agates the S/N from one iteration to the other, so that  if  this
S/N degrades, the method stops. Default is 0 (no propagation, and
hence no test on S/N).  The value should be < 1.0 in all cases.

\end{verbatim}
\subsubsection{CLEAN\_\_INFLATE}
\index{�!CLEAN\_\_INFLATE}
\begin{verbatim}

  Maximum Inflation factor for UV__RESTORE  (MULTISCALE  method).
If  the  number of true (i.e. pixel based) Clean components found
by MULTISCALE is larger than CLEAN__INFLATE times the  number  of
compressed  (i.e.  those  with  the smoothing factor information)
components, expansion of the compressed components  will  not  be
possible, and UV__RESTORE will not be useable.

  A default of 50 is in general adequate.  Better solutions might
be found in the future, and this parameter suppressed. Apart from
memory usage, this number has no consequence on the algorithm.

\end{verbatim}
\subsubsection{METHOD}
\index{�!METHOD}
\begin{verbatim}

  Method used for the deconvolution. Can be HOGBOM, MULTI, MRC,
  SDI or CLARK.


\end{verbatim}
\subsubsection{Old\_\_Names:}
\index{�!Old\_\_Names:}
\begin{verbatim}

  Some  of the CLEAN parameters have kept their old names: MAJOR,
MINOR, ANGLE (which are also used by command FIT)  BLC,  TRC  and
BEAM__PATCH (which are seldom used)

  Others have equivalent short names: ARES, FRES, GAIN, NITER for
which the CLEAN__ prefix may be omitted.

\end{verbatim}
\subsubsection{BLC}
\index{�!BLC}
\begin{verbatim}

These are the (pixel) coordinates of the Bottom  Left  Corner  of
the  cleaning box.  The default (0,0) means the bottom left quar-
ter (Nx/4,Ny/4).  If a SUPPORT is defined, BLC is ignored.

\end{verbatim}
\subsubsection{TRC}
\index{�!TRC}
\begin{verbatim}

These are the (pixel) coordinates of the Top Right Corner of  the
cleaning  box.   The  default  (0,0)  means the top right quarter
(3*Nx/4,3*Ny/4).  If a SUPPORT is defined, TRC is ignored.

\end{verbatim}
\subsubsection{MAJOR}
\index{�!MAJOR}
\begin{verbatim}

This is the major axis (FWHP) in user coordinates of the Gaussian
restoring  beam.  If  0,  the  program will fit a Gaussian to the
dirty beam. We strongly discourage to change the default value of
0.

\end{verbatim}
\subsubsection{MINOR}
\index{�!MINOR}
\begin{verbatim}

This is the minor axis (FWHP) in user coordinates of the Gaussian
restoring beam. If 0, the program will  fit  a  Gaussian  to  the
dirty beam. We strongly discourage to change the default value of
0.

\end{verbatim}
\subsubsection{ANGLE}
\index{�!ANGLE}
\begin{verbatim}

This is the position angle (from North towards East,  i.e.  anti-
clockwise)  of  the major axis of the Gaussian restoring beam (in
degrees).  If 0, the program will fit a  Gaussian  to  the  dirty
beam. We strongly discourage to change the default value of 0.

\end{verbatim}
\subsubsection{BEAM\_\_PATCH}
\index{�!BEAM\_\_PATCH}
\begin{verbatim}

The dirty beam patch to be used for the minor cycles in CLARK and
MRC method.  It should be large enough to avoid  doing  too  many
major  cycles,  but  has  practically no influence on the result.
This size should be specified in pixel units.  Reasonable  values
are  between  N/8 and N/4, where N is the number of map pixels in
the same dimension.  If set to N,  the  CLARK  algorithm  becomes
identical to the HOGBOM algorithm.




















































\end{verbatim}
