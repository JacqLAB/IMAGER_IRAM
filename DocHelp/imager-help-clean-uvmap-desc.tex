\subsubsection{MAP\_\_BEAM\_\_STEP}
\index{�!MAP\_\_BEAM\_\_STEP}
\begin{verbatim}

  MAP__BEAM__STEP   Integer

Number of channels per synthesized beam plane.

Default is 0, meaning only 1 beam plane for all channels.  N (>0)
indicates N consecutive channels will share the same dirty beam.

A value of -1 can be used to compute the number of  channels  per
beam plane to ensure the angular scale does not deviate more than
a fraction of the map cell at the map edge. This fraction is con-
trolled by variable MAP__PRECIS (default 0.1)

\end{verbatim}
\subsubsection{MAP\_\_CELL}
\index{�!MAP\_\_CELL}
\begin{verbatim}

      MAP__CELL[2]    Real

The  map  pixel size [arcsec]. It is recommended to use identical
values in X and Y.  A sampling of at least 3 pixel  per  beam  is
recommended  to ease the deconvolution. Enter 0,0 to let the task
find the best values.

\end{verbatim}
\subsubsection{MAP\_\_CENTER}
\index{�!MAP\_\_CENTER}
\begin{verbatim}

      MAP__CENTER     Character String

Specify the Map center and orientation in the same way as the ar-
guments of UV__MAP.

\end{verbatim}
\subsubsection{MAP\_\_CONVOLUTION}
\index{�!MAP\_\_CONVOLUTION}
\begin{verbatim}

    MAP__CONVOLUTION    Integer

Select the desired convolution function for gridding  in  the  UV
plane Choices are
        0    Default (currently 5)
        1    Boxcar
        2    Gaussian
        3    Sin(x)/x
        4    Gaussian * Sin(x)/x
        5    Spheroidal
Spheroidal functions is the optimal choice. So we  strongly  dis-
courage use of any other convolution function, which are here for
tests only.

\end{verbatim}
\subsubsection{MAP\_\_FIELD}
\index{�!MAP\_\_FIELD}
\begin{verbatim}

  MAP__FIELD[2]     Real

Field of view in X and Y in arcsec.  The field of view MAP__FIELD
has  precedence over the number of pixels MAP__SIZE to define the
actual map size when both are non-zero.

\end{verbatim}
\subsubsection{MAP\_\_POWER}
\index{�!MAP\_\_POWER}
\begin{verbatim}

      MAP__POWER[2]     Integer

Maximum exponent of  3  and  5  allowed  in  automatic  guess  of
MAP__SIZE.   MAP__SIZE  is decomposed in 2^k 3^p 5^q, and p and q
must be less or equal to MAP__POWER.

Default is 0: MAP__SIZE is just a power of 2. A value of 1 allows
approximation  of any map size to 20 %, while a value of 2 allows
\end{verbatim}
\subsection{0}
\index{0}
\begin{verbatim}
with  powers of 3 and 5, but limiting the map size can gain a lot
in the Cleaning process (which can scale as MAP__SIZE^4).


\end{verbatim}
\subsubsection{0 MAP\_\_PRECIS}
\index{0!MAP\_\_PRECIS}
\begin{verbatim}

  MAP__PRECIS    Real

Maximum mismatch in pixel at map edge between  the  true  synthe-
sized  beam (which would have been computed using the exact chan-
nel frequency) and the computed synthesized beam with   the  mean
frequency  of  the  channels  sharing the same beam. This is used
(with the actual image size) to derive the actual number of chan-
nels  which  can share the same beam, i.e. the effective value of
MAP__BEAM__STEP when MAP__BEAM__STEP is -1.

Default is 0.1

\end{verbatim}
\subsubsection{0 MAP\_\_ROBUST}
\index{0!MAP\_\_ROBUST}
\begin{verbatim}

  MAP__ROBUST     Real

Robust weighting factor. A number between 0 and +infty.

Robust weighting gives the natural weight to UV cells whose natu-
ral  weight  is lower than a given threshold. In contrast, if the
natural weight of the UV cell is larger than this threshold,  the
weight  is  set  to this (uniform) threshold. The UV cell size is
defined by MAP__UVCELL and the threshold value is in MAP__ROBUST.

0 means natural weighting, which is optimal  for  point  sources.
The Robust weighting factor controls the resolution: better reso-
lution is obtained for small values (at the  expense  of  noise),
resolution  approaching  the  natural  weighting scheme for large
values.  Larger UV cell size give higher angular resolution  (but
again more noise).

MAP__ROBUST around .5 to 1 is a good compromise between noise in-
crease and angular resolution.

\end{verbatim}
\subsubsection{0 MAP\_\_ROUNDING}
\index{0!MAP\_\_ROUNDING}
\begin{verbatim}

  MAP__ROUNDING     Real

Maximum error between optimal size (MAP__FIELD /  MAP__CELL)  and
rounded  (as  a power of 2^k 3^p 5^q) MAP__SIZE to round by floor
(thus limiting the field of  view),  instead  of  ceiling  (which
guarantees a larger field of view, but leads to bigger images).

Default is 0.05.


\end{verbatim}
\subsubsection{0 MAP\_\_SHIFT}
\index{0!MAP\_\_SHIFT}
\begin{verbatim}

  MAP__SHIFT        Logical

Obsolescent, superseded by MAP__CENTER, or the UV__MAP arguments.

Logical variable indicating whether map center (i.e. phase track-
ing center) or orientation should be changed.

\end{verbatim}
\subsubsection{0 MAP\_\_SIZE}
\index{0!MAP\_\_SIZE}
\begin{verbatim}

  MAP__SIZE[2]      Integer

Number of pixels in X and Y. It should preferentially be a  power
of  two, (although this is not strictly required) to speed-up the
FFT. MAP__SIZE*MAP__CELL should be at least twice the size of the
field-of-view  (primary  beam size for a single field). Enter 0,0
to let the command find a sensible map size.

MAP__SIZE is not used if MAP__FIELD is non zero.

Odd values are forbidden.

Default is 0,0, i.e. UV__MAP will guess the most appropriate val-
ues which depend on MAP__ROUNDING and MAP__POWER.


\end{verbatim}
\subsubsection{0 MAP\_\_TAPEREXPO}
\index{0!MAP\_\_TAPEREXPO}
\begin{verbatim}

  MAP__TAPEREXPO    Real

Taper  exponent.  The  default  is  2 (indicating a Gaussian) but
smoother or sharper functions can be used. 1 would give an  Expo-
nential, 4 would be getting close to square profile...

\end{verbatim}
\subsubsection{0 MAP\_\_TRUNCATE}
\index{0!MAP\_\_TRUNCATE}
\begin{verbatim}

  MAP__TRUNCATE    Real

Mosaic  truncation  level in PerCent.  Default value is 0.2. Cur-
rent value can be  overriden  by  option  /TRUNCATE  in  commands
UV__MAP or PRIMARY.

\end{verbatim}
\subsubsection{0 MAP\_\_UVTAPER}
\index{0!MAP\_\_UVTAPER}
\begin{verbatim}

  MAP__UVTAPER[3]  Real

Parameters of the tapering function (Gaussian if MAP__TAPEREXPO =
\end{verbatim}
\subsubsection{0 ):}
\index{0!):}
\begin{verbatim}
position angle [deg].

\end{verbatim}
\subsubsection{MAP\_\_UVCELL}
\index{0!MAP\_\_UVCELL}
\begin{verbatim}

  MAP__UVCELL   Real

UV cell size for robust weighting [m].  Should be of the order of
half the dish diameter (7.5 m for PdBI), or smaller or even larg-
er.  It controls the beam shape in Robust weighting.

\end{verbatim}
\subsubsection{MAP\_\_VERSION}
\index{0!MAP\_\_VERSION}
\begin{verbatim}

  MAP__VERSION  Integer

[EXPERT  Only]  Code  indicating which version of the UV__MAP and
UV__RESTORE algorithm should be used. 0 is  optimal.  -1  is  the
"historical"  (pre-2016)  version.  1  is an intermediate version
used during multi-frequency beams development.



\end{verbatim}
\subsubsection{Old\_\_Names:}
\index{0!Old\_\_Names:}
\begin{verbatim}

NAME is no longer used, and WEIGHT__MODE is obsolete.
MAP__RA          [  hours]  RA of map center
MAP__DEC         [    deg]  Dec of map center
MAP__ANGLE       [    deg]  Map position angle
MAP__SHIFT       [Yes/No ]  Shift phase center
are obsolescent, superseded by MAP__CENTER. They are provided on-
ly for compatibility with older scripts.

WCOL (the Weight channel) and MCOL[2] (the channel range) are ob-
solete also. WCOL has no meaning when more than 1  beam  must  be
produced  for  all channels, and should be set to 0.  MCOL is su-
perseded by the /RANGE option facility.

 NAME        [       ]  Label of the dirty image and beam plots
 UV__TAPER    [m,m,deg]  UV-apodization by convolution with a Gaussian
 WEIGHT__MODE [       ]  Weighting mode (NA|UN)
 UV__CELL     [m, ??  ]  UV cell size and threshold for Robust weighting
 MAP__FIELD   [ arcsec]  Map field of view
 MAP__CELL    [ arcsec]  Map cell size
 MAP__SIZE    [ pixels]  Map size in pixels (if MAP__FIELD is zero)
 MCOL        [       ]  First and Last channel to map
 WCOL        [       ]  Channel from which the weights are taken
 CONVOLUTION [       ]  Convolution function (5)
 UV__SHIFT    [       ]  Change the map phase center or map orientation?
 MAP__RA      [       ]  RA of map phase center
 MAP__DEC     [       ]  Dec of map phase center
 MAP__ANGLE   [    deg]  Map position angle
 MAP__BEAM__STEP [     ]  Number of channels per synthesized beam plane

\end{verbatim}
\subsubsection{convolution}
\index{0!convolution}
\begin{verbatim}

  Older variable name for MAP__CONVOLUTION

\end{verbatim}
\subsubsection{map\_\_angle}
\index{0!map\_\_angle}
\begin{verbatim}

  MAP__ANGLE      Real

Position Angle of the direction which will  become  the  apparent
North in the map. Used only if UV__SHIFT is YES.

Superseded by MAP__CENTER.

\end{verbatim}
\subsubsection{map\_\_dec}
\index{0!map\_\_dec}
\begin{verbatim}

  MAP__DEC     Real

Dec of map center. Used only if UV__SHIFT is YES.

Superseded by MAP__CENTER.

\end{verbatim}
\subsubsection{map\_\_ra}
\index{0!map\_\_ra}
\begin{verbatim}

  MAP__RA      Real

RA of map center. Used only if UV__SHIFT is YES.

Superseded by MAP__CENTER.

\end{verbatim}
\subsubsection{mcol}
\index{0!mcol}
\begin{verbatim}

  mcol[2]   Integer

[Deprecated]  First  and  Last channel to image. Values of 0 mean
imaging all the planes.  See UV__MAP /RANGE for a  more  flexible
way to specify the channel range.

\end{verbatim}
\subsubsection{uv\_\_cell}
\index{0!uv\_\_cell}
\begin{verbatim}

  Older variables for MAP__UVCELL (uv_cell[1]) and MAP__ROBUST

\end{verbatim}
\subsubsection{uv\_\_shift}
\index{0!uv\_\_shift}
\begin{verbatim}

  Older  variable  name  of MAP__SHIFT (this one is also obsoles-
cent)

\end{verbatim}
\subsubsection{uv\_\_taper}
\index{0!uv\_\_taper}
\begin{verbatim}

  Older variable name of MAP__UVTAPER

\end{verbatim}
\subsubsection{taper\_\_expo}
\index{0!taper\_\_expo}
\begin{verbatim}

  Older variable name for MAP__TAPEREXPO

\end{verbatim}
\subsubsection{wcol}
\index{0!wcol}
\begin{verbatim}

  WCOL      Integer

[Obsolescent] The channel from which the weight should be  taken.
WCOL  set  to  0  means using a default channel. WCOL has no real
meaning in all cases where more than one beam is computed for all
channels.

\end{verbatim}
\subsubsection{weight\_\_mode}
\index{0!weight\_\_mode}
\begin{verbatim}

  weight_mode      Character

[Deprecated]  Weighting mode: Natural (optimum in terms of sensi-
tivity) or robust (usually lower  sidelobes  and  higher  spatial
resolution)  weighting.  This was needed in Mapping to toggle be-
tween Natural and Robust weighting, while IMAGER does that  based
on MAP__ROBUST value.








































\end{verbatim}
