\section{Visual Checks and Image Displays}

The ultimate (and often sole) way to evaluate the data quality
and suitability for scientific analysis is to visualize it.
\imager{} offer simple, yet powerfull, tools to do so.

\subsection{the \texttt{UV\_PREVIEW} command}
With large datasets, image can be long. \imager{} offers a
simple, fast pre-imaging viewer through command \com{UV\_PREVIEW}.

This command will display the spectra obtained towards the
phase center at several spatial scales (the default is for 4 tapers).
It will also attempt to detect spectral features, by analyzing for
each spectrum the noise statistics and the outliers. It performs
automatic line identifications, using a database in the \lang{LINEDB}
format in file \texttt{\$HOME/image.linedb} .
Potential spectral lines in the band are displayed in blue, and
detected ones in red.
\com{UV\_PREVIEW} also warns the user about improper scaling
of the data weights.
The line emission region is indicated in grey.

The result of this automatic signal detection and line identification
is saved in a SIC structure named \sicvar{PREVIEW\%}, that is 
automatically used by commands \comm{UV\_BASELINE}{/CHANNEL} and
\comm{UV\_FILTER}{/CHANNEL} to respectively remove the continuum and
filter the line emission to produce a continuum data set. 
The detected line frequencies are stored in
 \sicvar{PREVIEW\%FOUND\%FREQ} and their names in
\sicvar{PREVIEW\%FOUND\%LINES}. This can be used to reference
the velocity scale of the data to one of the detected lines, by
using command \comm{SPECIFY}{FREQUENCY}\texttt{'PREVIEW\%FOUND\%FREQ[1]'} 
for example before further processing.

An example is shown in Fig\ref{fig:preview}
\begin{figure}
  \centering
  \includegraphics[width=16cm]{preview.pdf}
  \caption{The \com{UV\_PREVIEW} output
\label{fig:preview}}
\end{figure}


\subsection{the \texttt{SHOW} command}

In general, command \com{SHOW} allows a per-plane display of any SIC 
3-D image variable, with contours overlaid on bitmap for each plane: 
e.g.\\
\texttt{SHOW DIRTY 30 -30} 
will display contour and bitmaps of each channel of the \texttt{DIRTY}
image, starting for channel 30 and ending 30 channels before the last one.
See Fig.\ref{fig:show} for an example. 

For \uv{}  data, \comm{SHOW}{UV} can plot visibility values
such as amplitude as a function of time, baseline length, etc...,
again on a per channel basis. 

\com{SHOW} can also display more specific issues:
\begin{itemize}\itemsep 0pt
\item \comm{SHOW}{CCT} will display the cumulative flux as a function 
of number of clean components (Fig.\ref{fig:cct}). 
\item \comm{SHOW}{COVERAGE} will display the \uv{} coverage (it
assumes there is only one, and not one per channel, because the
display time is long, see Fig.\ref{fig:coverage})
\item \comm{SHOW}{SELFCAL} behaves as \comm{SELFCAL}{SHOW}
\item \comm{SHOW}{FIELDS} display the fields of a Mosaic.
\item \comm{SHOW}{NOISE} display the histogram of the intensity
distribution for each channel, estimating the noise by fitting
a Gaussian in these histograms (see Fig.\ref{fig:noise}.
\end{itemize}

\begin{figure}
  \centering
  \includegraphics[width=16cm]{show.pdf}
  \caption{The \com{SHOW} output.
\label{fig:show}}
\end{figure}
\begin{figure}
  \centering
  \includegraphics[width=12cm]{cct.pdf}
  \caption{The \comm{SHOW}{CCT} output.
\label{fig:cct}}
\end{figure}
\begin{figure}
  \centering
  \includegraphics[width=12cm]{coverage.pdf}
  \caption{The \comm{SHOW}{COVERAGE} output. Colors indicate
  different dates.
\label{fig:coverage}}
\end{figure}
\begin{figure}
  \centering
  \includegraphics[width=16cm]{noise.pdf}
  \caption{The \comm{SHOW}{NOISE} output.
\label{fig:noise}}
\end{figure}

A direct use on Gildas 3-D image files is also possible:
\com{SHOW}\texttt{Filename.ext} will directly display the
file if possible.

\subsection{the \texttt{VIEW} command}

The \com{VIEW} command is a powerful alternative to \com{SHOW}, the
later being inefficient when the number of spectral line channels
is large. 

\com{VIEW} provides a 4 panel display for 3-D data cubes, with
the current channel bitmap, the integrated area bitmap, the current
spectrum and the integrated intensity spectrum.  The spectra
can be displayed with 2 simultaneous frequency windows, a broad
and a zoomed one, allowing browsing through a large number of
channels. Spectral line identification can be 

\comm{VIEW}{CCT} will display the cumulative flux of Clean components
for all channels in just one panel, instead of a per-channel panel
for \comm{SHOW}{CCT}

\begin{figure}
  \centering
  \includegraphics[width=16cm]{view.pdf}
  \caption{The \com{VIEW} output.
\label{fig:view}}
\end{figure}

\subsection{the \texttt{SELFCAL SHOW} command}

Verifying the convergence of a self-calibration is important.
Figure \ref{fig:selfphase} shows an example, while Fig.\ref{fig:selftot}
shows the total phase correction between the original data and
the last iteration. The displayed range can be controlled
by variables \sicvar{SELF\_PRANGE[2]} for the Phase, 
\sicvar{SELF\_ARANGE[2]} for the Amplitude, and \sicvar{SELF\_TRANGE[2]} 
for the Time. Error bars are displayed if \sicvar{ERROR\_BARS} is \texttt{YES},
as shown in Fig.\ref{fig:selfamp} for amplitude.

\begin{figure}
  \centering
  \includegraphics[width=12cm]{self-phase.pdf}
  \caption{The \comm{SELFCAL}{SHOW} output after a phase calibration,
  showing the convergence of the corrections between the last 2
  iterations.
\label{fig:selfphase}}
\end{figure}
\begin{figure}
  \centering
  \includegraphics[width=12cm]{self-phase-total.pdf}
  \caption{The \comm{SELFCAL}{SHOW}4 output after a phase calibration,
  showing the difference between iteration 4 and the original
  data.  
\label{fig:selftot}}
\end{figure}
\begin{figure}
  \centering
  \includegraphics[width=12cm]{self-ampli-total.pdf}
  \caption{The \comm{SELFCAL}{SHOW}4 output after an amplitude self calibration,
  showing the difference between iteration 4 and the original
  data, with the error bars.  
\label{fig:selfamp}}
\end{figure}
