\subsection{Language}
\index{Language}
\begin{verbatim}
    APPLY         : Apply gain solution to current UV Data
    FLUX_SCALE    : Adjust flux scale on a daily basis
    MODEL         : Compute a UV model from the Clean Components Table
    SOLVE         : Solve for complex gains using the UV model
    UV_SELF       : Build the Self Calibration UV Table and dirty image

\end{verbatim}
\subsection{APPLY}
\index{APPLY}
\begin{verbatim}
        [CALIBRATE\]APPLY [AMPLI|PHASE [gain]] [/FLAG]

      Apply  gain solution computed by MODEL and SOLVE (which are called im-
    plicitely by SELFCAL)  or obtained by READ CGAINS to the current UV  da-
    ta.  The optional arguments indicate whether this should be an AMPLITUDE
    or PHASE gain, and what gain value is used (in range 0 to 1).
      If no argument is given, the current SELF_MODE (see HELP  SELFCAL)  is
    used, and the gain is 1.0.


    The /FLAG option controls whether data without a valid gain solution are
    kept unchanged or flagged.

\end{verbatim}
\subsubsection{APPLY /FLAG}
\index{APPLY!/FLAG}
\begin{verbatim}

        [CALIBRATE\]APPLY [AMPLI|PHASE [gain]] /FLAG

      Apply gain solution (in AMPLITUDE or PHASE) and flag  data  without  a
    corresponding valid gain solution.


\end{verbatim}
\subsection{FLUX\_SCALE}
\index{FLUX\_SCALE}
\begin{verbatim}
        [CALIBRATE\]FLUX Find|Apply|List|Calibrate [VarName]

    A  set  of  commands to check flux calibration on a day to day basis.  A
    supervising procedure, check_flux.map, allows  self-calibration  of  the
    flux  by using as intermediate model a table generated by the Clean Com-
    ponent list.

    FLUX_SCALE FIND DateTolerance
    determines, by linear regression, the best scaling factor to match  date
    by date the UV data set with the MODEL data set. Dates are assumed iden-
    tical if separated by less than DateTolerance

    FLUX_SCALE APPLY VarName
    apply previously determined flux scale factors to the  MODEL  data  set,
    previously read by READ MODEL.  This is in general used only in an iter-
    ative search way, e.g. by procedure check_flux. The  resulting  UV  data
    set is loaded into the specified VarName SIC variable.

    FLUX_SCALE LIST
    print out dates, baselines and determined flux factors

    FLUX_SCALE CALIBRATE
    apply  previously determined flux scale factors to the UV data set. This
    may then be written using command WRITE UV .



\end{verbatim}
\subsection{MODEL}
\index{MODEL}
\begin{verbatim}
        [CALIBRATE\]MODEL [MaxIter] [/MINVAL Value [Unit]]

    Compute visibilities on the current UV sampling  using  a  source  model
    made of the MaxIter first Clean Components, or of all pixel values above
    the given Value if /MINFLUX is present.

\end{verbatim}
\subsubsection{MODEL /MINVAL}
\index{MODEL!/MINVAL}
\begin{verbatim}
        [CALIBRATE\]MODEL [MaxIter] /MINVAL Value [Unit]

    Construct the source model using all Clean Components until MaxIter (all
    if  MaxIter  is  0  or not specified). These components are stacked on a
    grid, and then all pixels above the given Value are taken as source mod-
    el to derive visibilities.

    Unit can be Jy, mJy, K or sigma. The default value is Jy.

\end{verbatim}
\subsection{SOLVE}
\index{SOLVE}
\begin{verbatim}
        [CALIBRATE\]SOLVE Time SNR [Reference]
          /MODE [Phase|Amplitude] [Antenna|Baseline] [Flag|Keep]

    Solve  the baseline or antenna based gains using the current UV data and
    current MODEL.

    Time is the integration time for the solution.  SNR is the minimum  Sig-
    nal to Noise Ratio required to find a solution.

\end{verbatim}
\subsubsection{SOLVE /MODE}
\index{SOLVE!/MODE}
\begin{verbatim}
        [CALIBRATE\]SOLVE Time SNR [Reference]
          /MODE [Phase|Amplitude] [Antenna|Baseline] [Flag|Keep]

    Dependin on the /MODE arguments, the gains can be antenna-based or base-
    line-based, and include Phase or Amplitude, and data  without  solutions
    either KEEPed or FLAGged,


\end{verbatim}
\subsection{UV\_SELF}
\index{UV\_SELF}
\begin{verbatim}
        [CALIBRATE\]UV_SELF [CenterX CenterY UNIT [Angle]]
      [/RANGE [Min Max Type]] [/RESTORE]

    Use  (and  if  specified  and/or needed create) the "Self Calibrated" UV
    dataset to make a dirty image, instead of using the current UV table.

    UV_SELF utilizes UV_MAP for imaging. See HELP UV_MAP for parameters.

\end{verbatim}
\subsubsection{UV\_SELF /RANGE}
\index{UV\_SELF!/RANGE}
\begin{verbatim}
        [CALIBRATE\]UV_SELF [CenterX CenterY UNIT [Angle]] /RANGE  [Min  Max
    Type]

    Create and image the "Self Calibrated" UV data.

    The "Self Calibrated" UV dataset is created from the current UV data set
    by extracting the range of channels specified by  the  /RANGE  arguments
    Min   Max  Type.  Type can be CHANNEL, VELOCITY or FREQUENCY.  If /RANGE
    has no argument, all channels are averaged together.

    It is then updated by command SOLVE at each self-calibration loop.   See
    SOLVE and CLEAN\SELFCAL for details.

\end{verbatim}
\subsubsection{UV\_SELF /RESTORE}
\index{UV\_SELF!/RESTORE}
\begin{verbatim}
        [CALIBRATE\]UV_SELF /RESTORE

    As UV_RESTORE but for the self-calibrated UV table.

    Restores  the Clean image from the Clean Component Table by removing the
    components from the Self-calibrated UV data and  imaging  the  residuals
    before adding them to the convolved Clean components.

    See UV_RESTORE for details.


\end{verbatim}
