\section{IMAGER versus CASA}
\label{app:imager-casa}

\subsection{Imaging Philosophy and Data Architecture}
CASA intends to solve the \textit{Measurement Equation}, whatever the
complexity of this process.  It is a all-in-one package for this purpose,
where calibration and imaging are deeply intermixed and use a
unified data format.
As a result, a CASA Measurement Set is a complex architecture
encompassing relations between many components stored as Tables
in a directory-like tree. It can handle calibrated data, calibration
tables, multisource data sets, raw data and final images
in the same architecture,
allowing to retain all information to process complex images, such
as multi-frequency synthesis of polarized emission observed in
a mosaic of fields.

On the contrary, \gildas{} is designed to break the process in 
totally independent steps. For IRAM data, calibration is done
in one program (CLIC for interferometry with \NOEMA{} or 
CLASS for single-dish with the 30-m),
and imaging in another (here \imager{}), with a clear intermediate step
to change from the calibration data format (using the CLASSIC container)
to UV Tables or CLASS Tables. In particular, \imager{} does not
handle polarization transparently at the current time: 
polarization states must be imaged independently.
The Gildas Data Format stores a limited
number of informations in a single binary data file with a compact binary
header, and is only suited for calibrated data.  UV Tables
are little more elaborate than simple images, but still limited
in complexity. 

\subsection{Frequency and Velocity scales}
In the \ALMA{} use of CASA, all observations are kept in the Observatory
frequency frame, and only converted to  a celestial reference frame
(such as the Local Standard of Rest, \texttt{LSRK}) at later stages, during
imaging if requested.

\gildas{} is more analysis oriented, and contains a dual interpretation 
of the frequency axis.  This axis can either be interpreted with
a Velocity scale, usually in the \texttt{LSRK} frame relative to a spectral
line rest frequency, or as a Rest Frequency, with the astronomical
source velocity specified. The choice of representation depends
on the astronomer's science objective:  astronomical object study
(in which case the Velocity representation is more appropriate)
or chemical composition study (in which case the Rest Frequency
representation is preferred).

\subsection{Transfering UV data from \casa{}}

The basic idea of data transfer at this stage is to transfer
a whole spectral window to \gildas{}, and use the simpler and faster
tools available in \imager{} for extracting channels, resampling,
subtracting continuum, etc... rather than doing that in \casa{}.

\subsection{In CASA}
The steps in CASA involve
\begin{enumerate}
\item Separating spectral windows and different sources into 
independent (temporary) MS
\item Getting rid of flagged data 
\item Setting the velocity reference frame and correcting
for Doppler motions
\item Exporting to UVFITS 
\item Removing the intermediate MS
\end{enumerate}
The intermediate MS is created using \textit{mstransform},  with 
\texttt{keepflags=False, regridms=True, outframe='LSRK'}. Source and 
spectral window identification is done using keywords \texttt{spw} and 
\texttt{field}, and an appropriate rest frequency is specified (in MHz) 
using \texttt{restfreq}. Finally, \textit{exportuvfits}  is used on the 
simple intermediate MS. \imager{} provides to \casa{} a tool named
\textit{casagildas()} that scans the content of the Measurement Set
(using \textit{listobs()} to automatically do this on all combinations
of spectral windows and sources.

\subsection{In \imager{} }

The script \texttt{@ fits\_to\_uvt} handles the conversion from
UVFITS to \uv{} table format.
The steps in GILDAS involve
\begin{enumerate}
\item Converting UVFITS to UVT
\item Collapsing the polarization information
\item Adjusting the weights to properly estimate the noise
\item Identify and flag bad data
\item Setting the frequency and velocity references
\end{enumerate}
It also handles some nasty details, like the source coordinates
being hidden in different places depending on the \casa{} version.

\begin{itemize}
\item Step 1 \\
\texttt{fits 'name'.uvfits to 'name'.uvt}\\
will create a GILDAS UV table from the UVFITS file. The signal
is assumed to be unpolarized at this stage. With the /STOKES
option, polarization information would be carried on properly at this stage.
\item Step 2 
This step is only needed with the /STOKES option. It could be
done manually by \\
\texttt{run uv\_splitpolar} \\
that collapses the polarization information. The desired 
polarization state should be set to \texttt{NONE} to optimize signal to noise
ratio for unpolarized sources, to \texttt{I} if polarization
is a concern.
\item Step 3
This step is only carried on if the /WEIGHT option is present. 
Depending on CASA version, the weights are not handled in the same way. 
In Casa 3.4, they are approximately correct. In Casa 4.1, they are off 
by a large factor (perhaps the number of channels...).  Task 
\texttt{uv\_noise} utilizes the many channels available (3840 per 
spectral baseband) to compute a statistic per visibility, and adjust 
the weight through a median scaling factor. Task \texttt{uv\_noise} 
will also flag data with ``unusual'' weights (deviating from the median 
by more than some factor (user specified, default 3)).\\ \item Step 4 
\\ Task \texttt{uv\_trim} will remove the flagged data from the UV 
table, saving space.
\item In practice, the optional Steps 2 to 4are handled by a single
task called \texttt{uv\_casa}, which saves 2 intermediate files
(and thus 2 read and 2 writes of large files).
\item Step 5\\
At this stage, one could start usual imaging using the
standard Mapping commands \texttt{READ UV 'name'; UV\_MAP}. 
Commands \texttt{SPECIFY FREQUENCY Value} and \texttt{SPECIFY 
VELOCITY Value} will set the desired correspondence  between
Velocity and Frequency axis.
\end{itemize}



\subsection{Transfering Image data}

In the Quality Assessment step 2 (QA2), the ALMA ARC staff usually provides one
or more deconvolved images as FITS files.  These FITS files can readily
be converted into \gildas{} images with the \sic{} command \com{FITS}: \\
  \texttt{FITS 'name'.fits TO 'name'.gdf}\\
They will however have a frequency
axis labelled in FREQUENCY, while \gildas{} usually works with this
axis labelled in VELOCITY. 
They can also be accessed as \sic{} Image variables using command
\comm{DEFINE}{IMAGE}

In \imager{}, direct display of these FITS files is normally
possible with commands \com{SHOW} and \com{VIEW}.
