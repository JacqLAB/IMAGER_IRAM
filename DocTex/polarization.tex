\section{Polarization Handling}
\label{sec:polar}

\imager{} recognizes and handles polarization at different levels.
Support for polarization is currently experimental and limited, 
but continuously improving.

\paragraph{Importing data}
When importing data, the \texttt{fits\_to\_uvt} script assumes by default
the data is unpolarized and produces the pseudo-polarisation
state "None" from the UVFITS file, by a properly weighted combination
of the two parallel hand states if more than one state is present.

Full polarization information can be preserved by adding the 
\texttt{/STOKES} option to the \texttt{@ fits\_to\_uvt} command. 

The \com{READ}\texttt{ UV} command will read data with any
polarization state(s).

\paragraph{Processing data}
On the contrary, practically all \imager{} commands cannot handle 
more than 1 polarization state. 

Most commands will flatly refuse to handle data with more than 
1 polarization state (e.g. \com{UV\_FILTER}, \com{UV\_RESAMPLE}, etc\ldots).

For debugging purpose, some commands like \com{UV\_PREVIEW}, \com{UV\_MAP}
or \com{UV\_STAT} will operate with more than 1 polarization state,
but will not produce meaningful results (only a subset of the data
may be treated).

So far, there are only two commands that fully support polarized
data: \com{UV\_TIME} and \com{STOKES}.
\begin{itemize}
\item \com{STOKES} is the primary command that allows to derive or extract 
a UV data with only one Stokes parameter from a UV data set with 
several polarization states. \imager{} can then process the 
individual Stokes parameters separately.
\item For convenience (because polarized data is obviously in general
bigger), the \com{UV\_TIME} command can be used for time averaging
prior to use of the \com{STOKES} command.
\end{itemize}

Some imaging strategies cannot be used on polarized data. Self-calibration
requires a totally different approach that is not available in \imager{}.
Also, the automatic definition of supports (\comm{MASK}{/THRESHOLD}) 
can only be done on Stokes I or parallel hand polarization states, not
on Stokes Q,U,V or cross-hand states. The same applies to spectral
line identifications.
 
\paragraph{Analysing data}
Sorry, so far \imager{} offers no integrated tools to derive e.g.
polarization fraction, or polarization vector directions. This may
come later !...
 
\subsection{The STOKES command}

The \comm{STOKES}{/FILE} command operates only on files. It allows
to extract a UV data with visibilities for one output (pseudo-)Stokes 
parameter from UV data with visibilities with 1,2 or 4 (pseudo-)Stokes
parameters. Besides the standard Stokes parameters I, Q, U, V, RR, 
LL, RL, LR (Left and Right circular), XX, YY, XY and YX (X and Y
linear) which are defined in the Sky frame,
command \com{STOKES} recognizes pseudo-Stokes values NONE, HH, VV,
HV and VH which are the linear polarization states in the frame
of the antennas (Horizontal and Vertical pure states).

Conversion from the H-V pseudo-Stokes polarization states to any standard
Stokes parameter is made by the \com{STOKES} command by applying 
the rotation due to the parallactic angle. For this, the UV data set
must contain the \texttt{PARA\_ANGLE} extra column. If it is not present,
it can be added to the data set by command \comm{UV\_ADD}{/FILE}.
That command can also insert the Doppler correction as an extra
column.

Ultimately, a similar DD, GG, DG and GD pseudo-Stokes polarization set
will be added for circular polarization (D is the first letter for
Droite in French, and G for Gauche). 
