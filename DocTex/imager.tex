\documentclass[11pt]{article}
%\usepackage[utf8]{inputenc}
\usepackage{natbib}
\usepackage{color}
\include{gildas-def} % GILDAS ( + at some point IMAGER) specific definitions
%
\pdfminorversion=7

% \newcommand{\lang}[1]{\texttt{\MakeUppercase{#1}$\backslash$}}

\newcommand{\lmv}{\texttt{lmv}}
\newcommand{\clean}{\texttt{CLEAN}}
\newcommand{\uv}{\textit{uv}}
\newcommand{\emm}[1]{\ensuremath{#1}}   % Ensures math mode.
\newcommand{\emr}[1]{\emm{\mathrm{#1}}} % Uses math roman fonts.
\newcommand{\nds}[1]{\emm{\displaystyle#1}} % Normal display size for array and tabular environment.
\newcommand{\paren}[1]  {\nds{\left(  #1 \right) }} % Parenthesis. 
\newcommand{\cbrace}[1] {\nds{\left\{ #1 \right\}}} % Curly Braces.
\newcommand{\abs}[1]{\emm{\left| #1 \right|}} % Absolute value.
\newcommand{\sicvar}[1] {\mbox {\tt #1}}
\newcommand{\lang}[1]{\mbox {\texttt{\MakeUppercase{#1}\textbackslash}}}
\newcommand{\imager} {\mbox{{{\sc imager}}}}

% Use COLOR only in PDF - does not translate (yet) correctly in HTML version
%begin{latexonly}
\renewcommand{\sic}    {\mbox{{\color{red} \sc sic}}}
\renewcommand{\greg}   {\mbox{{\color{red} \sc greg}}}
\renewcommand{\clic}   {\mbox{{\color{red} \sc clic}}}
\renewcommand{\mapping}{\mbox{{\color[rgb]{1,0,0}{\sc mapping}}}}
\renewcommand{\com}[1] {\mbox {{\color{magenta}\tt #1}}\index{#1}}
\renewcommand{\comm}[2]{\mbox {{\color{magenta}\tt #1 #2}}\index{#1!#2}}
%
\renewcommand{\imager} {\mbox{{\color[rgb]{1,0,0}{\sc imager}}}}
\renewcommand{\sicvar}[1] {\mbox {\color{blue}\tt #1}}
\renewcommand{\lang}[1]{\mbox {\color{magenta}\texttt{\MakeUppercase{#1}\textbackslash}}}
%end{latexonly}
%
%\newcommand{\PdBI}{\mbox{PdBI}}
\newcommand{\NOEMA}{\mbox{NOEMA}}
%\newcommand{\ALMA}{\mbox{ALMA}}
\newcommand{\IRAM}{\mbox{IRAM}}

\begin{htmlonly}
%\renewcommand{\sic}    {\mbox{{\sc sic}}}
%\renewcommand{\greg}   {\mbox{{\sc greg}}}
%\renewcommand{\clic}   {\mbox{{\sc clic}}}
%\renewcommand{\mapping}{\mbox{{{\sc mapping}}}}
%\renewcommand{\com}[1] {\mbox {\tt #1}\index{#1}}
%\renewcommand{\comm}[2]{\mbox {\tt #1 #2}\index{#1!#2}}
%\renewcommand{\imager} {\mbox{{{\sc imager}}}}
\renewcommand{\color}[1]{} % {\mbox {\tt #1 #2}\index{#1!#2}}
\end{htmlonly}

\makeindex{}

%%%%%%%%%%%%%%%%%%%%%%%%%%%%%%%%%%%%%%%%%%%%%%%%%%%%%%%%%%%%%%%%%%%%%%%%%%%

\begin{document}

% \frontmatter{} % For BOOKS only

\begin{center}
  \huge{
  \textbf{IMAGER} 
  } \\[2\bigskipamount]
  \Large{}
  A \gildas{} software\\[\bigskipamount]
  March 26th, 2019\\[\bigskipamount]
  Version 1.2\\[2\bigskipamount]
% I-IMAGER,  CLEAN\     Version 4.5-02 14-Mar-2019  S.Guilloteau
% I-IMAGER,  ADVANCED\  Version 1.6-05 28-Mar-2019  S.Guilloteau
% I-IMAGER,  CALIBRATE\ Version 1.2-01 20-Mar-2019  S.Guilloteau
  \large{}
  Questions? Comments? Bug reports? Mail to: \texttt{gildas@iram.fr}\\[2\bigskipamount]
  The \gildas{} team welcomes an acknowledgment in publications\\
  using \gildas{} software to reduce and/or analyze data.\\
  Please use the following reference in your publications:\\
  \texttt{http://www.iram.fr/IRAMFR/GILDAS}\\[2\bigskipamount]
\end{center}

\begin{description}
\item[Documentation] \mbox{}\\
  \credit{E.~Di~Folco$^{1}$}{S.~Guilloteau$^{1}$}{J.~Pety$^{2,3}$} 
\item[Software] \mbox{}\\
  \credit{S.~Guilloteau$^{3}$}{S.~Bardeau$^{2}$}{J.~Pety$^{2,3}$}
\end{description}
1. Laboratoire d'Astrophysique de Bordeaux
2. IRAM\\
3. Observatoire de Paris\\

\begin{rawhtml}
  Note: this is the on-line version of the IMAGER Manual.
  A <A HREF="../../pdf/imager.pdf"> PDF version</A>
  is also available.
\end{rawhtml}

\bigskip{} %

Related information is available in:
\begin{latexonly}
  \begin{itemize}
  \item{IRAM Plateau de Bure Interferometer: Introduction}
  \item{IRAM Plateau de Bure Interferometer: OBS Users Guide}
  \item{IRAM Plateau de Bure Interferometer: Atmospheric Calibration}
  \item{IRAM Plateau de Bure Interferometer: Calibration Cookbook}
  \item{CLIC: Continuum and Line Interferometric Calibration}
  % \item{MIS:  Millimeter Interferometry Simulation Tools}
  \item{GREG: Graphical Possibilities}
  \item{SIC:  Command Line Interpretor}
  \end{itemize}
\end{latexonly}
%
\begin{rawhtml}
  <UL>
  <LI> IRAM Plateau de Bure Interferometer: <A HREF="../noema-intro-html/noema-intro.html">
  Introduction</A>
  <LI> IRAM Plateau de Bure Interferometer: <A HREF="../obs-html/obs.html"> 
  OBS Users Guide</A>
  <LI> IRAM Plateau de Bure Interferometer: <A HREF="../pdbi-cookbook-html/pdbi-cookbook.html"> 
  Calibration CookBook</A>
  <LI> <A HREF="../clic-html/clic.html"> CLIC:</A>  
  Continuum and Line Interferometric Calibration 
  <LI> <A HREF="../mis-html/mis.html"> MIS:</A>
  Tools for Millimeter Interferometry Simulation
  <LI> <A HREF="../greg-html/greg.html"> GREG: Graphical Possibilities</A>
  <LI> <A HREF="../sic-html/sic.html">   SIC:  Command Line Interpretor</A>
  </UL>
\end{rawhtml}

\newpage
\tableofcontents{} %

% \mainmatter{} % for BOOKS only

\newpage
\section{Pre-amble}

%\textbf{This document is under construction; it is partly based on the older documentation of \mapping{}.}

\imager\ is an interferometric imaging package, tailored for usage
  simplicity and efficiency for multi-spectral data sets.

\textbf{\imager\  is *** NOT *** \mapping }

\imager{} was created because the initial infrastructure for imaging
in the \mapping{} program was not adapted to the implementation 
of imaging methods that became possible and/or were required for
\NOEMA{} (and useable for \ALMA{}), such as self-calibration,
wide band imaging, or routine processing of Mosaics and Short spacings.
 
The \imager{} design takes great care of efficiency, by using parallel 
programming and minimizing Input/Output on files. This lead to a 
concept with a single \com{READ} of data, a few (in general only 2) 
simple processing commands with built-in intelligent parameter guesses, 
a visual user control, and a single \com{WRITE} of the results once the 
user is satisfied of it.

We took the opportunity to revise and streamline the user interface,
providing access to all tools through simple commands. A special
effort was put on the visualization tools, which provide enhanced
speed and capabilities, yet preserving a very high level of compatibility
with those previously offered in \mapping{} .




\newpage
\section{Basic concepts}

\subsection{Objectives}

The main goals of \imager{} are
\begin{enumerate}
\item to offer a proper implementation of imaging in case of
wide relative bandwidth, where the natural angular resolution
changes with frequency.
\item to implement a simpler (and incidentally faster) scheme to process Mosaics,
including short spacings from single dish data
\item to minimize image sizes 
\item to minimize processing time by using parallel
programming as much as possible and reducing Input/Output to the strict minimum.
\item to simplify user interfaces, by providing sensible defaults.
\item to take advantage of improved capabilities of \NOEMA{} and \ALMA{}, by offering new
tools like self-calibration or wide bandwidth analysis.
\end{enumerate}

\imager{} was developed and optimized to handle large data files. Therefore, 
\imager{}\  works mostly on internal buffers and avoids as much as possible saving data to intermediate files. 
File saving is done ultimately once the data analysis process is complete, 
which offers an optimum use of the disk bandwidth.

\subsection{Overview of the data reduction and analysis}

Once the data has been acquired by an interferometer such as the 
NOrthern Extended Millimter Array  
(\NOEMA{}) or \ALMA{}, two different approaches may
be used for its reduction and analysis:
\begin{itemize}
\item The first possibility is to clearly separate 1) the calibration, 2)
  the imaging and deconvolution and 3) the analysis.
\item The second possibility is to merge in a single step calibration and
  imaging. This possibility is known as self-calibration.
\end{itemize}
While \casa{} uses the second paradigm, \gildas{} mainly implements 
the first approach, as the program and the data format used 
for each step is different. The calibration of \NOEMA{} data is
done inside \clic{} on the \NOEMA{} raw data format and the outcome is a
\uv{} table, which contains only calibrated visibilities of the
astronomical source.  The imaging and deconvolution is done inside
\imager{} on the calibrated \uv{} table and deliver mainly an \lmv{}
spectral cube (2 axes of coordinates and 1 axis of velocity/frequency).
%Finally, the \greg{} program implements several tools to visualize and
%analyze an \lmv{} spectral cube as those functionalities are not specific to
%an interferometer (\eg\ the user can use them as well on 30m spectral cube).
Finally, the \greg{} program implements several generic tools to visualize and 
analyze  \lmv{} spectral cubes, which are not specific to an interferometric use 
(\eg\ they can be used with the \IRAM{} 30\,m spectral cubes as well).
\imager{} includes \greg{} for the  visualization and analysis functionalities.

The choice of clearly separating calibration and imaging+deconvolution was
taken at start of the Plateau de Bure Interferometer (\PdBI{}), 
when the limiting number of antennas prevented 
the use of self-calibration. 
While many points of the calibration algorithms
inside \clic{} are specific to \NOEMA{} data (in particular its range of
Signal-to-Noise ratio), the algorithms of imaging+deconvolution can be used
in many different contexts and the visualization and analysis of spectra
cubes is mainly independent of the instrument that delivered the data. 
This last point implies that users can import data (mainly through FITS format)
in \imager{} for imaging and deconvolution, and in \greg{} for
visualization and analysis. But the reverse is also true. While calibration
of \NOEMA{} data should be done inside \clic{}, imaging+deconvolution and
visualization+analysis can be done in other softwares (\eg\ \miriad{},
\aips{}, \casa{} for the imaging and deconvolution
and \karma{} for the visualization and analysis).

With the improvement of \NOEMA{} (increase of the number of antennas and
better receiver sensitivities) and with the advent of a new generation of
interferometer (\ALMA{}), an additional step of self-calibration may
improve the consistency of the final results by imposing additional
consistent constraints on the calibration. This step is further presented in chapter 4.

\subsection{The structure of the \imager{} program}

\subsubsection{Structure and recommendations}

The \imager{} program supports
\begin{itemize}
\item The manipulation (\eg\ resampling), visualization and flagging of
  \uv{} tables;
\item The imaging of \uv{} tables in dirty maps and beams;
\item The deconvolution of dirty maps;
\item The inclusion of short-spacings; 
\item The visualization and analysis of spectra cubes;
\item The self-calibration.
%\item A simulator of ALMA continuum observations.
\end{itemize}

%While in \mapping these different steps can be performed through the use of 
%dedicated commands (e.g., the \clean language), tasks (activated through 
%the \com{RUN} command), or a collection of \sic procedures (the \com{GO} and \com{INPUT} 
%families, e.g., \comm{GO}{CLEAN}), 
%there are two main types of tools specifically developed in \imager:
It consists in a collection of commands, either dedicated to image and deconvolution
  (the \lang{clean} language) or implementing basic functionalities (the
  \lang{sic}, \lang{greg}, or \lang{calibrate}  families of languages).
%\item A collection of widgets (?) grouped in the \imager{} main menu to
%  interface most of the above possibilities.


%We recommend the following general strategy to select the ``right'' way of
%doing one operation
%\begin{itemize}
%\item If a command of the \clean{} language implements this operation,
%  always use it even though this possibility may also be implemented as a
%  procedure and/or a task (\eg\ \verb|CLEAN\UV_MAP|, \comm{GO}{UV\_MAP} and
%  \comm{RUN}{UV\_MAP}). This is because our minimum goal is the maintenance
%  of the \clean{} language, not of the tasks (as long as the tasks do not
%  share the same code as the \clean{} language).
%\item If a \com{GO} procedure or a widget and a task implement this
%  operation, always prefer the use of the procedure or the widget over the
%  use of the task as procedure or the widget were designed to ease the
%  interaction with the task, \ie\ they wrap the call of the task anyhow.
%\item If this operation is coded only as a task or a procedure, avoid
%  reinvent the wheel, just use it.
%\end{itemize}
%There may be exceptions to those recommendations: Always follow the
%detailed indications of the manual.

\subsection{Imaging/deconvolution: a brief sequence of commands}
For the user, \imager{} reduces the number of actions to the strict minimum. 
The imaging sequence is always the same: 
\begin{verbatim}
       1- Reading data
       READ UV MyData.uvt /RANGE Min Max Type
       ! here, optionally use UV_TIME, UV_COMPRESS, UV_BASELINE to average data
       ! or UV_FILTER, UV_CONT to filter lines or remove continuum
       2- Imaging
       UV_MAP         
       3- Deconvolving
       CLEAN           
       ! here, optionally use UV_RESTORE
       4- Looking at the result
       VIEW CLEAN      ! or SHOW CLEAN     
       5- Writing the result 
       WRITE * MyData  
\end{verbatim}



\begin{itemize}
\item \textbf{Step 1:} Reading  the  specified  internal  buffer (here 
UV) from the input file (.uvt), loading only the channels falling in 
the range defined by the variables Min and Max, of Type \com{CHANNEL}, 
\com{VELOCITY} or \com{FREQUENCY}. \imager{} recognizes whether the UV 
table is for a single field or a mosaic. The only difference between 
the single field and mosaic cases is that \imager{} yields a Sky 
brightness image for Mosaics, while the computed sky brightness of a 
single field is not automatically corrected for the primary beam 
attenuation. Imaging for multiple fields will be presented in chapter 
4. Single Dish data can also be loaded  in the following way : 
\comm{READ}{SINGLE} File.
\item \textbf{Step 2:} Computing a  dirty  map  and  beam from a UV data. 
\com{UV\_MAP} processes  single fields as well as Mosaics.
\item \textbf{Step 3:} Deconvolving the \sicvar{DIRTY} image map (a Single-field or Mosaic) 
using the dirty \sicvar{BEAM} with the current \sicvar{METHOD}. The default 
for the SIC variable \sicvar{METHOD} is \com{HOGBOM},  the other 
supported methods being \com{CLARK}, \com{MRC}, \com{MULTI} and 
\com{SDI}. See \comm{CLEAN}{?} for the other SIC variables 
controlling the deconvolution process. The outputs are the \sicvar{CLEAN} and 
\sicvar{RESIDUAL} images, and the Clean Component Table \sicvar{CCT}, all being stored 
in dedicated SIC variables.
\item \textbf{Step 4:} Plotting the result in the specified internal 
buffer (\sicvar{CLEAN}). Optionaly, the user can restrict the plot to a 
subset of channels through the optional arguments First and Last. 
\comm{SHOW}{CLEAN} can also be used instead, and produces a different 
type of plot.
\item \textbf{Step 5:}  Writing all modified image-like buffers (not 
the UV  tables) under the common file name ''\texttt{MyData}''. In the case of the 
present example, the following files are produced: \texttt{MyData.lmv, 
MyData.lmv-clean, MyData.cct, MyData.beam}, which correspond to the 
buffers: \sicvar{DIRTY}, \sicvar{CLEAN}, \sicvar{CCT}, and \sicvar{BEAM}, respectively. 
\com{WRITE}\texttt{UV MyData} would only write the internal buffer (\sicvar{UV}) in the 
file \texttt{MyData.uvt} (the default extension corresponds to the specified 
buffer).
\end{itemize}    

\subsubsection{Implementation issues}

The new implementation of the \com{UV\_MAP} command uses most of the 
older code, but re-arranged such that ensembles of contiguous channels 
(``chunks'') are treated at once and share the same synthesized beam. 
Deconvolution with \com{CLEAN} then proceeds by using the synthesized 
beam with the appropriate frequency for each channel. The user can 
control the ``chunk'' size, and hence the precision of the process 
given the desired field of view.

As a result of the new concept, beams (whether primary or synthesized) 
can be 4-D arrays, as they may depend on Frequency and Field.

% MOVE THIS TO DECONVOLUTION CHAPTER
%\begin{verbatim}
%IMAGER> clean ?
%CLEAN deconvolves the DIRTY image map using the dirty BEAM
% 
%* The outputs are the CLEAN and RESIDUAL images, and the 
%  Clean Component Table, CCT
% 
%  Cleaning method                   CLEAN_METHOD    [ hogbom ]     METHOD 
%  Loop gain                         CLEAN_GAIN      [ 0.2 ]        GAIN   
%  Fractional residual               CLEAN_FRES      [ 0 ]          FRES
%  Absolute residual (Jy)            CLEAN_ARES      [ 0 ]          ARES
%  Number of Clean components        CLEAN_NITER     [ 0 ]          NITER 
%  Minimum number of components      CLEAN_NKEEP     [ 70 ]   
%  Number of positive components     CLEAN_POSITIVE  [ 0 ]    
% 
%* Mosaic mode is OFF 
%     BLC [ 0 0 ]               TRC [ 0 0 ]
%     MAJOR [ 0 arc sec]        MINOR [ 0 arc sec]
%     ANGLE [ 0 deg E from N]   BEAM_PATCH [ 0 0 ]
%\end{verbatim}

\subsection{Usage of the HELP}

A simple call to \com{HELP} will display the various languages (e.g., 
\lang{SIC}, \lang{GREG}, \lang{CALIBRATE}, \lang{CLEAN}) accessible to 
the help documentation and list some commands with available 
documentation. Note that \com{CLEAN} is a command and \lang{CLEAN} a language 
(i.e., a family of commands). The language of a given command is 
recalled in the help of each command. 

Example: the command \com{APPLY} belongs to the language \lang{CALIBRATE}, 
it has one argument which can be \com{AMPLI} or \com{PHASE} exclusively, 
one optional argument \com{gain}, and one option \com{/FLAG}.
\begin{verbatim} 
IMAGER> help apply
        [CALIBRATE\]APPLY [AMPLI|PHASE [gain]] [/FLAG]
\end{verbatim} 

A brief description of the imager program can be obtained through:
\begin{verbatim} 
IMAGER> help imager
USER\IMAGER = "@ welcome.ima"
 
      IMAGER is a interferometric imaging package, tailored for usage
  simplicity and efficiency for multi-spectral data sets.
 
     The basic concept of IMAGER is the use of a simple
         READ data  - ACTION(s) - [SHOW or VIEW] - WRITE
  sequence of commands, minimizing the data I/O as much as possible.
  Automatic guesses of appropriate default values for the ACTIONs
  parameters is implemented whenever possible.
 
Additional Help Available:
 Actions      MAPPING      UV_Handling  MAP_Handling
\end{verbatim} 

To further explore the difference between \imager\ and \mapping :
\begin{verbatim} 
IMAGER> help imager mapping
USER\IMAGER = "@ welcome.ima"
IMAGER MAPPING
     Caution:  IMAGER is *** NOT *** MAPPING
 
     Although the underlying algorithms are the same as in the
  MAPPING program, the concepts are quite different.
 
    The user interface to MAPPING is file-oriented. The user interacts
  with MAPPING generally by a script, packaging a complex sequence of
  commands, tasks, reads and writes to intermediate files. Fine control
  of the script parameters are in general done through widgets.
 
    The basic concept of IMAGER is opposite: a single READ of data, a
  few simple processing commands with built-in intelligent parameter
  guesses, a visual user control, and a single WRITE of the results
  once the user is satisfied of it.
 \end{verbatim}
 
Finding documentation and help for the \imager\ commands can be done in three different ways:  
\begin{itemize}
\item a simple call to the \com{HELP} command provides a description of 
the command and its arguments and options
\item the command name followed by one or more questions marks will 
display some partial help on the command and its most useful 
parameters (''\com{Command ?}''), its second level parameters for advanced 
users (''\com{Command ??}''), all its parameters (''\com{Command ???}). 
\item \com{INPUT Command} provides a list of default values 
for the most commonly used parameters.
\end{itemize}

Documentation on subtopics of a given command (e.g., Variables, Arguments, or Results) 
can be obtained though: \com{HELP} command subtopic. (Warning: subtopic is case sensitive!). 
The list of available subtopics is found at the bottom of the documentation of each command: 
\begin{verbatim} 
IMAGER> help uv_map
[...] 
Additional Help Available:
 Mosaics      /FIELDS      /TRUNCATE    Variables    MAP_BEAM_STE MAP_CELL
 MAP_CENTER   MAP_CONVOLUT MAP_FIELD    MAP_POWER    MAP_PRECIS   MAP_ROBUST
 MAP_ROUNDING MAP_SHIFT    MAP_SIZE     MAP_TAPEREXP MAP_TRUNCATE MAP_UVTAPER
 MAP_UVCELL   MAP_VERSION  MAP_WEIGHT   MCOL         WCOL         Old_Names:
 convolution  map_angle    map_dec      map_ra       uv_taper     uv_cell
 taper_expo   weight_mode
\end{verbatim} 

Example: the following command will list the control variables of the 
\com{UV\_MAP} function and describe the associated parameter(s): 
\begin{verbatim} 
IMAGER> help uv_map Variables
UV_MAP Variables
                [CLEAN\]UV_MAP ?
        Will list all MAP_* variables controlling the UV_MAP parameters.
 
    The  list  of  control  variables  is (by alphabetic order, with the old
    names used by Mapping on the right)
    New names       [   unit]       -- Description --    % Old Name
    MAP_BEAM_STEP   [       ]  Number of channels per single dirty beam
    MAP_CELL        [ arcsec]  Image pixel size
    MAP_CENTER      [ string]  RA, Dec of map center, and Position Angle
    MAP_CONVOLUTION [       ]  Convolution function    % CONVOLUTION
    MAP_FIELD       [ arcsec]  Map field of view
    MAP_POWER       [       ]  Maximum exponent of 3 and 5 allowed in MAP_SIZE
    MAP_PRECIS      [       ]  Fraction of pixel tolerance on beam matching
    MAP_ROBUST      [       ]  Robustness factor        % UV_CELL[2]
    MAP_ROUNDING    [       ]  Precision of MAP_SIZE
    MAP_SIZE        [       ]  Number of pixels
    MAP_TAPEREXPO   [       ]  Taper exponent           % TAPER_EXPO
    MAP_TRUNCATE    [      %]  Mosaic truncation level
    MAP_UVCELL      [      m]  UV cell size             % UV_CELL[1]
    MAP_UVTAPER     [m,m,deg]  Gaussian taper           % UV_TAPER
    MAP_VERSION     [       ]  Code version (0 new, -1 old)
 
    NAME is no longer used, and WEIGHT_MODE is obsolete.
    MAP_RA          [  hours]  RA of map center
    MAP_DEC         [    deg]  Dec of map center
    MAP_ANGLE       [    deg]  Map position angle
    MAP_SHIFT       [Yes/No ]  Shift phase center
    are obsolescent, superseded by MAP_CENTER. 
    They are  provided  only  for compatibility with older scripts.
\end{verbatim} 

A more detailed description (type, size) of a given variable can be obtained 
through ''help command variable'', as in this example:
\begin{verbatim} 
IMAGER> help uv_map map_uvtaper
UV_MAP MAP_UVTAPER
 
      MAP_UVTAPER[3]  Real
 
    Parameters of the tapering function (Gaussian if MAP_TAPEREXPO = 2): ma-
    jor axis at 1/e level [m], minor axis at 1/e level [m], and position an-
    gle [deg].
\end{verbatim} 
MAP\_UVTAPER requires 3 values of type Real.
\vspace{0.5cm}

The default values of the useful parameters are checked through 
\comm{Command}{?}\footnote{Users familiar with \mapping{} can still use 
\com{INPUT Command} instead, although the output format may be slightly 
different.}
\begin{verbatim} 
IMAGER> uv_map ?
 
 UV_MAP makes a dirty image and a dirty beam from the UV data
 
* Variable MAP_CENTER controls shifting and rotation
* MAP_CELL[2], MAP_SIZE[2], MAP_FIELD[2] control the map sampling
* MAP_UVTAPER[3], MAP_UVCELL and MAP_ROBUST
     control the beam shape and weighting scheme
* MAP_BEAM_STEP and MAP_PRECIS control the dirty beam precision
 
  Map Size (pixels)                 MAP_SIZE        [ 0 0 ]
  Field of view (arcsec)            MAP_FIELD       [ 0 0 ]
  Pixel size (arcsec)               MAP_CELL        [ 0 0 ] 
  Map center                        MAP_CENTER      [  ]      
  Robust weighting parameter        MAP_ROBUST      [ 0 ]
  UV cell size (meter)              MAP_UVCELL      [ 7.5 ]   
  UV Taper (m,m,deg)                MAP_UVTAPER     [ 0 0 0 ] MAP_TAPEREXPO [ 2 ]
  Channels per single beam          MAP_BEAM_STEP   [ 0 ]
  Tolerange at map edge (pixels)    MAP_PRECIS      [ 0.1]
  Rounding method                   MAP_POWER       [ 2 ]     MAP_ROUNDING [ 0.05 ]
  Gridding Convolution method       MAP_CONVOLUTION [ 5 ]
\end{verbatim} 
Example: The default Gridding Convolution method is number 5 (Spheroidal).

\subsubsection{Notes for \mapping{} users}

The names of variables and most commands have been kept from 
\mapping{}, old names appear in the \com{HELP} whenever they have been 
replaced. In addition, for the sake of compatibility, \sic{} procedures 
can still be activated as in \mapping{} for the sake of compatibility, 
although most of the \com{GO} procedures should have been replaced by a 
dedicated \imager{} command or procedure. For instance, \comm{GO}{PLOT}  
(or its variants \comm{GO}{BIT}, \comm{GO}{NICE} and \comm{GO}{MAP}) and 
\comm{GO}{UVSHOW} offer 
similar features to the \com{SHOW} command, but take data from files or 
SIC image variables, depending on variables \sicvar{NAME} and \sicvar{TYPE}. 


%\subsection{Widget description}
%TO BE DONE
%\subsection{A bit of history}
%
%Implementation of imaging and deconvolution algorithms inside \gildas{}
%started in the early nighties. The first implementation was made as a
%collection of independent programs, called tasks in the \gildas{}
%environment, and activated through the \com{RUN} command in the \graphic{}
%program. The main advantage was the ease of programing, the main drawback
%was the lack of user-friendliness. To tackle this drawback, two different
%approaches were used.
%\begin{itemize}
%\item First, the calling of tasks were hidden through the call to \sic{}
%  procedures (the \com{GO} and \com{INPUT} families of scripts) whose
%  behaviors was modified by global variables. The activation of the tasks
%  and the development of the \sic{} procedures happened in a preexisting
%  program, called \graphic{} which also contained the tools to visualize
%  and analyze the spectra cubes.
%\item Second, a single, big program, called \imager{}, was developed with
%  flexibility in mind, \eg\ the possibility to interactively define
%  supports where to search for clean components. The support for
%  deconvolution of mosaics was built only in \imager{}. The same procedure
%  names were used in \imager{} and \graphic{} to obtain the same
%  look-and-feel.
%\end{itemize}
%Up to 2003, there thus were two \gildas{} program, \ie\ \imager{} and
%\graphic{}, which offered slightly different services with procedures
%sharing the same names. The status of the \graphic{} program was difficult
%to understand as it shared similarities with the \greg{} program (which
%defines all the drawing commands of \gildas{}) and with the \imager{}
%program. We thus decided during the 2003 change of \gildas{} architecture
%to transfer the visualization and analysis capacities of \graphic{} in
%\greg{} and the imaging and deconvolution capacities of \graphic{} in
%\imager{}.  The \graphic{} program was deprecated and we made the \greg{}
%and \imager{} program able to understand the \texttt{graphic} extension of
%procedure files for backward compatibility. The major drawback of this
%decision is the fact that we currently have in the same program (\ie\ 
%\imager{}) both tasks, procedures and commands to do similar but slightly
%different things.  Not much happened following this step due to manpower
%shortage in the \gildas{} team. Our goal in the coming years is to clean
%this situation by ensuring that both tasks and commands use the same
%FORTRAN code.  Our first main step is to update the documentation.

%\section{Cookbook for the impatient ones}
%
%This section presents typical deconvolution sessions on a real example
%available in the \gildas{} distribution.
%
%\subsection{Imaging and deconvolution pipeline}
%
%\begin{verbatim}
%       1 sic copy gag_demo:demo.uvt 1mm.uvt
%       2 let name 1mm
%       3 go uvcov 
%       4 go uvall
%       5 input uvall
%       6 let xtype weight
%       7 go uvall
%       8 go image
%       9 let type beam
%      10 go bit
%      11 let type lmv
%      12 go bit
%      13 let type lmv-res
%      14 go bit
%      15 let type lmv-clean
%      16 go bit
%      17 exit
%\end{verbatim}
%Comments:
%\begin{description}
%\item[Step 1] Copy a demonstration \uv{} table in the \gildas{}
%  distribution in the current directory.
%\item[Steps 2-7] Visualization of different aspects of \uv{} data directly
%  from the file. Step 3 shows the \uv{} coverage. Step 4 displays the
%  scatter plots of the amplitude vs spatial frequency of the \uv{}
%  visibilities (default of the \comm{GO}{UVALL} procedure). Step 5 displays
%  the value of the parameters which modify the behavior of the
%  \comm{GO}{UVALL} procedure. Steps 6 and 7 display the scatter plots of
%  the amplitude vs weight of the \uv{} visibilities.
%\item[Step 8] Imaging and deconvolution pipeline with visualization of the
%  deconvolved image using default parameters. The result is clearly not
%  optimal in this case (see next section).  All the results are directly
%  written as files on disk with the following file name conventions: dirty
%  beam: \texttt{1mm.beam}, dirty image: \texttt{1mm.lmv}, clean image:
%  \texttt{1mm.lmv-clean}, clean residuals: \texttt{1mm.lmv-res}.
%\item[Step 9-14] Successive visualization of the dirty beam (steps 9 and
%  10), dirty image (steps 11 and 12), clean residuals (steps 13 and 14) and
%  clean image (steps 15 and 16).
%\end{description}
%
%\subsection{Finely tuned imaging and deconvolution}
%
%\begin{verbatim}
%       1 lut rainbow3  
%       2 read uv 1mm
%       3 uv_show
%       4 uv_stat weight
%       5 input uv_map
%       6 let weight_mode UN
%       7 let uv_cell 7.5 1
%       8 uv_map
%       9 show beam
%      10 show dirty
%      11 input clean
%      12 let niter 1000
%      13 hogbom /flux 0 0.6
%      14 show residual
%      15 let niter 2000
%      16 hogbom /flux 0 0.6
%      17 show residual
%      18 let niter 4000
%      19 hogbom /flux 0 0.6
%      20 show residual
%      21 show clean
%      22 support
%      23 hogbom /flux 0 0.6
%      24 show residual
%      25 show clean
%      26 write * 1mm
%%      27 write dirty 1mm
%%      28 write clean 1mm
%%      29 write residual 1mm
%%      30 write cct 1mm
%      31 exit
%\end{verbatim}
%Comments:
%\begin{description}
%\item[Step 1] Select a color lookup table which nicely displays the
%  features of the studied source.
%\item[Step 2] Read \uv{} data from the \texttt{1mm.uvt} file to an internal
%  \imager{} buffer.
%\item[Step 3] Displays the scatter plot of the amplitude vs spatial
%  frequency of the \uv{} visibilities. This commands is similar to the
%  \comm{GO}{UVALL} command (see previous section), except that it works
%  only on the data previously loaded in the internal buffer.
%\item[Step 4] Predicts the synthesized beam, expected noise level, and
%  recommended pixel size for different values of the robust weighting
%  threshold. This helps the user to select the threshold used in the
%  imaging steps (2nd parameter of the \texttt{uv\_cell} variable).
%\item[Steps 5-10] Compute a tailored dirty beam and dirty image. Step 5
%  displays the \sic{} variables that customizes the behavior of the
%  \com{UV\_MAP} command. Step 6 and 7 selects robust weighting (instead of
%  the default natural weighting) and the associated threshold. Step 8
%  actually computes the results which are stored in internal buffers and
%  visualized in steps 9 and 10.
%\item[Steps 11-14] First deconvolution on internal buffers. Resulting clean
%  residuals, clean image and clean component tables are also stored in
%  internal buffers. Step 11 displays the \sic{} variables that customizes
%  the behavior of all the clean deconvolution algorithms. Steps 12 select
%  the stopping criterion by enabling a maximum of 1000 clean components.
%  Step 13 launches the deconvolution using the simplest \clean{} algorithm
%  with simultaneous plot of the cumulative flux as a function of the number
%  of found clean components. Step 14 displays the residuals, \ie\ remaining
%  undeconvolved signal.
%\item[Steps 15-24] Successive tries of the deconvolution to ensure deep
%  enough cleaning, just by changing the stopping criterion (here the total
%  number of clean components).
%\item[Step 25] Displays the resulting clean image.
%\item[Steps 26-30] Write the results (dirty beam, dirty image, clean image,
%  clean residuals and clean component table) on disk files for use in
%  future sessions.
%\end{description}
%
%\subsection{Noise estimation and plotting}
%
%\begin{verbatim}
%       1 let name 1mm
%       2 let type lmv-clean
%       3 go cct
%       4 go noise
%       5 let spacing '3*noise'
%       6 go bit
%       7 hardcopy 1mm-clean
%       8 exit
%\end{verbatim}
%Comments:
%\begin{description}
%\item[Steps 1-3] Visualize the cumulative flux as a function of the number
%  of found clean components to get an idea of the cleaning convergence.
%\item[Step 4] Computes an experimental noise value which takes into account
%  possible remaining side lobes after deconvolution.
%\item[Step 5] Sets the spacing between contour levels to 3 times the
%  experimental noise value (the \texttt{noise} \sic{} variable was defined
%  by the \comm{GO}{NOISE} procedure).
%\item[Step 6] Visualizes the clean image with the right values for the
%  contour levels.
%\item[Step 7] Makes a color Post-Script file, named \texttt{1mm-clean.eps},
%  of the plot.
%\end{description}



\newpage
\section{The input data: UV Tables}
\label{sec:uvtables}

The main goal of \imager{} is to convert interferometric measurements 
stored in \uv{} tables into images suitable for astrophysical interpretation. 
\uv{} tables contain a set of visibilities. \uv{} tables being the
starting point, \imager{} contains a number of commands to read then
and handle them.

\subsection{In a nutshell}

For \NOEMA{}, create with \clic{} a UV table from the current
index of cross-correlation observations and selected spectral window, 
and read it with \imager{}
\begin{verbatim}
$ clic
CLIC> (FILE ; FIND ; SET )   ! Build your index and spectral Window selection
CLIC> TABLE MyTable.uvt [NEW] [/MOSAIC]  !  Create a UV table
CLIC> EXIT
$ imager
IMAGER> READ MyTable.uvt [/Options ]
\end{verbatim}

For \ALMA{}, create with \casa{} a list of UVFITS files from a 
Measuremet Set, convert them with \imager{}
\begin{verbatim}
$ casa
CASA <2>: vis='MyMeasurementSet.ms'
CASA <3>: casagildas()
CASA <4>: execfile('casas2uvfits.py')
CASA <5>: exit()
$ imager
IMAGER> sic find *.uvfits
IMAGER> for string /in dir%file
IMAGER>   @ fits_to_uvt 'string'
IMAGER  next
\end{verbatim}
Conventions for file naming are described in Section \ref{uv:casa}.
Each \uv{} table can later be read separately.


\subsection{UV table description}

\subsubsection{UV table data format}

A \uv{} table is a specific 2-D Gildas table, with a few additional
informations in the header, and a special interpretation of the data
organisation.

In a standard \uv{} table, each \textit{line} describes a visibility. 
Here a \textit{line} designate either the first or second axis of the 
table, and a \textit{column} the other one. \uv{} tables may appear in 
both orders. The default one is \textit{line} on 1st axis (\texttt{.uvt} 
ordering, used by most application). The \texttt{.tuv} ordering 
obtained by a 21 transposition is used essentially for display, as in 
this case the \textit{column} as the same meaning as for the 
\com{COLUMN} of \greg\ .

The number of lines of a \uv{} table is thus the number of visibilities described
in the table. Each \textit{column} of the table stores a particular property of the
visibilities, namely:
\begin{description}\itemsep 0pt
\item[Column 1] U in meters;
\item[Column 2] V in meters;
\item[Column 3] W in meters or Scan number;
\item[Column 4] Observation date (integer \class{}/\clic{} Day
  Number\footnote{The \class{}/\clic{} is a "radio Julian date" (or "Jansky
    Julian date"), which starts as $-2^{15}$ on the date of the first radio
    observation by Karl Jansky. It is thus the Modified Julian date minus
    60549.  That choice was made to maximize the time interval over which
    radio astronomical data could be usefully stored in an
    \texttt{integer*2}, back when 2 bytes of header space per spectrum were
    a significant consideration.  This date has little meaning outside the
    rather sparse community of souls gathered around the \class{} and
    \clic{} programs, however...});
\item[Column 5] Time in seconds since 0:00 UT of above date;
\item[Column 6] Number of the first antenna used to measure the visibility;
\item[Column 7] Number of the second antenna used to measure the visibility;
\item[Column 8] Real part for the first frequency channel;
\item[Column 9] Imaginary part for the first frequency channel;
\item[Column 10] Weight for the first frequency channel;
\item[Columns 11-13] Same as column 8-10 but for the second frequency
  channel, or for the second Stokes parameter of this channel.
\item[...] etc\ldots for all channels
\item[Columns N-ntrail+1 ... N] Trailing columns after the channel visibilities.
\end{description}
If a \uv{} table describes \texttt{nvis} visibility spectra composed of
\texttt{nchan} frequency channels, each with \texttt{nstokes} Stokes parameters,
the size of the table will thus be:
\texttt{nvis} lines of \texttt{7+3*nchan*nstokes+ntrail} columns, where \texttt{ntrail}
is the number of trailing columns.

\subsubsection{\uv{} header}

A \uv{} table header contains all the informations of a GDF header but some
of these informations have a special meaning in this context. Command
\com{HEADER} is the standard way inside \gildas{} to display in a human
readable way the header of GDF file. For instance, the command \\
\texttt{IMAGER> header gag\_demo:demo-line.uvt}  \\
would display
  %I-GIO_RIH,  File is  [EEEI to IEEE] , Header Version 1 (32 bit)
\begin{verbatim}
 1  W-GDF,  UNKNOWN Velocity type defaulted to LSR
 2  File : /Users/guilloteau/gildas/gildas-exe-dev/demo/demo-line.uvt  REAL*4
 3  Size        Reference Pixel           Value                  Increment       
 4       103   16.0000000000000       220398.688000000     -0.183792725205421
 5      9146   0.00000000000000       0.00000000000000       1.00000000000000
 6  Blanking value and tolerance      1.23455997E+34   0.0000000
 7  Source name         GG_TAU
 8  Map unit            Jy
 9  Axis type           UV-DATA      RANDOM
10  Coordinate system   EQUATORIAL          Velocity    LSR
11  Right Ascension   04:32:30.34200        Declination       17:31:40.5230
12  Lii        0.000000000000000            Bii       0.000000000000000
13  Equinox            2000.0000
14  Projection type     AZIMUTHAL           Angle     0.000000000000000
15  Axis 0     A0     04:32:30.34200        Axis 0     D0     17:31:40.5230
16  Baselines               0.0       0.0
17  Axis 1 Line Name    13CO(21)            Rest Frequency   220398.6880000000
18  Resolution in Velocity   0.25000000     in Frequency        -0.18379273
19  Offset in Velocity        6.3000002     Doppler Velocity     -40.755900
20  Beam                   0.00                0.00                 0.00
21  NO Noise level
22  NO Proper motion
23  NO Telescope section
24  UV Data    Channels:     32, Stokes: 1 None        Visibilities:        9146
25  Column            1 (Size 1) contains U           
26  Column            2 (Size 1) contains V           
27  Column            4 (Size 1) contains DATE        
28  Column            5 (Size 1) contains TIME        
29  Column            6 (Size 1) contains IANT        
30  Column            7 (Size 1) contains JANT        
31  Column            3 (Size 1) contains SCAN        
\end{verbatim}
Comments:
\begin{description}\itemsep 0pt
\item[Line 1] Indicates the velocity frame. If not present in the table
(as here), it is assumed to be LSR.
\item[Line 2] Indicates the filename associated to the currently displayed
  header.
\item[Lines 3-5] Display the dimensions of the associated array. Here it is
  a rank 2 array of dimension 9146 lines times 9146 lines, \ie\ 9146 
  visibility spectra of 32 frequency channels.  Line
  4 describes the frequency axis of the visibility spectra stored in the
  \uv{} table. Be careful that this is a convention, \ie\ it must be
  decoded using the particular form of the table. In our case, each spectra
  has 32 frequency channels of width -183.8~kHz, the frequency of the
  reference pixel 16.0 corresponding to 220398.688~MHz. This last
  frequency is the frequency delivered by the correlator, \ie\ seen by the
  observatory. In particular, this is the frequency that must be used to
  compute the primary beam of the interferometer.
\item[Line 8] Indicates the unit of the real and imaginary parts of the
  visibilities, normally the Jansky (Jy).
\item[Line 9] Indicates that this is \uv{} table (\texttt{UV-DATA} and
  \texttt{RANDOM}).
\item[Lines 10-13] Describe the coordinate system.
\item[Lines 14-15] Describe the projection system. In the \uv{} table
  format, \texttt{A0} and \texttt{D0} indicate the phase center while
  \texttt{Right Ascension} and \texttt{Declination} indicate where the
  antenna pointed when acquiring the signal. These information are in
  general identical for single field imaging and different for mosaicing.
\item[Lines 16] Indicates the baseline range in meters (m). 
\item[Lines 17-19] Describe additional information about the frequency axis
  of the visibility spectra. In particular, the rest frequency (here
  220398.688~MHz, that of the $^{13}$CO J=2-1 line) corresponding to a 
  velocity of 6.3~km/s in the velocity frame indicated at line 1 (in general LSR).
\item[Line 20] Indicates the primary beam size of the interferometer in
  radian. This is an obsolescent way to pass the size of the interferometer
  antennas.
\item[Line 21] The noise section has no meaning for the UV table.
\item[Line 22] If present, proper motions are given in mas/yr. The epoch
is used as the time origin.
\item[Line 23] If the TELESCOPE section is present, this line would 
indicate telescope name, its geographic coordinates and the antenna 
diameter (in m). This is the new way to specify the primary beam.
\item[Line 24] UV data section: number of channels, number of Stokes
parameters and number of visibilities.
\item[Line 25 to end] Special columns description, including
the 7 first ones and the \texttt{ntrail} trailing ones.
\end{description}
In particular, \textbf{Mosaic} \uv{} tables contain two trailing 
columns named \texttt{L\_PHASE\_OFF, M\_PHASE\_OFF}
for the so-called ''Phase Offset Mosaics'', or 
\texttt{X\_POINT\_OFF, Y\_POINT\_OFF} for the ''Pointing Offset Mosaics'',
which contains the angular offsets of the field centers with respect
to the Phase reference. 

\subsection{NOEMA: UV Tables from CLIC}
\label{uv:clic}

For \NOEMA{}, creating UV tables suitable for \imager{} is straightforward
in \clic{}. From a list of selected cross-correlation scans, the command 
\texttt{TABLE} creates a  \uv{} table with the appropriate format
(or adds to an existing one). Mosaics are treated in much the same way:
it is sufficient to add the \texttt{/MOSAIC} option to it.

\subsection{\ALMA : UV Tables from \casa{}}
\label{uv:casa}

Importing \uv{} data from \casa{} to IMAGER is a little more complex, because
of the totally different design philosophy of the two packages.
\casa{} intends to solve the \textit{Measurement Equation}, whatever the
complexity of this process.  It is a all-in-one package for this purpose,
where calibration and imaging are deeply intermixed and use a
unified data format.
As a result, a \casa{} Measurement Set is a complex architecture
encompassing relations between many components stored as Tables
in a directory-like tree. It can handle calibrated data, calibration
tables, multisource data sets, raw data in the same architecture,
allowing to retain all information to process complex images, such
as multi-frequency synthesis of polarized emission observed in
a mosaic of fields.

On the contrary, \imager{} works on calibrated data only, and
with a single source (though possibly a mosaic) and single spectral setup at once.

The importation process goes through UVFITS data files, produced by
\texttt{exportuvfits} from \casa{}, and imported through the \com{FITS}
command of \sic{}.  However, the UVFITS format is an incomplete standard, and 
to recover properly all associated informations, these two steps have
been encapsulated in sophisticated scripts on each side.

\paragraph{On the \casa{} side}, the proper exportation is done by invoking the 
\texttt{casagildas()} tool. \index{casagildas} 
\footnote{This tool is made available to \casa{} by 
\imager{}  (\imager{} does not need to be active, but must have been 
executed once before)}. \texttt{casagildas()} scans the Measurement Set 
using the \texttt{listobs()} tool, then runs a \gildas{} task named 
\texttt{casa\_uvfits} which scans the output of 
\texttt{listobs()} to prepare a Python script named  
\texttt{casa2uvfits.py}.

It is then up to the user to execute this Python script (as gently 
reminded by the  \texttt{casagildas()} tool) that creates one UVFITS 
file per Source and Spectral window in the input Measurement Set. Thus, 
the sequence
\begin{verbatim}
  vis='MyMeasurementSet.ms'   # Setup the input Measurement Set filename
  casagildas()                # List its content and create the export script
  execfile('casa2uvfits.py')  # Execute the export script
\end{verbatim}
will create a set of
\begin{verbatim}
  Source-Frequency.uvfits
\end{verbatim}
files, where \texttt{Source} is the source name and \texttt{Frequency} is 
the central frequency of the spectral window in MHz, for all combinations
of sources and spectral windows.


\paragraph{On the \imager{} side}, the UVFITS files have first to be
converted to \uv{} tables. This can be done by a simple script
\begin{verbatim}
  sic find *.uvfits
  for string /in dir%file
    @ fits_to_uvt 'string'
  next
\end{verbatim}
Each \texttt{Source-Frequency.uvfits} file will be converted to a 
\texttt{Source-Frequency.uvt} \uv{} table (the original 
\texttt{.uvfits} file is kept too).

The \texttt{fits\_to\_uvt} \index{fits\_to\_uvt} script is based on the \com{FITS} command
of \sic{}, but augmented with a number of tests to recognize the
proper layout of the UVFITS file, as this layout depends on which
\casa{} version was used, and whether it is a single source or
multi source file.

The \texttt{fits\_to\_uvt} script has a number of options. Help can
be obtained by typing
\begin{verbatim}
  @ fits_to_uvt ?
\end{verbatim}


\paragraph{Current assumptions and limitations}
\gildas{} works in the \texttt{LSR} velocity frame and has limited 
polarization capabilities (as of Dec 2018, \imager{} can only process 
one polarization state at a time, see Section \ref{sec:polar} for 
details). Thus the \textit{casagildas() -}\texttt{fits\_to\_uvt} 
pipeline makes several assumptions:
\begin{itemize}
\item FDM (Frequency Division Mode) spectral windows are converted to 
LSR frame using the \textit{mstransform} tool
\item TDM (Time Division Mode) spectral windows, which have low 
spectral resolution (15 MHz), are assumed to be pure continuum
data, and remain in the default frame of the original measurement
set.
\item The conversion from UVFITS to \uv{} tables assumes that the data 
is unpolarized\footnote{unless the \texttt{/STOKES} option is present, 
see Section \ref{sec:polar}}, and merges the initial polarization 
states in an optimal way from the signal to noise point of view (i.e. 
using the respective weights of the two parallel hand states). 
\item \textit{casagildas()} takes the same input parameters
as the \textit{listobs()} facility. Source and/or spectral
window selection can thus be made by the user at this stage
\item \textit{casagildas()} takes the same input parameters
as the \textit{mstransform()} facility. Time integration
may be done at this stage, but this may limit the performance
of self-calibration at a later stage.
\end{itemize}

\subsection{Reading UV tables}

\uv{} tables are simply read into \imager{} by \comm{READ}{UV}.
The rest frequency to be used to compute the velocity scale
is normally found in the table, but can be overridden 
using the \comm{READ}{/FREQUENCY} option.
By default, the whole table is read, but a subset of the
channels can be read using the \comm{READ}{/RANGE} option (with
the velocity scale as above).

\subsection{UV Table handling}

Besides the \comm{READ}{UV} and \comm{WRITE}{UV} commands to read
or write \uv{} tables, \imager{} has a number of commands to manipulate
the current \uv{} table buffer. These commands have names starting
by \texttt{UV\_}. Most of them are in the \lang{CLEAN} language, some
in the \lang{ADVANCED} one. 

\imager{} works using UV buffers. Most commands only work on 
the current UV buffer, but some of them keep track of the previous
buffer to allow the user to revert the operation.

\begin{description}\itemsep 0pt

\item[Data inspection and editing]:
\begin{itemize}\itemsep 0pt
\item \comm{SHOW}{COVERAGE} display the \uv{} coverage
\item \comm{SHOW}{UV} display the \uv{} data
\item \com{UV\_FLAG} allow flagging visibilities
\item \com{UV\_PREVIEW} provides a quick view of the visibilities as a function
of frequencies, and attempts to automatically find the continuum level
and parts of the bandwidth with spectral line emissions.
\end{itemize}

\item[Data size reduction routines]:
\begin{itemize}\itemsep 0pt
\item \com{UV\_COMPRESS} is a simple spectral smoothing, providing only channel 
averaging by integer number of channels.
\item \com{UV\_RESAMPLE} provides a more flexible spectral smoothing and resampling
facility. 
\item \com{UV\_TIME}  can be used to time-average the UV data set, leading
to faster processing. However, using \com{UV\_TIME} too early may limit
your ability to perform accurate phase self-calibration.
\end{itemize}

\item[Continuum processing commands]:
\begin{itemize}\itemsep 0pt
\item \com{UV\_BASELINE} allows to remove the continuum baseline, by
0th or 1st order baseline fitting of each visibility. 
\item Conversely, \com{UV\_FILTER} will filter the spectral line
range to leave only the channels with continuum emission.
Both \com{UV\_BASELINE} and \com{UV\_FILTER} can use the results provided
by \com{UV\_PREVIEW} to specify where spectral lines may be found.
\item \com{UV\_CONTINUUM} converts a spectral line \uv{} table into a bandwidth 
synthesis continuum \uv{} table. \com{UV\_CONTINUUM} requires some knowledge of 
the field of view to evaluate how many channels should be averaged 
together. This is done using the same parameters (\sicvar{MAP\_FIELD}, 
or the product of \sicvar{MAP\_SIZE} by \sicvar{MAP\_CELL}) and subroutines as 
for commands \comm{UV\_STAT}{SETUP} and \com{UV\_MAP}.
\item \comm{UV\_MERGE}{/FILE} can merge several UV continuum 
tables with a specified spectral index to optimize the sensitivity.
\end{itemize}

\item[Image preparation]:
\begin{itemize}\itemsep 0pt
\item \com{UV\_CHECK} inspects the \uv{} data to figure out how many
different synthesized beams are needed.
\item \com{UV\_SHORT} adds the short (or zero) spacing information provided by
an additional single dish data, read by \comm{READ}{SINGLE}.
\item \com{UV\_STAT} evaluates the impact of robust weighting
and tapering on the synthesized beam. It provides recommendations for
the image and pixel sizes.
\item \com{UV\_TRUNCATE} restricts the \uv{} baseline length range.
\end{itemize}

\item[Miscellaneous]:
\begin{itemize}\itemsep 0pt
\item \com{UV\_DEPROJECT} de-projects the $(u,v)$ coordinates given
a specified phase center, orientation and inclination. This can be
useful for inclined, flattened, nearly axi-symmetric structures such as
proto-planetary disks or galaxies. 
\item \com{UV\_RADIAL} computes the azimutal average of the 
visibilities. It is useful for rotationally symmetric structures such 
as proto-planetary disks or circumstellar envelopes, for example.
\item \com{UV\_REWEIGHT} changes the visibility weights.
\item \com{UV\_SHIFT} changes the phase center
\item \comm{UV\_MERGE}{/FILE} can merge several UV tables, Line or Continuum.
It also allows stacking different spectral lines.
\end{itemize}

\end{description} 

The remaining \com{UV\_...} commands are related to imaging and 
deconvolution: \com{UV\_MAP} computes the dirty image, 
\com{UV\_RESTORE} computes the Clean image from a Clean component list 
by removal of the Clean components in the \uv{} plane, and imaging of 
the residuals. \com{UV\_RESIDUAL} just computes the residuals by 
subtraction of the Clean components. 

Finally, \com{UV\_SELF}, in the \lang{CALIBRATE} language, is a 
specific variant of \com{UV\_MAP} used to compute the intermediate 
images required for self-calibration. It is not intended for direct use 
by normal users.



\newpage
\section{Single-field imaging and deconvolution}
\label{sec:single}

\subsection{In a nutshell}

\begin{verbatim}
  1  read uv YourData
  2  uv_stat
  3  uv_map
  4  clean
  5  view clean
  6  write * YourResult
\end{verbatim}
\begin{enumerate}\itemsep 0pt
\item Read your \uv{} data
\item Have a look at its header, and get recommendations on
image characteristics.
\item Image it
\item Deconvolve
\item see the result
\item Save the result if OK.
\end{enumerate}
You are done. And this is often good. However, it may take a while, and
the angular resolution and/or the brightness sensitivity may not be optimal.
So, it may be worth for you to read the information below and adjust
the control variables of \com{UV\_MAP} 

\subsection{Measurement equation and other definitions}
\label{sub:single:principle}

The measurement equation of an instrument is the relationship between the
sky intensity and the measured quantities. The measurement equation for a
millimeter interferometer is to a good approximation (after calibration)
\begin{equation}
  V(u,v) = \mbox{FT}\cbrace{B_\emr{primary}.I_\emr{source}}(u,v)+N
\end{equation}
where $\mbox{FT}\cbrace{F}(u,v)$ is the bi-dimensional Fourier transform of
the function $F$ taken at the spatial frequency $(u,v)$, 
$I_{\emr{source}}$
the sky intensity distribution, $B_{\emr{primary}}$ the primary beam of the
interferometer (almost a Gaussian whose FWHM is the natural resolution of the
single-dish antenna composing the interferometer), $N$ some thermal noise
and $V(u,v)$ the calibrated visibility at the spatial frequency $(u,v)$.
This measurement equation implies different kinds of problems.
\begin{enumerate}
\item The presence of noise leads to sensitivity problems.
\item The presence of the Fourier transform implies that visibilities
  belongs to the Fourier space while most (radio)astronomers are used to
  interpret images. A step of \emph{imaging} is thus required to go
  from the \uv{} plane to the image plane.
\item The multiplication of the sky intensity by the primary beam implies a
  distortion of the information about the intensity distribution of the
  source.
\item Finally, the main problem implied by this measurement equation is
  certainly the irregular, limited sampling of the \uv{} plane because it
  implies that the information about the source intensity distribution is
  incomplete.
\end{enumerate}
Deconvolution techniques are needed to overcome the incomplete
sampling of the \uv{} plane. To show how this can be done, we need additional definitions
\begin{itemize}
\item Let us call $V = \mbox{FT}\cbrace{B_\emr{primary}.I_\emr{source}}$ the
  continuous visibility function.
\item The sampling function $S$ is defined as
  \begin{itemize}
  \item $S(u,v) = 1/\sigma^2$ at $(u,v)$ spatial frequencies where
    visibilities are measured by the interferometer. $\sigma{}$ is the rms
    noise predicted from the system temperature, antenna efficiency,
    integration time and bandwidth. The sampling function thus contains
    information on the relative weights of each visibility.
  \item $S(u,v) = 0$ elsewhere.
  \end{itemize}
\item We finally call $B_{\emr{dirty}} = \mbox{FT}^{-1} \cbrace{S}$ the dirty
  beam.
\end{itemize}
If we forget about the noise, we can thus rewrite the measurement equation
as
\begin{equation}
  I_\emr{dirty} = \mbox{FT}^{-1} \cbrace{S.V}.
\end{equation}
Using the property \#1 of the Fourier transform (see Appendix), we obtain
\begin{equation}
  I_\emr{dirty} = B_\emr{dirty} \ast \cbrace{B_\emr{primary}.I_\emr{source}},
\end{equation}
where $\ast$ is the convolution symbol. Thus, the incompleteness of the
\uv{} sampling translates into the image plane as a convolution by the dirty
beam, implying the need of deconvolution. From the last equation, it is
easy to show that the dirty beam is the point spread function of the
interferometer, \ie\ its response at a point source. Indeed, for a point
source at the phase center, $\cbrace{B_\emr{primary}.I_\emr{source}} =
I_{\emr{point}}$ at the phase center and 0 elsewhere and the convolution with
a point source is equal to the simple product: $I_{\emr{dirty}} =
B_{\emr{dirty}}.I_{\emr{point}} = B_{\emr{dirty}}$ for a point source of
intensity $I_{\emr{point}} = 1$~Jy.

We note that Fourier transform are in general done through Fast Fourier 
Transform, which implies first a stage of re-interpolation of the 
visibilities on a regular grid in the \uv{} plane, a process called 
\textit{gridding}. This gridding step introduces a convolution in the 
\uv{} space, and thus a multiplication by the Fourier transform of the 
gridding function in the image plane, which needs to be corrected
later by division by this Fourier transform. It can be shown that despite this 
step, the convolution property mentionned before still holds.
 
\subsection{Imaging}
\label{sub:single:imaging}

The process known as {\it imaging} consists in computing the dirty image
and the dirty beam from the measured visibilities and the sampling
function. 

\subsubsection{Image size and pixel size}


\paragraph{Link between image size and \uv{} cell size}

The gridding stage requires at least Nyquist sampling of the \uv{} 
plane to avoid the artifact known as aliasing. This sampling
depends on the source size. 

For the signal, the source size is limited by the primary beam, so that the 
Nyquist sampling in the \uv{} plane is obtained with a size of the grid 
cells equals to half the size of the antenna diameter. (this is the 
smallest spatial frequency that the interferometer can be sensitive to, 
\ie\ the natural resolution in the \uv{} plane). In the image plane, 
this implies to make an image at least twice as large as the primary 
beam size (see Fourier transform property \#2 in Appendix). 

Unfortunately, the spatial frequencies of the noise are not
bound:  the noise increases at the edges of the produced image 
because of the noise aliasing and gridding correction.

Unless you have good reasons (such as a strong confusion source close
to the primary beam), you should not choose too large an image size,
since that would slow down imaging and deconvolution.

\paragraph{Link between pixel size and largest \uv{} spatial frequency}

The largest sampled spatial frequency is directly linked to the synthesized
beam size (\ie\ the interferometer spatial resolution). The pixel size must
be at least 1/2 the synthesized beam size to ensure Nyquist sampling in the
image plane. However, Nyquist sampling is enough only when dealing with
linear processes while deconvolution techniques are non-linear.  It
is thus recommended to select a pixel size between 1/3 and 1/4 of the synthesized
beam to ease the deconvolution. Smaller pixel sizes would lead
to larger images, and unduly slow down the imaging and deconvolution process.

\subsubsection{Weighting and Tapering}

The use of the visibility weights $(1/\sigma^2)$ in the definition of the
sampling function is called natural weighting as it is natural to weight
each visibility by the inverse of noise variance. Natural weighting is also
the way to maximize the point source sensitivity in the final image.
However, the exact scaling of the sampling function is an additional degree
of freedom in the imaging process. In particular, the user may change this
scaling to give more or less weight to the long or short spatial frequencies.

We can thus introduce a weighting function $W(u,v)$ in the definitions of
$B_{\emr{dirty}}$ and $I_{\emr{dirty}}$
\begin{equation}
  B_\emr{dirty} = \mbox{FT}^{-1} \cbrace{W.S}
\end{equation}
and
\begin{equation}
  I_\emr{dirty} = \mbox{FT}^{-1} \cbrace{W.S.V}.
\end{equation}
There are two main categories of weighting functions
\begin{description}
\item[Robust weighting] In this case, $W$ is computed to enhance the
  contribution of the large spatial frequencies. This is done by first
  computing the natural weight in each cell of the \uv{} plane. Then $W$ is
  derived so that
  \begin{itemize}
  \item The product $W.S$ in a \uv{} cell is set to a constant if the
    natural weight is larger that a given threshold;
  \item $W = 1$ (\ie{} natural weighting) otherwise.
  \end{itemize}
  This decreases the weight of the well measured \uv{} cells (\ie{} very
  low noise cells) while it keeps natural weighting of the noisy cells.  It
  happens that the cells of the outer \uv{} plane (corresponding to the
  large interferometer configurations) are often noisier than the cells of
  the inner \uv{} plane (just because there are less cells in the inner
  \uv{} plane). Robust weighting thus increases the spatial resolution by
  emphasizing the large spatial frequencies at moderate cost in sensitivity 
  for point sources (but with a larger loss for extended sources, see below).
\item[Tapering] is the apodization of the \uv{} coverage by simple
  multiplication by a Gaussian
  \begin{equation}
    W = \exp\cbrace{-\frac{\paren{u^2+v^2}}{t^2}},
  \end{equation}
  where $t$ is the tapering distance. This multiplication in the \uv{}
  plane translates into a convolution by a Gaussian in the image plane,
  \ie\ a smoothing of the result. The only purpose of this is to increase
  the sensitivity to extended structure. Tapering should never be
  used \textbf{alone} as this somehow implies that you throw away large spatial
  frequencies measured by the interferometer. It is only a way to extract
  the most information from the given data set. If you need more sensitivity
  to extended structures, use compact configuration of the arrays 
  rather than extended configurations and tapering.
\end{description}
For more details on the whole imaging process the interested reader is
referred to \cite{guilloteau00}.

\subsubsection{Implementation (\comm{READ}{UV}, \com{UV\_MAP} and \com{UV\_STAT})}
\label{sub:single:implementation}

In \imager{}, gridding in the \uv{} plane and computation of the dirty
beam and image are coded in the \com{UV\_MAP} command. This command works
on an internal buffer containing the \uv{} table read from a file through
the \comm{READ}{UV} command. 

The \com{UV\_MAP} command is controlled by a set of SIC variables named 
with the prefix \texttt{MAP\_}. Suitable defaults are provided, so that 
only specific cases should require customization by the user. A 
description of the all variables can be obtained through \texttt{HELP UV\_MAP}.
\comm{UV\_MAP}{?} also gives their default and current values.

\textbf{Basic usage - image characterization:}\\
%\hspace{-1.0cm}
\begin{tabular}{lll}
 Name & Dim/Type & Description \\
 \sicvar{MAP\_CELL}   & [2] Real & Pixel size in arcsecond. Enter 0,0 to let the task find the best values. \\
 \sicvar{MAP\_FIELD}  &[2] Real & Image size in arcsecond.  \sicvar{MAP\_FIELD} has precedence over \\
      & & the number of pixels \sicvar{MAP\_SIZE} to define the  actual map size \\
      & & when both variables are non-zero. \\
 \sicvar{MAP\_SIZE}   & [2] Int  &   Image size in pixels \\
 \sicvar{MAP\_POWER}   & Int & Rounding scheme for default image size, to numbers like $2^{n}3^{p} 5^{q}$. \\
      & & $p$ and $q$ are less than or equal to \sicvar{MAP\_POWER}. \\
      & & Default value is 2, for smallest image size. For \sicvar{MAP\_POWER} = 0 \\
      & & \sicvar{MAP\_SIZE} is just a power of 2. \\
\sicvar{MAP\_ROUNDING} & Real & Maximum  error  between optimal size (\sicvar{MAP\_FIELD} / \sicvar{MAP\_CELL}) \\
      & & and rounded (as a power of $2^k 3^p 5^q$) \sicvar{MAP\_SIZE}. \\
\sicvar{MAP\_CENTER}  & Char.  &  A character string to specify the new Map center and the new map\\
      & & orientation, see the next subsection related to the definition of the \\
\phantom{MAP\_CONVOLUT}  & \phantom{Dim/Type} & projection center of the image. \\
\end{tabular}

\textbf{Weighting:}\\
\begin{tabular}{lll}
\sicvar{MAP\_ROBUST}  &  Real & Robust weighting factor, in range 0 - $+ \infty$. \\ 
	      &  & 0 means  Natural weighting (as $+ \infty$, actually).\\ 
	      &  & 0.5 or 1 is usually a good choice for Robust Weighting. \\
	      &  & Default is 0,i.e. natural weighting. (Old name \sicvar{UV\_CELL[1]}) \\
\sicvar{MAP\_UVTAPER} & [3] Real &  Array of 3 values controlling the UV taper: major/minor axis \\
                & & at 1/e level [m,m] (first two values), and position angle ([${\deg}$],  \\
                & & third value). By default (0,0,0). (Old name \sicvar{UV\_TAPER[3]}). \\
\sicvar{MAP\_UVCELL}  & [2] Real &  UV Cell size for Robust weighting. Default is 0, meaning that  \\
              & & the cell size is derived from the antenna diameter. \\
                & & (Old name \sicvar{UV\_CELL[2]}) \\
\sicvar{MAP\_TAPEREXPO} & Real &  the taper exponent. Default 2, indicating Gaussian function. \\
 \phantom{MAP\_CONVOLUT}          & \phantom{Dim/Type} & (Old name \sicvar{TAPER\_EXPO}). \\
\end{tabular}

\textbf{Advanced:}\\
\begin{tabular}{lll}
\sicvar{MAP\_BEAM\_STEP} & Int & Number of channels per common dirty beam, if $> 0$. \\
                & & If 0 (default value),  only one beam is produced in total. \\
                & & If $-1$, an automatic guess is performed from the map size and \\
                & & requested precision (\sicvar{MAP\_PRECIS}). \\
\sicvar{MAP\_PRECIS}  & Real & Position precision at the map edge, in fraction of pixel size, \\
                & & used (with the actual image size) to  derive  the  actual number of \\
                & & channels which can share the same beam.  Default value is 0.1. \\
\sicvar{MAP\_TRUNCATE} &  Real & For a Mosaic, truncate the primary beam to the specified level\\
\phantom{MAP\_CONVOLUT} & \phantom{Dim/Type} & (in fraction). Default value is 0.2.\\
%\hspace{0.2cm} \= \hspace{4.2cm} \= \kill
\end{tabular}

\textbf{Debug:}\\
\begin{tabular}{lll}
\sicvar{MAP\_CONVOLUTION}  & Int &  Gridding convolution mode in the $uv$ plane. \\
		& & (default 5 for speroidal functions) (old name \sicvar{CONVOLUTION}) \\
\sicvar{MAP\_VERSION} & Int &  Version of code to be used. This is a temporary variable to allow \\
                & &  comparison between the new and old codes without quitting \\
\phantom{MAP\_CONVOLUT}                & \phantom{Dim/Type} & \imager{}. \\
\end{tabular}


% The following variables are obsolescent or obsolete:
%\> MAP\_SHIFT    \>  Logical indicating whether the phase center must be shifted and/or \\
%                \> \> the image rotated (old name UV\_SHIFT). \\
% \> MAP\_ANGLE \>  Position angle of map axis when MAP\_SHIFT is set \\
% \> MAP\_RA      \>   Right ascension of new map center \\
% \> MAP\_DEC     \> Declination of new map center  \\
%\end{tabbing}

%In addition, the SIC variable WCOL indicates the weight channel, and the variable MCOL (a 2 dimension integer) specifies the channel range to be imaged. However, WCOL should in general be set to zero to allow the beam steps to be set. %(EDF: beam steps ??)

Parameters can be listed by the commands \texttt{''UV\_MAP ?''}  and \texttt{''CLEAN ?''}.
A thorough description of each parameter can be obtained by typing: \texttt{''help UV\_MAP MAP\_*''}
or \texttt{''help CLEAN CLEAN\_*''}.

\begin{verbatim}
IMAGER> UV_MAP ?
 UV_MAP makes a dirty image and a dirty beam from the UV data
 
* Variable MAP_CENTER controls shifting ang rotation
* MAP_CELL[2], MAP_SIZE[2], MAP_FIELD[2] control the map sampling
* MAP_UVTAPER[3], MAP_UVCELL and MAP_ROBUST
     control the beam shape and weighting scheme
* MAP_BEAM_STEP and MAP_PRECIS control the dirty beam precision
 
  Map Size (pixels)                 MAP_SIZE        [0 0]
  Field of view (arcsec)            MAP_FIELD       [0 0]
  Pixel size (arcsec)               MAP_CELL        [0 0] 
  Map center                        MAP_CENTER      [ ]      
  Robust weighting parameter        MAP_ROBUST      [0]
  UV cell size (meter)              MAP_UVCELL      [7.5]   
  UV Taper (m,m,deg)                MAP_UVTAPER     [0 0 0] MAP_TAPEREXPO [2] 
  Channels per single beam          MAP_BEAM_STEP   [0]
  Tolerance at map edge (pixels)    MAP_PRECIS      [0.1]
  Rounding method                   MAP_POWER       [2]     MAP_ROUNDING [0.05] 
  Gridding Convolution method       MAP_CONVOLUTION [5]
\end{verbatim}

%\begin{verbatim}
%IMAGER> help UV_MAP MAP_beam_step  
%UV_MAP MAP_BEAM_STEP
% 
%      MAP_BEAM_STEP   Integer
% 
%    Number of channels per synthesized beam plane.
% 
%    Default  is 0, meaning only 1 beam plane for all channels.  N (>0) indi-
%    cates N consecutive channels will share the same dirty beam.
% 
%    A value of -1 can be used to compute the number  of  channels  per  beam
%    plane  to ensure the angular scale does not deviate more than a fraction
%    of the map cell at the map edge. This fraction is controlled by variable
%    MAP_PRECIS (default 0.1)
%\end{verbatim}

\subsubsection{Defining the Projection Center of the image}
\label{sub:single:center}

The command \com{UV\_MAP} handles phase tracking center through its arguments, 
or through the string variable \sicvar{MAP\_CENTER}.

\begin{verbatim}
UV_MAP [CenterX CenterY UNIT  [Angle]]  [/FIELDS  FieldList]  [/TRUNCATE Percent]
\end{verbatim}
    
%The old command UV\_SHIFT was introduced to phase shift Mosaics to a
%common phase center in a Mosaic. It also works for single fields if needed,
%but this command is deprecated as the mechanism should be modified
%for large field mosaics.
%
%The old code can still be executed by first setting MAP\_VERSION = -1. 
%MAP\_VERSION set to 0 (the default) uses the new code, and MAP\_VERSION = 1 allows access to an intermediate version. For phase center shifting, the old code still requires the use of the
%variables MAP\_RA, MAP\_DEC, MAP\_ANGLE and UV\_SHIFT.
%(EDF: clarify whether MAP\_SHIFT variable can be used or not , and in which version of the code -- it exists in version 0, although declared as obsolescent, but MAP\_RA and MAP\_DEC are not described in the help)

\subsubsection{Typical imaging session}
\label{sub:single:example}

\begin{verbatim}
       1 read uv gag_demo:demo-single
       2 uv_map ?
       3 uv_stat weight
       4 let map_robust 0.5
       5 uv_map ?
       6 uv_map
       7 show beam
       8 show dirty
       9 let map_size 128
      10 uv_map
      11 show beam
      12 show dirty
      13 hardcopy demo-dirty /dev eps
      14 write dirty demo
      15 write beam demo
\end{verbatim}
Comments:
\begin{description}\itemsep 0pt
  \item[Step 1] Read the \texttt{demo.uvt} \uv{} table in an internal buffer.
  \item[Step 2] Check current state of the variables that control the 
  imaging.
  \item[Steps 3-5] Select the robust weighting threshold (step 4) from 
  the result of the \com{UV\_STAT} command (step 3) and recheck the 
  current state of the variables that control the imaging (step 5).
  \item[Steps 6-8] Image and plot the dirty beam and image.
  \item[Steps 9-12] Try a smaller size of the map as the default imaged 
  field-of-view looked too large from previous plots.
  \item[Steps 13] Make a Post-Script file from the dirty image.
  \item[Steps 14-15] Write dirty image and beam in \texttt{demo.lmv} 
  and \texttt{demo.beam} files for deconvolution in a future \imager{} 
  session. These steps are optional as you may directly proceed to the 
  deconvolution stage without writing the files.
\end{description}

\subsection{Deconvolution}
\label{sub:single:deconvolution}

Once the dirty beam and the dirty image have been calculated, we want to
derive an astronomically meaningful result, ideally the sky brightness.
However, it is extremely difficult to recover the intrinsic brightness
distribution with an interferometer. Mathematically, the
incomplete sampling of the \uv{} plane implies that there is an infinite
number of intensity distributions which are compatible with the constraints
given by the measured visibilities.  Fortunately, physics allow us to
select some solutions from the infinite number that mathematics authorize.
The goal of deconvolution is thus to find a sensible intensity distribution
compatible with the measured visibilities. To reach this goal,
deconvolution needs 1) some \emph{a priori}, physically valid, assumptions
about the source intensity distribution and 2) as much knowledge as
possible about the dirty beam and the noise properties (in radioastronomy,
both are well known). The best solution would obviously be to avoid
deconvolution, \ie\ to get a Gaussian dirty beam. For instance, the design
of the compact configuration of \ALMA{} has been thought with this goal in
mind. However, this goal is out of reach for today's millimeter
interferometers, even \ALMA{}.

The simplest \emph{a priori} knowledge that the user can feed to
deconvolution algorithm is a rough idea of the emitting region in the
source.  The user defines a support inside which the signal is to be found
while the outside is only made of sidelobes. The definition of a support
considerably helps the convergence of deconvolution algorithms because it
decreases the complexity of the problem (\ie\ the size of the space to be
searched for solutions). However, it can introduce important biases in the
final solution if the support excludes part of the sky region that is really
emitting. Support must be thus used with caution.

\subsection{The family of \clean{} algorithms (\com{HOGBOM}, \com{CLARK},
  \com{MX}, \com{SDI}, \com{MRC}, \com{MULTI})}
\label{sub:single:clean}

Radio astronomy interferometry made a significant step forward with the
introduction of a robust deconvolution algorithm, known as \clean{}, by
\cite{hogbom74}.

\subsubsection{\clean{} ideas}

The family of \clean{} algorithms is based on the \emph{a priori}
assumption that the sky intensity distribution is a collection of point
sources. The algorithms have three main steps
\begin{description}
\item[Initialization] \mbox{}
  \begin{itemize}
  \item of the residual map to the dirty map;
  \item and of the clean component list to a \texttt{NULL} (\ie\ zero)
    value.
  \end{itemize}
\item[Iterative search] for point sources on the residual map. As those
  point sources are found,
  \begin{itemize}
  \item they are subtracted from the residual map;
  \item and then they are logged in the clean component list.
  \end{itemize}
\item[Restoration] of the clean map 1) by convolution of the clean
  component list with the clean beam, \ie\ a Gaussian whose size matches
  the synthesized beam size and 2) by addition of the residual map.
\end{description}

\paragraph{Stopping criteria}

Several criteria may be used to stop the iterative search of this
``matching pursuit'':
\begin{enumerate}
\item When the maximum of the absolute value of the residual map is lower
  than a fraction of the noise. This stopping criterion is adapted to
  \emph{noise limited situations}, \ie\ when empirical measures of the
  noise in the cleaned image give a value similar to the noise value
  estimated from the system temperatures.
\item When the maximum of the absolute value of the residual map is lower
  than a fraction of the maximum intensity of the original dirty map.  This
  stopping criterion is adapted to \emph{dynamic range limited situations}, \ie\ 
  when some part of the source is so intense that the associated side lobes
  are larger than the thermal noise. In this case, any empirical measure of
  the noise in the cleaned image will give a value larger than the noise
  value estimated from the system temperatures.
\item The total number of clean components. This is a sanity criterium in
  case the other ones would be badly tuned.
\item When the total flux remains stable.
\end{enumerate}
Choosing the good stopping criterion is important because the 
deconvolution must go deep enough to recover weak extended flux  but 
\clean{} algorithms start to diverge when the noise is cleaned too 
deep. Criterium 4 is thus in general preferable, but may lead to 
insufficient cleaning when the dirty beam is poor (by lack of \uv{} 
coverage and/or because of phase noise). If (\# 4) fails, a good 
compromise is to clean down to or slightly below (typically 
$0.8\sigma$) the noise level.

\paragraph{Stability criterion}

Clean convergence is controlled by the usual \sicvar{ARES} (\#1, 
maximum Absolute RESidual value) , \sicvar{FRES} (\#2, maximum 
Fractional RESidual value) and \sicvar{NITER} (\#3, maximum Number of 
ITERations) criteria, plus \sicvar{CLEAN\_NCYCLE} for methods with 
major cycles. A fourth criterium  (\#4) is \textit{convergence}, which  
is controlled by \sicvar{CLEAN\_NKEEP}, a number of components. 
Deconvolution of a given channel stops if the cumulative flux at 
iteration number \textit{N} is smaller (resp. larger) than at iteration 
\textit{N-}\texttt{CLEAN\_NKEEP} for positive signals (resp. negative). In 
essence, \sicvar{CLEAN\_NKEEP} is the number of components when the 
signal is just above the noise. Experimentation with various types of 
images has shown that \sicvar{CLEAN\_KEEP}\textit{=70} is a good compromise. 

However, criteria \#1-3 can be set to 0, allowing \imager{} to 
automatically guess when to stop. In this case, \imager{} uses
an absolute residual threshold equals to the noise level (available
in \sicvar{dirty\%gil\%noise}), and estimates a (conservative) maximum
number of Clean components. 

\paragraph{Formation of the \clean{} map}

The clean component list may be searched on an arbitrarily fine spatial 
grid without too much physical sense as the interferometer has a finite 
spatial resolution.  The convolution by the clean beam thus 
reintroduces the finite resolution of the observation, an information 
which is missing from the list of clean components alone. This step is 
often called \emph{a posteriori} regularization.

The shape (principally its size) of the clean beam used in the 
restoration step plays an important role. The clean beam is usually a 
fit of the main lobe (\ie\ the inner part) of the dirty beam. This 
ensures that 1) the flux density estimation\footnote{Some odd dirty 
beams may lead to incorrect flux measurements. For example, if the 
dirty beam has a very narrow central peak superimposed on a rather 
broad plateau, the volume of the Gaussian fitted to the central peak 
does not match that of the dirty beam, and the flux scale will be 
incorrect. Data-reweighting is required to cure these peculiar 
situations.} will be correct and 2) the addition of the residual map to 
the convolved list of clean component makes sense (\ie\ the unit of the 
clean and residual maps approximately matches). 

The final addition of the residual map plays a double role. First, it 
is a first order correction to insufficient deconvolution. Second, it 
enables noise estimate on the cleaned image since the residual image 
should be essentially noise when the deconvolution has converged.

Super-resolution is the fact of restoring with a clean beam size 
smaller that the fit of the main lobe of the dirty beam. The underlying 
idea is to get a bit finer spatial resolution. However, it is a bad 
practice because it breaks the flux estimation and the usefulness of 
the addition of the residual maps. It is better to use robust weighting 
to emphasize the largest measured spatial frequencies.


\subsubsection{Basic \clean{} algorithms (\com{HOGBOM}, \com{CLARK} and \com{MX})}

The main difference between the different basic \clean{} algorithms is 
the strategy for searching the point sources.

\paragraph{\com{HOGBOM}}

The simplest strategy of the iterative search was introduced by
\cite{hogbom74}. It works as follows
\begin{enumerate}
\item Localization of the strongest intensity pixel in the current residual
  map: $\emr{max}(\abs{I_\emr{res}})$.
\item Add $\gamma.\emr{max}(\abs{I_\emr{res}})$ and its spatial position to
  the clean component list.
\item Convolution of $\gamma.\emr{max}(\abs{I_\emr{res}})$ by the dirty
  beam.
\item Subtract the resulting convolution from the residual map in order to
  clean out the side lobes associated to the localized clean component.
\end{enumerate}
$\gamma$ is the loop gain. It controls the convergence of the method. In
theory, $ 0 < \gamma < 2$. $ \gamma =1$ would in principle give
faster convergence, since the remaining flux at one position is $ \propto
(1-\gamma)^{n_{{\mathrm{comp}}}}$, where $n_{\mathrm{comp}}$ is the number of
clean components found at this position. But, in practice, one should use $
\gamma \simeq 0.1 - 0.2$, depending on sidelobe levels, source structure
and dynamic range.  Indeed, deviations (such as thermal noise, phase noise
or calibration errors) from an ideal convolution equation force to use low
gain values in order to avoid non linear amplifications of errors.

An important property of \com{HOGBOM} algorithm is that only the inner
quarter of the dirty image can be properly cleaned when dirty beam and
images are computed on the same spatial grid. Indeed, the subtraction of
the dirty sidelobes associated to any clean component is possible only in
the spatial extent of the dirty beam image. When the user defines a support
(\emph{a priori} knowledge), the cleaned region becomes even smaller than
the inner quarter of the dirty map.

\paragraph{\com{CLARK}}

The most popular variant to the \com{HOGBOM} algorithm is due to
\cite{clark80}. The iterative search for point sources involves minor and
major cycles.
\begin{description}
\item[In minor cycles,] an \com{HOGBOM} search is performed with two
  limitations: 1) Only the brightest pixels are considered in the above
  step 1, and 2) the convolution of the found point sources (step 3 above)
  is done with a spatially truncated dirty beam 
  \footnote{It is also theoretically possible to do so with an intensity truncated beam.}. 
  Both limitations fasten
  the search but may lead to difficult convergence in cases where the
  secondary side lobes are a large fraction (\eg\ 40\%) of the main side
  lobe.
\item[In major cycles,] the clean components found in the last minor cycle
  are removed in a single step from the residual map in the Fourier plane.
  The use of the Fourier transform enable to clean slightly more than the
  inner quarter of the map.
\end{description}
\com{CLARK} is faster than \com{HOGBOM}

\paragraph{\com{MX}}

The \com{MX} (or Cotton-Schwab, from the names of its authors) 
algorithm, due to \citet{schwab84}, is a variant of the
\com{CLARK} algorithm in which the clean components are removed from the
\uv{} table at each major cycle. This is the most precise way of removing
the found clean components because it avoids aliasing of the dirty
sidelobes. A direct consequence is that this method enables to clean the
largest region of the dirty map. However, this may be a relatively slow
algorithm because the imaging step must be redone at each major cycle,
although this speed issue could be compensated by the ability to
use smaller images.

\subsubsection{Advanced \clean{} algorithms to deal with extended
  structures (\com{SDI}, \com{MULTI} and \com{MRC})}

When the spatial dynamic of the imaged source is large (\ie\ when the ratio
of the largest source structure over the synthesized resolution is large),
the basic \clean{} algorithms may (rarely) turn smooth area of the source
into a serie of ridges and stripes. Indeed, when the dirty beam pattern is
subtracted from a smooth feature of the dirty map, the sidelobes patterns
appear in the residual map. The search for the next clean component will
then pick first the pixels in the sidelobes pattern amplifying this
pattern. Several variants of \clean{} have been devised to solve this
problem.

\paragraph{\com{SDI}}

In the \clean{} variant proposed by \cite{steer84}, extended features
(instead of point sources) around the current maximum of the residual map
are selected and removed in a single step. The simplest implementation
redefines the notion of minor and major cycles of the \com{CLARK}
algorithm. In the minor cycles, only the selection of the clean components
is done by including all the pixels in the residual map that rise above a
contour set at some fraction of the current peak level. In major cycles,
all those components are removed together in the Fourier plane. The
\com{SDI} algorithm may be instable if the fraction used for the selection
of the clean components is badly chosen.

\paragraph{\com{MRC}}

The Multi Resolution Clean \citep[\texttt{MRC},][]{wakker88} is the first
try to introduce the notion of cleaning at different scales. \com{MRC}
works on two intermediate maps (strictly speaking \com{MRC} is a
double--resolution \clean{} algorithm).  The first map is a smoothed
version of the dirty map and the second map, called difference map, is
obtain by subtraction of the smoothed map from the original dirty map.
Since the measurement equation is linear, both maps can be cleaned
independently (using a smoothed and a difference dirty beam, respectively).
The underlying idea is that extended sources in the dirty map will look
like more "point-like" with respect to the smoothed dirty beam in the
smoothed map. \com{MRC} is faster than the basic \clean{} algorithms
because fewer clean components are needed to reproduce an extended source
feature in the smoothed map than in the original map.

\paragraph{\com{MULTI}}

The Multi-scale \clean\ algorithm (\cite{cornwell0X}) has also been 
designed to improve the performance of \clean\ for extended sources.  
It is a straightforward extension of \clean\ that models the sky 
brightness by the summation of blobs of emission having different size 
scales.  It is equivalent to simultaneously deconvolve images obtained 
with different synthesized beams derived from the highest resolution 
one by convolution kernels. This algorithms works simultaneously in a 
range of specified scales. Multi-scale \clean\ can produce good images 
with a loop gain of 0.5 or even higher. 
 
 
The implementation of Multi-scale \clean\ in  \imager{} slightly 
differs from that of \casa\ . It is less optimized in terms of speed, 
but uses a better convergence scheme in which the scale chosen at each 
iteration is the one with best signal to noise ratio. Accordingly, it 
is more stable. Only 3 scales are used so far in \imager , with a size 
ratio controlled by \sicvar{CLEAN\_SMOOTH}.

\subsubsection{Implementation and typical use}

%\subsection{CLEAN} from SG doc
Deconvolution parameters are controlled by \sicvar{CLEAN\_*} variables. 
Progress has been made on automatic guess for Cleaning parameters. The 
table below presents the current naming scheme, with previous or 
equivalent names mentioned in parentheses, since these names were (or 
are still) used by several older packages such as \mapping{}, \aips{} 
or \casa{}. The equivalent ''old'' names (mentioned in Upper case 
below) will remain as aliases, while those mentioned in mixed case have 
disappeared as they were seldom used before. 
 
\begin{tabular}{lll}
  \sicvar{CLEAN\_ARES} &    Absolute residual  (ARES) \\
  \sicvar{CLEAN\_FRES} &    Fractional residual (FRES) \\
  \sicvar{CLEAN\_GAIN} &    Loop gain  (GAIN) \\
  \sicvar{CLEAN\_INFLATE} &  Inflation factor allowed to display MultiScale clean components \\
  \sicvar{CLEAN\_METHOD} &   Cleaning Method  (METHOD) \\
  \sicvar{CLEAN\_NCYCLE} &   Maximum number of Major cycles (Nmajor) \\
  \sicvar{CLEAN\_NITER} &   Maximum number of iterations (NITER) \\
  \sicvar{CLEAN\_NGOAL} &  A number of components for ALMA joint deconvolution only (Ngoal)\\
  \sicvar{CLEAN\_NKEEP} &   Number of iterations used to check convergence (see below)\\
  \sicvar{CLEAN\_POSITIVE} &  Minimum number of positive Clean components \\
  \sicvar{CLEAN\_RATIO} &  Ratio for  Dual Resolution clean (Ratio) \\
  \sicvar{CLEAN\_RESTORE} &   Minimum primary beam threshold for restoring  (Restore\_W) \\
  \sicvar{CLEAN\_SEARCH} &   Minimum primary beam threshold for searching (Search\_W) \\
  \sicvar{CLEAN\_SMOOTH} &  Smoothing factor for Multi Scale Clean (Smooth) \\
  \sicvar{CLEAN\_SPEEDY} &  Speeding factor for Clark (Spexp) \\
  \sicvar{CLEAN\_WORRY} &  "Worry" factor for Clark (Worry) \\
\end{tabular}


\paragraph{Implementation}

In \imager{}, the variants of the \clean{} algorithms discussed above are
coded as the following commands: \com{HOGBOM}, \com{CLARK}, \com{MX},
\com{SDI}, \com{MULTI} and \com{MRC}. All those commands work on two
internal buffers containing the dirty beam and dirty image. Both buffers
are created directly from \uv{} table through the \com{UV\_MAP} command, or
they can be loaded from files through the \comm{READ}{BEAM} and
\comm{READ}{DIRTY} commands. The behavior of those commands is controlled
through the following common \sic{} variables:
\begin{description}\itemsep 0pt
\item[Iterative search] \mbox{}
  \begin{description}\itemsep 0pt
  \item[\sicvar{CLEAN\_POSITIVE}] Number of positive clean components to be found
    before enabling the search for negative components. Default is 0.
  \item[\sicvar{CLEAN\_GAIN}] Loop gain. Default is 0.2, good compromise between
    stability and speed.
  \end{description}

\item[Stopping criteria] \mbox{}
  \begin{description}\itemsep 0pt
  \item[\sicvar{CLEAN\_NITER}] Maximum number of clean components. Default is 0.
  \item[\sicvar{CLEAN\_FRES}] Maximum amplitude of the absolute value of the
    residual image. This maximum is expressed as a fraction of the peak
    intensity of the dirty image. Default value is 0.
  \item[\sicvar{CLEAN\_ARES}] Maximum amplitude of the absolute value of the
    residual image. This maximum is expressed in the image units (Jy/Beam).
    Default value is 0.
    \item[\sicvar{CLEAN\_NKEEP}] Minimum number of Clean components be-
    fore testing if Cleaning has converged. Default value is 70. 
  \end{description}
  
\item[Support] \mbox{}
  \begin{description}\itemsep 0pt
  \item[\sicvar{BLC} and \sicvar{TRC}] Bottom Left Corner and Top Right
    Corner of a square support in pixel units. Default is 0, which means 
    using only the inner quarter if no other support is defined.
  \item[\com{SUPPORT}] A command that defines the support where
    to search for clean components. The support can be a Mask, or a
    Polygon. For a Polygon, the definition can be interactive,
    using the \greg{}  cursor. This definition can be stored in a file
    through the \comm{WRITE}{SUPPORT} command and read back in memory from
    the file with the \com{SUPPORT} command. The polygon support definition is
    stored in the \sicvar{SUPPORT\%} structure. Command \comm{SUPPORT}{/MASK}
    instructs \imager{} to use the Mask instead of the polygon for the
    Clean support.
  \item[\com{MASK}]  Command  \com{MASK} is used to define a Mask-like
    support. This can be interactive, or automatic using a thresholding
    technique in command \comm{MASK}{THRESHOLD}. The computed Mask 
    can be saved by command \comm{WRITE}{MASK}. The Mask can also be
    read by command \comm{READ}{MASK}. Command \comm{MASK}{USE} is 
    equivalent to command \comm{SUPPORT}{/MASK}, and instructs 
    \imager{} to use the Mask instead of the polygon for the Clean 
    support.
  \end{description}

\item[Clean beam parameters] \mbox{}
  \begin{description}\itemsep 0pt
  \item[\texttt{MAJOR, MINOR} and \texttt{ANGLE}] FWHM size of the major
    and minor axes (in arcsec) and position angle (in degree) of the
    Gaussian used to restore the clean image from the clean component list.
    Default is all parameters at 0, meaning use the fit of the main lobe of
    the dirty image. Changing the default value of those parameters is
    dangerous.
  \end{description}
\end{description}
Other variables control specific aspects of a subclass of the \clean{}
algorithm:
\begin{description}\itemsep 0pt
\item[\sicvar{CLEAN\_NCYCLE}] Maximum number of major cycles in all algorithms
  using this notion (\com{CLARK}, \com{MX}, \com{SDI}). Default is 50.
\item[\sicvar{BEAM\_PATCH}] Size (in pixel units) of the dirty beam used to
  deconvolve the residual image in minor cycles. It is used in \com{CLARK} and 
  \com{MRC} algorithms only. Default value is 0. This is for development only.
\item[\sicvar{CLEAN\_SMOOTH}] Smoothing factor between different scales in 
the MULTISCALE methods. Default value is sqrt(3).
\item[\sicvar{CLEAN\_RATIO}] Smoothing factor between different scales in 
the MRC method. Default value is 0, for which the code automatically
derives the best power of 2 adequate for the current problem.
\end{description}

\subsubsection{Typical deconvolution session}

\begin{verbatim}
       1 read beam demo
       2 read dirty demo
       3 clean ?
       4 hogbom /flux 0 1
       5 show residual
       6 show clean
       7 write clean demo
       8 let name demo
       9 show noise
      10 let ares 0.5*noise
      11 clean ?
      12 hogbom /flux 0 1
      13 let niter 2000
      14 clean ?
      15 hogbom /flux 0 1
      16 show residual
      17 show clean
      18 for iplane 1 to 10
      19    show clean iplane
      20    support
      21    hogbom iplane /flux 0 1
      22    write support "demo-"'iplane'
      23 next iplane
      24 show residual
      25 view cct
      26 view clean
      27 write residual demo
      28 write clean demo
      29 write cct demo
\end{verbatim}
Comments:
\begin{description}\itemsep 0pt
\item[Steps 1-2] Read dirty beam and dirty image from the
  \texttt{demo.beam} and \texttt{demo.lmv} files. Those steps are not
  needed if the dirty beam and image are already stored in the internal
  buffer, \ie\ if you have imaged the \uv{} table just before in the same
  \imager{} session.
\item[Steps 3-6] Print the current state of the control parameters,
  deconvolve the dirty image using the \com{HOGBOM} algorithm (step 3) and
  look at the results (residual and clean images).  The \texttt{/flux 0 1}
  option pop-up the visualization of the cumulative flux deconvolved as the
  clean components are found.
\item[Steps 8-12] Estimate the empirical noise through the \comm{SHOW}{NOISE}
  command after this first deconvolution and set the \texttt{ares} stopping
  criterion accordingly. Check that the new value of \texttt{ares} has been
  correctly set (step 11) and restart deconvolution.
\item[Steps 13-17] Increase the number of clean components as the previous
  deconvolution stopped before the residual image reached the
  \texttt{ares} value. Restart deconvolution and look at results.
\item[Steps 18-23] Attempt to improve deconvolution by definition of a
  support per plane and deconvolve this plane accordingly. The support is
  stored in a file for further re-use. The deconvolution results are then displayed.
\item[Steps 24-26] Display the residual images, visualize the cumulative 
  flux as a function of the clean component number and visualize the clean 
  spectra cube in an interactive way.
\item[Steps 27-29] Write residual image, clean image and clean component
  list in \texttt{demo.lmv-res}, \texttt{demo.lmv-clean} and
  \texttt{demo.cct} files for later use. 
\end{description}
Typical deconvolution session using other \clean{} algorithm would look
very similar. The main difference would be the possible tuning of other
control parameters. A deconvolution session using \com{MX} would start
differently as the imaging and deconvolution are done in the same step:
\begin{verbatim}
       1 read uv demo
       2 mx ?
       3 mx /flux 0 1
       4 show residual
       5 show clean
       6 write * demo
%       6 write beam demo
%       7 write dirty demo
%       8 write clean demo
%       9 write residual demo
%      10 write cct demo
\end{verbatim}
Comments:
\begin{description}\itemsep 0pt
\item[Step 1] Read the \texttt{demo.uvt} \uv{} table in an internal buffer.
\item[Step 2] Check current state of the variables that control the
  imaging and deconvolution.
\item[Steps 3-5] Deconvolve and look at the results.
\item[Steps 6-10] Write all the internal buffers on disk files. 
\end{description}
All the tuning of the typical imaging and deconvolution sessions could be
used in this \com{MX} session although they are not repeated here.

\subsection{Practical advices}
\label{sub:single:advice}

\subsubsection{Comparison of deconvolution algorithms}

\com{HOGBOM} is the basic \clean{} algorithm. It is robust but slow. 
\com{CLARK} introduces a faster strategy for the search and removal of 
clean component. However, it can be instable when dirty sidelobes are 
high, or the phase noise still significant. \com{MX} cleans the largest 
region of the dirty map because the source removal happens in the \uv{} 
plane. For the same map size, it is slower than \com{CLARK} because of 
the repeated imaging step, but smalle imager sizes can be used.
It shares some of the \com{CLARK}  instabilities because it uses 
the same search strategy, but the removal strategy counteracts this.

\com{HOGBOM}, \com{CLARK} and \com{MX} may introduce artifacts as parallel
stripes in the clean map when dealing with smooth, extended structures.
\com{SDI}, \com{MRC} and \com{MULTI} introduce (in principle) a better
handling of those extended sources. \com{SDI} is a rough attempt while
\com{MRC} and \com{MULTI} introduce the notion of cleaning at different
spatial scales.

\subsubsection{A few (obvious) practical recommendations}
\begin{description}\itemsep 0pt
\item[Map size] Make an image about twice the size of the primary beam (\eg\ 
  $2\times55''$ at 90~GHz and $2\times22''$ at 230~GHz for \NOEMA{} antenna)
  to ensure that all the area of the primary beam (inner quarter of the
  dirty map) will be cleaned whatever the deconvolution algorithm is used.
  However, avoid making a too large dirty image because the \clean{}
  algorithms will then try to deconvolve region outside the primary beam
  area where the noise dominates.
\item[Support] Start your first deconvolution \emph{without} any support to
  avoid biasing your clean image. If the source is spatially bound, you
  can define a support around the source and restart the deconvolution with
  this \emph{a priori} information. Be careful to check that there is no
  low signal-to-noise extended structure that could contain a large
  fraction of the source flux outside your support... Avoid defining a
  support too close to the natural edges of your source. Indeed,
  deconvolving noisy regions around your source is advisable because it
  ensures that you do not bias your deconvolution too much.
\item[Stopping criterion] Choose the right stopping criterion.\\
  Use the stability \sicvar{CLEAN\_NKEEP} parameter preferentially, i.e. 
  keep \sicvar{CLEAN\_ARES}, \sicvar{CLEAN\_FRES} and \sicvar{CLEAN\_NITER}
  to zero. If it does not work, then
  \begin{itemize}\itemsep 0pt
  \item Estimate an empirical noise on your first deconvolved cleaned image
    with \com{STATISTIC}, \com{CLEAN}, or \comm{SHOW}{NOISE}.
  \item If this empirical noise value is larger than the value computed from the
    visibility weights (this noise value is one of the outputs of the
    \com{UV\_MAP} command), your observation is probably dynamic range limited,
    \ie\ you have a bright source whose leftover dirty sidelobes are much larger
    that the thermal noise. In this case, set \texttt{ARES} to 0 and
    \texttt{FRES} to a fraction which depends on the sidelobe level of your
    dirty beam.
  \item Else you are in the noise limited case. Set \texttt{FRES} to 0 and
    \texttt{ARES} to a fraction of the empirical noise value (typically
    0.5).
  \end{itemize}
\item[Convergence checks] Ensure that your deconvolution converged by
  checking that:
  \begin{itemize}\itemsep 0pt
  \item The cumulative flux as a function of the number of clean component
    has reached a roughly constant level (use \texttt{/FLUX} option of the
    deconvolution commands to see this curves, or \comm{SHOW}{CCT}).
  \item The residuals are similar or smaller in the source region (where Clean
  components were found) compared to elsewhere.
  \end{itemize}
  If not, change the values of the stopping criterion, in particular the
  number of clean components (\sicvar{CLEAN\_NITER}).
\item[Deconvolution methods] If you want a robust result in all cases,
  start with \com{HOGBOM}. If you prefer obtaining a quick result, use
  \com{CLARK} but you then first need to check that the dirty sidelobes are
  not too large on the dirty beam. If you obtain stripes in your clean
  image:
  \begin{itemize}\itemsep 0pt
  \item First check that your deconvolution converged.
  \item Then check that there is no spurious visibilities that should 
   be flagged : use command \com{UV\_FLAG} as a last resort.
  \item If it is clear that you have an extended source structure, you
    should first ask yourself whether you are in the wide-field imaging
    case and act accordingly (see next chapter). Else you can try a
    \clean{} variant which better deals with cases that implies a large
    spatial dynamic. This is rare at \NOEMA{}, but may happen with \ALMA .
  \end{itemize}
\item[Outside help] Always consult an expert until you become one.
\end{description}



\newpage

\section{Wide-field imaging and deconvolution}

We are often asked why the wide-field imaging and deconvolution steps are
more difficult than their equivalent for single-field. The main answer is
that doing wide-field observations with an interferometer is kind of
paradoxical. Indeed, (sub)millimeter interferometers are before all tuned
to get the best possible spatial resolution. A natural consequence is the
lack of measurement of the low spatial frequencies which are extremely
important in wide-field observations. Hence the paradox.  

Progress in the design (ALMA was designed with wide-field imaging as a 
main goal) or in performances (NOEMA and the 30-m) has led to 
wide-field images being now customary.  The tools have become much 
simpler and user-friendly (see below !) but because of its paradoxical 
nature, wide-field imaging with an interferometer implies a 
knowledgeable use of those tools.

\subsection{In a nutshell}

\begin{verbatim}
  1  read uv gag_demo:demo-mosaic-v22.uvt
 (2)  read single gag_demo:demo-single-v22.tab
 (3)  uv_short
  4  uv_map
  5  clean
  6  write * MyDemo
\end{verbatim}
\begin{enumerate}\itemsep 0pt
\item Read your \uv{} data
\item Optionally, read your single-dish data
\item Optionally, merge it with with the \uv{} data set,
by using the single-dish data to provide short spacings
for the interferometer data.
\item Image as usual
\item deconvolve as usual
\item Save the result.
\end{enumerate}
You are done. If the \uv{} data set is a \textbf{Mosaic} data set,
it works as if it is a single-field. But, now, if the result is crazy, do not blame the
software. Rather read carefully the information below: \textbf{Mosaics} and
\com{UV\_SHORT} are simple to use, but can be tricky to use well !...

\subsection{General considerations about wide-field imaging}

The measurement equation for a millimeter interferometer is to a good
approximation (after calibration)
\begin{equation}
  V(u,v) = \mbox{FT}\cbrace{B_\emr{primary}.I_\emr{source}}(u,v)+N
\end{equation}
where $\mbox{FT}\cbrace{F}(f)$ is the bi-dimensional Fourier transform of
the function $F$ taken at the spatial frequency $f$, $I_{\emr{source}}$ the
sky intensity, $B_{\emr{primary}}$ the primary beam of the interferometer
(\ie\ a Gaussian of FWHM the natural resolution of the single-dish antenna
composing the interferometer), $N$ some thermal noise and $V(u,v)$ the
calibrated visibility at the spatial frequency $u,v$. The product of the
sky intensity by the primary beam, which ``quickly'' decreases to zero,
implies that an interferometer looking at a particular direction of the sky
will have its field-of-view limited by the size of the primary beam.
 
To image a field-of-view larger than the primary beam size, the antennae of
an interferometer will be successively pointed in different directions of
the sky typically separated by half the size of the primary beam. This
process is called mosaicing and the result requires specific imaging and
deconvolution steps. Another possibility is to acquire data as the
interferometer antenna continuously slew through a portion of the sky. This
second observing mode is called interferometric On-The-Fly (OTF). While
mosaicing is standard at NOEAM (see section~\ref{sec:mosaicing}), some
efforts are currently done to commission the OTF observing mode.

Mosaicing and OTF clearly belongs to wide-field imaging. However
considerations about wide-field imaging start as soon as the size of the
source is larger than about 1/3 to 1/2 of the interferometer primary beam.
Indeed, a multiplicative interferometer (\eg\ all interferometer in the
(sub)mm range) is a bandpass instrument, \ie\ it filters not only the large
spatial frequencies (this is the effect of the finite resolution of the
instrument) but also the small spatial frequencies (all the frequencies
smaller than typically the diameter of the interferometer antennas). An
important consequence is that a multiplicative interferometer do \emph{not}
measure the total flux of the observed source. This derives immediately
from the following property of the Fourier Transform: The Fourier transform
of a function evaluated at zero spacial frequency is equal to the integral
of your function. Adapting this to our notation, this gives
\begin{equation}
  V(u=0,v=0) \stackrel{\mbox{FT}}{\rightleftharpoons} \sum_{ij~\in~\emr{image}} \cbrace{B_\emr{primary}.I_\emr{source}}_{ij}.
\end{equation}
\ie\ the visibility at the center of the \uv{} plane is the total intensity
of the source. As a multiplicative interferometer filters out in particular
$V(u=0,v=0)$, the information about the total flux of the observed source
is lost. In summary, a multiplicative interferometer only gives information
about the way the flux of the source is distributed in the spatial
frequencies larger than the primary beam but no information about the total
flux.

Deconvolution algorithms use, in one way or another, the information of the
flux at the smallest \emph{measured} spatial frequencies to extrapolate the
total flux of the source. This works correctly when the size of the source
is small compared to the primary beam of the interferometer. The extreme
case is a point source at the phase center for which the amplitude of all
the visibilities is constant and equal to the total flux of the source:
Extrapolation is then exact. However, the larger the size of the source,
the worst the extrapolation, which then underestimates the total source
flux. This is the well-known problem of the missing flux that observers
sometimes note when comparing the flux of the source delivered by a mm
interferometer with the flux observed with a single-dish antenna. The
transition between right and wrong extrapolation is not well documented. It
depends on the repartition of the flux with spatial frequencies but also of
the signal-to-noise ratio of the measured spatial frequencies. It is often
agreed that the transition happens for sizes between 1/3 and 1/2 of the
interferometer primary beam. For larger source size, information from a
single-dish telescope is needed to fill in the missing information and to
thus obtain a correct result. This is the object of
section~\ref{sec:short-spacings}.

\subsection{Mosaicing}
\label{sec:mosaicing}

\subsubsection{Observations and processing}

In a single-field observation, an interferometer tracks a particular
direction of the sky, named the phase center. The portion of the sky which
can be image around this direction is directly linked to the size of the
primary beam. The easiest way to image field-of-view larger than the
primary beam size is to track one direction of the sky after another until
the desired field-of-view is filled with small images made around many
different tracking directions. This observing mode is called mosaicing and
the tracked observations which constitute the mosaic are called fields.

There are many constraints to optimize mosaicing.
\begin{description}
\item[Nyquist sampling of the mosaic field-of-view and mosaic pattern] The
  mosaic field-of-view must at least be Nyquist-sampled to obtain a
  reliable image. Each observed field can produce a reliable image of the
  same shape than the primary beam, \ie\ a circular Gaussian (This assumes
  that the short-spacing problem has been solved). Nyquist sampling thus
  implies that the mosaic fields follow an hexagonal compact pattern as
  this ensures a distance between all neighboring fields of half the
  primary beam size. When the total observing time is fixed, Nyquist
  sampling is the best compromise between sensitivity and total
  field-of-view. Indeed, the distance between neighboring fields could be
  less (in which case the mosaic would be oversampled) than half the
  primary beam size. In this case, the sensitivity on each pixel of the
  final image would increase with the share of the time spent to observe
  this direction.
\item[Uniform imaging properties and quick loop around the fields] Getting
  uniform imaging properties is a desirable feature in the final result.
  This implies that a \uv{} coverage and a noise level as uniform as
  possible among the different fields. Quickly looping around the different
  fields is the easiest way to reach this goal. However, dead time to
  travel from one field to another must almost be minimized. At NOEMA,
  the compromise is to pause at least 1 minute on each field and to try to
  loop over all the fields between two calibrations every 20 minutes.
  Hence, mosaic done in a single observing run is made of at most 20
  fields.  Larger mosaic must be observed by group of fields in different
  observing runs.
\end{description}

\subsubsection{Imaging}

When combining together (dirty or clean) images, it is important to correct
the primary beam attenuation to avoid modulation of the signal in the
combined image. If we forget for the moment the dirty beam convolution, the
images associated to each fields are noisy measurement of the same quantity
(the sky brightness distribution) weighted by the primary beam. The best
estimation of the measured quantity is thus given by the least mean square
formula
\begin{equation}
  \label{eq:mosaic:signal}
  \displaystyle %
    M(\alpha,\delta) = \frac{\displaystyle\sum\nolimits_i \frac{B_i(\alpha,\delta)}{\sigma_i^2}\,F_i(\alpha,\delta)}
      {\displaystyle\sum\nolimits_i \frac{B_i(\alpha,\delta)^2}{\sigma_i^2}},
\end{equation}
where $M(\alpha,\delta)$ is the brightness of the dirty/cleaned mosaic
image in the direction $(\alpha,\delta)$, $B_i$ are the response functions
of the primary antenna beams in the tracking direction of field $i$, $F_i$
are the brightness distributions of the individual dirty/cleaned maps, and
$\sigma_i$ are the corresponding noise values. As may be seen on this
equation, the intensity distribution of the mosaic is corrected for primary
beam attenuation. This implies that noise is inhomogeneous. Indeed, if
$N(\alpha,\delta)$ is the noise distribution and $\sigma(\alpha,\beta)$ is
its standard deviation in the direction $(\alpha,\beta)$, we have
\begin{equation}
  \label{eq:mosaic:noise}
    N(\alpha,\delta) = \frac{\sum\nolimits_i \frac{B_i(\alpha,\delta)}{\sigma_i^2}\,N_i(\alpha,\delta)}
      {\sum\nolimits_i \frac{B_i(\alpha,\delta)^2}{\sigma_i^2}},
\end{equation}
and
\begin{equation}
  \label{eq:mosaic:sigma}
    \sigma(\alpha,\delta) = %
    \frac{\sqrt{\sum\nolimits_i \frac{B_i(\alpha,\delta)}{\sigma_i^2}}}
    {\sum\nolimits_i \frac{B_i(\alpha,\delta)^2}{\sigma_i^2}} = %
    \frac{1}{\sqrt{\sum\nolimits_i \frac{B_i(\alpha,\delta)^2}{\sigma_i^2}}}
\end{equation}
Not only, the noise strongly increases near the edges of the mosaic
field-of-view. But also, the center of each field is contaminated by
increased noise level coming from the external regions of the neighboring
fields. Indeed, the noise corrected for the primary beam attenuation is
largely increasing where the primary beam is going to zero. To limit these
effects, both the primary beams used in the above formula and the resulting
mosaic are truncated.

\subsubsection{Deconvolution}

Standard \clean{} algorithms must be slightly modified to work on a dirty
mosaics. Indeed, the use of truncated primary beam in the above equations
is only a first order measure to avoid noise artifacts. However, the noise
level still increases at the edges of the mosaic, implying that at some
point the \clean{} algorithms will confuse noise peaks at the mosaic edges
with true signal. To avoid this, the iterative search is made on a
signal-to-noise image $M(\alpha,\beta)/\sigma(\alpha,\beta)$ instead of the
residual image. At the restoration step, the clean component list is used
to produce the residual map and clean map. Nothing particular is done with
the remaining signal-to-noise image.

\subsubsection{Typical use}

The processing of mosaics for NOEMA is essentially similar to that
of single fields. There are only two small changes
\begin{description}\itemsep 0pt
\item[Creation of \uv{} table]  A mosaic UV table should be created
using the /MOSAIC option of command TABLE in CLIC.
\item[Imaging] is done through \com{UV\_MAP} as for a single field.
However, the process is different. The command takes into account
the various fields, the primary beams, and select an optimum projection
center (phase center). The later may need to be specified by the user
using the \sicvar{MAP\_CENTER} string, or as argument to \com{UV\_MAP}
\item[Deconvolution] is also similar, but not all algorithms are
available. Only  HOGBOM and CLARK are possible so far.
The change of behavior of the \clean{} algorithms is visualized
through the change of prompt from \texttt{IMAGER>} to \texttt{MOSAIC}.
Two additional variables are used for mosaic deconvolution
\begin{description}\itemsep 0pt
\item[\sicvar{CLEAN\_SEARCH}] The minimum fraction of a primary beam
below which no Clean component is searched for.
\item[\sicvar{CLEAN\_RESTORE}] The minimum fraction of a primary beam
below which no image restoration is performed. Below this threshold,
the Clean image is blanked.
\end{description}
Finally, note that the mosaic deconvolution produces sky brightness
images while single-field deconvolution produces images attenuated by the
primary beam.
\end{description}
Although \imager{} makes no specific assumption about the \uv{} coverage
of individual fields, mosaicing deconvolution will work better
if all fields are equivalent in \uv{} coverage and noise level
\footnote{IS THAT TRUE ? : An additional subtlety of the current \imager{} implementation of
mosaicing is that \mapping{} assumes that the individual fields of the
mosaic have similar noise level.}

A mosaicing session would thus just be like a single-field
imaging:
\begin{verbatim}
       1 read uv gag_demo:demo-mosaic 
       2 uv_map
       3 hogbom /flux 0 10
       4 show residual
       5 show clean
       6 write * demo
\end{verbatim}
Comments:
\begin{description}\itemsep 0pt
\item[Step 1] Read the UV table
\item[Step 2] Image the mosaic
\item[Step 3] Deconvolve
\item[Steps 4-5] Look at the result
\item[Steps 6] Save the result
\end{description}

\section{Short and Zero spacings} 
\label{sec:short-spacings}

\subsection{In a nutshell}

\begin{verbatim}
  1  read uv gag_demo:demo-mosaic-v22.uvt
  2  read single gag_demo:demo-single-v22.tab
  3  uv_short
  4  uv_map; clean
  5  view clean
  6  write * MyDemo
\end{verbatim}
\begin{enumerate}\itemsep 0pt
\item Read your \uv{} data
\item read your single-dish data
\item Merge it with with the \uv{} data set,
\item Image and deconvolve as usual
\item check the result
\item Save it. 
\end{enumerate}
You are done. This works for mosaics as well as single fields.
But, again, if the result are crazy, do not blame the software. 
Rather read carefully the information below: 
\com{UV\_SHORT} is simple to use, but can be tricky to use well !...

\subsection{Principle}
Let's note \textit{D} the diameter of the single-dish antenna (\textit{D\,=\,30}m for the
IRAM-30m telescope) used to produce the short-spacing information and $d$
the diameter of the interferometer antennas (\textit{d\,=\,15}m for \NOEMA{}). We
already mentioned that a multiplicative interferometer filters out all the
spatial frequencies smaller than $\sim d$ meters. When this information is
needed to get reliable results, the source should also be observed with a
single-dish antenna to produce the missing information.  The single-dish
antenna furnishes information about all spatial frequencies up to $\sim D$
meters (but this information is weighted by the single-dish beam shape,
\ie\ high frequencies are measured with a worse signal-to-noise ratio than
low frequencies). To recover all the information at spatial frequencies
smaller than \textit{d} meters, the diameter of single-dish antenna must be
larger or equal to the diameter of the interferometer antennae: \textit{D}$\ge$\textit{d}.

\subsection{Algorithms to merge single-dish and interferometer information}

The measurement equations of a single-dish and an interferometer are quite
different from each other. Indeed, the measurement equation of a
single-dish antenna is
\begin{equation}
  I_\emr{meas}^\emr{sd} = B_\emr{sd} \star I_\emr{source} + N,
\end{equation}
\ie\ the measured intensity ($I_{\emr{meas}}^{\emr{sd}}$) is the convolution of
the source intensity distribution ($I_{\emr{source}}$) by the single-dish
beam ($B_{\emr{sd}}$) plus some thermal noise, while the measurement equation
of an interferometer can be rewritten as
\begin{equation}
  I_\emr{meas}^\emr{id} = B_\emr{dirty} \star \cbrace{B_\emr{primary}.I_\emr{source}} + N,
\end{equation}
\ie\ the measured intensity ($I_{\emr{meas}}^{\emr{id}}$) is the convolution
of the source intensity distribution times the primary beam
($B_{\emr{primary}}.I_{\emr{source}}$) by the dirty beam ($B_{\emr{dirty}}$) plus
some thermal noise. $B_{\emr{sd}}$ has very similar properties than
$B_{\emr{primary}}$ and very different properties than $B_{\emr{dirty}}$.  In
radioastronomy, $B_{\emr{sd}}$ and $B_{\emr{primary}}$ both have (approximately)
Gaussian shapes. Moreover, the fact that we will use the single-dish information to
produce the short-spacing information filtered out by the interferometer
implies that $B_{\emr{sd}}$ and $B_{\emr{primary}}$ have similar full width at
half maximum. Now, $B_{\emr{dirty}}$ is quite far from a Gaussian shape with
the current generation of interferometer (in particular, it has large
sidelobes) and the primary side lobe of $B_{\emr{dirty}}$ has a full width at
half maximum close to the interferometer resolution, \ie\ much smaller than
the FWHM of $B_{\emr{sd}}$.

Merging both kinds of information obtained from such different measurement
equations thus asks for a dedicated processing. There are mainly two
families of short-spacing processing: the \textbf{hybridization} and the
\textbf{pseudo-visibility} techniques.

\subsubsection{Hybridization technique}

In this family, most of the processing is done on the interferometric data
alone. Indeed, the interferometric data is deconvolved and corrected for
the primary beam contribution to obtain
\begin{equation}
  I_\emr{clean}^\emr{id} = B_\emr{clean} \star I_\emr{source} + N',
  \label{eq:clean}
\end{equation}
where $B_{\emr{clean}}$ is a Gaussian of FWHM equal to the interferometer
resolution and $N'$ is some thermal noise corrected for the primary beam
contribution.  Two main facts are hidden in this formulation: 1) the
field-of-view of the observation is obviously limited to the observed
portion of the sky and 2) more importantly, the lack of short-spacings has
not yet been overcome and a better formulation would be
\begin{equation}
  I_\emr{clean}^\emr{id} = \mbox{Highpass-filter}\cbrace{B_\emr{clean} \star I_\emr{source}} + N'.
\end{equation}
The simplicity of equation~\ref{eq:clean} is thus slightly misleading but
we will keep it for the sake of simplicity. The hybridization method
consists in combining two images ($I_{\emr{meas}}^{\emr{sd}}$ and
$I_{\emr{clean}}^{\emr{id}}$) in the \uv{} plane.

\begin{enumerate}
\item Both images are first spatially regridded on the same fine grid.
\item The FFT of those two images are computed, and linearly combined by
  selecting the low spatial frequencies from FFT($I_{\emr{meas}}^{\emr{sd}}$)
  and the high spatial frequencies from FFT($I_{\emr{clean}}^{\emr{id}}$).
  The transition between low and high spatial frequency
\item The result is FFTed back to the image plane to produce a final,unique
  image, which takes into account both single-dish and interferometric
  information.
\end{enumerate}
The method has the following free parameters: the transition radius and the
detailed shape of that transition.  To avoid discontinuity, the transition
shape is chosen to be reasonably smooth. The spatial frequency of
transition is generally chosen to the smallest spatial frequency reliably
measured by the interferometer (\eg\ about 20~m for NOEMA).

\subsubsection{Pseudo-visibility technique}

\paragraph{General description}

In this family, the single-dish information is heavily processed before
merging with the interferometric information. The basic idea is to produce
from the single-dish observations pseudo-visibilities similar to the ones
that would be produced by the interferometer if they were not filtered out.
\begin{enumerate}\itemsep 0pt
\item The Single-Dish measurements are re-gridded and then FFTed into the
  \uv{} plane.
\item The data are deconvolved of the single-dish beam ($B_{\emr{sd}}$)
  convolution by division by its Fourier Transform (truncated to the
  antenna diameter).
\item The data are FFTed back to the image plane and multiplied by the
  interferometer primary beam, $B_{\emr{primary}}$.
\item The result is FFTed again in the \uv{} plane where the visibilities
  are sampled on a regular grid.
\item In the case of a mosaic, the two last operations are performed for
  each pointing center.
\end{enumerate}
Using the properties of the Fourier transform, we can rewrite the
measurement equation of an interferometer as
\begin{equation}
  V(u,v) = \cbrace{\mbox{FT}(B_\emr{primary}) \star \mbox{FT}(I_\emr{source})}(u,v)+N.
\end{equation}
This equation means that the visibility measured by an interferometer at
the spatial frequency $(u,v)$ is the convolution of the Fourier transform
of the source intensity distribution by the Fourier transform of the
primary beam. Hence, to get pseudo-visibilities truly consistent with
interferometric visibilities, we must be able to reliably compute the
convolution by the Fourier transform of the primary beam. This implies that
we can compute pseudo-visibilities only for spatial frequencies lower than
\textit{D-d}. The use of the IRAM-30m to produce the short-spacing information of
the \NOEMA{} is thus ideal as it enables to recover pseudo-visibilities up
to 15~m (=30\,\mbox{m}-15\,\mbox{m}). Once the pseudo-visibilities have
been computed, they are merged with the interferometric visibilities and
standard imaging and deconvolution are then applied to the merged data set.


\paragraph{Single-dish vs interferometer weight}

In all cases involving short spacings, the relative weight of the single
dish data to interferometer data is critical. Within the restrictions
imposed by the noise level, this relative weight is a free parameter. It is
all the more important that the Fourier transform of the \uv{} plane
density of weights is the dirty beam, a key parameter of the deconvolution.
The general goal is to have a dirty beam as close as possible to a
Gaussian. As the Fourier transform of a Gaussian is a Gaussian, we search
for obtaining a \uv{} plane density of weights as close as possible to a
Gaussian. In general, the short spacing frequencies are small compared to
the largest spatial frequency measured by an interferometer. This implies
we can use the linear approximation of a Gaussian, \ie\ a constant, in the
range of frequencies used for the short spacing processing. We thus end up
with the need to match (as far as possible) the Single-Dish and
interferometric densities of weights in the \uv{} plane. In practice, we
compute the density of weights from the single-dish in a \uv{} circle of
radius 1.25\,\textit{d} and we match it to the averaged density of weights from the
interferometer in a \uv{} ring between 1.25 and 2.5\,\textit{d}. Experience shows
that this gives the right order of magnitude for the relative weight and
that a large range of relative weight around this value gives very similar
final results.

When processing IRAM-30m data to combine to NOEMA data, this criterion
implies a large down-weighting of the IRAM-30m data which may make think
that too much observing time was used at the IRAM-30m data. However, just
using the above criterion to determine the observing time needed at the
IRAM-30m would in general lead to very low signal-to-noise ratio of the
single-dish map. In such a case, it is very difficult to detect problems
which would translate in strong artifacts in the IRAM-30m + NOEMA
combination. We recommend to ask for enough IRAM-30m time to get a
\emph{median} signal-to-noise ratio of 5 on the single-dish map. This ratio
should be achieved for all velocity channels of interest (which may include
the line wings).

\subsubsection{Comparison}

The simplicity of the hybridization technique is its main advantage. It is
simple to understand and simple to implement. However, this method works
badly in practice because it is truly difficult to obtain a
reliable deconvolution of interferometric data alone when short-spacing
information is important. An interferometer is a spatial pass-band
filter, filtering in particular the zero spacing. This implies that the total flux in the
dirty image is zero (\ie\ as much negative as positive flux in the dirty
image) but that the dirty beam integral is also zero (\ie\ as much negative
as positive sidelobes). Adding the short-spacing information (and in
particular the zero spacing) through the pseudo-visibility method, we
enforces the positivity of the dirty image total flux and of the dirty beam
integral. It is well-known that trying to deconvolve a mosaic built only
with interferometric data is quite difficult. It almost always requires the
definition of support where the \clean{} algorithms can search for clean
components with the clear risk to bias the final result. In contrast,
adding the short-spacing information through pseudo-visibilities enables an
almost straightforward \clean{} deconvolution \emph{without} the need of
any support.

For the sake of illustration, let us assume an intensity distribution 
made of a large scale structure (\eg{} a smoothly varying intensity) 
superimposed with a small scale distribution both in emission and 
absorption. An interferometer will filter out the smooth 
distribution. If there is no additional zero spacing 
information, the smooth distribution is completely lost with 
the important consequence that the final deconvoled image will have 
positive (emission) and negative (absorption) structures. Trying to 
reproduce both negative and positive structures is one of the most 
difficult task for deconvolution algorithms. In addition, the 
presence of large negative structures create instabilities in the 
algorithms of the \clean{} family (because it is difficult to 
distinguish between negative absorption structures and negative 
sidelobes of emission structures). Only the definition of support 
around positive emission peaks may succeed to stabilize the \clean{} 
algorithms with the drawback of biasing the result.

Both kind of algorithms (hybridization and pseudo-visibility) are
implemented in \gildas{}. However, we strongly recommend to use the
pseudo-visibility algorithm. That is why only the pseudo-visibility method
is packaged in a user-friendly way, in the \com{UV\_SHORT} command.
\citet{pety01b,pety01a,pety01c} showed through
simulations that 1) the pseudo-visibility algorithm implemented in
\gildas{} enable extremely reliable results (fidelities of a few thousands)
on ideal observations and 2) the accuracy of the wide-field imaging is
limited by pointing errors, amplitude calibration errors and atmospheric
phase noise (and not by the used algorithms), even for ALMA.

\subsection{Hybridization technique and ALMA}
A special case where Hybridization can be extremely useful is that
of ALMA observations involving the main 12-m array, the ACA and single-dish
data with the 12-m antennas.  In some circumstances, an optimal
result is obtained by hybridizing Cleaned images produced
from the mosaics obtained by combining (using the pseudo-visibility method)
ACA and Single-dish data
as short spacings, with another mosaic obtained with the 12-m
data and the Single-dish data together.

The deconvolution of each mosaic is stabilized by the addition
of the 12-m single-dish zero or short spacings, and these good
mosaics are then merged in an optimal way by using the best
region of the \uv{} plane that they sample.

\subsection{The Zero spacing: an important subset}

An important subset of the pseudo-visibility method is the production of
only the zero spacing. Indeed, the zero spacing is just the total flux of
the observed field-of-view. Hence, if the observed field-of-view is small
enough to fit in the single-dish beam (this is in particular always the
case if $D = d$), a single spectrum observed with the single-dish telescope
in the direction of the interferometer phase center may be used as zero
spacing, only a scaling from Kelvin to Jansky is needed. This is the poor
man solution as only part of the short spacing information is recovered
by this technique.

\subsection{Short Spacings in practice: command \texttt{UV\_SHORT}}

Our algorithm to produce the short-spacing information is coded in the
\com{UV\_SHORT} command.  \com{UV\_SHORT} will \textbf{add} the 
short spacing information to the current \uv{} table (read by
command \comm{READ}{UV} and optionally transformed by further 
\texttt{\color{magenta} UV\_...} processing commands). 

\com{UV\_SHORT} has a substantial number (17)
of control variables, but with experience, they have been reduced
to 5 significant ones, among which only 3 really matter in most cases
but often can be used with their default values:
\begin{description}\itemsep 0pt
\item{\sicvar{SHORT\_SD\_FACTOR}} The single-dish brightness unit to 
flux conversion factor. If set to zero, \com{UV\_SHORT} will attempt
to derive it from the information available in the single-dish data   
\item{\sicvar{SHORT\_UV\_TRUNC}} The longest baseline retained in
the pseudo-visibilities. It defaults to the maximum theoretically possible,
the single-dish diameter minus the interferometer diameter. Smaller
values are allowed, and even recommended if the pointing quality
of the single-dish data is moderate.
\item{\sicvar{SHORT\_SD\_WEIGHT}} The relative weight scaling factor
between the pseudo-visibilities and the interferometer visibilities.
\end{description}
The relative weight of these visibilities is derived by \com{UV\_SHORT}
in order to optimize the shape of the overall synthesized beam. 
\sicvar{SHORT\_SD\_WEIGHT} is a scale factor to this optimum weight,
which may need to differ from 1 in case of poor \uv{} coverage in
the interferometer data or 
noisy single-dish data (it should be lower than 1 in this case).


\texttt{UV\_SHORT ?} will list these 3 major ones, and \texttt{UV\_SHORT ??} the
2 remaining main ones:
\begin{description}\itemsep 0pt
\item{\sicvar{SHORT\_TOLE}} The position tolerance in the single-dish map
\item{\sicvar{SHORT\_MIN\_WEIGHT}}  The minimum (relative) weight for a spectrum
in the single-dish map to be included.    
\end{description}
as well as four optional ones needed only if the original single-dish
and \uv{} data lacks the proper information (antenna diameter and beam sizes)

The \com{UV\_SHORT} command starts from data in a the format produced by the \class{} command
\texttt{TABLE}
command, and read in \imager{} through command \comm{READ}{SINGLE}. 
Basically, this is a GDF table containing one line per spectrum,
the columns representing the lambda offset, beta offset, weight, and the
spectrum intensities. \footnote{This format is subject to change: Please, refer to
the \com{TABLE} documentation for up-to-date information}. This data
must match spectrally the velocity sampling of the interferometric
data. This can be obtained using the \texttt{/RESAMPLING} option of
command \texttt{TABLE} in \class{}.

The \comm{READ}{SINGLE} and \com{UV\_SHORT} commands also support
a 3-D data cube (as produced by e.g. command \texttt{XY\_MAP} in \class{})
as input instead of a \class{} table. Again, the
velocity axis must match that of the interferometric data.

\emph{\color{red} temporary results as SIC variables and associated WRITE commands
to save them ?}

\com{UV\_SHORT} will automatically produce the Zero spacing from
the single-dish data when the data does not allow other short spacings
to be evaluated.

Finally, as \com{UV\_SHORT} \textbf{adds} the 
short spacing information, \comm{UV\_SHORT}{/REMOVE} allows to remove
it (there is no direct ``replace'' possibility).

\subsection{Practical considerations}

\subsubsection{When are short-spacing information needed?}

\begin{itemize}
\item If the source size is smaller than 1/3 the primary beam size,
  short-spacing information is superfluous.
\item If the source size is between 1/3 and 1/2 the primary beam size of
  \NOEMA{} antennas, a single spectra obtained at the IRAM-30m telescope in
  the direction of the source can be used to produce the zero spacing
  information with the \com{UV\_ZERO} task. Indeed, the IRAM-30m diameter
  being twice the diameter of the \NOEMA{} antenna, all the flux of the
  source will be measured by a single IRAM-30m spectrum only if the size of
  the source is smaller than 1/2 the primary beam size of \PdBI{} antennae.
\item If the source size is larger than 1/2 the primary beam size of
  \NOEMA{} antennas, short-spacing information under the form of an IRAM-30m
  map is almost always mandatory. The only exception could be wide-field
  imaging of a region made of unresolved or small (compared to the primary
  beam size) sources as it may happen when mapping close-by external
  galaxies for instance. However, adding short-spacing will anyway help the
  deconvolution. 
\item Short-spacing information is only useful if the brightness of the
  extended component is above the noise level. This requires a prior knowledge
  of the total flux in the imaged area to be determined. However, this
  information may be available from previous low-sensitivity single-dish
  observations. Checking this can avoid wasting a lot of telescope
  (and astronomer) time.
\end{itemize}
A generalization to \ALMA{} (12 m antenna) and ACA (7 m antennas)
is straightforward.


\subsubsection{How to optimize single-dish observations?}

One of the main difficulty of the short-spacing problematic is the need of
observations from a single-dish telescope at least as big as the
interferometer antennas\footnote{If there were no pointing errors,
a single-dish of the same size as the interferometer antennas
would be strictly sufficient.}. In this respect, the IRAM-30m and \NOEMA{} are very
complementary. Nevertheless, when observing with the single-dish telescope,
a few precautions are needed to avoid contaminating the interferometric
data with possible artifacts of single-dish data.
\begin{itemize}\itemsep 0pt
\item The field-of-view of the single-dish map must be twice the
  field-of-view covered by the mosaic. The only exception to this rule
  happens when the source intensity decreases to zero in a smaller
  field-of-view. Indeed, there is no point in observing an empty sky.
\item The observing strategy must enforce Nyquist sampling (or better) of
  the source at the resolution of the single-dish telescope.
\item A particular care should be taken of the pointing, tracking and
  amplitude calibration and baseline removal as those are critical issues
  in obtaining a high quality single-dish map to produce short-spacing
  information. For instance, data with too large tracking errors should be
  discarded.
\item Among ``baseline'' issues, the presence of continuum sources
  is to be treated with care. Continuum is difficult to measure with
  single-dish telescopes, and a (linear or polynomial) spectral baseline
  is often fitted to avoid atmospheric contamination. In such cases,
  the combination should be made with interferometer data where the 
  continuum has been removed, and added back later\ldots 
\item We advise to make many On-The-Fly coverages of the observed
  field-of-view to get homogeneous observing conditions. Scanning in
  perpendicular directions is needed to decrease stripping.  
\end{itemize}
Sometimes, single-dish telescope time is scarce and some of the above
criteria can not be fulfilled. In those cases, you can still try to use
your single-dish observations and our algorithm will try to make its best
to get a sensible result. However, any artifact in the combination may
directly come from wrong single-dish observations. In other words, do
\emph{not} blame the software unless you are sure of the quality of the
quality of your single-dish (and interferometric) observations...



\newpage
\section{Self calibration}

\subsection{In a nutshell}

\begin{verbatim}
  1  read uv YourData
  2  selfcal phase
  3  selfcal summary
  4  selfcal show
  5  selfcal apply
  6  uv_map
  7  clean
  6  write * YourSelf 
\end{verbatim}
\begin{enumerate}\itemsep 0pt
\item Read your \uv{} data
\item Self-calibrate the phase
\item Get a summary of the result
\item Show the phase correction between the last 2 iterations
\item Apply the self-calibration
\item Image as usual
\item Clean as usual
\item Save the result if it is worthwhile...
\end{enumerate}
You are done. But, now, if the result is crazy, do not blame the
software. Rather read carefully the information below: \com{SELFCAL}
is simple to use, but there are some pitfalls... 


\subsection{Principle}

The self-calibration idea is based on the fact that the dominant error terms
are antenna-based, while source information is baseline-based. 
With $N$ antennas, one gets at any time $N (N-1) /2$ visibility measurements, 
but $N$ amplitude gains, and only $N-1$ error terms for the morphology of the source (phase gains). 
The $N-1$ number is because only relative phases count. The absolute flux
scale is a separate problem, and therefore also $N-1$ relative amplitude gains count.

The measured visibilities 
on baselines from antenna $i$ to antenna $j$ at time $t$ are, 
from the simplified measurement equation:
\begin{equation}
    V_{\rm obs}(i,j,t) = G(i,t) G^{{\star}}(j,t)  V_{\rm true}(i,j) + Noise
\end{equation}    
where $G(i,t)$ is the complex (phase and amplitude) gain for the 
antenna $i$ at time $t$. The true visibility $V_{\rm true}(i,j)$ only 
depends on the baseline $(i,j)$, not on the time.
  
Given a source model $V_{mod}(i,j)$, one can derive the antenna gain
products at time $t$, based on the system:
 \begin{equation}
  \frac{V_{\rm obs}(i,j,t)}{V_{\rm mod}(i,j)} = G(i,t) G^{{\star}}(j,t) 
  \end{equation}
which is an over-constrained process, since there are $N(N-1)/2$ 
constraints for $N-1$ unknowns. Solving for this over-constrained 
problem is similar to deriving the amplitude and phase solution from a 
calibrator observation. In the calibrator case (i.e., an unresolved 
source like a distant bright quasar), $V_{\rm mod(i,j)} = (1.0,0.0)$ 
(constant amplitude, zero phase), so there is no risk of noise 
amplification in the process. 
%The antenna gain corrections are determined 
%by minimizing the distance between the observed data and the ''model'':
%\begin{equation}
%    S_{k} = \Sigma_{i,j} w_{i,j} \| V_{\rm obs}(i,j,tk) - G(i,tk) G^{{\star}}(j,tk)  V_{\rm true}(i,j)\|^{2}
%\end{equation}

For any (not a point-like) source, $V_{\rm mod}$ must be guessed. 
Self-calibration will use your source to improve the calibration of the 
antenna-based (complex) gains as a function of time. The practice is to 
proceed iteratively, based on a preliminary deconvolution solution. Let 
$V_{\rm obs}(k)$ be the ``observed'' visibilities at iteration $k$, with 
$V_{\rm obs(k=0)} = V_{\rm obs} $ the raw calibrated visibilities. Some 
of the Clean components derived from $V_{obs}(k)$ are used to define 
''model' visibilities $V_{\rm mod}(k)$. Then, solving for the antenna 
gains, one obtains: 

\begin{equation}
V_{\rm obs}(k+1) = \frac {V_{obs}(k)}{(G_i G^{{\star}}_j)}
\end{equation}

The model is thus progressively refined, and in the end, satisfies 
better the initial constraints on the source shape and on the antenna 
gains as a function of time provided by the measurements. Note that the 
absolute phase (and hence the position) can be lost in the 
self-calibration process and it should not be used for absolute 
astrometry.

There are two types of self-calibration: phase and amplitude self-calibration. 
The amplitude gain is a more complex problem than the phase gain.
Amplitude gains can (and often do) vary with time, but from the
measurement equation, a scale factor in the amplitude gain can be
exchanged by a scale factor on the source flux. It is thus customary
to re-normalize the gains so that the source flux is conserved
in the process. An alternate (perhaps not strictly equivalent) solution 
is to ensure that the time averaged product of the amplitude gains is 1.
The two approaches differ by the averaging process.

For any typical source, $V_{\rm mod}$ is non zero and of magnitude 
smaller than 1 (using the total flux as a scale factor) since the 
source is partially resolved. So in computing $V_{obs}/V_{\rm mod}$, 
there is noise amplification. It may even be the case that $V_{mod}$ is 
zero (case of an extended, over-resolved emission), and thus some 
(long) baselines will yield no direct constraint on the antenna gains 
$G(i) G^{{\star}}(j)$. But this should not matter too much for 
self-calibration, for two reasons. First, other (i.e., shorter) 
baselines may provide contraints on the gains. Second, if all 
$V_{\rm obs}$ for an antenna are
close to zero, it implies $V_{\rm mod}$ must be close to zero too, 
so an error on the phase of those visibilities (as well as on
its magnitude) is not so important. 

Self-calibration is related to the ``closure'' relations. For any triplet 
of antennas, the phase of the triple product $V_{ij} V_{jk} V_{ki}$ is 
independent of the antenna errors, and thus is (within the noise) a 
bias free constraint on the source. Similarly, for any quadruplet of 
antennas, the amplitude of the ratio $(V_{ij} V_{kl}) / (V_{ik} 
V_{jl})$  is independent of the antenna errors. But here, the noise 
amplification can be large because of the likelihood to have two small 
visibilities. For this reason, amplitude self-calibration requires in 
practice higher signal to noise ratios than phase calibration in the 
initial deconvolved data set used as a model. 
%EDF: Typical numbers here ?, 

Among the advantages of self-calibration, one may emphasize that 
antenna gains are derived at the correct time of the science object 
observation, while they must be interpolated in the classical 
calibration approach. Both atmospheric and electronic noises are 
supposed to vary with time, although with different timescales. Gains 
are also computed in the correct direction on the celestial sphere, 
while the calibrator-based approach introduces differences in the 
pointing direction with respect to the science object. The robustness 
of the approach increases with the number of baselines. 

In order to implement self-calibration, it is however necessary that 
the signal to noise ratio be large enough (the process will require a 
sufficient bright source). Self-calibration can especially bring 
significant improvements to the calibration solution in the case of 
higher than expected background noise, or in the presence of 
convolutional artifacts around objects, especially point sources.

\subsection{Implementation}

The typical procedure for self-calibration consists in an iterative 
process based on the following steps:
\begin{enumerate}\itemsep 0pt
\item \comm{UV\_MAP}{/SELF} $+$ \com{CLEAN} $+$ 
\comm{UV\_RESTORE}{/SELF}: from the classically calibrated (and 
preliminarily flagged) data, define an initial source model through a 
conservative (not too deep) first deconvolution : the model consists of 
a reasonably modest number of CLEAN components.
\item \com{SOLVE}: determine an estimate of the antenna gains (best fit 
to the observed visibilities)
\item \com{APPLY}: apply the derived gains to correct the observed data
\item \com{STATISTIC}: compare to the initial Image, and estimate the 
improvement through an adequate quality assessment (e.g., improvement 
of the dynamic range)
\item If necessary, re-build a new model from these corrected data, and 
iterate until the solution is satisfactory.
\end{enumerate}

Phase-only self-calibration is less stringent on the signal-to-noise 
ratio (SNR) threshold than amplitude self-calibration, and it should 
therefore be attempted first. For phase self-calibration to work, the 
SNR values in the initial data should be at least of $SNR > 3$ per 
antenna (in a solution interval shorter than the time for significant 
phase variations for all baselines to a single antenna). The SNR 
threshold in the initial image depends on the number of antennas and on 
the adopted time averaging. Depending on the complexity of the source 
(and the contribution from extended emission), all available baselines 
may not be considered in the process, and specific preliminary flagging 
could be necessary. Amplitude errors tend to be negligible for dynamic 
ranges below about 500. Amplitude self-calibration will thus 
be eventually attempted in a subsequent step.

Self Calibration is available in \imager{} through the command 
\com{SELFCAL}, which uses an iterative scheme driven by a script 
(gag\_pro:p\_selfcal.ima).  From the above principle, the script 
controls
  a) the number of iterations
  b) the number of selected clean components at each iteration
  c) the time scale of the solution, i.e. the integration time
  over which the gains are assumed to be constant

The parameters of the command \com{SELFCAL} are available as 
\sicvar{SELF\_Names} \sic{} variables.  The script uses the commands 
\com{UV\_MAP}, \com{CLEAN}, and  \com{SOLVE} and \com{APPLY} from the 
\lang{CALIBRATE} language. By default, a solution is searched for the 
phase calibration only (\sicvar{SELF\_MODE} $=$ \texttt{PHASE}), and 
the number of iterations is 3 (\sicvar{SELF\_LOOP}). For each 
iteration: 
\begin{itemize}\itemsep 0pt
\item All components found by \com{CLEAN} are kept by default 
(\sicvar{SELF\_NITER}$=0$), but for simple source structures, 10 components only 
may be enough (the maximum number \sicvar{NITER} of  CLEAN components to 
subtract is automatically guessed by the program in the default 
process, see \sicvar{CLEAN\_NITER} and other usual clean convergence control 
variables),
\item the minimum flux density per pixel to be considered by CLEAN can 
be defined (SELF\_MINFLUX $=0$) 
\item The ''integration'' times (gain averaging) are fixed to a default 
value SELF\_{TIMES}$=45$\,s (minimum value for NOEMA, while it is only 
6\,s for ALMA). This can be adapted for each loop (in general, one 
should start with larger solution times, depending on the SNR values 
and try to decrease it in order to better sample the atmospheric 
fluctuations). 
\end{itemize}
The number of iterations can be changed by resizing the 
\sicvar{SELF\_TIMES}, \sicvar{SELF\_NITER}, or \sicvar{SELF\_MINFLUX} 
arrays (the number of loops \sicvar{SELF\_LOOP} is then automatically 
recomputed). You should make sure that these 3 arrays have the same 
dimension. If any of the \sicvar{SELF\_NITER} and/or 
\sicvar{SELF\_MINFLUX} array are constant then their dimension is 
accordingly changed by \com{SELFCAL} each time one of these arrays is 
resized. For instance, the following command allows to define 5 loops 
with an integration time for solution of 45\,sec:
\begin{verbatim}
let Self_times 45 45 45 45 45 /resize
\end{verbatim}
\sicvar{SELF\_NITER} and \sicvar{SELF\_MINFLUX} are automatically enlarged
to the same size if, and only if, they were already constant.
By default, all channels are averaged to compute a 
''continuum'' image, but  the range of adequate channels can be 
specified through \sicvar{SELF\_CHANNEL}. 

\comm{SELFCAL}{PHASE} will compute a phase calibration, but will not 
apply it. One needs to call \com{SELFCAL} once more with the argument 
APPLY in order to apply the solution (the script does really apply the 
solution if and only if the previously found solution can be considered 
as a good one (see SELF\_{STATUS} argument value). 

\comm{SELFCAL}{APPLY} automatically saves the parameters and results in 
the \textit{selfcal.last} file. By default, data are flagged if no 
sufficiently good solution is found. \comm{SELFCAL}{APPLY} keep tracks 
of whether the solution has already been applied through 
\sicvar{SELF\_APPLIED}. \comm{SELFCAL}{APPLY} will refuse to apply 
``bad'' solutions: solutions are declared ``bad'' if the improvement in 
dynamic range and noise level is insufficient (i.e. below a precision 
level controlled by \sicvar{SELF\_PRECISION}). In this case, the 
\sicvar{SELF\_STATUS} variable is negative. In this case, the user can 
still decide to apply the solution directly using command \com{APPLY}, 
but the \sicvar{SELF\_APPLIED} variable will not be updated.

The merit criteria for the quality assessment of the computed solution 
are the final dynamic range, and the Clean map noise at each iteration. 
These quantities are stored in the \sicvar{SELF\_DYNAMIC} and 
\sicvar{SELF\_RMSCLEAN} variables (at each iteration). The dynamic is 
defined as the ratio of the peak flux density value to the noise in the 
clean map (automatically estimated with the command \com{STATISTIC}). 
The  minimum signal to noise ratio value (for an antenna) for a valid 
solution is \sicvar{SELF\_SNR}$=6$ by default. 

\com{SELFCAL} can be controlled through a widget, using
command \comm{SELFCAL}{/WIDGET} (see Figure \ref{fig:selfcal}) 
\begin{figure}
  \centering
  \includegraphics[width=10cm]{selfcal-widget.png}
  \caption{The Self-Calibration widget
\label{fig:selfcal}}
\end{figure}


It is possible to visualize the computed corrections with the command 
\comm{SELFCAL}{SHOW}. The solution computed with the \com{SOLVE} 
command is written in a \textit{'self\_sname'.tab} file. By default, 
the difference between the last two iterations is displayed. For 
\texttt{PHASE}, the phase difference should be close to zero if the 
solution converged, and for \texttt{AMPLI} the amplitude values close 
to 1. It is also possible to show the difference between two specified 
iterations. In addition, the command \comm{SELFCAL}{SUMMARY} will 
display the results of the process in terms of resulting noise and 
improved dynamic range. If the solution is satisfactory, the command 
\comm{SELFCAL}{SAVE} can be used to save both the results and the 
parameters in the \textit{selfcal.last} file. (\comm{SELFCAL}{APPLY} 
performs an implicit \comm{SELFCAL}{SAVE}.)

\subsection{Basic use }
\subsubsection{Timescale for averaging solution interval}

The solution interval, or timescale used to average the solution for 
the gain variations, is the result of a tradeoff between the timescale 
of the true gain variations (e.g., changes in the atmospheric 
conditions or electronics) and the data averaging which is necessary to 
reach a minimum SNR value for the visibilities. In principle, this 
timescale should be smaller than the coherence time
of the atmospheric fluctuations for the phase solution, typically
one minute. It is in general longer for the amplitude gains, since here
these are changes in the atmospheric transparency or antenna gains
which matter. It is 
therefore recommended to start iterations with a not too short 
averaging time, and to decrease it in a second step if the 
self-calibration was successful. It is also recommended to use the same 
integration time for the last two iterations, in order to ease the 
interpretation of the results and of the \comm{SELFCAL}{SHOW} display.
 
The current self-calibration method computes baseline-based
gains (observed complex visibility divided by model prediction) 
for each visibility, and performs the time averaging on these
baseline-based gains. Antenna based gains are derived from
the time-averaged baseline-based gains.  This is in principle an 
optimal method, since the model is not noisy: the linearity of the
first step guarantees a noise decrease as square root of time.

%Boxcar smoothing was used in AIPS in the early VLA times. CASA uses the CLIC engine in a unknown way.

\subsubsection{Quality assessment and data flagging}

\begin{description}\itemsep 0pt
\item [Validation of the solution] 
One of the difficulties of self-calibration is to evaluate whether
it has improved the image or not. The self-calibration 
solution is biased towards the assumed model. If used with insufficient
signal to noise, it will tend to produce a point source at the
initial peak position, and the bias will be of order of the noise.
This may be inappropriate. 
Currently, the validity of the self-calibration solution is based on 
the estimated signal to noise ratio for the gains at each time step. 
If that SNR is below a user-controlled threshold (by default, \sicvar{SELF\_SNR}$=6$), 
the corresponding data is flagged (default value: \sicvar{SELF\_FLAG}$=$\texttt{YES}) 
or kept WITHOUT self calibration (if \sicvar{SELF\_FLAG}=\texttt{NO}). 
%
\item [Flagging or not flagging ?] 
The decision to flag or not results from a trade-off:
\begin{itemize}\itemsep 0pt
\item \sicvar{SELF\_FLAG} $=$ \texttt{Yes} : will result in no contamination by bad data, but
may lead to lower angular resolution since long baselines may be flagged.
\item \sicvar{SELF\_FLAG} $=$ \texttt{No}: will avoid loosing all long baselines (where the SNR is lower)
\end{itemize}
Both options may be explored, and it is recommended to check afterwards the final angular 
resolution with and without flagging.
\end{description}

The following scheme is proposed to check the validity of the self-calibration solution:
\begin{itemize}\itemsep 0pt
\item read the status using \comm{SELFCAL}{SUMMARY}
\item use \comm{SELFCAL}{SHOW} to verify if the solution is converged
\item if it looks good, but noise is still far from theoretical, try 
again to self-calibrate with a shorter integration time (SELF\_TIMES). 
\item if it is not good, try to increase \sicvar{SELF\_TIMES} and find an 
optimum value. For ALMA data, typical values may be in the range 
$6-60$\,s, and for NOEMA in the range $45-120$\,s. Alternatively, you 
can also try to decrease \sicvar{SELF\_SNR} to lower values, but never
less than 3.
\end{itemize}

From our experience, the number of loops \sicvar{SELF\_NLOOP} does not impact 
much the quality of the solution, and 2 to 3 iterations are usually 
sufficient.

%Propositions and discussion:
%Min/max values should be displayed:  Nyquist sampling time could be automatically calculated 
%how is the maximum expected dynamic range computed ? (smaller than the one obtained after selfcal ! ;)
%sometimes Expected rms noise is       NaN microJy/beam (SNR>=4)
%comment afficher les donnees flagguees (visualiser quelles bases on ete flagguees)

Caveat: the comparison with theoretical noise relies on a proper 
scaling of the weights of the UV data. This was fine for the previous
IRAM correlator, but data exported from CASA is not always correct in this respect.
It also remains to be checked with the PolyFIX NOEMA correlator. 
\com{UV\_PREVIEW} can warn you about potential issues in this respect. 
The appropriate scaling factor can be specified by variable \sicvar{SELF\_SNOISE}.

\subsubsection{Advanced use}

\paragraph{Amplitude self calibration}
The amplitude calibration (\comm{SELFCAL}{AMPLI}) is a secondary step in the self-calibration process.
In general, it should only be attempted if the phase calibration was 
already excellent, i.e., once you obtained the best solution by 
adjusting the \sicvar{SELF\_TIMES} parameter for the 
\comm{SELFCAL}{PHASE} command. If possible or needed, the amplitude 
self-calibration should use a \textit{longer timescale} than the phase 
calibration (typically, \sicvar{SELF\_TIMES}$=120$s). \com{SELFCAL} 
automatically adjusts the gains so that their mean is 1, in order to 
avoid changing the flux scale. In practice, it is useless if the expected noise
limited dynamic range is less than about 300.

\paragraph{Support restriction, flux threshold}
Support restriction in the \com{CLEAN} process may be useful to build a 
simpler model for very complex, extended sources only.  Similarly, limiting
the flux per pixel in the model may help.


\clearpage
\newpage
\section{Visual Checks and Image Displays}

The ultimate (and often sole) way to evaluate the data quality
and suitability for scientific analysis is to visualize it.
\imager{} offer simple, yet powerfull, tools to do so.

\subsection{The \texttt{UV\_PREVIEW} command}
With large datasets, image can be long. \imager{} offers a
simple, fast pre-imaging viewer through command \com{UV\_PREVIEW}.

This command will display the spectra obtained towards the
phase center at several spatial scales (the default is for 4 tapers).
It will also attempt to detect spectral features, by analyzing for
each spectrum the noise statistics and the outliers. It performs
automatic line identifications, using database(s) in the \lang{LINEDB}
format selected by command \com{CATALOG}.
Potential spectral lines in the band are displayed in blue, and
detected ones in red.
\com{UV\_PREVIEW} also warns the user about improper scaling
of the data weights.
The line emission region is indicated in grey.

The result of this automatic signal detection and line identification
is saved in a SIC structure named \sicvar{PREVIEW\%}, that is 
automatically used by commands \comm{UV\_BASELINE}{/CHANNELS} and
\comm{UV\_FILTER}{/CHANNELS} to respectively remove the continuum and
filter the line emission to produce a continuum data set. 
The detected line frequencies are stored in
 \sicvar{PREVIEW\%FOUND\%FREQ} and their names in
\sicvar{PREVIEW\%FOUND\%LINES}. This can be used to reference
the velocity scale of the data to one of the detected lines, by
using command \comm{SPECIFY}{FREQUENCY}\texttt{'PREVIEW\%FOUND\%FREQ[1]'} 
for example before further processing.

An example is shown in Fig\ref{fig:preview}
\begin{figure}[!h]
  \centering
  \includegraphics[width=16cm]{preview.png}
  \caption{The \com{UV\_PREVIEW} output
\label{fig:preview}}
\end{figure}


\subsection{the \texttt{SHOW} command}

In general, command \com{SHOW} allows a per-plane display of any SIC 
3-D image variable, with contours overlaid on bitmap for each plane: 
e.g.\\
\texttt{SHOW DIRTY 30 -30} 
will display contour and bitmaps of each channel of the \texttt{DIRTY}
image, starting for channel 30 and ending 30 channels before the last one.
See Fig.\ref{fig:show} for an example. 
\begin{figure}
  \centering
  \includegraphics[width=16cm]{show.png}
  \caption{The \com{SHOW} output.
\label{fig:show}}
\end{figure}

For \uv{}  data, \comm{SHOW}{UV} can plot visibility values
such as amplitude as a function of time, baseline length, etc...,
again on a per channel basis. 

\com{SHOW} can also display more specific issues:
\begin{itemize}\itemsep 0pt
\item \comm{SHOW}{CCT} will display the cumulative flux as a function 
of number of clean components (Fig.\ref{fig:cct}). 
\begin{figure}
  \centering
  \includegraphics[width=12cm]{cct.png}
  \caption{The \comm{SHOW}{CCT} output.
\label{fig:cct}}
\end{figure}
\item \comm{SHOW}{COVERAGE} will display the \uv{} coverage (it
assumes there is only one, and not one per channel, because the
display time is long, see Fig.\ref{fig:coverage})
\begin{figure}
  \centering
  \includegraphics[width=12cm]{coverage.png}
  \caption{The \comm{SHOW}{COVERAGE} output. Colors indicate
  different dates.
\label{fig:coverage}}
\end{figure}
\item \comm{SHOW}{SELFCAL} behaves as \comm{SELFCAL}{SHOW}
\item \comm{SHOW}{FIELDS} displays the fields of a Mosaic.
\item \comm{SHOW}{NOISE} displays the histogram of the intensity
distribution for each channel, estimating the noise by fitting
a Gaussian in these histograms (see Fig.\ref{fig:noise}).
\begin{figure}
  \centering
  \includegraphics[width=16cm]{noise.png}
  \caption{The \comm{SHOW}{NOISE} output.
\label{fig:noise}}
\end{figure}
\end{itemize}

A direct use on Gildas 3-D image files is also possible:
\com{SHOW}\texttt{ Filename.ext} will directly display the
file if possible. It also works for simple FITS files in which
the data array is in the HDU.

\subsection{the \texttt{VIEW} command}

The \com{VIEW} command is a powerful alternative to \com{SHOW}, the
later being inefficient when the number of spectral line channels
is large. 

\com{VIEW} provides a 4 panel display for 3-D data cubes, with
the current channel bitmap, the integrated area bitmap, the current
spectrum and the integrated intensity spectrum.  The spectra
can be displayed with 2 simultaneous frequency windows, a broad
and a zoomed one, allowing browsing through a large number of
channels. Spectral line identification can be added by typing
L when the cursor is on one of the 2 broad-band spectra.

\begin{figure}
  \centering
  \includegraphics[width=16cm]{view.png}
  \caption{The \com{VIEW} output.
\label{fig:view}}
\end{figure}

\comm{VIEW}{CCT} will display the cumulative flux of Clean components
for all channels in just one panel, instead of a per-channel panel
for \comm{SHOW}{CCT}


\subsection{the \texttt{SELFCAL SHOW} command}

Verifying the convergence of a self-calibration is important.
Figure \ref{fig:selfphase} shows an example, while Fig.\ref{fig:selftot}
shows the total phase correction between the original data and
the last iteration. 
\begin{figure}
  \centering
  \includegraphics[width=12cm]{self-phase.png}
  \caption{The \comm{SELFCAL}{SHOW} output after a phase calibration,
  showing the convergence of the corrections between the last 2
  iterations.
\label{fig:selfphase}}
\end{figure}
\begin{figure}
  \centering
  \includegraphics[width=12cm]{self-phase-total.png}
  \caption{The \comm{SELFCAL}{SHOW} 4 output after a phase calibration,
  showing the difference between iteration 4 and the original
  data.  
\label{fig:selftot}}
\end{figure}

The displayed range can be controlled
by variables \sicvar{SELF\_PRANGE[2]} for the Phase, 
\sicvar{SELF\_ARANGE[2]} for the Amplitude, and \sicvar{SELF\_TRANGE[2]} 
for the Time. Error bars are displayed if \sicvar{ERROR\_BARS} is \texttt{YES},
as shown in Fig.\ref{fig:selfamp} for amplitude.
\begin{figure}
  \centering
  \includegraphics[width=12cm]{self-ampli-total.png}
  \caption{The \comm{SELFCAL}{SHOW} 4 output after an amplitude self calibration,
  showing the difference between iteration 4 and the original
  data, with the error bars.  
\label{fig:selfamp}}
\end{figure}


\clearpage
\newpage
\section{The "all in one" imager suite}
\label{sec:all}

With \ALMA{} or \NOEMA{}, you may end up with an observational data set
that contains several sources, each of them observed with
a number of spectral windows of different spectral resolutions,
covering many spectral lines. 

The bookkeeping of such data sets can be intricate. To help the
users to focus on the science, we have developped a suite of
scripts that automates the whole imaging process in this case.

These scripts have names starting with \texttt{all-}  and control
parameters are available in the \sicvar{ALL\%} global structure.

The principle is to gather all UV tables in a sub-directory (named
\texttt{./RAW/} by default), and to store the results of the
various processing steps (Self Calibration,  Spectral line
extraction, Imaging) in different sub-directories (respectively 
named \texttt{./SELF/, ./TABLES/, ./MAPS/} by default).

Naming conventions apply to identify which data set covers which 
spectral line. An automatic determination of the continuum
level is performed: this implies that no spectral line
confusion should occur.  The case of spectral confusion will be
handled later.

\subsection{Preparing the data}

For \NOEMA{}, the UV data can be created from the \texttt{.IPB,.hpb}
raw data files using the \clic{}  script \texttt{@ all-tables}. The 
resulting UV tables (\texttt{.uvt} files) should be placed in your
working directory.

For \ALMA{},  UVFITS files should be created from the original Measurement
Set (\texttt{.ms} directory) using the \texttt{casagildas()} Python tool 
in CASA, and placed in your working directory. 

Once your working directory is loaded with the \texttt{.uvt} or 
\texttt{.uvfits} files, you can start using the \texttt{all-widget}
script.


\subsection{The \texttt{all-} scripts}

\subsubsection{Getting started: \texttt{@ all-widget}}

\texttt{all-widget.ima} creates a Widget that is used to customize
the process and launch the various steps. It also creates
(through a call to \texttt{@ all-create}) the \sicvar{ALL\%} structure
and its components.
\begin{figure}[!h]
  \centering
  \includegraphics[width=14cm]{all-widget.png}
  \caption{The All-In-One control widget
\label{fig:allwidget}}
\end{figure}

\subsubsection{\texttt{ORGANIZE} step}
The \texttt{ORGANIZE} button will move the initial files
(\texttt{.uvfits} and \texttt{.uvt} files) into an appropriate sub-directory
structure. By default, \texttt{.uvfits} files will be in ./UVFITS/
(name controlled by \sicvar{all\%uvfits})
sub-directory, while \texttt{.uvt} files will be in ./RAW/ sub-directory
(name controlled by \sicvar{all\%raw}).

If absent, the \texttt{.uvt} files will be created from the \texttt{.uvfits} ones.

\subsubsection{\texttt{FIND} step}

The \texttt{FIND} button will explore the directory containing
the initial UV tables (\texttt{./RAW/} by default) to identify
\textit{Wide Band} UV tables, i.e. those that cover enough 
bandwidth to provide enough sensitivity
for self-calibration on continuum flux.

\textbf{CURRENT LIMITATIONS} \\
- No check is made on the continuum flux.\\
- No provision is made to use spectral line flux instead.\\

\subsubsection{\texttt{SELF} step}

The \texttt{SELF} button will use the list of \textit{Wide Band}
UV tables to compute a (phase and amplitude) Self calibration
solution, and apply it to all UV tables. It will identify which
UV tables correspond to a given Wide Band one, so that the proper
self calibration solution is applied.

The self calibrated UV tables are placed in another sub-directory
(\texttt{./SELF/} by default, controlled by \sicvar{all\%self}).

A prefix (controlled by \sicvar{all\%prefix\_self}) can be added
to the file names to remind the user that they have been self-calibrated.

\subsubsection{\texttt{TIME} step}

The \texttt{TIME} button will time average the self calibrated
UV tables to save space and speed up further processing.

\subsubsection{\texttt{IMAGING} step}

The \texttt{IMAGING} button will scan all spectral windows to
identify whether they can be used to produce a Continuum or
spectral line image. 

  Low resolution spectral windows, identified as those whose
resolution is coarser than \sicvar{ALL\%MINFRES} will be used to
produce continuum images, by filtering any detected spectral
signature in the data before.

  High resolution spectral windows, identified as those whose
resolution is better than \sicvar{ALL\%MINFRES}, will be scanned
for line identifications. For each spectral line in the current 
catalog(s) (defined by command \com{CATALOG}) 
that fall in a spectral window, a continuum-free UV table will be
created, covering the velocity range specified by the user (by
variable \sicvar{ALL\%RANGE}) around the line frequency. The naming
convention is the following:
\begin{verbatim}
  original-molecule-I-X
\end{verbatim}
where \\
- \texttt{original} is the name of the initial high resolution
spectral window\\
- \texttt{molecule} is the name of the spectral line in the catalog\\
- \texttt{I} is a sequence number, incremented each time a new
line is found from the same original UV table.\\
- \texttt{X} is a character code, equal to \texttt{D} if the line
is suspected to be detected, and to \texttt{U} if not. When several
lines are too close together, the \texttt{D} status may be incorrectly
affected, but this is just a naming convention, not a strong result.

  In addition an
\begin{verbatim}
  original-C
\end{verbatim}
file that contains line-free emission only is created for each
original UV table. 

  The imaging results are stored in a specific sub-directory 
(\texttt{./MAPS/} by default, controlled by \sicvar{all\%maps}). Only 
the Clean Component Table (\texttt{.cct}) and Clean image 
(\texttt{.lmv-clean}) files are written.

\subsubsection{\texttt{TABLES} step}

The \texttt{TABLES} button perform the same scanning and
identificatio process as the \texttt{IMAGING} button, but stores
the resulting UV Tables (instead of the .cct and .lmv-clean files for
the \texttt{IMAGING} button) in a sub-directory (\texttt{./TABLES/}
by default, controlled by \sicvar{all\%tables}).

  This steps is optional, and only needed if the user intends to 
perform some UV data analysis (like UV plane fitting, line stacking,
continuum spectral index determination, direct modeling, etc...).

\section{Really Huge Problems}

\imager{} assumes everything fits into memory. This is in general
quite fine for \NOEMA{}  data. However, for \ALMA{} data, if you are 
working with a too small computer (such as my laptop, which is 
otherwise fine), you may be lacking physical RAM, and \imager{} may 
become really inefficient by using Virtual Memory instead.

To avoid time losses, \imager{} prevents reading UV data whose size
exceeds the available RAM, and warns the user if it exceeds half
of the RAM.  To treat these cases, \imager{} provides instead
a number of tools that can work sequentially on the data set,
instead of loading it in memory all at once. 

\begin{enumerate}
\item Working by subsets:\\
  the \comm{READ}{/RANGE} command allows to select an ensemble of 
  channels from a UV data. If this ensemble is small enough, \imager{} 
  can work.  At the end the \comm{WRITE}{/APPEND} and 
  \comm{WRITE}{/REPLACE} command will allow to put these channels at 
  their proper places in a deconvolved data cube.
\item Working on UV data files:\\
  Some operations on UV data, like time averaging, separation of line 
  from continuum emission, or self-calibration, and of course, spectral 
  resampling, are best done using all valid channels to avoid loosing 
  sensitivity. 
  To allow \imager{} to do them even for large data
  files, most UV-related commands have a \texttt{/FILE} option which
  instructs the command to work from the corresponding data file,
  instead of the UV buffers. This includes \comm{UV\_PREVIEW}{/FILE}, 
  \comm{UV\_BASELINE}{/FILE}, \comm{UV\_FILTER}{/FILE} and especially 
  the \comm{UV\_SPLIT}{/FILE} commands. Time averaging can be performed 
  by \comm{UV\_TIME}{/FILE}, and prior sorting by time order can be
  done by \comm{UV\_SORT}{/FILE}. Spectral range extraction is possible by 
  \comm{UV\_EXTRACT}{/FILE}, spectral resampling by \comm{UV\_COMPRESS}{/FILE}
  \comm{UV\_RESAMPLE}{/FILE}  and \comm{UV\_HANNING}{/FILE}.
\end{enumerate}
By using the above commands, all operations can be done in a
quasi sequential way, avoiding to load in memory whole data sets.

Two commands have no equivalent using the \imager{} buffers, and work
only on files. Their \texttt{/FILE} option is used to provide an
homogeneous syntax, but is  mandatory.
\begin{enumerate}
\item the \comm{UV\_MERGE}{/FILE}\\
that allows to merge together an arbitrary number of UV tables,
in spectral line (with spectral resampling) or continuum (with 
flux scaling according to a spectral index) modes.
\item the \comm{UV\_SPLIT}{/FILE}\\
that combines the capabilities of \com{UV\_BASELINE} and 
\com{UV\_FILTER} in a single command, since both operations require 
the same parameters and provide complementary informations.
\item The \comm{UV\_SORT}{/FILE} command also has a different behaviour than
its memory only version \com{UV\_SORT}. While the later creates
a transposed version of the UV table, the former keeps the normal
organisation with the visibility axis first. 
\end{enumerate}



\clearpage
\newpage
\section{Polarization Handling}
\label{sec:polar}

\imager{} recognizes and handles polarization at different levels.
Support for polarization is currently experimental and limited, 
but continuously improving.

\paragraph{Importing data}
When importing data, the \texttt{fits\_to\_uvt} script assumes by default
the data is unpolarized and produces the pseudo-polarisation
state "None" from the UVFITS file, by a properly weighted combination
of the two parallel hand states if more than one state is present.

Full polarization information can be preserved by adding the 
\texttt{/STOKES} option to the \texttt{@ fits\_to\_uvt} command. 

The \com{READ}\texttt{ UV} command will read data with any
polarization state(s).

\paragraph{Processing data}
On the contrary, practically all \imager{} commands cannot handle 
more than 1 polarization state. 

Most commands will flatly refuse to handle data with more than 
1 polarization state (e.g. \com{UV\_FILTER}, \com{UV\_RESAMPLE}, etc\ldots).

For debugging purpose, some commands like \com{UV\_PREVIEW}, \com{UV\_MAP}
or \com{UV\_STAT} will operate with more than 1 polarization state,
but will not produce meaningful results (only a subset of the data
may be treated).

So far, there are only two commands that fully support polarized
data: \com{UV\_TIME} and \com{STOKES}.
\begin{itemize}
\item \com{STOKES} is the primary command that allows to derive or extract 
a UV data with only one Stokes parameter from a UV data set with 
several polarization states. \imager{} can then process the 
individual Stokes parameters separately.
\item For convenience (because polarized data is obviously in general
bigger), the \com{UV\_TIME} command can be used for time averaging
prior to use of the \com{STOKES} command.
\end{itemize}

Some imaging strategies cannot be used on polarized data. Self-calibration
requires a totally different approach that is not available in \imager{}.
Also, the automatic definition of supports (\comm{MASK}{/THRESHOLD}) 
can only be done on Stokes I or parallel hand polarization states, not
on Stokes Q,U,V or cross-hand states. The same applies to spectral
line identifications.
 
\paragraph{Analysing data}
Sorry, so far \imager{} offers no integrated tools to derive e.g.
polarization fraction, or polarization vector directions. This may
come later !...
 
\subsection{The STOKES command}

The \comm{STOKES}{/FILE} command operates only on files. It allows
to extract a UV data with visibilities for one output (pseudo-)Stokes 
parameter from UV data with visibilities with 1,2 or 4 (pseudo-)Stokes
parameters. Besides the standard Stokes parameters I, Q, U, V, RR, 
LL, RL, LR (Left and Right circular), XX, YY, XY and YX (X and Y
linear) which are defined in the Sky frame,
command \com{STOKES} recognizes pseudo-Stokes values NONE, HH, VV,
HV and VH which are the linear polarization states in the frame
of the antennas (Horizontal and Vertical pure states).

Conversion from the H-V pseudo-Stokes polarization states to any standard
Stokes parameter is made by the \com{STOKES} command by applying 
the rotation due to the parallactic angle. For this, the UV data set
must contain the \texttt{PARA\_ANGLE} extra column. If it is not present,
it can be added to the data set by command \comm{UV\_ADD}{/FILE}.
That command can also insert the Doppler correction as an extra
column.

Ultimately, a similar DD, GG, DG and GD pseudo-Stokes polarization set
will be added for circular polarization (D is the first letter for
Droite in French, and G for Gauche). 


%\input{in-out-imager}

%% \section{OTF}


%% \chapter{Imaging simulation}

% {Internal Helps}
\clearpage
\newpage

\section{CLEAN Language Internal Help} \label{CLEANH}
\subsection{Language}
\index{Language}
\begin{verbatim}
    ALMA          : Joint deconvolution of ALMA and ACA dirty images
    CLEAN         : Deconvolve a dirty image using the current METHOD
    CLARK         : Deconvolve a dirty image using the CLARK clean algorithm
    DUMP          : Dump the control parameters of the deconvolution algorithms
    FIT           : Fit the dirty beam
    HOGBOM        : Deconvolve a dirty image using the basic clean algorithm
    MAP_RESAMPLE  :  Resample a Map on a different Velocity/Frequency scale
    MAP_INTEGRATE :  Compute a Moment 0 Map
    MAP_COMPRESS  :  Average a Map by several channels
    MOSAIC        : Toggle the mosaic mode
    MULTI         : Deconvolve a dirty image using the MULTI-SCALE Clean
    MRC           : Deconvolve a dirty image using the MULTI-RESOLUTION Clean
    MX            : Iteratively image and deconvolve a dirty image
    PRIMARY       : Apply primary beam correction
    READ          : Read the input files in internal buffers
    SDI           : Deconvolve a dirty image using the Steer-Dewdney-Ito Clean
    SHOW          : Display (in a plot) some internal buffer result
    SPECIFY       : Change the Frequency / Velocity scale or the Telescope
    STATISTIC     : Compute statistics on image
    STOKES        : Extract one polarization state from a polarized UV table
    SUPPORT       : Define the support used to search clean components
    UV_BASELINE   : Subtract a continuum baseline from a Line UV data
    UV_CHECK      : Check UV data for null visibilities or per channel flags.
    UV_COMPRESS   : Compress a Line UV data into another Line UV data
    UV_CONTINUUM  : Compress a Line UV data into a Continuum UV data
    UV_FILTER     : Filter out (line) channels
    UV_FLAG       : Interactively flag UV data
    UV_MAP        : Build the dirty image and beam from a UV table
    UV_RESAMPLE   : Resample (in Velocity) the UV data
    UV_RESIDUAL   : Subtract Clean Component from the UV Data
    UV_RESTORE    : Restore a Clean image from UV data and Clean Components
    UV_SHIFT      : Shift a UV table to common phase center
    UV_SORT       : Sort and Transpose UV data for plotting
    UV_STAT       : Gives beam sizes and noise properties as a function of
                     tapering or robust weighting parameter
    UV_TIME       : Time average the current UV data
    UV_TRUNCATE   : Truncate the baseline range of the UV data
    VIEW          : Show (in a GO VIEW-like plot) some internal buffer result
    WRITE         : Save internal buffers or Image variables into output files

\end{verbatim}
\subsection{ALMA}
\index{ALMA}
\begin{verbatim}
        [CLEAN\]ALMA  [FirstPlane  [LastPlane]]  [/PLOT Clean|Residu] [/FLUX
    Fmin Fmax] [/QUERY] [/NOISE] [/METHOD]

    Joint deconvolution methods specific to ALMA+ACA observations.

    The ALMA simulator can be accessed either by  typing  "@  alma"  at  the
    prompt or from the MAPPING main menu.


\end{verbatim}
\subsection{CLARK}
\index{CLARK}
\begin{verbatim}
        [CLEAN\]CLARK  [FirstPlane  [LastPlane]] [/PLOT Clean|Residu] [/FLUX
    Fmin Fmax] [/QUERY]

    A Major-Minor cycles CLEAN method, originally developed by  B.Clark,  in
    which  clean components are selected using a limited beam patch, and de-
    convolved through Fourier transform at each major cycle. In mosaic  mode
    (See  command MOSAIC), a mosaic clean is performed. Rings and/or stripes
    may appear on extended sources.  Faster than the Hogbom method for  sin-
    gle  fields  but  maybe  slower  for mosaics. The strategy to search for
    CLEAN components in CLARK does not work properly when the secondary side
    lobes  are  too  large  (e.g. larger than 0.3), or in case of high phase
    noise.

    Clean the specified plane interval (default:  planes  between  variables
    FIRST and LAST). If only FirstPlane is specified, Clean only that plane.

    If option /PLOT is given, a display of the CLEAN or RESIDUAL map will be
    shown  at  each major cycle, depending on the argument (default: Residu-
    al). The user will be prompted for continuation when the  /QUERY  option
    is  present.  The  cumulative, already cleaned flux is displayed in real
    time in an additional window while cleaning goes on when the  /FLUX  op-
    tion  is  present.  Parameters of the /FLUX option are then used to give
    the flux limits for this display.

    The user can control the algorithm through SIC variables. New values can
    be given using "LET VARIABLE value". For ease of use, and whenever it is
    possible, a sensible value of each parameter will automatically be  com-
    puted from the context if the value of the corresponding variable is set
    to its default value, i.e. zero value and empty string. A few  variables
    are initialized to "reasonable" values.

        [CLEAN\]CLEAN ?
    Will  list  all main CLEAN_* variables controlling the CLEAN parameters.
    HELP CLEAN Variables will give a more complete list.

\end{verbatim}
\subsubsection{CLARK Variables:}
\index{CLARK!Variables:}
\begin{verbatim}

    Basic parameters
    CLEAN_GAIN       [       ] Loop gain
    CLEAN_NITER      [       ] Maximum number of clean components
    CLEAN_FRES       [      %] Maximum value of residual (Fraction of peak)
    CLEAN_ARES       [Jy/Beam] Maximum value of residual (Absolute)
    CLEAN_POSITIVE   [       ] Minimum number of positive components at start
    CLEAN_NKEEP      [       ] Min number of components before convergence

    Old names like in MAPPING
    BLC              [  pixel] Bottom left corner of cleaning box
    TRC              [  pixel] Top right corner of cleaning box
    MAJOR            [ arcsec] Clean beam major axis
    MINOR            [ arcsec] Clean beam minor axis
    ANGLE            [ degree] Position angle of clean beam
    BEAM_PATCH       [  pixel] Size of cleaning beam ** not clear **

    Method dependent parameters
    CLEAN_INFLATE    [      ] Maximum Inflation factor for UV_RESTORE (MuLTISCAL
    CLEAN_NCYCLE     [      ] Max number of Major Cycles (SDI & CLARK methods)
    CLEAN_NGOAL      [      ] Max number of comp. in Cycles (ALMA method)
    CLEAN_RESTORE    [      ] Threshold for restoring a Mosaic (def 0.2)
    CLEAN_SEARCH     [      ] Threshold to search Clean Comp. in a Mosaic (def 0
    CLEAN_SIDELOBE   [      ] Min threshold to fit the synthesized beam
    CLEAN_SMOOTH     [      ] Smoothing ratio (MRC and MULTISCALE)
    CLEAN_SPEEDY     [      ] Speed-up factor (CLARK)
    CLEAN_WORRY      [      ] Worry factor (MULTISCALE)
\end{verbatim}
\subsubsection{CLARK CLEAN\_ARES}
\index{CLARK!CLEAN\_ARES}
\begin{verbatim}

    This is the minimal flux in the dirty map that the program will consider
    as  significant.   Alternatively,  the  threshold  can be specified as a
    fraction of the peak flux using CLEAN_FRES.  Once this  level  has  been
    reached the program stops subtracting, and starts the restoration phase.
    The unit for this parameter is the map unit  (typically  Jy/Beam).   The
    parameter  should  usually  be of the order of magnitude of the expected
    noise in the clean map.

    If 0, CLEAN_FRES will be used instead. If all of CLEAN_NITER, CLEAN_ARES
    and CLEAN_FRES are 0, an absolute residual equal to the noise level will
    be used for CLEAN_ARES.

    Short form is ARES.

\end{verbatim}
\subsubsection{CLARK CLEAN\_FRES}
\index{CLARK!CLEAN\_FRES}
\begin{verbatim}

    This is the minimal fraction of the peak flux in the dirty map that  the
    program  will  consider  as  significant.   Alternatively,  an  absolute
    threshold can be specified using CLEAN_ARES.  Once this level  has  been
    reached the program stops subtracting, and starts the restoration phase.
    This parameter is normalized to 1 (neither in % nor in db).   It  should
    usually  be of the order of magnitude of the inverse of the expected dy-
    namic range of the intensity.

    If 0, CLEAN_ARES will be used instead. If all of CLEAN_NITER, CLEAN_ARES
    and CLEAN_FRES are 0, an absolute residual equal to the noise level will
    be used for CLEAN_ARES.

    Short form is FRES.

\end{verbatim}
\subsubsection{CLARK CLEAN\_GAIN}
\index{CLARK!CLEAN\_GAIN}
\begin{verbatim}

    This is the gain of the subtraction loop.  It should typically be chosen
    in the range 0.05 and 0.3.  Higher values give faster convergence, while
    lower values give a better restitution of the extended structure. A sen-
    sible default is 0.2.

    Short form is GAIN.

\end{verbatim}
\subsubsection{CLARK CLEAN\_NITER}
\index{CLARK!CLEAN\_NITER}
\begin{verbatim}

    This is the maximum number of components the program will accept to sub-
    tract.  Once it has been reached, the  program  starts  the  restoration
    phase.

    If 0, the program will guess a number, based on the image size and maxi-
    mum signal-to-noise  ratio,  and  specified  residual  level  CLEAN_ARES
    and/or CLEAN_FRES.

    Short form is NITER.

\end{verbatim}
\subsubsection{CLARK CLEAN\_NKEEP}
\index{CLARK!CLEAN\_NKEEP}
\begin{verbatim}

    This is an integer specifying the minimum number of Clean components be-
    fore testing if Cleaning has converged. The convergence is criterium  is
    a  comparison  of the cumulative flux evolution separated by CLEAN_NKEEP
    components. If th

    IF CLEAN_NKEEP is 0, CLEAN will ignore this convergence  criterium,  and
    continue  clean until the CLEAN_NITER, CLEAN_ARES or CLEAN_FRES criteria
    indicate to stop.

    With CLEAN_NKEEP > 0, CLEAN will explore  the  stability  of  the  total
    clean  flux over the last CLEAN_NKEEP  iterations. For a positive (resp.
    negative) source, if the Clean flux becomes smaller (resp. larger)  than
    the Clean flux CLEAN_NKEEP iterations earlier, CLEAN will stop.

    Using  CLEAN_NKEEP  about  70 is a reasonable value.  Some special cases
    (faint extended sources) may require larger values of CLEAN_NKEEP.

\end{verbatim}
\subsubsection{CLARK CLEAN\_POSITIVE}
\index{CLARK!CLEAN\_POSITIVE}
\begin{verbatim}

    The minimum number of positive components before negative ones  are  se-
    lected.

\end{verbatim}
\subsubsection{CLARK CLEAN\_RESTORE}
\index{CLARK!CLEAN\_RESTORE}
\begin{verbatim}

      Fraction  of  peak  response of the primary beams coverage under which
    the Sky brightness image is blanked in a Mosaic deconvolution.

    The default is 0.2.

\end{verbatim}
\subsubsection{CLARK CLEAN\_SEARCH}
\index{CLARK!CLEAN\_SEARCH}
\begin{verbatim}


      Fraction of peak response of the primary beams coverage  beyond  which
    no Clean component is searched in a Mosaic deconvolution.

    The default is 0.2.

\end{verbatim}
\subsubsection{CLARK CLEAN\_SIDELOBE}
\index{CLARK!CLEAN\_SIDELOBE}
\begin{verbatim}

    Minimal  relative  intensity to consider for fitting the syntheized beam
    to obtain the Clean beam parameters (MAJOR, MINOR  and  ANGLE)  when  0.
    The default is 0.35.

    In  case  of  poor UV coverage, CLEAN_SIDELOBE should be higher than the
    maximum sidelobe level to perform a good Gaussian fit. Some particularly
    bad UV coverage may not allow any good fit at all, however.

\end{verbatim}
\subsubsection{CLARK CLEAN\_NGOAL}
\index{CLARK!CLEAN\_NGOAL}
\begin{verbatim}

      Number  of clean components to be selected in a Cycle in the ALMA het-
    erogeneous array cleaning method.

\end{verbatim}
\subsubsection{CLARK CLEAN\_NCYCLE}
\index{CLARK!CLEAN\_NCYCLE}
\begin{verbatim}

      Maximum number of Major Cycles for the SDI and CLARK methods.

\end{verbatim}
\subsubsection{CLARK CLEAN\_SMOOTH}
\index{CLARK!CLEAN\_SMOOTH}
\begin{verbatim}

      Smoothing factor between different scales in the  MRC  and  MULTISCALE
    methods.  The default is sqrt(3).

\end{verbatim}
\subsubsection{CLARK CLEAN\_SPEEDY}
\index{CLARK!CLEAN\_SPEEDY}
\begin{verbatim}

      Speed-up factor for the CLARK major cycles. The default is 1.0.  Larg-
    er values may be used, but at the expense of possible  instabilities  of
    the algorithm.

\end{verbatim}
\subsubsection{CLARK CLEAN\_WORRY}
\index{CLARK!CLEAN\_WORRY}
\begin{verbatim}

      Worry  factor  in the MULTISCALE method for convergence. It propagates
    the S/N from one iteration to the other, so that if this  S/N  degrades,
    the  method  stops.  Default  is 0 (no propagation, and hence no test on
    S/N).  The value should be < 1.0 in all cases.

\end{verbatim}
\subsubsection{CLARK CLEAN\_INFLATE}
\index{CLARK!CLEAN\_INFLATE}
\begin{verbatim}

      Maximum Inflation factor for UV_RESTORE (MULTISCALE method).   If  the
    number  of  true (i.e. pixel based) Clean components found by MULTISCALE
    is larger than CLEAN_INFLATE times the number of compressed (i.e.  those
    with the smoothing factor information) components, expansion of the com-
    pressed components will not be possible, and UV_RESTORE will not be use-
    able.

      A  default  of  50  is in general adequate.  Better solutions might be
    found in the future, and this parameter suppressed.  Apart  from  memory
    usage, this number has no consequence on the algorithm.

\end{verbatim}
\subsubsection{CLARK METHOD}
\index{CLARK!METHOD}
\begin{verbatim}

      Method used for the deconvolution. Can be HOGBOM, MULTI, MRC,
      SDI or CLARK.


\end{verbatim}
\subsubsection{CLARK Old\_Names:}
\index{CLARK!Old\_Names:}
\begin{verbatim}

      Some  of the CLEAN parameters have kept their old names: MAJOR, MINOR,
    ANGLE (which are also used by  command  FIT)  BLC,  TRC  and  BEAM_PATCH
    (which are seldom used)

      Others  have equivalent short names: ARES, FRES, GAIN, NITER for which
    the CLEAN_ prefix may be omitted.

\end{verbatim}
\subsubsection{CLARK BLC}
\index{CLARK!BLC}
\begin{verbatim}

    These are the (pixel) coordinates of  the  Bottom  Left  Corner  of  the
    cleaning  box.   The actual cleaning support will be the intersection of
    the specified window with the inner quarter of the map and with any user
    defined polygon.

\end{verbatim}
\subsubsection{CLARK TRC}
\index{CLARK!TRC}
\begin{verbatim}

    These  are the (pixel) coordinates of the Top Right Corner of the clean-
    ing box.  The actual cleaning window will be  the  intersection  of  the
    specified window with the inner quarter of the map and with any user de-
    fined polygon.


\end{verbatim}
\subsubsection{CLARK MAJOR}
\index{CLARK!MAJOR}
\begin{verbatim}

    This is the major axis  (FWHP)  in  user  coordinates  of  the  Gaussian
    restoring beam. If 0, the program will fit a Gaussian to the dirty beam.
    We strongly discourage to change the default value of 0.

\end{verbatim}
\subsubsection{CLARK MINOR}
\index{CLARK!MINOR}
\begin{verbatim}

    This is the minor axis  (FWHP)  in  user  coordinates  of  the  Gaussian
    restoring beam. If 0, the program will fit a Gaussian to the dirty beam.
    We strongly discourage to change the default value of 0.

\end{verbatim}
\subsubsection{CLARK ANGLE}
\index{CLARK!ANGLE}
\begin{verbatim}

    This is the position angle (from North towards East, i.e. anticlockwise)
    of  the  major  axis of the Gaussian restoring beam (in degrees).  If 0,
    the program will fit a Gaussian to the dirty beam. We strongly  discour-
    age to change the default value of 0.

\end{verbatim}
\subsubsection{CLARK BEAM\_PATCH}
\index{CLARK!BEAM\_PATCH}
\begin{verbatim}

    The  dirty  beam  patch to be used for the minor cycles in CLARK and MRC
    method.  It should be large enough to avoid doing too many major cycles,
    but  has  practically  no  influence on the result.  This size should be
    specified in pixel units.  Reasonable values are between  N/8  and  N/4,
    where N is the number of map pixels in the same dimension.  If set to N,
    the CLARK algorithm becomes identical to the HOGBOM algorithm.



\end{verbatim}
\subsection{CLEAN}
\index{CLEAN}
\begin{verbatim}
        [CLEAN\]CLEAN [FirstPlane [LastPlane]] [/PLOT Clean|Residu]
      [/FLUX Fmin Fmax] [/QUERY] [/NITER NiterList] [/ARES AresList]

    Deconvolve a Mosaic or Single-field using the  current  METHOD  (in  SIC
    variable  METHOD).  See INPUT CLEAN for the other SIC variables control-
    ling the deconvolution process.

    Clean the specified plane interval (default:  planes  between  variables
    FIRST and LAST). If only FirstPlane is specified, Clean only that plane.

    Supported methods are CLARK, HOGBOM, MRC, MULTISCALE and SDI.  See  help
    of  each  of  these commands for further details on each algorithm. This
    command allows  a  per-plane  definition  of  the  convergence  criteria
    CLEAN_NITER and CLEAN_ARES.

    The user can control the algorithm through SIC variables. New values can
    be given using "LET VARIABLE value". For ease of use, and whenever it is
    possible,  a sensible value of each parameter will automatically be com-
    puted from the context if the value of the corresponding variable is set
    to  its default value, i.e. zero value and empty string. A few variables
    are initialized to "reasonable" values.

        [CLEAN\]CLEAN ?
    Will list all main CLEAN_* variables controlling the  CLEAN  parameters.
    HELP CLEAN Variables will give a more complete list.

\end{verbatim}
\subsubsection{CLEAN /FLUX}
\index{CLEAN!/FLUX}
\begin{verbatim}
        [CLEAN\]CLEAN   [FirstPlane[LastPlane]]   /FLUX   Fmin  Fmax  [/PLOT
    Clean|Residu] [/QUERY] [/NITER NiterList] [/ARES AresList]

    Display the cumulative Clean flux as Clean progresses.  This  option  is
    inactive in Parallel mode.

\end{verbatim}
\subsubsection{CLEAN /PLOT}
\index{CLEAN!/PLOT}
\begin{verbatim}
        [CLEAN\]CLEAN  [FirstPlane  [LastPlane]]  /PLOT  Clean|Residu [/FLUX
    Fmin Fmax] [/QUERY] [/NITER NiterList] [/ARES AresList]

    Display the iterated Clean or Residual image for Cleaning methods  which
    have  major  cycles  (CLARK or SDI). This option is inactive in Parallel
    mode.

\end{verbatim}
\subsubsection{CLEAN /QUERY}
\index{CLEAN!/QUERY}
\begin{verbatim}
        [CLEAN\]CLEAN [FirstPlane  [LastPlane]]  /PLOT  Clean|Residu  /QUERY
    [/FLUX Fmin Fmax] [/NITER NiterList] [/ARES AresList]

    *** Obsolescent ***

    Prompt  for  continuation when a Major cycle is complete. This option is
    inactive in Parallel mode.

\end{verbatim}
\subsubsection{CLEAN /NITER}
\index{CLEAN!/NITER}
\begin{verbatim}

        [CLEAN\]CLEAN  [FirstPlane  [LastPlane]]  /NITER  NiterList   [/PLOT
    Clean|Residu] [/FLUX Fmin Fmax] [/QUERY] [/ARES AresList]

    Use a per-plane value for the number of iterations, instead of the glob-
    al NITER variable.  NiterList should be a 1-D integer array of dimension
    the number of channels.

    This option is only available through the CLEAN command, not through the
    specific command of each method.

\end{verbatim}
\subsubsection{CLEAN /ARES}
\index{CLEAN!/ARES}
\begin{verbatim}

        [CLEAN\]CLEAN  [FirstPlane  [LastPlane]]   /ARES   AresList   [/PLOT
    Clean|Residu] [/FLUX Fmin Fmax] [/QUERY] [/NITER NiterList]

    Use  a  per-plane value for the absolute residual used to stop cleaning,
    instead of the global ARES variable.  AresList should be a 1-D real  ar-
    ray of dimension the number of channels.

    This option is only available through the CLEAN command, not through the
    specific command of each method.

\end{verbatim}
\subsubsection{CLEAN Variables:}
\index{CLEAN!Variables:}
\begin{verbatim}

    Basic parameters
    CLEAN_GAIN       [       ] Loop gain
    CLEAN_NITER      [       ] Maximum number of clean components
    CLEAN_FRES       [      %] Maximum value of residual (Fraction of peak)
    CLEAN_ARES       [Jy/Beam] Maximum value of residual (Absolute)
    CLEAN_POSITIVE   [       ] Minimum number of positive components at start
    CLEAN_NKEEP      [       ] Min number of components before convergence

    Old names like in MAPPING
    BLC              [  pixel] Bottom left corner of cleaning box
    TRC              [  pixel] Top right corner of cleaning box
    MAJOR            [ arcsec] Clean beam major axis
    MINOR            [ arcsec] Clean beam minor axis
    ANGLE            [ degree] Position angle of clean beam
    BEAM_PATCH       [  pixel] Size of cleaning beam ** not clear **

    Method dependent parameters
    CLEAN_INFLATE    [      ] Maximum Inflation factor for UV_RESTORE (MuLTISCAL
    CLEAN_NCYCLE     [      ] Max number of Major Cycles (SDI & CLARK methods)
    CLEAN_NGOAL      [      ] Max number of comp. in Cycles (ALMA method)
    CLEAN_RESTORE    [      ] Threshold for restoring a Mosaic (def 0.2)
    CLEAN_SEARCH     [      ] Threshold to search Clean Comp. in a Mosaic (def 0
    CLEAN_SIDELOBE   [      ] Min threshold to fit the synthesized beam
    CLEAN_SMOOTH     [      ] Smoothing ratio (MRC and MULTISCALE)
    CLEAN_SPEEDY     [      ] Speed-up factor (CLARK)
    CLEAN_WORRY      [      ] Worry factor (MULTISCALE)

\end{verbatim}
\subsubsection{CLEAN CLEAN\_ARES}
\index{CLEAN!CLEAN\_ARES}
\begin{verbatim}

    This is the minimal flux in the dirty map that the program will consider
    as  significant.   Alternatively,  the  threshold  can be specified as a
    fraction of the peak flux using CLEAN_FRES.  Once this  level  has  been
    reached the program stops subtracting, and starts the restoration phase.
    The unit for this parameter is the map unit  (typically  Jy/Beam).   The
    parameter  should  usually  be of the order of magnitude of the expected
    noise in the clean map.

    If 0, CLEAN_FRES will be used instead. If all of CLEAN_NITER, CLEAN_ARES
    and CLEAN_FRES are 0, an absolute residual equal to the noise level will
    be used for CLEAN_ARES.

    Short form is ARES.

\end{verbatim}
\subsubsection{CLEAN CLEAN\_FRES}
\index{CLEAN!CLEAN\_FRES}
\begin{verbatim}

    This is the minimal fraction of the peak flux in the dirty map that  the
    program  will  consider  as  significant.   Alternatively,  an  absolute
    threshold can be specified using CLEAN_ARES.  Once this level  has  been
    reached the program stops subtracting, and starts the restoration phase.
    This parameter is normalized to 1 (neither in % nor in db).   It  should
    usually  be of the order of magnitude of the inverse of the expected dy-
    namic range of the intensity.

    If 0, CLEAN_ARES will be used instead. If all of CLEAN_NITER, CLEAN_ARES
    and CLEAN_FRES are 0, an absolute residual equal to the noise level will
    be used for CLEAN_ARES.

    Short form is FRES.

\end{verbatim}
\subsubsection{CLEAN CLEAN\_GAIN}
\index{CLEAN!CLEAN\_GAIN}
\begin{verbatim}

    This is the gain of the subtraction loop.  It should typically be chosen
    in the range 0.05 and 0.3.  Higher values give faster convergence, while
    lower values give a better restitution of the extended structure. A sen-
    sible default is 0.2.

    Short form is GAIN.

\end{verbatim}
\subsubsection{CLEAN CLEAN\_NITER}
\index{CLEAN!CLEAN\_NITER}
\begin{verbatim}

    This is the maximum number of components the program will accept to sub-
    tract.  Once it has been reached, the  program  starts  the  restoration
    phase.

    If 0, the program will guess a number, based on the image size and maxi-
    mum signal-to-noise  ratio,  and  specified  residual  level  CLEAN_ARES
    and/or CLEAN_FRES.

    Short form is NITER.

\end{verbatim}
\subsubsection{CLEAN CLEAN\_NKEEP}
\index{CLEAN!CLEAN\_NKEEP}
\begin{verbatim}

    This is an integer specifying the minimum number of Clean components be-
    fore testing if Cleaning has converged. The convergence is criterium  is
    a  comparison  of the cumulative flux evolution separated by CLEAN_NKEEP
    components. If th

    IF CLEAN_NKEEP is 0, CLEAN will ignore this convergence  criterium,  and
    continue  clean until the CLEAN_NITER, CLEAN_ARES or CLEAN_FRES criteria
    indicate to stop.

    With CLEAN_NKEEP > 0, CLEAN will explore  the  stability  of  the  total
    clean  flux over the last CLEAN_NKEEP  iterations. For a positive (resp.
    negative) source, if the Clean flux becomes smaller (resp. larger)  than
    the Clean flux CLEAN_NKEEP iterations earlier, CLEAN will stop.

    Using  CLEAN_NKEEP  about  70 is a reasonable value.  Some special cases
    (faint extended sources) may require larger values of CLEAN_NKEEP.

\end{verbatim}
\subsubsection{CLEAN CLEAN\_POSITIVE}
\index{CLEAN!CLEAN\_POSITIVE}
\begin{verbatim}

    The minimum number of positive components before negative ones  are  se-
    lected.

\end{verbatim}
\subsubsection{CLEAN CLEAN\_RESTORE}
\index{CLEAN!CLEAN\_RESTORE}
\begin{verbatim}

      Fraction  of  peak  response of the primary beams coverage under which
    the Sky brightness image is blanked in a Mosaic deconvolution.

    The default is 0.2.

\end{verbatim}
\subsubsection{CLEAN CLEAN\_SEARCH}
\index{CLEAN!CLEAN\_SEARCH}
\begin{verbatim}


      Fraction of peak response of the primary beams coverage  beyond  which
    no Clean component is searched in a Mosaic deconvolution.

    The default is 0.2.

\end{verbatim}
\subsubsection{CLEAN CLEAN\_SIDELOBE}
\index{CLEAN!CLEAN\_SIDELOBE}
\begin{verbatim}

    Minimal  relative  intensity to consider for fitting the syntheized beam
    to obtain the Clean beam parameters (MAJOR, MINOR  and  ANGLE)  when  0.
    The default is 0.35.

    In  case  of  poor UV coverage, CLEAN_SIDELOBE should be higher than the
    maximum sidelobe level to perform a good Gaussian fit. Some particularly
    bad UV coverage may not allow any good fit at all, however.

\end{verbatim}
\subsubsection{CLEAN CLEAN\_NGOAL}
\index{CLEAN!CLEAN\_NGOAL}
\begin{verbatim}

      Number  of clean components to be selected in a Cycle in the ALMA het-
    erogeneous array cleaning method.

\end{verbatim}
\subsubsection{CLEAN CLEAN\_NCYCLE}
\index{CLEAN!CLEAN\_NCYCLE}
\begin{verbatim}

      Maximum number of Major Cycles for the SDI and CLARK methods.

\end{verbatim}
\subsubsection{CLEAN CLEAN\_SMOOTH}
\index{CLEAN!CLEAN\_SMOOTH}
\begin{verbatim}

      Smoothing factor between different scales in the  MRC  and  MULTISCALE
    methods.  The default is sqrt(3).

\end{verbatim}
\subsubsection{CLEAN CLEAN\_SPEEDY}
\index{CLEAN!CLEAN\_SPEEDY}
\begin{verbatim}

      Speed-up factor for the CLARK major cycles. The default is 1.0.  Larg-
    er values may be used, but at the expense of possible  instabilities  of
    the algorithm.

\end{verbatim}
\subsubsection{CLEAN CLEAN\_WORRY}
\index{CLEAN!CLEAN\_WORRY}
\begin{verbatim}

      Worry  factor  in the MULTISCALE method for convergence. It propagates
    the S/N from one iteration to the other, so that if this  S/N  degrades,
    the  method  stops.  Default  is 0 (no propagation, and hence no test on
    S/N).  The value should be < 1.0 in all cases.

\end{verbatim}
\subsubsection{CLEAN CLEAN\_INFLATE}
\index{CLEAN!CLEAN\_INFLATE}
\begin{verbatim}

      Maximum Inflation factor for UV_RESTORE (MULTISCALE method).   If  the
    number  of  true (i.e. pixel based) Clean components found by MULTISCALE
    is larger than CLEAN_INFLATE times the number of compressed (i.e.  those
    with the smoothing factor information) components, expansion of the com-
    pressed components will not be possible, and UV_RESTORE will not be use-
    able.

      A  default  of  50  is in general adequate.  Better solutions might be
    found in the future, and this parameter suppressed.  Apart  from  memory
    usage, this number has no consequence on the algorithm.

\end{verbatim}
\subsubsection{CLEAN METHOD}
\index{CLEAN!METHOD}
\begin{verbatim}

      Method used for the deconvolution. Can be HOGBOM, MULTI, MRC,
      SDI or CLARK.


\end{verbatim}
\subsubsection{CLEAN Old\_Names:}
\index{CLEAN!Old\_Names:}
\begin{verbatim}

      Some  of the CLEAN parameters have kept their old names: MAJOR, MINOR,
    ANGLE (which are also used by  command  FIT)  BLC,  TRC  and  BEAM_PATCH
    (which are seldom used)

      Others  have equivalent short names: ARES, FRES, GAIN, NITER for which
    the CLEAN_ prefix may be omitted.

\end{verbatim}
\subsubsection{CLEAN BLC}
\index{CLEAN!BLC}
\begin{verbatim}

    These are the (pixel) coordinates of  the  Bottom  Left  Corner  of  the
    cleaning  box.   The actual cleaning support will be the intersection of
    the specified window with the inner quarter of the map and with any user
    defined polygon.

\end{verbatim}
\subsubsection{CLEAN TRC}
\index{CLEAN!TRC}
\begin{verbatim}

    These  are the (pixel) coordinates of the Top Right Corner of the clean-
    ing box.  The actual cleaning window will be  the  intersection  of  the
    specified window with the inner quarter of the map and with any user de-
    fined polygon.


\end{verbatim}
\subsubsection{CLEAN MAJOR}
\index{CLEAN!MAJOR}
\begin{verbatim}

    This is the major axis  (FWHP)  in  user  coordinates  of  the  Gaussian
    restoring beam. If 0, the program will fit a Gaussian to the dirty beam.
    We strongly discourage to change the default value of 0.

\end{verbatim}
\subsubsection{CLEAN MINOR}
\index{CLEAN!MINOR}
\begin{verbatim}

    This is the minor axis  (FWHP)  in  user  coordinates  of  the  Gaussian
    restoring beam. If 0, the program will fit a Gaussian to the dirty beam.
    We strongly discourage to change the default value of 0.

\end{verbatim}
\subsubsection{CLEAN ANGLE}
\index{CLEAN!ANGLE}
\begin{verbatim}

    This is the position angle (from North towards East, i.e. anticlockwise)
    of  the  major  axis of the Gaussian restoring beam (in degrees).  If 0,
    the program will fit a Gaussian to the dirty beam. We strongly  discour-
    age to change the default value of 0.

\end{verbatim}
\subsubsection{CLEAN BEAM\_PATCH}
\index{CLEAN!BEAM\_PATCH}
\begin{verbatim}

    The  dirty  beam  patch to be used for the minor cycles in CLARK and MRC
    method.  It should be large enough to avoid doing too many major cycles,
    but  has  practically  no  influence on the result.  This size should be
    specified in pixel units.  Reasonable values are between  N/8  and  N/4,
    where N is the number of map pixels in the same dimension.  If set to N,
    the CLARK algorithm becomes identical to the HOGBOM algorithm.



\end{verbatim}
\subsection{DUMP}
\index{DUMP}
\begin{verbatim}
        [CLEAN\]DUMP [U]

    Dump on screen the control parameters of the different CLEAN  deconvolu-
    tion algorithms, mainly for debugging purpose. "DUMP U" dumps the param-
    eters as input by the user while "DUMP" dumps the parameters really used
    and/or modified by the deconvolution algorithm.

\end{verbatim}
\subsection{FIT}
\index{FIT}
\begin{verbatim}
        [CLEAN\]FIT [Field]

    Fit  the  dirty  beam to obtain the clean beam parameters.  This is done
    automatically by all CLEAN algorithm when needed.  In mosaic  mode,  you
    can specify on which field the fit has to be made, using the first argu-
    ment.

    Fixed values: Clean beam parameters can also be specified by the  users,
    by  setting  the  variables  MAJOR, MINOR and ANGLE to values other than
    their default values (0,0,0). In this case, no  automatic  fit  will  be
    performed.  We strongly discourage the use of those variables, which may
    result in a very improper flux scaling if the beam size  is  inappropri-
    ate. This mode should be reserved to special cases where the beam cannot
    be properly fitted, or to have a circular beam by taking  as  beam  size
    the geometrical mean of the fitted major and minor sizes.

\end{verbatim}
\subsubsection{FIT CLEAN\_SIDELOBE}
\index{FIT!CLEAN\_SIDELOBE}
\begin{verbatim}

    Minimal  relative  intensity to consider for fitting the syntheized beam
    to obtain the Clean beam parameters (MAJOR, MINOR  and  ANGLE)  when  0.
    The default is 0.30.

    In  case  of  poor UV coverage, CLEAN_SIDELOBE should be higher than the
    maximum sidelobe level to perform a good Gaussian fit. Some particularly
    bad UV coverage may not allow any good fit at all, however.

\end{verbatim}
\subsection{HOGBOM}
\index{HOGBOM}
\begin{verbatim}
        [CLEAN\]HOGBOM [FirstPlane [LastPlane]] [/FLUX Fmin Fmax]

    See  HELP  CLEAN  for  the  SIC variables controlling the deeconvolution
    process.

    The simplest CLEAN algorithm, originally developed by Hogbom. In  mosaic
    mode  (See  command  MOSAIC), a mosaic clean is performed.  Rings and or
    strips may appear on extended sources. It is slower  than  CLARK  for  a
    single  field  but  maybe  faster  for a mosaic. It is extremely robust.
    Cleaning can be interrupted by pressing C at any time.

    Clean the specified plane interval (default:  planes  between  variables
    FIRST and LAST). If only FirstPlane is specified, Clean only that plane.

    The cumulative, already cleaned flux is displayed in real time in an ad-
    ditional window while cleaning goes on when the /FLUX option is present.
    Parameters of the /FLUX option are then used to give the flux limits for
    this display.

    The user can control the algorithm through SIC variables. New values can
    be given using "LET VARIABLE value". For ease of use, and whenever it is
    possible,  a sensible value of each parameter will automatically be com-
    puted from the context if the value of the corresponding variable is set
    to  its default value, i.e. zero value and empty string. A few variables
    are initialized to "reasonable" values.

        [CLEAN\]CLEAN ?
    Will list all main CLEAN_* variables controlling the  CLEAN  parameters.
    HELP CLEAN Variables will give a more complete list.

\end{verbatim}
\subsubsection{HOGBOM Variables:}
\index{HOGBOM!Variables:}
\begin{verbatim}

    Basic parameters
    CLEAN_GAIN       [       ] Loop gain
    CLEAN_NITER      [       ] Maximum number of clean components
    CLEAN_FRES       [      %] Maximum value of residual (Fraction of peak)
    CLEAN_ARES       [Jy/Beam] Maximum value of residual (Absolute)
    CLEAN_POSITIVE   [       ] Minimum number of positive components at start
    CLEAN_NKEEP      [       ] Min number of components before convergence

    Old names like in MAPPING
    BLC              [  pixel] Bottom left corner of cleaning box
    TRC              [  pixel] Top right corner of cleaning box
    MAJOR            [ arcsec] Clean beam major axis
    MINOR            [ arcsec] Clean beam minor axis
    ANGLE            [ degree] Position angle of clean beam
    BEAM_PATCH       [  pixel] Size of cleaning beam ** not clear **

    Method dependent parameters
    CLEAN_INFLATE    [      ] Maximum Inflation factor for UV_RESTORE (MuLTISCAL
    CLEAN_NCYCLE     [      ] Max number of Major Cycles (SDI & CLARK methods)
    CLEAN_NGOAL      [      ] Max number of comp. in Cycles (ALMA method)
    CLEAN_RESTORE    [      ] Threshold for restoring a Mosaic (def 0.2)
    CLEAN_SEARCH     [      ] Threshold to search Clean Comp. in a Mosaic (def 0
    CLEAN_SIDELOBE   [      ] Min threshold to fit the synthesized beam
    CLEAN_SMOOTH     [      ] Smoothing ratio (MRC and MULTISCALE)
    CLEAN_SPEEDY     [      ] Speed-up factor (CLARK)
    CLEAN_WORRY      [      ] Worry factor (MULTISCALE)
\end{verbatim}
\subsubsection{HOGBOM CLEAN\_ARES}
\index{HOGBOM!CLEAN\_ARES}
\begin{verbatim}

    This is the minimal flux in the dirty map that the program will consider
    as significant.  Alternatively, the threshold  can  be  specified  as  a
    fraction  of  the  peak flux using CLEAN_FRES.  Once this level has been
    reached the program stops subtracting, and starts the restoration phase.
    The  unit  for  this parameter is the map unit (typically Jy/Beam).  The
    parameter should usually be of the order of magnitude  of  the  expected
    noise in the clean map.

    If 0, CLEAN_FRES will be used instead. If all of CLEAN_NITER, CLEAN_ARES
    and CLEAN_FRES are 0, an absolute residual equal to the noise level will
    be used for CLEAN_ARES.

    Short form is ARES.

\end{verbatim}
\subsubsection{HOGBOM CLEAN\_FRES}
\index{HOGBOM!CLEAN\_FRES}
\begin{verbatim}

    This  is the minimal fraction of the peak flux in the dirty map that the
    program  will  consider  as  significant.   Alternatively,  an  absolute
    threshold  can  be specified using CLEAN_ARES.  Once this level has been
    reached the program stops subtracting, and starts the restoration phase.
    This  parameter  is normalized to 1 (neither in % nor in db).  It should
    usually be of the order of magnitude of the inverse of the expected  dy-
    namic range of the intensity.

    If 0, CLEAN_ARES will be used instead. If all of CLEAN_NITER, CLEAN_ARES
    and CLEAN_FRES are 0, an absolute residual equal to the noise level will
    be used for CLEAN_ARES.

    Short form is FRES.

\end{verbatim}
\subsubsection{HOGBOM CLEAN\_GAIN}
\index{HOGBOM!CLEAN\_GAIN}
\begin{verbatim}

    This is the gain of the subtraction loop.  It should typically be chosen
    in the range 0.05 and 0.3.  Higher values give faster convergence, while
    lower values give a better restitution of the extended structure. A sen-
    sible default is 0.2.

    Short form is GAIN.

\end{verbatim}
\subsubsection{HOGBOM CLEAN\_NITER}
\index{HOGBOM!CLEAN\_NITER}
\begin{verbatim}

    This is the maximum number of components the program will accept to sub-
    tract.   Once  it  has  been reached, the program starts the restoration
    phase.

    If 0, the program will guess a number, based on the image size and maxi-
    mum  signal-to-noise  ratio,  and  specified  residual  level CLEAN_ARES
    and/or CLEAN_FRES.

    Short form is NITER.

\end{verbatim}
\subsubsection{HOGBOM CLEAN\_NKEEP}
\index{HOGBOM!CLEAN\_NKEEP}
\begin{verbatim}

    This is an integer specifying the minimum number of Clean components be-
    fore  testing if Cleaning has converged. The convergence is criterium is
    a comparison of the cumulative flux evolution separated  by  CLEAN_NKEEP
    components. If th

    IF  CLEAN_NKEEP  is 0, CLEAN will ignore this convergence criterium, and
    continue clean until the CLEAN_NITER, CLEAN_ARES or CLEAN_FRES  criteria
    indicate to stop.

    With  CLEAN_NKEEP  >  0,  CLEAN  will explore the stability of the total
    clean flux over the last CLEAN_NKEEP  iterations. For a positive  (resp.
    negative)  source, if the Clean flux becomes smaller (resp. larger) than
    the Clean flux CLEAN_NKEEP iterations earlier, CLEAN will stop.

    Using CLEAN_NKEEP about 70 is a reasonable value.   Some  special  cases
    (faint extended sources) may require larger values of CLEAN_NKEEP.

\end{verbatim}
\subsubsection{HOGBOM CLEAN\_POSITIVE}
\index{HOGBOM!CLEAN\_POSITIVE}
\begin{verbatim}

    The  minimum  number of positive components before negative ones are se-
    lected.

\end{verbatim}
\subsubsection{HOGBOM CLEAN\_RESTORE}
\index{HOGBOM!CLEAN\_RESTORE}
\begin{verbatim}

      Fraction of peak response of the primary beams  coverage  under  which
    the Sky brightness image is blanked in a Mosaic deconvolution.

    The default is 0.2.

\end{verbatim}
\subsubsection{HOGBOM CLEAN\_SEARCH}
\index{HOGBOM!CLEAN\_SEARCH}
\begin{verbatim}


      Fraction  of  peak response of the primary beams coverage beyond which
    no Clean component is searched in a Mosaic deconvolution.

    The default is 0.2.

\end{verbatim}
\subsubsection{HOGBOM CLEAN\_SIDELOBE}
\index{HOGBOM!CLEAN\_SIDELOBE}
\begin{verbatim}

    Minimal relative intensity to consider for fitting the  syntheized  beam
    to  obtain  the  Clean  beam parameters (MAJOR, MINOR and ANGLE) when 0.
    The default is 0.35.

    In case of poor UV coverage, CLEAN_SIDELOBE should be  higher  than  the
    maximum sidelobe level to perform a good Gaussian fit. Some particularly
    bad UV coverage may not allow any good fit at all, however.

\end{verbatim}
\subsubsection{HOGBOM CLEAN\_NGOAL}
\index{HOGBOM!CLEAN\_NGOAL}
\begin{verbatim}

      Number of clean components to be selected in a Cycle in the ALMA  het-
    erogeneous array cleaning method.

\end{verbatim}
\subsubsection{HOGBOM CLEAN\_NCYCLE}
\index{HOGBOM!CLEAN\_NCYCLE}
\begin{verbatim}

      Maximum number of Major Cycles for the SDI and CLARK methods.

\end{verbatim}
\subsubsection{HOGBOM CLEAN\_SMOOTH}
\index{HOGBOM!CLEAN\_SMOOTH}
\begin{verbatim}

      Smoothing  factor  between  different scales in the MRC and MULTISCALE
    methods.  The default is sqrt(3).

\end{verbatim}
\subsubsection{HOGBOM CLEAN\_SPEEDY}
\index{HOGBOM!CLEAN\_SPEEDY}
\begin{verbatim}

      Speed-up factor for the CLARK major cycles. The default is 1.0.  Larg-
    er  values  may be used, but at the expense of possible instabilities of
    the algorithm.

\end{verbatim}
\subsubsection{HOGBOM CLEAN\_WORRY}
\index{HOGBOM!CLEAN\_WORRY}
\begin{verbatim}

      Worry factor in the MULTISCALE method for convergence.  It  propagates
    the  S/N  from one iteration to the other, so that if this S/N degrades,
    the method stops. Default is 0 (no propagation, and  hence  no  test  on
    S/N).  The value should be < 1.0 in all cases.

\end{verbatim}
\subsubsection{HOGBOM CLEAN\_INFLATE}
\index{HOGBOM!CLEAN\_INFLATE}
\begin{verbatim}

      Maximum  Inflation  factor for UV_RESTORE (MULTISCALE method).  If the
    number of true (i.e. pixel based) Clean components found  by  MULTISCALE
    is  larger than CLEAN_INFLATE times the number of compressed (i.e. those
    with the smoothing factor information) components, expansion of the com-
    pressed components will not be possible, and UV_RESTORE will not be use-
    able.

      A default of 50 is in general adequate.   Better  solutions  might  be
    found  in  the  future, and this parameter suppressed. Apart from memory
    usage, this number has no consequence on the algorithm.

\end{verbatim}
\subsubsection{HOGBOM METHOD}
\index{HOGBOM!METHOD}
\begin{verbatim}

      Method used for the deconvolution. Can be HOGBOM, MULTI, MRC,
      SDI or CLARK.


\end{verbatim}
\subsubsection{HOGBOM Old\_Names:}
\index{HOGBOM!Old\_Names:}
\begin{verbatim}

      Some of the CLEAN parameters have kept their old names: MAJOR,  MINOR,
    ANGLE  (which  are  also  used  by  command FIT) BLC, TRC and BEAM_PATCH
    (which are seldom used)

      Others have equivalent short names: ARES, FRES, GAIN, NITER for  which
    the CLEAN_ prefix may be omitted.

\end{verbatim}
\subsubsection{HOGBOM BLC}
\index{HOGBOM!BLC}
\begin{verbatim}

    These  are  the  (pixel)  coordinates  of  the Bottom Left Corner of the
    cleaning box.  The actual cleaning support will be the  intersection  of
    the specified window with the inner quarter of the map and with any user
    defined polygon.

\end{verbatim}
\subsubsection{HOGBOM TRC}
\index{HOGBOM!TRC}
\begin{verbatim}

    These are the (pixel) coordinates of the Top Right Corner of the  clean-
    ing  box.   The  actual  cleaning window will be the intersection of the
    specified window with the inner quarter of the map and with any user de-
    fined polygon.


\end{verbatim}
\subsubsection{HOGBOM MAJOR}
\index{HOGBOM!MAJOR}
\begin{verbatim}

    This  is  the  major  axis  (FWHP)  in  user coordinates of the Gaussian
    restoring beam. If 0, the program will fit a Gaussian to the dirty beam.
    We strongly discourage to change the default value of 0.

\end{verbatim}
\subsubsection{HOGBOM MINOR}
\index{HOGBOM!MINOR}
\begin{verbatim}

    This  is  the  minor  axis  (FWHP)  in  user coordinates of the Gaussian
    restoring beam. If 0, the program will fit a Gaussian to the dirty beam.
    We strongly discourage to change the default value of 0.

\end{verbatim}
\subsubsection{HOGBOM ANGLE}
\index{HOGBOM!ANGLE}
\begin{verbatim}

    This is the position angle (from North towards East, i.e. anticlockwise)
    of the major axis of the Gaussian restoring beam (in  degrees).   If  0,
    the  program will fit a Gaussian to the dirty beam. We strongly discour-
    age to change the default value of 0.

\end{verbatim}
\subsubsection{HOGBOM BEAM\_PATCH}
\index{HOGBOM!BEAM\_PATCH}
\begin{verbatim}

    The dirty beam patch to be used for the minor cycles in  CLARK  and  MRC
    method.  It should be large enough to avoid doing too many major cycles,
    but has practically no influence on the result.   This  size  should  be
    specified  in  pixel  units.  Reasonable values are between N/8 and N/4,
    where N is the number of map pixels in the same dimension.  If set to N,
    the CLARK algorithm becomes identical to the HOGBOM algorithm.



\end{verbatim}
\subsection{MAP\_COMPRESS}
\index{MAP\_COMPRESS}
\begin{verbatim}
        [CLEAN\]MAP_COMPRESS WhichOne Nc

    Resample  (in  frequency/velocity)  the  images  (computed  by UV_MAP or
    CLEAN, or loaded by READ WhichOne) by averaging NC adjacent channels.

    WhichOne indicates which image must be compressed (DIRTY,  CLEAN,  SKY).
    WHichOne = * can be used to treat all the possible images.

\end{verbatim}
\subsection{MAP\_INTEGRATE}
\index{MAP\_INTEGRATE}
\begin{verbatim}
        [CLEAN\]MAP_INTEGRATE WhichOne Min Max Type

    Compute  the  integrated  intensity map(s) over the specified range from
    the current image()s (computed by UV_MAP or CLEAN,  or  loaded  by  READ
    WhichOne). Type can be VELOCITY FREQUENCY or CHANNELS.

    WhichOne  indicates  which image must be compressed (DIRTY, CLEAN, SKY).
    WhichOne = * can be used to treat all the possible images.

\end{verbatim}
\subsection{MAP\_RESAMPLE}
\index{MAP\_RESAMPLE}
\begin{verbatim}
        [CLEAN\]MAP_RESAMPLE WhichOne Nc Ref Val Inc

    Resample the images (computed by UV_MAP or CLEAN, or loaded by READ  Wi-
    chOne) on a different velocity scale.
         Nc   new number of channels
         Ref  New reference pixel
         Val  New velocity at reference pixel
         Inc  Velocity increment
    WhichOne  indicates  which  image must be compressed (DIRTY, CLEAN, SKY)
    WHichOne = * can be used to treat all the possible images.

\end{verbatim}
\subsection{SPECIFY}
\index{SPECIFY}
\begin{verbatim}
        [CLEAN\]SPECIFY FREQUENCY|VELOCITY|TELESCOPE Value

        SPECIFY FREQUENCY Value
    Modify the rest frequency and recompute the velocity scale  accordingly.
    Value is the new rest frequency in MHz

        SPECIFY VELOCITY Value
    Modify  the  source  velocity and recompute the rest frequency scale ac-
    cordingly Value is the new velocity in km/s.

        SPECIFY TELESCOPE Name

    Add or Replace the telescope section with the  parameters  (name,  size,
    position)  for the specified telescope name, essentially to get the most
    appropriate beam parameter.

    A Telescope section is required for MOSAIC. The beamsize will depend  on
    telescope diameter and frequency, with a telescope dependent factor. The
    default beam size is 1.13 Lambda/D.

\end{verbatim}
\subsection{MOSAIC}
\index{MOSAIC}
\begin{verbatim}
        [CLEAN\]MOSAIC On|Off

    Turn on or off the mosaic mode for deconvolution. Note that a READ  PRI-
    MARY command, or a UV_MAP command working on a Mosaic UV Table, automat-
    ically switches on the mosaic mode. The program prompt changes to inform
    the user of the current operating mode for deconvolution.

\end{verbatim}
\subsection{MRC}
\index{MRC}
\begin{verbatim}
        [CLEAN\]MRC  [FirstPlane  [LastPlane]]  [/PLOT  Clean|Residu] [/FLUX
    Fmin Fmax] [/QUERY]

    Perform a Multi-Resolution CLEAN on the current dirty image.   MRC  does
    not support mosaics for theoretical reasons.

    Clean  the  specified  plane interval (default: planes between variables
    FIRST and LAST). If only FirstPlane is specified, Clean only that plane.

    If option /PLOT is given, a display of the CLEAN or RESIDUAL map will be
    shown at each major cycle, depending on the argument  (default:  Residu-
    al).  The  user will be prompted for continuation when the /QUERY option
    is present. The cumulative, already cleaned flux is  displayed  in  real
    time  in  an additional window while cleaning goes on when the /FLUX op-
    tion is present. Parameters of the /FLUX option are then  used  to  give
    the  flux  limits  for this display. A summary plot with the Difference,
    Smooth, and total CLEANed maps is also displayed.

    The user can control the algorithm through SIC variables. New values can
    be given using "LET VARIABLE value". For ease of use, and whenever it is
    possible, a sensible value of each parameter will automatically be  com-
    puted from the context if the value of the corresponding variable is set
    to its default value, i.e. zero value and empty string. A few  variables
    are initialized to "reasonable" values.

        [CLEAN\]CLEAN ?
    Will  list  all main CLEAN_* variables controlling the CLEAN parameters.
    HELP CLEAN Variables will give a more complete list.

\end{verbatim}
\subsubsection{MRC Variables:}
\index{MRC!Variables:}
\begin{verbatim}

    Basic parameters
    CLEAN_GAIN       [       ] Loop gain
    CLEAN_NITER      [       ] Maximum number of clean components
    CLEAN_FRES       [      %] Maximum value of residual (Fraction of peak)
    CLEAN_ARES       [Jy/Beam] Maximum value of residual (Absolute)
    CLEAN_POSITIVE   [       ] Minimum number of positive components at start
    CLEAN_NKEEP      [       ] Min number of components before convergence

    Old names like in MAPPING
    BLC              [  pixel] Bottom left corner of cleaning box
    TRC              [  pixel] Top right corner of cleaning box
    MAJOR            [ arcsec] Clean beam major axis
    MINOR            [ arcsec] Clean beam minor axis
    ANGLE            [ degree] Position angle of clean beam
    BEAM_PATCH       [  pixel] Size of cleaning beam ** not clear **

    Method dependent parameters
    CLEAN_INFLATE    [      ] Maximum Inflation factor for UV_RESTORE (MuLTISCAL
    CLEAN_NCYCLE     [      ] Max number of Major Cycles (SDI & CLARK methods)
    CLEAN_NGOAL      [      ] Max number of comp. in Cycles (ALMA method)
    CLEAN_RESTORE    [      ] Threshold for restoring a Mosaic (def 0.2)
    CLEAN_SEARCH     [      ] Threshold to search Clean Comp. in a Mosaic (def 0
    CLEAN_SIDELOBE   [      ] Min threshold to fit the synthesized beam
    CLEAN_SMOOTH     [      ] Smoothing ratio (MRC and MULTISCALE)
    CLEAN_SPEEDY     [      ] Speed-up factor (CLARK)
    CLEAN_WORRY      [      ] Worry factor (MULTISCALE)
    RATIO      [       ] Smoothing factor (default 0: guess, otherwise must be 2
\end{verbatim}
\subsubsection{MRC CLEAN\_ARES}
\index{MRC!CLEAN\_ARES}
\begin{verbatim}

    This is the minimal flux in the dirty map that the program will consider
    as  significant.   Alternatively,  the  threshold  can be specified as a
    fraction of the peak flux using CLEAN_FRES.  Once this  level  has  been
    reached the program stops subtracting, and starts the restoration phase.
    The unit for this parameter is the map unit  (typically  Jy/Beam).   The
    parameter  should  usually  be of the order of magnitude of the expected
    noise in the clean map.

    If 0, CLEAN_FRES will be used instead. If all of CLEAN_NITER, CLEAN_ARES
    and CLEAN_FRES are 0, an absolute residual equal to the noise level will
    be used for CLEAN_ARES.

    Short form is ARES.

\end{verbatim}
\subsubsection{MRC CLEAN\_FRES}
\index{MRC!CLEAN\_FRES}
\begin{verbatim}

    This is the minimal fraction of the peak flux in the dirty map that  the
    program  will  consider  as  significant.   Alternatively,  an  absolute
    threshold can be specified using CLEAN_ARES.  Once this level  has  been
    reached the program stops subtracting, and starts the restoration phase.
    This parameter is normalized to 1 (neither in % nor in db).   It  should
    usually  be of the order of magnitude of the inverse of the expected dy-
    namic range of the intensity.

    If 0, CLEAN_ARES will be used instead. If all of CLEAN_NITER, CLEAN_ARES
    and CLEAN_FRES are 0, an absolute residual equal to the noise level will
    be used for CLEAN_ARES.

    Short form is FRES.

\end{verbatim}
\subsubsection{MRC CLEAN\_GAIN}
\index{MRC!CLEAN\_GAIN}
\begin{verbatim}

    This is the gain of the subtraction loop.  It should typically be chosen
    in the range 0.05 and 0.3.  Higher values give faster convergence, while
    lower values give a better restitution of the extended structure. A sen-
    sible default is 0.2.

    Short form is GAIN.

\end{verbatim}
\subsubsection{MRC CLEAN\_NITER}
\index{MRC!CLEAN\_NITER}
\begin{verbatim}

    This is the maximum number of components the program will accept to sub-
    tract.  Once it has been reached, the  program  starts  the  restoration
    phase.

    If 0, the program will guess a number, based on the image size and maxi-
    mum signal-to-noise  ratio,  and  specified  residual  level  CLEAN_ARES
    and/or CLEAN_FRES.

    Short form is NITER.

\end{verbatim}
\subsubsection{MRC CLEAN\_NKEEP}
\index{MRC!CLEAN\_NKEEP}
\begin{verbatim}

    This is an integer specifying the minimum number of Clean components be-
    fore testing if Cleaning has converged. The convergence is criterium  is
    a  comparison  of the cumulative flux evolution separated by CLEAN_NKEEP
    components. If th

    IF CLEAN_NKEEP is 0, CLEAN will ignore this convergence  criterium,  and
    continue  clean until the CLEAN_NITER, CLEAN_ARES or CLEAN_FRES criteria
    indicate to stop.

    With CLEAN_NKEEP > 0, CLEAN will explore  the  stability  of  the  total
    clean  flux over the last CLEAN_NKEEP  iterations. For a positive (resp.
    negative) source, if the Clean flux becomes smaller (resp. larger)  than
    the Clean flux CLEAN_NKEEP iterations earlier, CLEAN will stop.

    Using  CLEAN_NKEEP  about  70 is a reasonable value.  Some special cases
    (faint extended sources) may require larger values of CLEAN_NKEEP.

\end{verbatim}
\subsubsection{MRC CLEAN\_POSITIVE}
\index{MRC!CLEAN\_POSITIVE}
\begin{verbatim}

    The minimum number of positive components before negative ones  are  se-
    lected.

\end{verbatim}
\subsubsection{MRC CLEAN\_RESTORE}
\index{MRC!CLEAN\_RESTORE}
\begin{verbatim}

      Fraction  of  peak  response of the primary beams coverage under which
    the Sky brightness image is blanked in a Mosaic deconvolution.

    The default is 0.2.

\end{verbatim}
\subsubsection{MRC CLEAN\_SEARCH}
\index{MRC!CLEAN\_SEARCH}
\begin{verbatim}


      Fraction of peak response of the primary beams coverage  beyond  which
    no Clean component is searched in a Mosaic deconvolution.

    The default is 0.2.

\end{verbatim}
\subsubsection{MRC CLEAN\_SIDELOBE}
\index{MRC!CLEAN\_SIDELOBE}
\begin{verbatim}

    Minimal  relative  intensity to consider for fitting the syntheized beam
    to obtain the Clean beam parameters (MAJOR, MINOR  and  ANGLE)  when  0.
    The default is 0.35.

    In  case  of  poor UV coverage, CLEAN_SIDELOBE should be higher than the
    maximum sidelobe level to perform a good Gaussian fit. Some particularly
    bad UV coverage may not allow any good fit at all, however.

\end{verbatim}
\subsubsection{MRC CLEAN\_NGOAL}
\index{MRC!CLEAN\_NGOAL}
\begin{verbatim}

      Number  of clean components to be selected in a Cycle in the ALMA het-
    erogeneous array cleaning method.

\end{verbatim}
\subsubsection{MRC CLEAN\_NCYCLE}
\index{MRC!CLEAN\_NCYCLE}
\begin{verbatim}

      Maximum number of Major Cycles for the SDI and CLARK methods.

\end{verbatim}
\subsubsection{MRC CLEAN\_SMOOTH}
\index{MRC!CLEAN\_SMOOTH}
\begin{verbatim}

      Smoothing factor between different scales in the  MRC  and  MULTISCALE
    methods.  The default is sqrt(3).

\end{verbatim}
\subsubsection{MRC CLEAN\_SPEEDY}
\index{MRC!CLEAN\_SPEEDY}
\begin{verbatim}

      Speed-up factor for the CLARK major cycles. The default is 1.0.  Larg-
    er values may be used, but at the expense of possible  instabilities  of
    the algorithm.

\end{verbatim}
\subsubsection{MRC CLEAN\_WORRY}
\index{MRC!CLEAN\_WORRY}
\begin{verbatim}

      Worry  factor  in the MULTISCALE method for convergence. It propagates
    the S/N from one iteration to the other, so that if this  S/N  degrades,
    the  method  stops.  Default  is 0 (no propagation, and hence no test on
    S/N).  The value should be < 1.0 in all cases.

\end{verbatim}
\subsubsection{MRC CLEAN\_INFLATE}
\index{MRC!CLEAN\_INFLATE}
\begin{verbatim}

      Maximum Inflation factor for UV_RESTORE (MULTISCALE method).   If  the
    number  of  true (i.e. pixel based) Clean components found by MULTISCALE
    is larger than CLEAN_INFLATE times the number of compressed (i.e.  those
    with the smoothing factor information) components, expansion of the com-
    pressed components will not be possible, and UV_RESTORE will not be use-
    able.

      A  default  of  50  is in general adequate.  Better solutions might be
    found in the future, and this parameter suppressed.  Apart  from  memory
    usage, this number has no consequence on the algorithm.

\end{verbatim}
\subsubsection{MRC METHOD}
\index{MRC!METHOD}
\begin{verbatim}

      Method used for the deconvolution. Can be HOGBOM, MULTI, MRC,
      SDI or CLARK.


\end{verbatim}
\subsubsection{MRC Old\_Names:}
\index{MRC!Old\_Names:}
\begin{verbatim}

      Some  of the CLEAN parameters have kept their old names: MAJOR, MINOR,
    ANGLE (which are also used by  command  FIT)  BLC,  TRC  and  BEAM_PATCH
    (which are seldom used)

      Others  have equivalent short names: ARES, FRES, GAIN, NITER for which
    the CLEAN_ prefix may be omitted.

\end{verbatim}
\subsubsection{MRC BLC}
\index{MRC!BLC}
\begin{verbatim}

    These are the (pixel) coordinates of  the  Bottom  Left  Corner  of  the
    cleaning  box.   The actual cleaning support will be the intersection of
    the specified window with the inner quarter of the map and with any user
    defined polygon.

\end{verbatim}
\subsubsection{MRC TRC}
\index{MRC!TRC}
\begin{verbatim}

    These  are the (pixel) coordinates of the Top Right Corner of the clean-
    ing box.  The actual cleaning window will be  the  intersection  of  the
    specified window with the inner quarter of the map and with any user de-
    fined polygon.


\end{verbatim}
\subsubsection{MRC MAJOR}
\index{MRC!MAJOR}
\begin{verbatim}

    This is the major axis  (FWHP)  in  user  coordinates  of  the  Gaussian
    restoring beam. If 0, the program will fit a Gaussian to the dirty beam.
    We strongly discourage to change the default value of 0.

\end{verbatim}
\subsubsection{MRC MINOR}
\index{MRC!MINOR}
\begin{verbatim}

    This is the minor axis  (FWHP)  in  user  coordinates  of  the  Gaussian
    restoring beam. If 0, the program will fit a Gaussian to the dirty beam.
    We strongly discourage to change the default value of 0.

\end{verbatim}
\subsubsection{MRC ANGLE}
\index{MRC!ANGLE}
\begin{verbatim}

    This is the position angle (from North towards East, i.e. anticlockwise)
    of  the  major  axis of the Gaussian restoring beam (in degrees).  If 0,
    the program will fit a Gaussian to the dirty beam. We strongly  discour-
    age to change the default value of 0.

\end{verbatim}
\subsubsection{MRC BEAM\_PATCH}
\index{MRC!BEAM\_PATCH}
\begin{verbatim}

    The  dirty  beam  patch to be used for the minor cycles in CLARK and MRC
    method.  It should be large enough to avoid doing too many major cycles,
    but  has  practically  no  influence on the result.  This size should be
    specified in pixel units.  Reasonable values are between  N/8  and  N/4,
    where N is the number of map pixels in the same dimension.  If set to N,
    the CLARK algorithm becomes identical to the HOGBOM algorithm.



\end{verbatim}
\subsubsection{MRC RATIO}
\index{MRC!RATIO}
\begin{verbatim}

    Used smoothing factor, which must be a power of 2. The default is 0,  to
    indicate that the actual value must be estimated from the image size.

\end{verbatim}
\subsection{MULTI}
\index{MULTI}
\begin{verbatim}
        [CLEAN\]MULTI [FirstPlane [LastPlane]] [/FLUX Fmin Fmax]

    Multiscale CLEAN algorithm.

    Clean  the  specified  plane interval (default: Planes between variables
    FIRST and LAST). The cumulative, already cleaned flux  is  displayed  in
    real  time in an additional window while cleaning goes on when the /FLUX
    option is present. Parameters of the /FLUX option are then used to  give
    the flux limits for this display.

    MULTI does not yet work on mosaics.

    The user can control the algorithm through SIC variables. New values can
    be given using "LET VARIABLE value". For ease of use, and whenever it is
    possible,  a sensible value of each parameter will automatically be com-
    puted from the context if the value of the corresponding variable is set
    to  its default value, i.e. zero value and empty string. A few variables
    are initialized to "reasonable" values.

        [CLEAN\]CLEAN ?
    Will list all main CLEAN_* variables controlling the  CLEAN  parameters.
    HELP CLEAN Variables will give a more complete list.
\end{verbatim}
\subsubsection{MULTI Variables:}
\index{MULTI!Variables:}
\begin{verbatim}

    Basic parameters
    CLEAN_GAIN       [       ] Loop gain
    CLEAN_NITER      [       ] Maximum number of clean components
    CLEAN_FRES       [      %] Maximum value of residual (Fraction of peak)
    CLEAN_ARES       [Jy/Beam] Maximum value of residual (Absolute)
    CLEAN_POSITIVE   [       ] Minimum number of positive components at start
    CLEAN_NKEEP      [       ] Min number of components before convergence

    Old names like in MAPPING
    BLC              [  pixel] Bottom left corner of cleaning box
    TRC              [  pixel] Top right corner of cleaning box
    MAJOR            [ arcsec] Clean beam major axis
    MINOR            [ arcsec] Clean beam minor axis
    ANGLE            [ degree] Position angle of clean beam
    BEAM_PATCH       [  pixel] Size of cleaning beam ** not clear **

    Method dependent parameters
    CLEAN_INFLATE    [      ] Maximum Inflation factor for UV_RESTORE (MuLTISCAL
    CLEAN_NCYCLE     [      ] Max number of Major Cycles (SDI & CLARK methods)
    CLEAN_NGOAL      [      ] Max number of comp. in Cycles (ALMA method)
    CLEAN_RESTORE    [      ] Threshold for restoring a Mosaic (def 0.2)
    CLEAN_SEARCH     [      ] Threshold to search Clean Comp. in a Mosaic (def 0
    CLEAN_SIDELOBE   [      ] Min threshold to fit the synthesized beam
    CLEAN_SMOOTH     [      ] Smoothing ratio (MRC and MULTISCALE)
    CLEAN_SPEEDY     [      ] Speed-up factor (CLARK)
    CLEAN_WORRY      [      ] Worry factor (MULTISCALE)
    CLEAN_SMOOTH     [       ] Smoothing factor (default sqrt(3))
\end{verbatim}
\subsubsection{MULTI CLEAN\_ARES}
\index{MULTI!CLEAN\_ARES}
\begin{verbatim}

    This is the minimal flux in the dirty map that the program will consider
    as significant.  Alternatively, the threshold  can  be  specified  as  a
    fraction  of  the  peak flux using CLEAN_FRES.  Once this level has been
    reached the program stops subtracting, and starts the restoration phase.
    The  unit  for  this parameter is the map unit (typically Jy/Beam).  The
    parameter should usually be of the order of magnitude  of  the  expected
    noise in the clean map.

    If 0, CLEAN_FRES will be used instead. If all of CLEAN_NITER, CLEAN_ARES
    and CLEAN_FRES are 0, an absolute residual equal to the noise level will
    be used for CLEAN_ARES.

    Short form is ARES.

\end{verbatim}
\subsubsection{MULTI CLEAN\_FRES}
\index{MULTI!CLEAN\_FRES}
\begin{verbatim}

    This  is the minimal fraction of the peak flux in the dirty map that the
    program  will  consider  as  significant.   Alternatively,  an  absolute
    threshold  can  be specified using CLEAN_ARES.  Once this level has been
    reached the program stops subtracting, and starts the restoration phase.
    This  parameter  is normalized to 1 (neither in % nor in db).  It should
    usually be of the order of magnitude of the inverse of the expected  dy-
    namic range of the intensity.

    If 0, CLEAN_ARES will be used instead. If all of CLEAN_NITER, CLEAN_ARES
    and CLEAN_FRES are 0, an absolute residual equal to the noise level will
    be used for CLEAN_ARES.

    Short form is FRES.

\end{verbatim}
\subsubsection{MULTI CLEAN\_GAIN}
\index{MULTI!CLEAN\_GAIN}
\begin{verbatim}

    This is the gain of the subtraction loop.  It should typically be chosen
    in the range 0.05 and 0.3.  Higher values give faster convergence, while
    lower values give a better restitution of the extended structure. A sen-
    sible default is 0.2.

    Short form is GAIN.

\end{verbatim}
\subsubsection{MULTI CLEAN\_NITER}
\index{MULTI!CLEAN\_NITER}
\begin{verbatim}

    This is the maximum number of components the program will accept to sub-
    tract.   Once  it  has  been reached, the program starts the restoration
    phase.

    If 0, the program will guess a number, based on the image size and maxi-
    mum  signal-to-noise  ratio,  and  specified  residual  level CLEAN_ARES
    and/or CLEAN_FRES.

    Short form is NITER.

\end{verbatim}
\subsubsection{MULTI CLEAN\_NKEEP}
\index{MULTI!CLEAN\_NKEEP}
\begin{verbatim}

    This is an integer specifying the minimum number of Clean components be-
    fore  testing if Cleaning has converged. The convergence is criterium is
    a comparison of the cumulative flux evolution separated  by  CLEAN_NKEEP
    components. If th

    IF  CLEAN_NKEEP  is 0, CLEAN will ignore this convergence criterium, and
    continue clean until the CLEAN_NITER, CLEAN_ARES or CLEAN_FRES  criteria
    indicate to stop.

    With  CLEAN_NKEEP  >  0,  CLEAN  will explore the stability of the total
    clean flux over the last CLEAN_NKEEP  iterations. For a positive  (resp.
    negative)  source, if the Clean flux becomes smaller (resp. larger) than
    the Clean flux CLEAN_NKEEP iterations earlier, CLEAN will stop.

    Using CLEAN_NKEEP about 70 is a reasonable value.   Some  special  cases
    (faint extended sources) may require larger values of CLEAN_NKEEP.

\end{verbatim}
\subsubsection{MULTI CLEAN\_POSITIVE}
\index{MULTI!CLEAN\_POSITIVE}
\begin{verbatim}

    The  minimum  number of positive components before negative ones are se-
    lected.

\end{verbatim}
\subsubsection{MULTI CLEAN\_RESTORE}
\index{MULTI!CLEAN\_RESTORE}
\begin{verbatim}

      Fraction of peak response of the primary beams  coverage  under  which
    the Sky brightness image is blanked in a Mosaic deconvolution.

    The default is 0.2.

\end{verbatim}
\subsubsection{MULTI CLEAN\_SEARCH}
\index{MULTI!CLEAN\_SEARCH}
\begin{verbatim}


      Fraction  of  peak response of the primary beams coverage beyond which
    no Clean component is searched in a Mosaic deconvolution.

    The default is 0.2.

\end{verbatim}
\subsubsection{MULTI CLEAN\_SIDELOBE}
\index{MULTI!CLEAN\_SIDELOBE}
\begin{verbatim}

    Minimal relative intensity to consider for fitting the  syntheized  beam
    to  obtain  the  Clean  beam parameters (MAJOR, MINOR and ANGLE) when 0.
    The default is 0.35.

    In case of poor UV coverage, CLEAN_SIDELOBE should be  higher  than  the
    maximum sidelobe level to perform a good Gaussian fit. Some particularly
    bad UV coverage may not allow any good fit at all, however.

\end{verbatim}
\subsubsection{MULTI CLEAN\_NGOAL}
\index{MULTI!CLEAN\_NGOAL}
\begin{verbatim}

      Number of clean components to be selected in a Cycle in the ALMA  het-
    erogeneous array cleaning method.

\end{verbatim}
\subsubsection{MULTI CLEAN\_NCYCLE}
\index{MULTI!CLEAN\_NCYCLE}
\begin{verbatim}

      Maximum number of Major Cycles for the SDI and CLARK methods.

\end{verbatim}
\subsubsection{MULTI CLEAN\_SMOOTH}
\index{MULTI!CLEAN\_SMOOTH}
\begin{verbatim}

      Smoothing  factor  between  different scales in the MRC and MULTISCALE
    methods.  The default is sqrt(3).

\end{verbatim}
\subsubsection{MULTI CLEAN\_SPEEDY}
\index{MULTI!CLEAN\_SPEEDY}
\begin{verbatim}

      Speed-up factor for the CLARK major cycles. The default is 1.0.  Larg-
    er  values  may be used, but at the expense of possible instabilities of
    the algorithm.

\end{verbatim}
\subsubsection{MULTI CLEAN\_WORRY}
\index{MULTI!CLEAN\_WORRY}
\begin{verbatim}

      Worry factor in the MULTISCALE method for convergence.  It  propagates
    the  S/N  from one iteration to the other, so that if this S/N degrades,
    the method stops. Default is 0 (no propagation, and  hence  no  test  on
    S/N).  The value should be < 1.0 in all cases.

\end{verbatim}
\subsubsection{MULTI CLEAN\_INFLATE}
\index{MULTI!CLEAN\_INFLATE}
\begin{verbatim}

      Maximum  Inflation  factor for UV_RESTORE (MULTISCALE method).  If the
    number of true (i.e. pixel based) Clean components found  by  MULTISCALE
    is  larger than CLEAN_INFLATE times the number of compressed (i.e. those
    with the smoothing factor information) components, expansion of the com-
    pressed components will not be possible, and UV_RESTORE will not be use-
    able.

      A default of 50 is in general adequate.   Better  solutions  might  be
    found  in  the  future, and this parameter suppressed. Apart from memory
    usage, this number has no consequence on the algorithm.

\end{verbatim}
\subsubsection{MULTI METHOD}
\index{MULTI!METHOD}
\begin{verbatim}

      Method used for the deconvolution. Can be HOGBOM, MULTI, MRC,
      SDI or CLARK.


\end{verbatim}
\subsubsection{MULTI Old\_Names:}
\index{MULTI!Old\_Names:}
\begin{verbatim}

      Some of the CLEAN parameters have kept their old names: MAJOR,  MINOR,
    ANGLE  (which  are  also  used  by  command FIT) BLC, TRC and BEAM_PATCH
    (which are seldom used)

      Others have equivalent short names: ARES, FRES, GAIN, NITER for  which
    the CLEAN_ prefix may be omitted.

\end{verbatim}
\subsubsection{MULTI BLC}
\index{MULTI!BLC}
\begin{verbatim}

    These  are  the  (pixel)  coordinates  of  the Bottom Left Corner of the
    cleaning box.  The actual cleaning support will be the  intersection  of
    the specified window with the inner quarter of the map and with any user
    defined polygon.

\end{verbatim}
\subsubsection{MULTI TRC}
\index{MULTI!TRC}
\begin{verbatim}

    These are the (pixel) coordinates of the Top Right Corner of the  clean-
    ing  box.   The  actual  cleaning window will be the intersection of the
    specified window with the inner quarter of the map and with any user de-
    fined polygon.


\end{verbatim}
\subsubsection{MULTI MAJOR}
\index{MULTI!MAJOR}
\begin{verbatim}

    This  is  the  major  axis  (FWHP)  in  user coordinates of the Gaussian
    restoring beam. If 0, the program will fit a Gaussian to the dirty beam.
    We strongly discourage to change the default value of 0.

\end{verbatim}
\subsubsection{MULTI MINOR}
\index{MULTI!MINOR}
\begin{verbatim}

    This  is  the  minor  axis  (FWHP)  in  user coordinates of the Gaussian
    restoring beam. If 0, the program will fit a Gaussian to the dirty beam.
    We strongly discourage to change the default value of 0.

\end{verbatim}
\subsubsection{MULTI ANGLE}
\index{MULTI!ANGLE}
\begin{verbatim}

    This is the position angle (from North towards East, i.e. anticlockwise)
    of the major axis of the Gaussian restoring beam (in  degrees).   If  0,
    the  program will fit a Gaussian to the dirty beam. We strongly discour-
    age to change the default value of 0.

\end{verbatim}
\subsubsection{MULTI BEAM\_PATCH}
\index{MULTI!BEAM\_PATCH}
\begin{verbatim}

    The dirty beam patch to be used for the minor cycles in  CLARK  and  MRC
    method.  It should be large enough to avoid doing too many major cycles,
    but has practically no influence on the result.   This  size  should  be
    specified  in  pixel  units.  Reasonable values are between N/8 and N/4,
    where N is the number of map pixels in the same dimension.  If set to N,
    the CLARK algorithm becomes identical to the HOGBOM algorithm.



\end{verbatim}
\subsubsection{MULTI SMOOTH}
\index{MULTI!SMOOTH}
\begin{verbatim}

    Smoothing  ratio  between  the different scales. The default is sqrt(3),
    but larger values should be used for large images with wide spatial  dy-
    namic range.

\end{verbatim}
\subsection{MX}
\index{MX}
\begin{verbatim}
        [CLEAN\]MX [FirstPlane [LastPlane]] [/PLOT Clean|Residu] [/FLUX Fmin
    Fmax] [/QUERY]

    Make and deconvolve maps starting from a UV table.  It  combines  UV_MAP
    and CLEAN in a single step.

    The mapping process is identical to UV_MAP.  It makes a map from UV data
    by griding the UV data using a convolving function, and then Fast Fouri-
    er  Transforming the individual channels.  However, MX always produces a
    single beam for all channels, thus neglecting frequency  change  between
    channels.   MX  enables to shift the map center and rotate the image, by
    shifting the phase tracking center and rotating the  UV  coordinates  of
    the input UV table.

    The  CLEAN  algorithm is similar to the CLARK method, but with major cy-
    cles operating directly on the ungridded UV table rather than in the im-
    age  plane. Accordingly, aliasing affects only of the residuals, not the
    clean components. It is thus more accurate but also slower than CLARK as
    it  asks  for the gridding step at each major cycle.  MX also shares the
    same limitation as CLARK on large sidelobes.

    The user can control the algorithm through SIC variables. New values can
    be given using "LET VARIABLE value". For ease of use, and whenever it is
    possible, a sensible value of each parameter will automatically be  com-
    puted from the context if the value of the corresponding variable is set
    to its default value, i.e. zero value and empty string. A few  variables
    are initialized to "reasonable" values.

        [CLEAN\]CLEAN ?
    Will  list  all main CLEAN_* variables controlling the CLEAN parameters.
    HELP CLEAN Variables will give a more complete list.

                [CLEAN\]UV_MAP ?
        Will list all MAP_* variables controlling the UV_MAP parameters.

    The list of control variables is (by  alphabetic  order,  with  the  old
    names used by Mapping on the right)
    New names       [   unit]       -- Description --    % Old Name
    MAP_BEAM_STEP   [       ]  Number of channels per single dirty beam
    MAP_CELL        [ arcsec]  Image pixel size
    MAP_CENTER      [ string]  RA, Dec of map center, and Position Angle
    MAP_CONVOLUTION [       ]  Convolution function    % CONVOLUTION
    MAP_FIELD       [ arcsec]  Map field of view
    MAP_POWER       [       ]  Maximum exponent of 3 and 5 allowed in MAP_SIZE
    MAP_PRECIS      [       ]  Fraction of pixel tolerance on beam matching
    MAP_ROBUST      [       ]  Robustness factor        % UV_CELL[2]
    MAP_ROUNDING    [       ]  Precision of MAP_SIZE
    MAP_SIZE        [       ]  Number of pixels
    MAP_TAPEREXPO   [       ]  Taper exponent           % TAPER_EXPO
    MAP_TRUNCATE    [      %]  Mosaic truncation level
    MAP_UVCELL      [      m]  UV cell size             % UV_CELL[1]
    MAP_UVTAPER     [m,m,deg]  Gaussian taper           % UV_TAPER
    MAP_VERSION     [       ]  Code version (0 new, -1 old)

    NAME is no longer used, and WEIGHT_MODE is obsolete.
    MAP_RA          [  hours]  RA of map center
    MAP_DEC         [    deg]  Dec of map center
    MAP_ANGLE       [    deg]  Map position angle
    MAP_SHIFT       [Yes/No ]  Shift phase center
    are  obsolescent,  superseded  by MAP_CENTER. They are provided only for
    compatibility with older scripts.
\end{verbatim}
\subsubsection{MX Variables:}
\index{MX!Variables:}
\begin{verbatim}

    Basic parameters
    CLEAN_GAIN       [       ] Loop gain
    CLEAN_NITER      [       ] Maximum number of clean components
    CLEAN_FRES       [      %] Maximum value of residual (Fraction of peak)
    CLEAN_ARES       [Jy/Beam] Maximum value of residual (Absolute)
    CLEAN_POSITIVE   [       ] Minimum number of positive components at start
    CLEAN_NKEEP      [       ] Min number of components before convergence

    Old names like in MAPPING
    BLC              [  pixel] Bottom left corner of cleaning box
    TRC              [  pixel] Top right corner of cleaning box
    MAJOR            [ arcsec] Clean beam major axis
    MINOR            [ arcsec] Clean beam minor axis
    ANGLE            [ degree] Position angle of clean beam
    BEAM_PATCH       [  pixel] Size of cleaning beam ** not clear **

    Method dependent parameters
    CLEAN_INFLATE    [      ] Maximum Inflation factor for UV_RESTORE (MuLTISCAL
    CLEAN_NCYCLE     [      ] Max number of Major Cycles (SDI & CLARK methods)
    CLEAN_NGOAL      [      ] Max number of comp. in Cycles (ALMA method)
    CLEAN_RESTORE    [      ] Threshold for restoring a Mosaic (def 0.2)
    CLEAN_SEARCH     [      ] Threshold to search Clean Comp. in a Mosaic (def 0
    CLEAN_SIDELOBE   [      ] Min threshold to fit the synthesized beam
    CLEAN_SMOOTH     [      ] Smoothing ratio (MRC and MULTISCALE)
    CLEAN_SPEEDY     [      ] Speed-up factor (CLARK)
    CLEAN_WORRY      [      ] Worry factor (MULTISCALE)
\end{verbatim}
\subsubsection{MX MAP\_BEAM\_STEP}
\index{MX!MAP\_BEAM\_STEP}
\begin{verbatim}

      MAP_BEAM_STEP   Integer

    Number of channels per synthesized beam plane.

    Default is 0, meaning only 1 beam plane for all channels.  N (>0)  indi-
    cates N consecutive channels will share the same dirty beam.

    A  value  of  -1  can be used to compute the number of channels per beam
    plane to ensure the angular scale does not deviate more than a  fraction
    of the map cell at the map edge. This fraction is controlled by variable
    MAP_PRECIS (default 0.1)

\end{verbatim}
\subsubsection{MX MAP\_CELL}
\index{MX!MAP\_CELL}
\begin{verbatim}

          MAP_CELL[2]    Real

    The map pixel size [arcsec]. It is recommended to use  identical  values
    in  X  and Y.  A sampling of at least 3 pixel per beam is recommended to
    ease the deconvolution. Enter 0,0 to let the task find the best  values.

\end{verbatim}
\subsubsection{MX MAP\_CENTER}
\index{MX!MAP\_CENTER}
\begin{verbatim}

          MAP_CENTER     Character String

    Specify  the Map center and orientation in the same way as the arguments
    of UV_MAP.

\end{verbatim}
\subsubsection{MX MAP\_CONVOLUTION}
\index{MX!MAP\_CONVOLUTION}
\begin{verbatim}

        MAP_CONVOLUTION    Integer

    Select the desired convolution function for gridding  in  the  UV  plane
    Choices are
            0    Default (currently 5)
            1    Boxcar
            2    Gaussian
            3    Sin(x)/x
            4    Gaussian * Sin(x)/x
            5    Spheroidal
    Spheroidal  functions  is  the optimal choice. So we strongly discourage
    use of any other convolution function, which are here for tests only.

\end{verbatim}
\subsubsection{MX MAP\_FIELD}
\index{MX!MAP\_FIELD}
\begin{verbatim}

      MAP_FIELD[2]     Real

    Field of view in X and Y in arcsec.  The field  of  view  MAP_FIELD  has
    precedence  over  the number of pixels MAP_SIZE to define the actual map
    size when both are non-zero.

\end{verbatim}
\subsubsection{MX MAP\_POWER}
\index{MX!MAP\_POWER}
\begin{verbatim}

          MAP_POWER[2]     Integer

    Maximum exponent of 3 and 5 allowed  in  automatic  guess  of  MAP_SIZE.
    MAP_SIZE is decomposed in 2^k 3^p 5^q, and p and q must be less or equal
    to MAP_POWER.

    Default is 0: MAP_SIZE is just a power of 2. A value of 1 allows approx-
    imation of any map size to 20 %, while a value of 2 allows 10 % approxi-
    mation. Fast Fourier Transform are slightly slower with powers of 3  and
    5,  but  limiting  the  map  size can gain a lot in the Cleaning process
    (which can scale as MAP_SIZE^4).


\end{verbatim}
\subsubsection{MX MAP\_PRECIS}
\index{MX!MAP\_PRECIS}
\begin{verbatim}

      MAP_PRECIS    Real

    Maximum mismatch in pixel at map edge between the true synthesized  beam
    (which  would  have been computed using the exact channel frequency) and
    the computed synthesized beam with  the mean frequency of  the  channels
    sharing  the same beam. This is used (with the actual image size) to de-
    rive the actual number of channels which can share the same  beam,  i.e.
    the effective value of MAP_BEAM_STEP when MAP_BEAM_STEP is -1.

    Default is 0.1

\end{verbatim}
\subsubsection{MX MAP\_ROBUST}
\index{MX!MAP\_ROBUST}
\begin{verbatim}

      MAP_ROBUST     Real

    Robust weighting factor. A number between 0 and +infty.

    Robust  weighting  gives  the  natural  weight to UV cells whose natural
    weight is lower than a given threshold.  In  contrast,  if  the  natural
    weight  of  the UV cell is larger than this threshold, the weight is set
    to this (uniform) threshold. The UV cell size is defined  by  MAP_UVCELL
    and the threshold value is in MAP_ROBUST.

    0  means  natural weighting, which is optimal for point sources. The Ro-
    bust weighting factor controls the resolution: better resolution is  ob-
    tained  for small values (at the expense of noise), resolution approach-
    ing the natural weighting scheme for large values.  Larger UV cell  size
    give higher angular resolution (but again more noise).

    MAP_ROBUST  around  .5  to 1 is a good compromise between noise increase
    and angular resolution.

\end{verbatim}
\subsubsection{MX MAP\_ROUNDING}
\index{MX!MAP\_ROUNDING}
\begin{verbatim}

      MAP_ROUNDING     Real

    Maximum error between optimal size (MAP_FIELD /  MAP_CELL)  and  rounded
    (as  a  power  of 2^k 3^p 5^q) MAP_SIZE to round by floor (thus limiting
    the field of view), instead of ceiling (which guarantees a larger  field
    of view, but leads to bigger images).

    Default is 0.05.


\end{verbatim}
\subsubsection{MX MAP\_SHIFT}
\index{MX!MAP\_SHIFT}
\begin{verbatim}

      MAP_SHIFT        Logical

    Obsolescent, superseded by MAP_CENTER, or the UV_MAP arguments.

    Logical variable indicating whether map center (i.e. phase tracking cen-
    ter) or orientation should be changed.

\end{verbatim}
\subsubsection{MX MAP\_SIZE}
\index{MX!MAP\_SIZE}
\begin{verbatim}

      MAP_SIZE[2]      Integer

    Number of pixels in X and Y. It should preferentially be a power of two,
    (although   this   is  not  strictly  required)  to  speed-up  the  FFT.
    MAP_SIZE*MAP_CELL should be at least twice the size of the field-of-view
    (primary  beam  size  for  a single field). Enter 0,0 to let the command
    find a sensible map size.

    MAP_SIZE is not used if MAP_FIELD is non zero.

    Odd values are forbidden.

    Default is 0,0, i.e. UV_MAP will guess the most appropriate values which
    depend on MAP_ROUNDING and MAP_POWER.


\end{verbatim}
\subsubsection{MX MAP\_TAPEREXPO}
\index{MX!MAP\_TAPEREXPO}
\begin{verbatim}

      MAP_TAPEREXPO    Real

    Taper exponent. The default is 2 (indicating a Gaussian) but smoother or
    sharper functions can be used. 1 would give an Exponential, 4  would  be
    getting close to square profile...

\end{verbatim}
\subsubsection{MX MAP\_TRUNCATE}
\index{MX!MAP\_TRUNCATE}
\begin{verbatim}

      MAP_TRUNCATE    Real

    Mosaic truncation level in PerCent.  Default value is 0.2. Current value
    can be overriden by option /TRUNCATE in commands UV_MAP or PRIMARY.

\end{verbatim}
\subsubsection{MX MAP\_UVTAPER}
\index{MX!MAP\_UVTAPER}
\begin{verbatim}

      MAP_UVTAPER[3]  Real

    Parameters of the tapering function (Gaussian if MAP_TAPEREXPO = 2): ma-
    jor axis at 1/e level [m], minor axis at 1/e level [m], and position an-
    gle [deg].

\end{verbatim}
\subsubsection{MX MAP\_UVCELL}
\index{MX!MAP\_UVCELL}
\begin{verbatim}

      MAP_UVCELL   Real

    UV cell size for robust weighting [m].  Should be of the order  of  half
    the  dish diameter (7.5 m for PdBI), or smaller or even larger.  It con-
    trols the beam shape in Robust weighting.

\end{verbatim}
\subsubsection{MX MAP\_VERSION}
\index{MX!MAP\_VERSION}
\begin{verbatim}

      MAP_VERSION  Integer

    [EXPERT Only] Code indicating which version of the UV_MAP and UV_RESTORE
    algorithm  should  be  used.  0  is  optimal.  -1  is  the  "historical"
    (pre-2016) version. 1 is an intermediate version used during  multi-fre-
    quency beams development.

\end{verbatim}
\subsubsection{MX MCOL}
\index{MX!MCOL}
\begin{verbatim}

      MCOL[2]   Integer

    First  and  Last  channel  to  image.  Values  of 0 mean imaging all the
    planes.

\end{verbatim}
\subsubsection{MX WCOL}
\index{MX!WCOL}
\begin{verbatim}

      WCOL      Integer

    [Obsolescent] The channel from which the weight should be  taken.   WCOL
    set  to 0 means using a default channel. WCOL has no real meaning in all
    cases where more than one beam is computed for all channels.


\end{verbatim}
\subsubsection{MX Old\_Names:}
\index{MX!Old\_Names:}
\begin{verbatim}
     NAME        [       ]  Label of the dirty image and beam plots
     UV_TAPER    [m,m,deg]  UV-apodization by convolution with a Gaussian
     WEIGHT_MODE [       ]  Weighting mode (NA|UN)
     UV_CELL     [m, ??  ]  UV cell size and threshold for Robust weighting
     MAP_FIELD   [ arcsec]  Map field of view
     MAP_CELL    [ arcsec]  Map cell size
     MAP_SIZE    [ pixels]  Map size in pixels (if MAP_FIELD is zero)
     MCOL        [       ]  First and Last channel to map
     WCOL        [       ]  Channel from which the weights are taken
     CONVOLUTION [       ]  Convolution function (5)
     UV_SHIFT    [       ]  Change the map phase center or map orientation?
     MAP_RA      [       ]  RA of map phase center
     MAP_DEC     [       ]  Dec of map phase center
     MAP_ANGLE   [    deg]  Map position angle
     MAP_BEAM_STEP [     ]  Number of channels per synthesized beam plane

\end{verbatim}
\subsubsection{MX convolution}
\index{MX!convolution}
\begin{verbatim}

      Older variable name for MAP_CONVOLUTION

\end{verbatim}
\subsubsection{MX map\_angle}
\index{MX!map\_angle}
\begin{verbatim}

      MAP_ANGLE      Real

    Position Angle of the direction which will become the apparent North  in
    the map. Used only if UV_SHIFT is YES.

    Superseded by MAP_CENTER.

\end{verbatim}
\subsubsection{MX map\_dec}
\index{MX!map\_dec}
\begin{verbatim}

      MAP_DEC     Real

    Dec of map center. Used only if UV_SHIFT is YES.

    Superseded by MAP_CENTER.

\end{verbatim}
\subsubsection{MX map\_ra}
\index{MX!map\_ra}
\begin{verbatim}

      MAP_RA      Real

    RA of map center. Used only if UV_SHIFT is YES.

    Superseded by MAP_CENTER.

\end{verbatim}
\subsubsection{MX uv\_cell}
\index{MX!uv\_cell}
\begin{verbatim}

      Older variables for MAP_UVCELL (uv
\end{verbatim}
\subsubsection{MX uv\_shift}
\index{MX!uv\_shift}
\begin{verbatim}

      Older variable name of MAP_SHIFT (this one is also obsolescent)

\end{verbatim}
\subsubsection{MX uv\_taper}
\index{MX!uv\_taper}
\begin{verbatim}

      Older variable name of MAP_UVTAPER

\end{verbatim}
\subsubsection{MX taper\_expo}
\index{MX!taper\_expo}
\begin{verbatim}

      Older variable name for MAP_TAPEREXPO

\end{verbatim}
\subsubsection{MX weight\_mode}
\index{MX!weight\_mode}
\begin{verbatim}

      weightde      Character

    Weighting  mode:  Natural  (optimum  in  terms of sensitivity) or robust
    (usually lower sidelobes and higher spatial resolution) weighting.  This
    was  needed  in  Mapping to toggle between Natural and Robust weighting,
    while IMAGER does that based on MAP_ROBUST value.






\end{verbatim}
\subsubsection{MX CLEAN\_ARES}
\index{MX!CLEAN\_ARES}
\begin{verbatim}

    This is the minimal flux in the dirty map that the program will consider
    as  significant.   Alternatively,  the  threshold  can be specified as a
    fraction of the peak flux using CLEAN_FRES.  Once this  level  has  been
    reached the program stops subtracting, and starts the restoration phase.
    The unit for this parameter is the map unit  (typically  Jy/Beam).   The
    parameter  should  usually  be of the order of magnitude of the expected
    noise in the clean map.

    If 0, CLEAN_FRES will be used instead. If all of CLEAN_NITER, CLEAN_ARES
    and CLEAN_FRES are 0, an absolute residual equal to the noise level will
    be used for CLEAN_ARES.

    Short form is ARES.

\end{verbatim}
\subsubsection{MX CLEAN\_FRES}
\index{MX!CLEAN\_FRES}
\begin{verbatim}

    This is the minimal fraction of the peak flux in the dirty map that  the
    program  will  consider  as  significant.   Alternatively,  an  absolute
    threshold can be specified using CLEAN_ARES.  Once this level  has  been
    reached the program stops subtracting, and starts the restoration phase.
    This parameter is normalized to 1 (neither in % nor in db).   It  should
    usually  be of the order of magnitude of the inverse of the expected dy-
    namic range of the intensity.

    If 0, CLEAN_ARES will be used instead. If all of CLEAN_NITER, CLEAN_ARES
    and CLEAN_FRES are 0, an absolute residual equal to the noise level will
    be used for CLEAN_ARES.

    Short form is FRES.

\end{verbatim}
\subsubsection{MX CLEAN\_GAIN}
\index{MX!CLEAN\_GAIN}
\begin{verbatim}

    This is the gain of the subtraction loop.  It should typically be chosen
    in the range 0.05 and 0.3.  Higher values give faster convergence, while
    lower values give a better restitution of the extended structure. A sen-
    sible default is 0.2.

    Short form is GAIN.

\end{verbatim}
\subsubsection{MX CLEAN\_NITER}
\index{MX!CLEAN\_NITER}
\begin{verbatim}

    This is the maximum number of components the program will accept to sub-
    tract.  Once it has been reached, the  program  starts  the  restoration
    phase.

    If 0, the program will guess a number, based on the image size and maxi-
    mum signal-to-noise  ratio,  and  specified  residual  level  CLEAN_ARES
    and/or CLEAN_FRES.

    Short form is NITER.

\end{verbatim}
\subsubsection{MX CLEAN\_NKEEP}
\index{MX!CLEAN\_NKEEP}
\begin{verbatim}

    This is an integer specifying the minimum number of Clean components be-
    fore testing if Cleaning has converged. The convergence is criterium  is
    a  comparison  of the cumulative flux evolution separated by CLEAN_NKEEP
    components. If th

    IF CLEAN_NKEEP is 0, CLEAN will ignore this convergence  criterium,  and
    continue  clean until the CLEAN_NITER, CLEAN_ARES or CLEAN_FRES criteria
    indicate to stop.

    With CLEAN_NKEEP > 0, CLEAN will explore  the  stability  of  the  total
    clean  flux over the last CLEAN_NKEEP  iterations. For a positive (resp.
    negative) source, if the Clean flux becomes smaller (resp. larger)  than
    the Clean flux CLEAN_NKEEP iterations earlier, CLEAN will stop.

    Using  CLEAN_NKEEP  about  70 is a reasonable value.  Some special cases
    (faint extended sources) may require larger values of CLEAN_NKEEP.

\end{verbatim}
\subsubsection{MX CLEAN\_POSITIVE}
\index{MX!CLEAN\_POSITIVE}
\begin{verbatim}

    The minimum number of positive components before negative ones  are  se-
    lected.

\end{verbatim}
\subsubsection{MX CLEAN\_RESTORE}
\index{MX!CLEAN\_RESTORE}
\begin{verbatim}

      Fraction  of  peak  response of the primary beams coverage under which
    the Sky brightness image is blanked in a Mosaic deconvolution.

    The default is 0.2.

\end{verbatim}
\subsubsection{MX CLEAN\_SEARCH}
\index{MX!CLEAN\_SEARCH}
\begin{verbatim}


      Fraction of peak response of the primary beams coverage  beyond  which
    no Clean component is searched in a Mosaic deconvolution.

    The default is 0.2.

\end{verbatim}
\subsubsection{MX CLEAN\_SIDELOBE}
\index{MX!CLEAN\_SIDELOBE}
\begin{verbatim}

    Minimal  relative  intensity to consider for fitting the syntheized beam
    to obtain the Clean beam parameters (MAJOR, MINOR  and  ANGLE)  when  0.
    The default is 0.35.

    In  case  of  poor UV coverage, CLEAN_SIDELOBE should be higher than the
    maximum sidelobe level to perform a good Gaussian fit. Some particularly
    bad UV coverage may not allow any good fit at all, however.

\end{verbatim}
\subsubsection{MX CLEAN\_NGOAL}
\index{MX!CLEAN\_NGOAL}
\begin{verbatim}

      Number  of clean components to be selected in a Cycle in the ALMA het-
    erogeneous array cleaning method.

\end{verbatim}
\subsubsection{MX CLEAN\_NCYCLE}
\index{MX!CLEAN\_NCYCLE}
\begin{verbatim}

      Maximum number of Major Cycles for the SDI and CLARK methods.

\end{verbatim}
\subsubsection{MX CLEAN\_SMOOTH}
\index{MX!CLEAN\_SMOOTH}
\begin{verbatim}

      Smoothing factor between different scales in the  MRC  and  MULTISCALE
    methods.  The default is sqrt(3).

\end{verbatim}
\subsubsection{MX CLEAN\_SPEEDY}
\index{MX!CLEAN\_SPEEDY}
\begin{verbatim}

      Speed-up factor for the CLARK major cycles. The default is 1.0.  Larg-
    er values may be used, but at the expense of possible  instabilities  of
    the algorithm.

\end{verbatim}
\subsubsection{MX CLEAN\_WORRY}
\index{MX!CLEAN\_WORRY}
\begin{verbatim}

      Worry  factor  in the MULTISCALE method for convergence. It propagates
    the S/N from one iteration to the other, so that if this  S/N  degrades,
    the  method  stops.  Default  is 0 (no propagation, and hence no test on
    S/N).  The value should be < 1.0 in all cases.

\end{verbatim}
\subsubsection{MX CLEAN\_INFLATE}
\index{MX!CLEAN\_INFLATE}
\begin{verbatim}

      Maximum Inflation factor for UV_RESTORE (MULTISCALE method).   If  the
    number  of  true (i.e. pixel based) Clean components found by MULTISCALE
    is larger than CLEAN_INFLATE times the number of compressed (i.e.  those
    with the smoothing factor information) components, expansion of the com-
    pressed components will not be possible, and UV_RESTORE will not be use-
    able.

      A  default  of  50  is in general adequate.  Better solutions might be
    found in the future, and this parameter suppressed.  Apart  from  memory
    usage, this number has no consequence on the algorithm.

\end{verbatim}
\subsubsection{MX METHOD}
\index{MX!METHOD}
\begin{verbatim}

      Method used for the deconvolution. Can be HOGBOM, MULTI, MRC,
      SDI or CLARK.


\end{verbatim}
\subsubsection{MX Old\_Names:}
\index{MX!Old\_Names:}
\begin{verbatim}

      Some  of the CLEAN parameters have kept their old names: MAJOR, MINOR,
    ANGLE (which are also used by  command  FIT)  BLC,  TRC  and  BEAM_PATCH
    (which are seldom used)

      Others  have equivalent short names: ARES, FRES, GAIN, NITER for which
    the CLEAN_ prefix may be omitted.

\end{verbatim}
\subsubsection{MX BLC}
\index{MX!BLC}
\begin{verbatim}

    These are the (pixel) coordinates of  the  Bottom  Left  Corner  of  the
    cleaning  box.   The actual cleaning support will be the intersection of
    the specified window with the inner quarter of the map and with any user
    defined polygon.

\end{verbatim}
\subsubsection{MX TRC}
\index{MX!TRC}
\begin{verbatim}

    These  are the (pixel) coordinates of the Top Right Corner of the clean-
    ing box.  The actual cleaning window will be  the  intersection  of  the
    specified window with the inner quarter of the map and with any user de-
    fined polygon.


\end{verbatim}
\subsubsection{MX MAJOR}
\index{MX!MAJOR}
\begin{verbatim}

    This is the major axis  (FWHP)  in  user  coordinates  of  the  Gaussian
    restoring beam. If 0, the program will fit a Gaussian to the dirty beam.
    We strongly discourage to change the default value of 0.

\end{verbatim}
\subsubsection{MX MINOR}
\index{MX!MINOR}
\begin{verbatim}

    This is the minor axis  (FWHP)  in  user  coordinates  of  the  Gaussian
    restoring beam. If 0, the program will fit a Gaussian to the dirty beam.
    We strongly discourage to change the default value of 0.

\end{verbatim}
\subsubsection{MX ANGLE}
\index{MX!ANGLE}
\begin{verbatim}

    This is the position angle (from North towards East, i.e. anticlockwise)
    of  the  major  axis of the Gaussian restoring beam (in degrees).  If 0,
    the program will fit a Gaussian to the dirty beam. We strongly  discour-
    age to change the default value of 0.

\end{verbatim}
\subsubsection{MX BEAM\_PATCH}
\index{MX!BEAM\_PATCH}
\begin{verbatim}

    The  dirty  beam  patch to be used for the minor cycles in CLARK and MRC
    method.  It should be large enough to avoid doing too many major cycles,
    but  has  practically  no  influence on the result.  This size should be
    specified in pixel units.  Reasonable values are between  N/8  and  N/4,
    where N is the number of map pixels in the same dimension.  If set to N,
    the CLARK algorithm becomes identical to the HOGBOM algorithm.



\end{verbatim}
\subsection{PRIMARY}
\index{PRIMARY}
\begin{verbatim}
        [CLEAN\]PRIMARY [BeamSize] [/TRUNCATE Percent]

    Apply approximate primary beam correction to a single field  deconvolved
    (CLEAN)  image, in order to create the sky brightness image (named SKY).
    The primary beam model is a simple Gaussian.

    If BeamSize (in radian) is specified, uses the corresponding  half-power
    beam size to determine the Gaussian beam.

    If not, the parameters are taken from the telescope parameters, as found
    in the telescope section of the CLEAN image and the observing  frequency
    from the CLEAN image (e.g. for ALMA it uses 1.13 Lambda/D).

    The SKY image is written with extension .lmv-sky by command WRITE.


\end{verbatim}
\subsubsection{PRIMARY /TRUNCATE}
\index{PRIMARY!/TRUNCATE}
\begin{verbatim}
        [CLEAN\]PRIMARY [BeamSize] /TRUNCATE Percent

    Specify  the  truncation  level. Default (in % of peak beam response) is
    given by the MAP_TRUNCATE variable.  Values are blanked beyond this.

\end{verbatim}
\subsection{READ}
\index{READ}
\begin{verbatim}
        [CLEAN\]READ Buffer File [/COMPACT] |/FREQUENCY RestFreq]
      [/RANGE Min Max Type] [/NOTRAIL]

    Read the specified internal buffer (BEAM,  CCT,  CGAINS,  CLEAN,  DIRTY,
    FIELDS,  MASK,  MODEL,  PRIMARY, RESIDUAL, SKY, SUPPORT, SINGLEDISH, UV)
    from input File.  Default extension is .uvt for UV,  CGAINS  and  MODEL,
    and  for  the  others,  in  order,  .beam, .cct, .lmv-clean, .lmv, .msk,
    .lobe, .lmv-res, be indicated through the /RANGE option, even for UV ta-
    bles.

    The  corresponding  buffer  is available as a SIC image-like variable of
    the same name, so that one can use e.g. HEADER DIRTY command.

    READ * Name   will attempt to read all existing files of same Name  with
    the standard file types corresponding to the respective buffers.

    The  MODEL UV table is for use in conjunction with commands in the CALI-
    BRATE\ language.

    The /COMPACT option is used to load  the  ACA-specific  internal  buffer
    used in the ALMA joint deconvolution method.

    The  /NOTRAIL allows to ignore trailing columns (normally not recommend-
    ed. Used for debug only).

\end{verbatim}
\subsubsection{READ Optimisation}
\index{READ!Optimisation}
\begin{verbatim}

        Reading can be lengthy, especially for ALMA data.  Mapping  attempts
    to  minimize  read  operations by checking if anything has changed since
    the last command.  This capability is enable if the  SIC  variable  MAP-
    PING_OPTIMIZE is non zero, disabled otherwise.
      Currently,  the READ command always display a message about what read-
    ing may have been skipped, and whether this  possible  optimization  has
    been overridden by user choice.

\end{verbatim}
\subsubsection{READ /COMPACT}
\index{READ!/COMPACT}
\begin{verbatim}
        [CLEAN\]READ Buffer File /COMPACT [/RANGE Min Max Type]

    Read  the  specified  internal  buffer (UV, MODEL, BEAM, PRIMARY, DIRTY,
    CLEAN, MASK, CCT) from input File to the "compact array" data area.

\end{verbatim}
\subsubsection{READ /FREQUENCY}
\index{READ!/FREQUENCY}
\begin{verbatim}
        [CLEAN\]READ Buffer File /FREQUENCY RestFreq [/RANGE Min Max Type]

    Read the specified internal buffer and reset the velocity scale  to  the
    corresponding  rest  frequencies. Velocities specified in the /RANGE Min
    Max VELOCITY option would then refer to this new frequency.

\end{verbatim}
\subsubsection{READ /NOTRAIL}
\index{READ!/NOTRAIL}
\begin{verbatim}
        [CLEAN\]READ Buffer File /NOTRAIL [/FREQUENCY Freq] [/RANGE Min  Max
    Type]

    When  reading  UV  Tables, ignores any trailing column. Trailing columns
    normally appear for mosaics. However, ALMA sometimes uses  known  proper
    motions  to shift (by small amounts) phase centers between two observing
    periods, yielding pseudo-mosaics with tiny  (in  general  insignificant)
    displacements.  The /NOTRAIL option allows to ignore these details.

\end{verbatim}
\subsubsection{READ /PLANES}
\index{READ!/PLANES}
\begin{verbatim}
        [CLEAN\]READ Buffer File /PLANES First Last

      ** OBSOLESCENT ** Use /RANGE for a simpler interface.

    Read  only  "channels"  between First and Last.  For UV tables with more
    than 1 Stokes parameter, which are **NOT** fully  supported  by  IMAGER,
    the meaning of "channel" is ambiguous.

\end{verbatim}
\subsubsection{READ /RANGE}
\index{READ!/RANGE}
\begin{verbatim}
        [CLEAN\]READ Buffer File /RANGE Min Max Type

    Load only the channels between the First and Last defined by Min Max and
    Type.  Type can be CHANNEL, VELOCITY or FREQUENCY.

    For type CHANNEL, Min and Max indicate offsets from Channel 1 and  Chan-
    nel Nchan (the number of channels in the data set). Thus Max can be neg-
    ative: it then indicates Last = Nchan-Max. Also Min=0 and Max=0  implies
    loading all the channels.

\end{verbatim}
\subsubsection{READ SINGLE}
\index{READ!SINGLE}
\begin{verbatim}
        [CLEAN\]READ SINGLE File[.ext] [/RANGE Min Max Type]

    Read  the  "Single  Dish" data set. It can be a Class table (.tab), or a
    3-D data cube (.lmv).

    The /RANGE option only works for a .lmv data cube so far.

\end{verbatim}
\subsection{SDI}
\index{SDI}
\begin{verbatim}
        [CLEAN\]SDI [FirstPlane  [LastPlane]]  [/PLOT  Clean|Residu]  [/FLUX
    Fmin Fmax] [/QUERY]

    Perform a Steer-Dewdney-Ito CLEAN. This clean method selects an ensemble
    of clean components and remove them at once using FFTs.  It  works  best
    for  extended  sources  and  UV coverages with short spacings. In such a
    case, it may avoid the "ringing" features which appear using  the  CLARK
    or  HOGBOM  techniques.  In  mosaic  mode (see command MOSAIC), a mosaic
    clean is performed.

    Clean the specified plane interval (default:  planes  between  variables
    FIRST and LAST). If only FirstPlane is specified, Clean only that plane.

    If option /PLOT is given, a display of the CLEAN or RESIDUAL map will be
    shown  at  each major cycle, depending on the argument (default: Residu-
    al). The user will be prompted for continuation when the  /QUERY  option
    is  present.  The  cumulative, already cleaned flux is displayed in real
    time in an additional window while cleaning goes on when the  /FLUX  op-
    tion  is  present.  Parameters of the /FLUX option are then used to give
    the flux limits for this display.

    The user can control the algorithm through SIC variables. New values can
    be given using "LET VARIABLE value". For ease of use, and whenever it is
    possible, a sensible value of each parameter will automatically be  com-
    puted from the context if the value of the corresponding variable is set
    to its default value, i.e. zero value and empty string. A few  variables
    are initialized to "reasonable" values.

        [CLEAN\]CLEAN ?
    Will  list  all main CLEAN_* variables controlling the CLEAN parameters.
    HELP CLEAN Variables will give a more complete list.
\end{verbatim}
\subsubsection{SDI Variables:}
\index{SDI!Variables:}
\begin{verbatim}

    Basic parameters
    CLEAN_GAIN       [       ] Loop gain
    CLEAN_NITER      [       ] Maximum number of clean components
    CLEAN_FRES       [      %] Maximum value of residual (Fraction of peak)
    CLEAN_ARES       [Jy/Beam] Maximum value of residual (Absolute)
    CLEAN_POSITIVE   [       ] Minimum number of positive components at start
    CLEAN_NKEEP      [       ] Min number of components before convergence

    Old names like in MAPPING
    BLC              [  pixel] Bottom left corner of cleaning box
    TRC              [  pixel] Top right corner of cleaning box
    MAJOR            [ arcsec] Clean beam major axis
    MINOR            [ arcsec] Clean beam minor axis
    ANGLE            [ degree] Position angle of clean beam
    BEAM_PATCH       [  pixel] Size of cleaning beam ** not clear **

    Method dependent parameters
    CLEAN_INFLATE    [      ] Maximum Inflation factor for UV_RESTORE (MuLTISCAL
    CLEAN_NCYCLE     [      ] Max number of Major Cycles (SDI & CLARK methods)
    CLEAN_NGOAL      [      ] Max number of comp. in Cycles (ALMA method)
    CLEAN_RESTORE    [      ] Threshold for restoring a Mosaic (def 0.2)
    CLEAN_SEARCH     [      ] Threshold to search Clean Comp. in a Mosaic (def 0
    CLEAN_SIDELOBE   [      ] Min threshold to fit the synthesized beam
    CLEAN_SMOOTH     [      ] Smoothing ratio (MRC and MULTISCALE)
    CLEAN_SPEEDY     [      ] Speed-up factor (CLARK)
    CLEAN_WORRY      [      ] Worry factor (MULTISCALE)
\end{verbatim}
\subsubsection{SDI CLEAN\_ARES}
\index{SDI!CLEAN\_ARES}
\begin{verbatim}

    This is the minimal flux in the dirty map that the program will consider
    as  significant.   Alternatively,  the  threshold  can be specified as a
    fraction of the peak flux using CLEAN_FRES.  Once this  level  has  been
    reached the program stops subtracting, and starts the restoration phase.
    The unit for this parameter is the map unit  (typically  Jy/Beam).   The
    parameter  should  usually  be of the order of magnitude of the expected
    noise in the clean map.

    If 0, CLEAN_FRES will be used instead. If all of CLEAN_NITER, CLEAN_ARES
    and CLEAN_FRES are 0, an absolute residual equal to the noise level will
    be used for CLEAN_ARES.

    Short form is ARES.

\end{verbatim}
\subsubsection{SDI CLEAN\_FRES}
\index{SDI!CLEAN\_FRES}
\begin{verbatim}

    This is the minimal fraction of the peak flux in the dirty map that  the
    program  will  consider  as  significant.   Alternatively,  an  absolute
    threshold can be specified using CLEAN_ARES.  Once this level  has  been
    reached the program stops subtracting, and starts the restoration phase.
    This parameter is normalized to 1 (neither in % nor in db).   It  should
    usually  be of the order of magnitude of the inverse of the expected dy-
    namic range of the intensity.

    If 0, CLEAN_ARES will be used instead. If all of CLEAN_NITER, CLEAN_ARES
    and CLEAN_FRES are 0, an absolute residual equal to the noise level will
    be used for CLEAN_ARES.

    Short form is FRES.

\end{verbatim}
\subsubsection{SDI CLEAN\_GAIN}
\index{SDI!CLEAN\_GAIN}
\begin{verbatim}

    This is the gain of the subtraction loop.  It should typically be chosen
    in the range 0.05 and 0.3.  Higher values give faster convergence, while
    lower values give a better restitution of the extended structure. A sen-
    sible default is 0.2.

    Short form is GAIN.

\end{verbatim}
\subsubsection{SDI CLEAN\_NITER}
\index{SDI!CLEAN\_NITER}
\begin{verbatim}

    This is the maximum number of components the program will accept to sub-
    tract.  Once it has been reached, the  program  starts  the  restoration
    phase.

    If 0, the program will guess a number, based on the image size and maxi-
    mum signal-to-noise  ratio,  and  specified  residual  level  CLEAN_ARES
    and/or CLEAN_FRES.

    Short form is NITER.

\end{verbatim}
\subsubsection{SDI CLEAN\_NKEEP}
\index{SDI!CLEAN\_NKEEP}
\begin{verbatim}

    This is an integer specifying the minimum number of Clean components be-
    fore testing if Cleaning has converged. The convergence is criterium  is
    a  comparison  of the cumulative flux evolution separated by CLEAN_NKEEP
    components. If th

    IF CLEAN_NKEEP is 0, CLEAN will ignore this convergence  criterium,  and
    continue  clean until the CLEAN_NITER, CLEAN_ARES or CLEAN_FRES criteria
    indicate to stop.

    With CLEAN_NKEEP > 0, CLEAN will explore  the  stability  of  the  total
    clean  flux over the last CLEAN_NKEEP  iterations. For a positive (resp.
    negative) source, if the Clean flux becomes smaller (resp. larger)  than
    the Clean flux CLEAN_NKEEP iterations earlier, CLEAN will stop.

    Using  CLEAN_NKEEP  about  70 is a reasonable value.  Some special cases
    (faint extended sources) may require larger values of CLEAN_NKEEP.

\end{verbatim}
\subsubsection{SDI CLEAN\_POSITIVE}
\index{SDI!CLEAN\_POSITIVE}
\begin{verbatim}

    The minimum number of positive components before negative ones  are  se-
    lected.

\end{verbatim}
\subsubsection{SDI CLEAN\_RESTORE}
\index{SDI!CLEAN\_RESTORE}
\begin{verbatim}

      Fraction  of  peak  response of the primary beams coverage under which
    the Sky brightness image is blanked in a Mosaic deconvolution.

    The default is 0.2.

\end{verbatim}
\subsubsection{SDI CLEAN\_SEARCH}
\index{SDI!CLEAN\_SEARCH}
\begin{verbatim}


      Fraction of peak response of the primary beams coverage  beyond  which
    no Clean component is searched in a Mosaic deconvolution.

    The default is 0.2.

\end{verbatim}
\subsubsection{SDI CLEAN\_SIDELOBE}
\index{SDI!CLEAN\_SIDELOBE}
\begin{verbatim}

    Minimal  relative  intensity to consider for fitting the syntheized beam
    to obtain the Clean beam parameters (MAJOR, MINOR  and  ANGLE)  when  0.
    The default is 0.35.

    In  case  of  poor UV coverage, CLEAN_SIDELOBE should be higher than the
    maximum sidelobe level to perform a good Gaussian fit. Some particularly
    bad UV coverage may not allow any good fit at all, however.

\end{verbatim}
\subsubsection{SDI CLEAN\_NGOAL}
\index{SDI!CLEAN\_NGOAL}
\begin{verbatim}

      Number  of clean components to be selected in a Cycle in the ALMA het-
    erogeneous array cleaning method.

\end{verbatim}
\subsubsection{SDI CLEAN\_NCYCLE}
\index{SDI!CLEAN\_NCYCLE}
\begin{verbatim}

      Maximum number of Major Cycles for the SDI and CLARK methods.

\end{verbatim}
\subsubsection{SDI CLEAN\_SMOOTH}
\index{SDI!CLEAN\_SMOOTH}
\begin{verbatim}

      Smoothing factor between different scales in the  MRC  and  MULTISCALE
    methods.  The default is sqrt(3).

\end{verbatim}
\subsubsection{SDI CLEAN\_SPEEDY}
\index{SDI!CLEAN\_SPEEDY}
\begin{verbatim}

      Speed-up factor for the CLARK major cycles. The default is 1.0.  Larg-
    er values may be used, but at the expense of possible  instabilities  of
    the algorithm.

\end{verbatim}
\subsubsection{SDI CLEAN\_WORRY}
\index{SDI!CLEAN\_WORRY}
\begin{verbatim}

      Worry  factor  in the MULTISCALE method for convergence. It propagates
    the S/N from one iteration to the other, so that if this  S/N  degrades,
    the  method  stops.  Default  is 0 (no propagation, and hence no test on
    S/N).  The value should be < 1.0 in all cases.

\end{verbatim}
\subsubsection{SDI CLEAN\_INFLATE}
\index{SDI!CLEAN\_INFLATE}
\begin{verbatim}

      Maximum Inflation factor for UV_RESTORE (MULTISCALE method).   If  the
    number  of  true (i.e. pixel based) Clean components found by MULTISCALE
    is larger than CLEAN_INFLATE times the number of compressed (i.e.  those
    with the smoothing factor information) components, expansion of the com-
    pressed components will not be possible, and UV_RESTORE will not be use-
    able.

      A  default  of  50  is in general adequate.  Better solutions might be
    found in the future, and this parameter suppressed.  Apart  from  memory
    usage, this number has no consequence on the algorithm.

\end{verbatim}
\subsubsection{SDI METHOD}
\index{SDI!METHOD}
\begin{verbatim}

      Method used for the deconvolution. Can be HOGBOM, MULTI, MRC,
      SDI or CLARK.


\end{verbatim}
\subsubsection{SDI Old\_Names:}
\index{SDI!Old\_Names:}
\begin{verbatim}

      Some  of the CLEAN parameters have kept their old names: MAJOR, MINOR,
    ANGLE (which are also used by  command  FIT)  BLC,  TRC  and  BEAM_PATCH
    (which are seldom used)

      Others  have equivalent short names: ARES, FRES, GAIN, NITER for which
    the CLEAN_ prefix may be omitted.

\end{verbatim}
\subsubsection{SDI BLC}
\index{SDI!BLC}
\begin{verbatim}

    These are the (pixel) coordinates of  the  Bottom  Left  Corner  of  the
    cleaning  box.   The actual cleaning support will be the intersection of
    the specified window with the inner quarter of the map and with any user
    defined polygon.

\end{verbatim}
\subsubsection{SDI TRC}
\index{SDI!TRC}
\begin{verbatim}

    These  are the (pixel) coordinates of the Top Right Corner of the clean-
    ing box.  The actual cleaning window will be  the  intersection  of  the
    specified window with the inner quarter of the map and with any user de-
    fined polygon.


\end{verbatim}
\subsubsection{SDI MAJOR}
\index{SDI!MAJOR}
\begin{verbatim}

    This is the major axis  (FWHP)  in  user  coordinates  of  the  Gaussian
    restoring beam. If 0, the program will fit a Gaussian to the dirty beam.
    We strongly discourage to change the default value of 0.

\end{verbatim}
\subsubsection{SDI MINOR}
\index{SDI!MINOR}
\begin{verbatim}

    This is the minor axis  (FWHP)  in  user  coordinates  of  the  Gaussian
    restoring beam. If 0, the program will fit a Gaussian to the dirty beam.
    We strongly discourage to change the default value of 0.

\end{verbatim}
\subsubsection{SDI ANGLE}
\index{SDI!ANGLE}
\begin{verbatim}

    This is the position angle (from North towards East, i.e. anticlockwise)
    of  the  major  axis of the Gaussian restoring beam (in degrees).  If 0,
    the program will fit a Gaussian to the dirty beam. We strongly  discour-
    age to change the default value of 0.

\end{verbatim}
\subsubsection{SDI BEAM\_PATCH}
\index{SDI!BEAM\_PATCH}
\begin{verbatim}

    The  dirty  beam  patch to be used for the minor cycles in CLARK and MRC
    method.  It should be large enough to avoid doing too many major cycles,
    but  has  practically  no  influence on the result.  This size should be
    specified in pixel units.  Reasonable values are between  N/8  and  N/4,
    where N is the number of map pixels in the same dimension.  If set to N,
    the CLARK algorithm becomes identical to the HOGBOM algorithm.



\end{verbatim}
\subsection{SHOW}
\index{SHOW}
\begin{verbatim}
        [CLEAN\]SHOW Variable [Arg1 [Arg2]]

    where
    Variable
        is an internal buffer to be plotted (BEAM, CCT, CLEAN, DIRTY,  etc..
    ) or a SIC image variable.

        [CLEAN\]SHOW ?

    will  list  the  names of recognized keywords. If Variable is not one of
    the recognized keywords, but an existing Image variable, this image will
    be displayed.

        Except for UV data where they have a different meaning,
    Arg1 and Arg2
        are optional arguments to restrict the range of channels to be plot-
    ted. They default to  FIRST and LAST variable values  respectively,  and
    Arg2 defaults to Arg1 if only Arg1 is specified.

    The buffer is displayed using the appropriate procedure
      p_show_map for data cubes
      p_show_cct for Clean Component Tables
      p_uvshow_sub for UV data
    Command VIEW offers a different style of display for cubes.

    The  UV data in the internal buffer is the one loaded by command READ UV
    File and optionally resampled by UV_RESAMPLE, UV_COMPRESS or transformed
    by UV_CONT. Flagged UV data  will appear in a different color.

    Note and Caution:
        GO  PLOT  (or its variants GO BIT, GO NICE and GO MAP) and GO UVSHOW
    offer similar features, but take data from files or SIC image variables,
    depending on variables NAME and TYPE.

\end{verbatim}
\subsubsection{SHOW COVERAGE}
\index{SHOW!COVERAGE}
\begin{verbatim}
        SHOW COVERAGE [Ant [Date]]

    Displays  the UV coverage.  Ant and Date are optional arguments indicat-
    ing which Antenna is to be highlighted, and for which date.  Date  is  a
    sequential number from 1 to the number of observing dates.

    For  spectral line UV data, SHOW COVERAGE will only show one UV coverage
    if FIRST and LAST are set to zero. It will show one per  channel  other-
    wise.

\end{verbatim}
\subsubsection{SHOW UV}
\index{SHOW!UV}
\begin{verbatim}
        SHOW UV [Ant [Date]]

    Displays the UV data.  Items to be displayed are controlled by XTYPE and
    YTYPE variables.  Ant and Date are optional arguments  indicating  which
    Antenna  is  to be highlighted, and for which date. Date is a sequential
    number from 1 to the number of observing dates.

    Structure uvshow% controls some additional options, such as coloring  of
    various dates, of data flagging, etc...

\end{verbatim}
\subsubsection{SHOW GO\_PLOT}
\index{SHOW!GO\_PLOT}
\begin{verbatim}
        GO  PLOT provides a somewhat different plotting mechanism, using the
    disk files specified by NAME and TYPE (or the SIC variable specified  by
    NAME if TYPE is empty)
      GO PLOT Variable [First [Last]]
    which  will plot channels of the specified data set.  First (default: 0)
    and Last (default: First) are optional arguments  indicating  the  first
    and  last  channel to be displayed. If not specified the variables FIRST
    and LAST are used.

    Most aspects of the display are controlled through the  same  SIC  vari-
    ables as the ones used by the "GO BIT|MAP|LMV" commands. Detailed infor-
    mation about those variables is found through the "INPUT LMV" command.

\end{verbatim}
\subsubsection{SHOW GO\_UVSHOW}
\index{SHOW!GO\_UVSHOW}
\begin{verbatim}
        GO UVSHOW

    Display UV data from the 'name'.uvt file. Control parameters are  avail-
    able as SIC variables. Use command INPUT UVSHOW to check them.

\end{verbatim}
\subsection{STATISTIC}
\index{STATISTIC}
\begin{verbatim}
        [CLEAN\]STATISTIC [Clean|Dirty|Residu] [Plane] [/WHOLE]

    Compute some basic statistics (min, max, mean, rms,...) on the specified
    buffer (default: Clean) and plane (default: variable FIRST). The current
    polygon  (see  command  SUPPORT) is used to define the area on which the
    pixels are to be taken  into  account,  except  when  option  /WHOLE  is
    present: The whole image is then considered.

\end{verbatim}
\subsection{STOKES}
\index{STOKES}
\begin{verbatim}
        [CLEAN\]STOKES Key UVin  [UVout]

    Derive  a  single  polarization  UV  table (UVout) with the polarization
    state specified by Key from a multi-polarization UV  table  (Uvin).   If
    UVout  is not specified, UVin is overwritten.  Key is any of the follow-
    ing: NONE, I, Q, U, V, LL, RR, HH, VV.

    A typical use is after command FITS on CASA data:
      FITS Fits.uvfits TO UVin.uvt
      STOKES NONE UVin

\end{verbatim}
\subsection{SUPPORT}
\index{SUPPORT}
\begin{verbatim}
        [CLEAN\]SUPPORT [Polygon] [/CURSOR] [/MASK] [/PLOT] [/RESET]
      [/VARIABLE] [/THRESHOLD [Raw Smooth Length Guard]]

    Define and/or plot the support inside which to search for  CLEAN  compo-
    nents. The support can be defined through a mask or a polygon, depending
    on the selected options.  selected.

    A polygon stored in a file, or in a Sic variable (/VARIABLE option), can
    be loaded as support. The /PLOT option can then be used to plot it.


\end{verbatim}
\subsubsection{SUPPORT /CURSOR}
\index{SUPPORT!/CURSOR}
\begin{verbatim}
        [CLEAN\]SUPPORT /CURSOR

    With option /CURSOR, it calls the interactive cursor to define the poly-
    gon summits.  Type any key to go to next summit, D to correct  the  last
    one and type E to end the polygon definition. The last polygon side will
    then appear. The polygon definition may be  aborted  by  typing  Q.  For
    graphical displays, you may use the mouse buttons for the commands.  The
    left mouse button draws a vertex, the middle mouse  button  deletes  the
    last vertex, and the right mouse button ends the polygon definition.

    The resulting support is available in the Sic structure SUPPORT:
      - SUPPORT%NXY  [Integer]  Number of summits
      - SUPPORT%X    [Double]   X coordinates
      - SUPPORT%Y    [Double]   Y coordinates

\end{verbatim}
\subsubsection{SUPPORT /RESET}
\index{SUPPORT!/RESET}
\begin{verbatim}
        [CLEAN\]SUPPORT /RESET

    Reset  any support to default. This deletes the current polygon support.
    This does not unload any mask defined  by READ MASK: it  can  be  re-in-
    stated by SUPPORT /MASK.

\end{verbatim}
\subsubsection{SUPPORT /MASK}
\index{SUPPORT!/MASK}
\begin{verbatim}
        [CLEAN\]SUPPORT /MASK

    Use  the  mask  defined by READ MASK as the clean support. This does not
    suppress the current polygon: it can be re-instated  by  SUPPORT  /PLOT.
    The  Mask can be a 3-D array: CLEAN will find out match which mask plane
    must be used for each spectral channel.

    Caution: READ MASK does not perform an implicit SUPPORT /MASK command.

\end{verbatim}
\subsubsection{SUPPORT /PLOT}
\index{SUPPORT!/PLOT}
\begin{verbatim}
        [CLEAN\]SUPPORT [Name] /PLOT

    Plot the current (or specified) polygon, or plot the current mask (if it
    is 2-D only).

\end{verbatim}
\subsubsection{SUPPORT /THRESHOLD}
\index{SUPPORT!/THRESHOLD}
\begin{verbatim}
        [CLEAN\]SUPPORT /THRESHOLD [Raw Smooth [Length [Guard]]

    Define  a  Mask from thresholding the CLEAN image. If the CLEAN image is
    3-D, the Mask will be 3-D.   `

    The method involves a first thresholding, followed by a smoothing to ex-
    tend the support and a second thresholding.

    The algorithm to define the mask is controlled by 4 parameters.
        Raw
    Initial  threshold  (in units of CLEAN image noise) under which the mask
    is set to 0.  Default is 6 sigma. The noise is taken from  the  computed
    Clean  noise  (clean%gil%rms)  if defined, or from the theoretical noise
    (dirty%gil%noise) if not.
        Smooth
    Threshold under which the Mask is set to 0 after smoothing. The  default
    is 2 sigma.
        Length
    FWHM  of  the smoothing gaussian to derive the smooth mask from the ini-
    tial mask.  The default is the Clean beam major axis.
        Guard
    Size of the guard band at edges where the mask is set to zero, in  units
    of  image  size. The default is 0.18, i.e. the mask can extends a little
    more than the inner quarter.  This is to avoid the edges where sidelobes
    aliasing occurs.

    The  computed mask can be displayed by SHOW MASK and saved by WRITE MASK
    for further use.

\end{verbatim}
\subsubsection{SUPPORT /VARIABLE}
\index{SUPPORT!/VARIABLE}
\begin{verbatim}
        [CLEAN\]SUPPORT VarName /VARIABLE

    Load the support from the Sic variable VarName. VarName can be an  array
    of the form:
             VarName[NXY,2]  Real or Double

    or a structure of the form (i.e. same as output):
             VarName%NXY             Integer or Long
             VarName%X[VarName%NXY]  Real or Double
             VarName%Y[VarName%NXY]  Real or Double


\end{verbatim}
\subsection{UV\_BASELINE}
\index{UV\_BASELINE}
\begin{verbatim}
        [CLEAN\]UV_BASELINE [Degree] [/CHANNELS Channel_List]
      [/FREQUENCY List Of Frequencies] [/RANGE Min Max [TYPE]]
      [/VELOCITY List of Velocities] [/WIDTH Width [TYPE]]

    Subtract a continuum from a line UV data set, by fitting  a baseline for
    each visibility.  The channels to be ignored in this process  (i.e.  the
    ones  including  the line emission) can be specified either by the /FRE-
    QUENCY or /VELOCITY options in combination with the /WIDTH option, or by
    a  channel  list  with /CHANNELS or a range (of channels, frequencies or
    velocities) with /RANGE.

\end{verbatim}
\subsubsection{UV\_BASELINE /CHANNELS}
\index{UV\_BASELINE!/CHANNELS}
\begin{verbatim}
        [CLEAN\]UV_BASELINE /CHANNELS Channel_List

    Channel_List must be a 1-D SIC variable containing the list of  channels
    to filter out.

\end{verbatim}
\subsubsection{UV\_BASELINE /FREQUENCY}
\index{UV\_BASELINE!/FREQUENCY}
\begin{verbatim}
        [CLEAN\]UV_BASELINE /FREQUENCY F1 [... [Fn]] [/WIDTH Width [TYPE]]

    Specify  around  which frequencies the line emission should be filtered.
    Frequencies F1 to Fn must be in MHz. The full  width  of  the  filtering
    window around every frequency can be set by option /WIDTH.  The optional
    argument TYPE indicates the type of width: FREQUENCY (in MHz),  VELOCITY
    (in  km/s)  or  CHANNEL (no unit), the default being FREQUENCY.  The de-
    fault width is the current channel width.

    Tip: it can be convenient to have a list of SIC variables containing the
    frequencies of the most intense spectral lines, e.g.
            HCO10 = 89188.52

\end{verbatim}
\subsubsection{UV\_BASELINE /RANGE}
\index{UV\_BASELINE!/RANGE}
\begin{verbatim}
        [CLEAN\]UV_BASELINE /FREQUENCY F1 [... [Fn]] /RANGE Min Max [TYPE]

    Indicate that channels between the First and Last defined by Min Max and
    Type contain line emission and should be ignored in  the  baseline  fit-
    ting.  Type can be CHANNEL, VELOCITY or FREQUENCY.

    For  type CHANNEL, Min and Max indicate offsets from Channel 1 and Chan-
    nel Nchan (the number of channels in the data set). Thus Max can be neg-
    ative:  it then indicates Last = Nchan-Max. Also Min=0 and Max=0 implies
    loading all the channels.

\end{verbatim}
\subsubsection{UV\_BASELINE /VELOCITY}
\index{UV\_BASELINE!/VELOCITY}
\begin{verbatim}
        [CLEAN\]UV_BASELINE /VELOCITY V1 [... [Vn]] [/WIDTH Width [TYPE]]

    Specify around which velocities the line emission  should  be  filtered.
    Velocities  V1  to  Vn  must be in km/s. The full width of the filtering
    window around every frequency can be set by option /WIDTH.  The optional
    argument  TYPE indicates the type of width: FREQUENCY (in MHz), VELOCITY
    (in km/s) or CHANNEL (no unit), the default being  FREQUENCY.   The  de-
    fault width is the current channel width.

\end{verbatim}
\subsubsection{UV\_BASELINE /WIDTH}
\index{UV\_BASELINE!/WIDTH}
\begin{verbatim}
        [CLEAN\]UV_BASELINE /FREQUENCY F1 [... [Fn]] /WIDTH Width [TYPE]

    Specify the full width of the window around every frequency given in the
    /FREQUENCY option.  The optional argument TYPE  indicates  the  type  of
    width:  FREQUENCY (in MHz), VELOCITY (in km/s) or CHANNEL (no unit), the
    default being FREQUENCY.  The  default  width  is  the  current  channel
    width.  The default width is the current channel width.

\end{verbatim}
\subsection{UV\_CHECK}
\index{UV\_CHECK}
\begin{verbatim}
        [CLEAN\]UV_CHECK  Beams|Nulls

      Check UV data for weight consistency.

      BEAMS
      List  which  channel  range can be processed with the same synthesized
    beam.

      NULLS
      Check if there are null visibilities with non-zero weights, and flag
      them if found.


\end{verbatim}
\subsection{UV\_COMPRESS}
\index{UV\_COMPRESS}
\begin{verbatim}
        [CLEAN\]UV_COMPRESS Nc

    Resample the UV data loaded by READ UV by averaging  NC  adjacent  chan-
    nels.  All  other UV commands except UV_RESAMPLE work on the "Resampled"
    UV table.

    The "Resampled" UV table is a simple copy of the original  one  after  a
    READ  UV command, or after a UV_RESAMPLE or UV_COMPRESS commands without
    arguments.



\end{verbatim}
\subsection{UV\_CONTINUUM}
\index{UV\_CONTINUUM}
\begin{verbatim}
        [CLEAN\]UV_CONTINUUM Naver [First Last]

    Transform the (presumably spectral line) UV data set loaded by  READ  UV
    into a "continuum" data set.

    The  transformation  selects  line  channels from First to Last, average
    them by groups of Naver contiguous channels, and concatenate the result-
    ing visibilities into a "continuum" UV table.

    For each of the (First-Last+1)/Naver channels, and for all visibilities,
    the U and V coordinates are rescaled to the  mean  observing  frequency,
    and  the  resulting  (single-channel) visibilities are concatenated into
    the "continuum" UV data set. The continuum dataset becomes  the  current
    UV data. Flagged channels are ignored: this allows to mask channels con-
    taining spectral lines (see UV_FILTER).

    Default for First and Last is 0 0, meaning all  channels  are  selected.
    Negative  value  for  Last indicate an offset from end channel (i.e. -20
    means ignore the last 20 channels).

\end{verbatim}
\subsection{UV\_FILTER}
\index{UV\_FILTER}
\begin{verbatim}
        [CLEAN\]UV_FILTER [/ZERO] [/CHANNELS Channel_List]
      [/FREQUENCY List Of Frequencies] [/RANGE Min Max [TYPE]]
      [/VELOCITY List of Velocities] [/WIDTH Width [TYPE]]

    "Filter" line emission, by  flagging  the  corresponding  channels.  The
    channels  can be specified either by the /FREQUENCY or /VELOCITY options
    in combination with the /WIDTH option, or by a channel list with  /CHAN-
    NELS or a range (of channels, frequencies or velocities) with /RANGE.

    By  default, channels to be filtered are flagged (weights becoming nega-
    tive).  Option /ZERO can be used to erase them: weights and visibilities
    are set to zero, see HELP UV_FILTER /ZERO.

\end{verbatim}
\subsubsection{UV\_FILTER /CHANNELS}
\index{UV\_FILTER!/CHANNELS}
\begin{verbatim}
        [CLEAN\]UV_FILTER [/ZERO] /CHANNELS Channel_List

    Channel_List  must be a 1-D SIC variable containing the list of channels
    to filter out.

\end{verbatim}
\subsubsection{UV\_FILTER /FREQUENCY}
\index{UV\_FILTER!/FREQUENCY}
\begin{verbatim}
        [CLEAN\]UV_FILTER [/ZERO] /FREQUENCY F1 [... [Fn]] [/WIDTH Width]

    Specify around which frequencies the line emission should  be  filtered.
    Frequencies  F1  to  Fn  must be in MHz. The full width of the filtering
    window around every frequency can be set by option /WIDTH.  The optional
    argument  TYPE indicates the type of width: FREQUENCY (in MHz), VELOCITY
    (in km/s) or CHANNEL (no unit), the default being  FREQUENCY.   The  de-
    fault width is the current channel width.

    Tip: it can be convenient to have a list of SIC variables containing the
    frequencies of the most intense spectral lines, e.g.
            HCO10 = 89188.52

\end{verbatim}
\subsubsection{UV\_FILTER /RANGE}
\index{UV\_FILTER!/RANGE}
\begin{verbatim}
        [CLEAN\]UV_FILTER [/ZERO] /RANGE Min Max [TYPE]

    Indicate that channels between the First and Last defined by Min Max and
    Type  contain  line  emission  and  should be filtered out.  Type can be
    CHANNEL, VELOCITY or FREQUENCY.

    For type CHANNEL, Min and Max indicate offsets from Channel 1 and  Chan-
    nel Nchan (the number of channels in the data set). Thus Max can be neg-
    ative: it then indicates Last = Nchan-Max. Also Min=0 and Max=0  implies
    loading all the channels.

\end{verbatim}
\subsubsection{UV\_FILTER /VELOCITY}
\index{UV\_FILTER!/VELOCITY}
\begin{verbatim}
        [CLEAN\]UV_FILTER /VELOCITY V1 [... [Vn]] [/WIDTH Width [TYPE]]

    Specify  around  which  velocities the line emission should be filtered.
    Velocities V1 to Vn must be in km/s. The full  width  of  the  filtering
    window around every frequency can be set by option /WIDTH.  The optional
    argument TYPE indicates the type of width: FREQUENCY (in MHz),  VELOCITY
    (in  km/s)  or  CHANNEL (no unit), the default being FREQUENCY.  The de-
    fault width is the current channel width.

\end{verbatim}
\subsubsection{UV\_FILTER /WIDTH}
\index{UV\_FILTER!/WIDTH}
\begin{verbatim}
        [CLEAN\]UV_FILTER [/ZERO] /FREQUENCY  F1  [...  [Fn]]  /WIDTH  Width
    [TYPE]

        [CLEAN\]UV_FILTER  [/ZERO]  /VELOCITY  V1  [...  [Vn]]  /WIDTH Width
    [TYPE]

    Specify the full width of the filtering window  around  every  frequency
    given  in  the  /FREQUENCY option or any velocity given in the /VELOCITY
    option.  The optional argument TYPE indicates the type  of  width:  FRE-
    QUENCY  (in  MHz),  VELOCITY (in km/s) or CHANNEL (no unit), the default
    being FREQUENCY.  The default width is the current channel width.

\end{verbatim}
\subsubsection{UV\_FILTER /ZERO}
\index{UV\_FILTER!/ZERO}
\begin{verbatim}
        [CLEAN\]UV_FILTER /ZERO [/CHANNELS Channel_List]
      [/FREQUENCY F1 [... [Fn]] [/VELOCITY V1 [... [Vn]]
      [/RANGE Min Max [TYPE]] [/WIDTH Width [TYPE]]

    Erase filtered channels (set weight and  visibilities  to  zero)  rather
    than  simply  flagging  them (weight set to negative value). This can be
    more convenient for further display, but is not reversible.

    By default, channels to be filtered are flagged (weights becoming  nega-
    tive,  and  data can be unflagged by UV_FLAG).  Flagged channels are ig-
    nored in the averaging process, such as UV_RESAMPLE or UV_CONT.   Howev-
    er,  as  UV_MAP (in general) uses only one channel to define the weights
    for all others, these flagged channels will nevertheless appear  in  the
    imaged data cube.

    See UV_CHECK for information about handling flagged channels and differ-
    ent beams in a single UV table.


\end{verbatim}
\subsection{UV\_FLAG}
\index{UV\_FLAG}
\begin{verbatim}
        [CLEAN\]UV_FLAG [/RESET]

    Display UV data and calls the cursor to interactively  select  a  region
    where  UV  data will be flagged (sign of weight is reversed). The /RESET
    option is used to unflag the data. UV data flagged using command UV_FLAG
    can  be  saved  on file with command WRITE UV File. Detailed information
    about the display control of the UV data may be  found  in  the  UV_SHOW
    help.

    The  user  can  control the algorithm through variables. Values of those
    variables can be checked using "EXAMINE VARIABLE".  New  values  can  be
    given using "LET VARIABLE value".

    DATE_START  [    ] The date of the first data to be flagged
    UT_START    [    ] The UT time of the first data to be flagged
    DATE_END    [    ] The date of the last data to be flagged
    UT_END      [    ] The time of the last data to be flagged
    BASELINE    [    ] Baseline to flagged
    CHANNEL     [    ] Channel range to be flagged/unflagged
    FLAG        [    ] Flag or unflag the specified time range

    Subsequent  mapping  with  UV_MAP will ignore flagged data. However, for
    multi-channel imaging, the weight column is (by default) taken from  one
    channel,  so  that  channel-based flagging will not be recognized in the
    default mode, but only in the "One Beam per Channel" mode.

\end{verbatim}
\subsubsection{UV\_FLAG DATE\_START}
\index{UV\_FLAG!DATE\_START}
\begin{verbatim}

    The date (DD-MMM-YYYY) of the first data to be flagged.

\end{verbatim}
\subsubsection{UV\_FLAG UT\_START}
\index{UV\_FLAG!UT\_START}
\begin{verbatim}

    The UT time (hh:mm:ss.ss) of the first data to be flagged.

\end{verbatim}
\subsubsection{UV\_FLAG DATE\_END}
\index{UV\_FLAG!DATE\_END}
\begin{verbatim}

    The date (DD-MMM-YYYY) of the last data to be flagged.

\end{verbatim}
\subsubsection{UV\_FLAG UT\_END}
\index{UV\_FLAG!UT\_END}
\begin{verbatim}

    The time (hh:mm:ss.ss) of the last data to be flagged.

\end{verbatim}
\subsubsection{UV\_FLAG BASELINE}
\index{UV\_FLAG!BASELINE}
\begin{verbatim}

    Baseline (ALL, 12, 13, 23, ...) to flagged.

\end{verbatim}
\subsubsection{UV\_FLAG FLAG}
\index{UV\_FLAG!FLAG}
\begin{verbatim}

    Flag (T) or unflag (F) the specified time range.

\end{verbatim}
\subsubsection{UV\_FLAG CHANNEL}
\index{UV\_FLAG!CHANNEL}
\begin{verbatim}

    Channel range to be flagged/unflagged.

\end{verbatim}
\subsection{UV\_MAP}
\index{UV\_MAP}
\begin{verbatim}
        [CLEAN\]UV_MAP [CenterX CenterY UNIT  [Angle]]  [/FIELDS  FieldList]
    [/TRUNCATE Percent]

    Compute  a  dirty  map  and  beam from a UV data. UV data must have been
    loaded from a UV table by command "READ UV File". UV_MAP processes  sin-
    gle fields as well as Mosaics.

    The user can control the algorithm through SIC variables. New values can
    be given using "LET VARIABLE value". For ease of use, and whenever it is
    possible,  a sensible value of each parameter will automatically be com-
    puted from the context if the value of the corresponding variable is set
    to  its default value, i.e. zero value and empty string. A few variables
    are initialized to "reasonable" values.

        [CLEAN\]UV_MAP ?
    Will list all MAP_* variables controlling the UV_MAP parameters

\end{verbatim}
\subsubsection{UV\_MAP Mosaics}
\index{UV\_MAP!Mosaics}
\begin{verbatim}
        [CLEAN\]UV_MAP [CenterX  CenterY  UNIT  [Angle]]  /FIELDS  FieldList
    [/TRUNCATE Percent]

    The  UV  data  can be a Mosaic UV table. In this case, UV_MAP will image
    the Mosaic, using appropriate primary beam size and truncation level.

    By default, the primary beam size is taken from  the  telescope  parame-
    ters,  either  as found in the telescope section of the UV table and the
    observing frequency (e.g. for ALMA it uses 1.13  Lambda/D).  If  absent,
    the  telescope section information can be added by command SPECIFY TELE-
    SCOPE.

    The truncation level is taken from variable  MAP_TRUNCATE  or  from  the
    /TRUNCATE option argument.


\end{verbatim}
\subsubsection{UV\_MAP /FIELDS}
\index{UV\_MAP!/FIELDS}
\begin{verbatim}
        [CLEAN\]UV_MAP  [CenterX  CenterY  UNIT  [Angle]]  /FIELDS FieldList
    [/TRUNCATE Percent]

    For a Mosaic, only image the fields specified in a 1-D Integer  variable
    FieldList.   SHOW  FIELDS  will highlight the list of fields selected by
    this command.

\end{verbatim}
\subsubsection{UV\_MAP /TRUNCATE}
\index{UV\_MAP!/TRUNCATE}
\begin{verbatim}
        [CLEAN\]UV_MAP [CenterX  CenterY  UNIT  [Angle]]  /FIELDS  FieldList
    /TRUNCATE Percent

    For  a Mosaic, truncate the primary beam to the specified level (in per-
    cent). SHOW FIELDS will use this level to show the beams.   The  default
    is to use MAP_TRUNCATE.

\end{verbatim}
\subsubsection{UV\_MAP Variables:}
\index{UV\_MAP!Variables:}
\begin{verbatim}
                [CLEAN\]UV_MAP ?
        Will list all MAP_* variables controlling the UV_MAP parameters.

    The  list  of  control  variables  is (by alphabetic order, with the old
    names used by Mapping on the right)
    New names       [   unit]       -- Description --    % Old Name
    MAP_BEAM_STEP   [       ]  Number of channels per single dirty beam
    MAP_CELL        [ arcsec]  Image pixel size
    MAP_CENTER      [ string]  RA, Dec of map center, and Position Angle
    MAP_CONVOLUTION [       ]  Convolution function    % CONVOLUTION
    MAP_FIELD       [ arcsec]  Map field of view
    MAP_POWER       [       ]  Maximum exponent of 3 and 5 allowed in MAP_SIZE
    MAP_PRECIS      [       ]  Fraction of pixel tolerance on beam matching
    MAP_ROBUST      [       ]  Robustness factor        % UV_CELL[2]
    MAP_ROUNDING    [       ]  Precision of MAP_SIZE
    MAP_SIZE        [       ]  Number of pixels
    MAP_TAPEREXPO   [       ]  Taper exponent           % TAPER_EXPO
    MAP_TRUNCATE    [      %]  Mosaic truncation level
    MAP_UVCELL      [      m]  UV cell size             % UV_CELL[1]
    MAP_UVTAPER     [m,m,deg]  Gaussian taper           % UV_TAPER
    MAP_VERSION     [       ]  Code version (0 new, -1 old)

    NAME is no longer used, and WEIGHT_MODE is obsolete.
    MAP_RA          [  hours]  RA of map center
    MAP_DEC         [    deg]  Dec of map center
    MAP_ANGLE       [    deg]  Map position angle
    MAP_SHIFT       [Yes/No ]  Shift phase center
    are obsolescent, superseded by MAP_CENTER. They are  provided  only  for
    compatibility with older scripts.

\end{verbatim}
\subsubsection{UV\_MAP MAP\_BEAM\_STEP}
\index{UV\_MAP!MAP\_BEAM\_STEP}
\begin{verbatim}

      MAP_BEAM_STEP   Integer

    Number of channels per synthesized beam plane.

    Default  is 0, meaning only 1 beam plane for all channels.  N (>0) indi-
    cates N consecutive channels will share the same dirty beam.

    A value of -1 can be used to compute the number  of  channels  per  beam
    plane  to ensure the angular scale does not deviate more than a fraction
    of the map cell at the map edge. This fraction is controlled by variable
    MAP_PRECIS (default 0.1)

\end{verbatim}
\subsubsection{UV\_MAP MAP\_CELL}
\index{UV\_MAP!MAP\_CELL}
\begin{verbatim}

          MAP_CELL[2]    Real

    The  map  pixel size [arcsec]. It is recommended to use identical values
    in X and Y.  A sampling of at least 3 pixel per beam is  recommended  to
    ease  the deconvolution. Enter 0,0 to let the task find the best values.

\end{verbatim}
\subsubsection{UV\_MAP MAP\_CENTER}
\index{UV\_MAP!MAP\_CENTER}
\begin{verbatim}

          MAP_CENTER     Character String

    Specify the Map center and orientation in the same way as the  arguments
    of UV_MAP.

\end{verbatim}
\subsubsection{UV\_MAP MAP\_CONVOLUTION}
\index{UV\_MAP!MAP\_CONVOLUTION}
\begin{verbatim}

        MAP_CONVOLUTION    Integer

    Select the desired convolution function for gridding  in  the  UV  plane
    Choices are
            0    Default (currently 5)
            1    Boxcar
            2    Gaussian
            3    Sin(x)/x
            4    Gaussian * Sin(x)/x
            5    Spheroidal
    Spheroidal functions is the optimal choice. So  we  strongly  discourage
    use of any other convolution function, which are here for tests only.

\end{verbatim}
\subsubsection{UV\_MAP MAP\_FIELD}
\index{UV\_MAP!MAP\_FIELD}
\begin{verbatim}

      MAP_FIELD[2]     Real

    Field  of  view  in  X and Y in arcsec.  The field of view MAP_FIELD has
    precedence over the number of pixels MAP_SIZE to define the  actual  map
    size when both are non-zero.

\end{verbatim}
\subsubsection{UV\_MAP MAP\_POWER}
\index{UV\_MAP!MAP\_POWER}
\begin{verbatim}

          MAP_POWER[2]     Integer

    Maximum  exponent  of  3  and  5 allowed in automatic guess of MAP_SIZE.
    MAP_SIZE is decomposed in 2^k 3^p 5^q, and p and q must be less or equal
    to MAP_POWER.

    Default is 0: MAP_SIZE is just a power of 2. A value of 1 allows approx-
    imation of any map size to 20 %, while a value of 2 allows 10 % approxi-
    mation.  Fast Fourier Transform are slightly slower with powers of 3 and
    5, but limiting the map size can gain a  lot  in  the  Cleaning  process
    (which can scale as MAP_SIZE^4).


\end{verbatim}
\subsubsection{UV\_MAP MAP\_PRECIS}
\index{UV\_MAP!MAP\_PRECIS}
\begin{verbatim}

      MAP_PRECIS    Real

    Maximum  mismatch in pixel at map edge between the true synthesized beam
    (which would have been computed using the exact channel  frequency)  and
    the  computed  synthesized beam with  the mean frequency of the channels
    sharing the same beam. This is used (with the actual image size) to  de-
    rive  the  actual number of channels which can share the same beam, i.e.
    the effective value of MAP_BEAM_STEP when MAP_BEAM_STEP is -1.

    Default is 0.1

\end{verbatim}
\subsubsection{UV\_MAP MAP\_ROBUST}
\index{UV\_MAP!MAP\_ROBUST}
\begin{verbatim}

      MAP_ROBUST     Real

    Robust weighting factor. A number between 0 and +infty.

    Robust weighting gives the natural weight  to  UV  cells  whose  natural
    weight  is  lower  than  a  given threshold. In contrast, if the natural
    weight of the UV cell is larger than this threshold, the weight  is  set
    to  this  (uniform) threshold. The UV cell size is defined by MAP_UVCELL
    and the threshold value is in MAP_ROBUST.

    0 means natural weighting, which is optimal for point sources.  The  Ro-
    bust  weighting factor controls the resolution: better resolution is ob-
    tained for small values (at the expense of noise), resolution  approach-
    ing  the natural weighting scheme for large values.  Larger UV cell size
    give higher angular resolution (but again more noise).

    MAP_ROBUST around .5 to 1 is a good compromise  between  noise  increase
    and angular resolution.

\end{verbatim}
\subsubsection{UV\_MAP MAP\_ROUNDING}
\index{UV\_MAP!MAP\_ROUNDING}
\begin{verbatim}

      MAP_ROUNDING     Real

    Maximum  error  between  optimal size (MAP_FIELD / MAP_CELL) and rounded
    (as a power of 2^k 3^p 5^q) MAP_SIZE to round by  floor  (thus  limiting
    the  field of view), instead of ceiling (which guarantees a larger field
    of view, but leads to bigger images).

    Default is 0.05.


\end{verbatim}
\subsubsection{UV\_MAP MAP\_SHIFT}
\index{UV\_MAP!MAP\_SHIFT}
\begin{verbatim}

      MAP_SHIFT        Logical

    Obsolescent, superseded by MAP_CENTER, or the UV_MAP arguments.

    Logical variable indicating whether map center (i.e. phase tracking cen-
    ter) or orientation should be changed.

\end{verbatim}
\subsubsection{UV\_MAP MAP\_SIZE}
\index{UV\_MAP!MAP\_SIZE}
\begin{verbatim}

      MAP_SIZE[2]      Integer

    Number of pixels in X and Y. It should preferentially be a power of two,
    (although  this  is  not  strictly  required)  to  speed-up   the   FFT.
    MAP_SIZE*MAP_CELL should be at least twice the size of the field-of-view
    (primary beam size for a single field). Enter 0,0  to  let  the  command
    find a sensible map size.

    MAP_SIZE is not used if MAP_FIELD is non zero.

    Odd values are forbidden.

    Default is 0,0, i.e. UV_MAP will guess the most appropriate values which
    depend on MAP_ROUNDING and MAP_POWER.


\end{verbatim}
\subsubsection{UV\_MAP MAP\_TAPEREXPO}
\index{UV\_MAP!MAP\_TAPEREXPO}
\begin{verbatim}

      MAP_TAPEREXPO    Real

    Taper exponent. The default is 2 (indicating a Gaussian) but smoother or
    sharper  functions  can be used. 1 would give an Exponential, 4 would be
    getting close to square profile...

\end{verbatim}
\subsubsection{UV\_MAP MAP\_TRUNCATE}
\index{UV\_MAP!MAP\_TRUNCATE}
\begin{verbatim}

      MAP_TRUNCATE    Real

    Mosaic truncation level in PerCent.  Default value is 0.2. Current value
    can be overriden by option /TRUNCATE in commands UV_MAP or PRIMARY.

\end{verbatim}
\subsubsection{UV\_MAP MAP\_UVTAPER}
\index{UV\_MAP!MAP\_UVTAPER}
\begin{verbatim}

      MAP_UVTAPER[3]  Real

    Parameters of the tapering function (Gaussian if MAP_TAPEREXPO = 2): ma-
    jor axis at 1/e level [m], minor axis at 1/e level [m], and position an-
    gle [deg].

\end{verbatim}
\subsubsection{UV\_MAP MAP\_UVCELL}
\index{UV\_MAP!MAP\_UVCELL}
\begin{verbatim}

      MAP_UVCELL   Real

    UV  cell  size for robust weighting [m].  Should be of the order of half
    the dish diameter (7.5 m for PdBI), or smaller or even larger.  It  con-
    trols the beam shape in Robust weighting.

\end{verbatim}
\subsubsection{UV\_MAP MAP\_VERSION}
\index{UV\_MAP!MAP\_VERSION}
\begin{verbatim}

      MAP_VERSION  Integer

    [EXPERT Only] Code indicating which version of the UV_MAP and UV_RESTORE
    algorithm  should  be  used.  0  is  optimal.  -1  is  the  "historical"
    (pre-2016)  version. 1 is an intermediate version used during multi-fre-
    quency beams development.

\end{verbatim}
\subsubsection{UV\_MAP MCOL}
\index{UV\_MAP!MCOL}
\begin{verbatim}

      MCOL[2]   Integer

    First and Last channel to image.  Values  of  0  mean  imaging  all  the
    planes.

\end{verbatim}
\subsubsection{UV\_MAP WCOL}
\index{UV\_MAP!WCOL}
\begin{verbatim}

      WCOL      Integer

    [Obsolescent]  The  channel from which the weight should be taken.  WCOL
    set to 0 means using a default channel. WCOL has no real meaning in  all
    cases where more than one beam is computed for all channels.


\end{verbatim}
\subsubsection{UV\_MAP Old\_Names:}
\index{UV\_MAP!Old\_Names:}
\begin{verbatim}
     NAME        [       ]  Label of the dirty image and beam plots
     UV_TAPER    [m,m,deg]  UV-apodization by convolution with a Gaussian
     WEIGHT_MODE [       ]  Weighting mode (NA|UN)
     UV_CELL     [m, ??  ]  UV cell size and threshold for Robust weighting
     MAP_FIELD   [ arcsec]  Map field of view
     MAP_CELL    [ arcsec]  Map cell size
     MAP_SIZE    [ pixels]  Map size in pixels (if MAP_FIELD is zero)
     MCOL        [       ]  First and Last channel to map
     WCOL        [       ]  Channel from which the weights are taken
     CONVOLUTION [       ]  Convolution function (5)
     UV_SHIFT    [       ]  Change the map phase center or map orientation?
     MAP_RA      [       ]  RA of map phase center
     MAP_DEC     [       ]  Dec of map phase center
     MAP_ANGLE   [    deg]  Map position angle
     MAP_BEAM_STEP [     ]  Number of channels per synthesized beam plane

\end{verbatim}
\subsubsection{UV\_MAP convolution}
\index{UV\_MAP!convolution}
\begin{verbatim}

      Older variable name for MAP_CONVOLUTION

\end{verbatim}
\subsubsection{UV\_MAP map\_angle}
\index{UV\_MAP!map\_angle}
\begin{verbatim}

      MAP_ANGLE      Real

    Position  Angle of the direction which will become the apparent North in
    the map. Used only if UV_SHIFT is YES.

    Superseded by MAP_CENTER.

\end{verbatim}
\subsubsection{UV\_MAP map\_dec}
\index{UV\_MAP!map\_dec}
\begin{verbatim}

      MAP_DEC     Real

    Dec of map center. Used only if UV_SHIFT is YES.

    Superseded by MAP_CENTER.

\end{verbatim}
\subsubsection{UV\_MAP map\_ra}
\index{UV\_MAP!map\_ra}
\begin{verbatim}

      MAP_RA      Real

    RA of map center. Used only if UV_SHIFT is YES.

    Superseded by MAP_CENTER.

\end{verbatim}
\subsubsection{UV\_MAP uv\_cell}
\index{UV\_MAP!uv\_cell}
\begin{verbatim}

      Older variables for MAP_UVCELL (uv
\end{verbatim}
\subsubsection{UV\_MAP uv\_shift}
\index{UV\_MAP!uv\_shift}
\begin{verbatim}

      Older variable name of MAP_SHIFT (this one is also obsolescent)

\end{verbatim}
\subsubsection{UV\_MAP uv\_taper}
\index{UV\_MAP!uv\_taper}
\begin{verbatim}

      Older variable name of MAP_UVTAPER

\end{verbatim}
\subsubsection{UV\_MAP taper\_expo}
\index{UV\_MAP!taper\_expo}
\begin{verbatim}

      Older variable name for MAP_TAPEREXPO

\end{verbatim}
\subsubsection{UV\_MAP weight\_mode}
\index{UV\_MAP!weight\_mode}
\begin{verbatim}

      weightde      Character

    Weighting mode: Natural (optimum in  terms  of  sensitivity)  or  robust
    (usually lower sidelobes and higher spatial resolution) weighting.  This
    was needed in Mapping to toggle between Natural  and  Robust  weighting,
    while IMAGER does that based on MAP_ROBUST value.







\end{verbatim}
\subsection{UV\_RESAMPLE}
\index{UV\_RESAMPLE}
\begin{verbatim}
        [CLEAN\]UV_RESAMPLE [Nc Ref Val Inc]

    Resample  the  UV  data loaded by READ UV on a different velocity scale.
    All other UV commands except UV_COMPRESS work on the "Resampled" UV  ta-
    ble.
         Nc   new number of channels
         Ref  New reference pixel
         Val  New velocity at reference pixel
         Inc  Velocity increment
    Any argument can be set to * for an automatic determination based on the
    values of the other arguments. The automatic determination  of  NC  pre-
    serves the velocity coverage.

    The  "Resampled"  UV  table is a simple copy of the original one after a
    READ UV command, or after a UV_RESAMPLE or UV_COMPRESS  command  without
    arguments.

\end{verbatim}
\subsection{UV\_RESIDUAL}
\index{UV\_RESIDUAL}
\begin{verbatim}
        [CLEAN\]UV_RESIDUAL [Niter]

      Subtract  the  Niter  first (default all) Clean Components from the UV
    data.  The residual UV data can be written by WRITE UV,  and  imaged  by
    UV_MAP.

\end{verbatim}
\subsection{UV\_RESTORE}
\index{UV\_RESTORE}
\begin{verbatim}
        [CLEAN\]UV_RESTORE

    Create  a  Clean image from the current UV data set and the Clean Compo-
    nent list.  The Clean Components are subtracted from the  UV  data  set,
    and  these  residuals are gridded and Fourier transformed to compute the
    Residual image. This Residual image is added to the Gaussian  beam  con-
    volved  image of the sum of Clean components. The results are similar to
    those of MX, since only the residual are aliased.

    This command can be used after HOGBOM, CLARK, MULTI, SDI or MX (although
    it is pointless after MX), but not MRC which has no notion of Clean Com-
    ponents.

\end{verbatim}
\subsection{UV\_REWEIGHT}
\index{UV\_REWEIGHT}
\begin{verbatim}
        [CLEAN\]UV_REWEIGHT Scale

    Scale the weights of the current UV data with the defined Scale  factor.
    Can  be  used to patch e.g. JVLA data files which may have only relative
    weights, not absolute values indicating the noise.

\end{verbatim}
\subsection{UV\_SHIFT}
\index{UV\_SHIFT}
\begin{verbatim}
        [CLEAN\]UV_SHIFT  [CenterX CenterY UNIT [Angle]]

    Shift the current UV table (single field or mosaic) to  a  common  phase
    center.  If no argument is specified, the string contained in MAP_CENTER
    is used instead.

    Although UV_MAP also provides a direct possibility to re-center the  im-
    age  on a specified projection (phase) center, the modified visibilities
    cannot be saved on file. It is sometimes required to do this for specif-
    ic use. UV_SHIFT provides this possibility, and the shifted UV table can
    be written using command WRITE UV.

    UV_DEPROJECT includes the UV_SHIFT capabilities, but also has additional
    functions.

\end{verbatim}
\subsection{UV\_SORT}
\index{UV\_SORT}
\begin{verbatim}
        [CLEAN\]UV_SORT TIME|BASE

    Sort and transpose the UV data set, loaded by command READ UV File.  Or-
    der is either TIME for Time-Baseline ordering,  BASE  for  Baseline-Time
    ordering.  The sorted UV data is then available in variable UVS for fur-
    ther plotting.

    NOTE: This is only done in an internal buffer.  WRITE  UV  will  **NOT**
    write this sorted, transposed, buffer.

\end{verbatim}
\subsection{UV\_STAT}
\index{UV\_STAT}
\begin{verbatim}
        [CLEAN\]UV_STAT CELL|HEADER|SETUP|TAPER|WEIGHT  [Step Start]

    UV_STAT  allows  the astronomer to select the best weighting and imaging
    parameters according to its personal trade off between  angular  resolu-
    tion, sensitivity and field of view.

      Default is HEADER+SETUP

\end{verbatim}
\subsubsection{UV\_STAT CELL}
\index{UV\_STAT!CELL}
\begin{verbatim}
        [CLEAN\]UV_STAT CELL [Step Start]

    Predict the synthesized beam, expected noise level, and recommended pix-
    el size for different values of  the uv cell  size  for  current  robust
    weighting parameters.

\end{verbatim}
\subsubsection{UV\_STAT HEADER}
\index{UV\_STAT!HEADER}
\begin{verbatim}
        [CLEAN\]UV_STAT HEADER

    Display  a  brief  summary of the UV data: number of antennas, observing
    dates, baseline ranges,  spectroscopic information.

\end{verbatim}
\subsubsection{UV\_STAT SETUP}
\index{UV\_STAT!SETUP}
\begin{verbatim}
        [CLEAN\]UV_STAT SETUP

    Display recommended values for the  imaging:  image  size,  pixel  size,
    field of view, largest angular scale, etc...

\end{verbatim}
\subsubsection{UV\_STAT TAPER}
\index{UV\_STAT!TAPER}
\begin{verbatim}
        [CLEAN\]UV_STAT TAPER [Step Start]

    For  TAPER,  beam sizes and noise level (in flux and brightness) will be
    computed for 9 different tapers (from Start to  Start*Step^9).   Default
    value  for  Step  is sqrt(2), Default value for Start is 50 m. Weighting
    mode, UV cell size  and  "robust"  parameter  are  taken  from  variable
    UV_CELL (i.e. one can combine Robust weighting and Tapering).

\end{verbatim}
\subsubsection{UV\_STAT WEIGHT}
\index{UV\_STAT!WEIGHT}
\begin{verbatim}
        [CLEAN\]UV_STAT WEIGHT [Step Start]

    For  WEIGHT, beam sizes and noise level (in flux and brightness) will be
    computed for 9 different "robust" weighting parameters  (from  Start  to
    Start*Step^9). Default value for Step is sqrt(10), and default value for
    Start is derived to center the "robust" parameter values around  1.   UV
    cell  size  is  taken  from variable UV_CELL[1], and Taper is taken from
    variable UV_TAPER.

\end{verbatim}
\subsection{UV\_TIME}
\index{UV\_TIME}
\begin{verbatim}
        [CLEAN]UV_TIME [Time] [/Weight Wcol]

    Average in time the current UV data set to reduce the number of visibil-
    ities.  Time must be in seconds. If not specified, an automatic guess is
    performed based on antenna size and baseline lengths.

\end{verbatim}
\subsubsection{UV\_TIME /WEIGHT}
\index{UV\_TIME!/WEIGHT}
\begin{verbatim}
        [CLEAN]UV_TIME Time /WEIGHT Wcol

    Select the weight column. Default is 0.

\end{verbatim}
\subsection{UV\_TRUNCATE}
\index{UV\_TRUNCATE}
\begin{verbatim}
        [CLEAN]UV_TRUNCATE Max [Min]

    Truncate the UV data, by removing baselines out of the  specified  range
    Min (default 0) and Max (in meter).

\end{verbatim}
\subsection{VIEW}
\index{VIEW}
\begin{verbatim}
        [CLEAN\]VIEW Variable [FirstPlane [LastPlane]]

    where
        is  the  internal  buffer  to  be  plotted (BEAM, CCT, CLEAN, DIRTY,
    FIELDS, MASK, etc.. ), or any Sic Image variable.


        [CLEAN\]VIEW ?

    will list the names of recognized keywords. If Variable is  not  one  of
    the recognized keywords, but an existing Image variable, this image will
    be displayed.

    FirstPlane and LastPlane
        are optional arguments to restrict the range of channels to be plot-
    ted  (default:  planes between variables FIRST and LAST). If only First-
    Plane is specified, SHOW only that plane.

    The plot is done by procedure p_view_cct for Clean  Components  CCT  and
    p_view_map  for data cubes. For data cubes, the behaviour is the same as
    GO VIEW with NAME set to the appropriate buffer name.

    VIEW is also controlled by a set of variables, available  in  the  View%
    global structure

\end{verbatim}
\subsubsection{VIEW Variables}
\index{VIEW!Variables}
\begin{verbatim}

    User control variables
    view%expand       0.8 Character and Tick expansion factor
    view%contour      NO  Contour the current channel map
    view%movie        0   Elapsed time in seconds for a movie (0 means guess)
    view%side         YES Display values in side Window
    view%status%rima  NO  Relative coordinates in Images
    view%status%rspe  NO  Relative coordinates in Spectra

    Returned  values  are  found  in view%current structure. The ZOOM action
    produces a sub-cube named EXTRACTED.  The SLICE action  produces  a  2-D
    data  set  named  SLICE.  As any other SIC Image variable, EXTRACTED and
    SLICE can be written on disk using command WRITE.

\end{verbatim}
\subsubsection{VIEW Keys}
\index{VIEW!Keys}
\begin{verbatim}

    Position-independent actions:
      Press E key: EXIT loop
      Press H key: HELP display
      Press M key: MOVIE
      Press P key: PRINT plot in "ha" subdirectory
      Press Q key: QUIT loop
      Press X key: EXTRACT on disk current zoomed region
      Press N key: Toggle  Narrow - Wide mode

    Cursor on images:
      Left  clic: Display spectrum at pointed position
      Right clic: Define a polygon
      Press C key: COORDINATES toggled from absolute to relative and back
      Press S key: SLICE definition (velocity-position)
      Press K key: KILL   pointed pixel
      Press U key: UNKILL pointed pixel
      Press Z key: ZOOM defined spatial region
      Press B key: BACK to full field of view
      Press V key: Display map coordinates at current position (the associated b
      Press W key: WRITE Integrated Area image
    Cursor on spectra:
      Left clic in Selected   Spectrum: Display selected velocity channel
      Left clic in Integrated Spectrum: Define velocity range
      Press C key: COORDINATES toggled from freq/velo to channels and back
      Press Z key: ZOOM defined velocity region
      Press B key: BACK to full velocity range
      Press W key: WRITE Integrated (and current) Spectrum
    Cursor outside plots:
      Press B key: BACK to full velocity range AND full field of view

    ZOOM action produces a sub-cube named EXTRACTED.  SLICE action  produces
    a  2-D  data set named SLICE. As any other SIC Image variable, EXTRACTED
    and SLICE can be written on disk using command WRITE.


\end{verbatim}
\subsection{WRITE}
\index{WRITE}
\begin{verbatim}
        [CLEAN\]WRITE Name File [/APPEND] [/RANGE Start End Kind] [/REPLACE]
    [/TRIM]

    WRITE  the  buffer  specified  by  Name  (UV, CGAINS, and BEAM, PRIMARY,
    DIRTY, CLEAN, RESIDUAL, SUPPORT, CCT, SKY) onto a File.  Default  exten-
    sions  are  .uvt  for  UV and CGAINS and .beam, .lmv, .lobe, .lmv-clean,
    .lmv-res, .pol, .cct, and lmv-sky respectively for  the  next  ones.  If
    Name does not refer to a known buffer, but to a SIC Image variable, this
    variable is written. The default extension is then .gdf.

    For UV data, the flagged data are written by default. They may be  omit-
    ted using the /TRIM option.

    WRITE  *  File can be used to write all modified image-like buffers (not
    the UV tables) under a common File name. This typically  include  .beam,
    .lmv  and  .lmv-clean, but also the .lmv-sky file if the PRIMARY command
    has been used after deconvolution.

\end{verbatim}
\subsubsection{WRITE /RANGE}
\index{WRITE!/RANGE}
\begin{verbatim}
        [CLEAN\]WRITE Buffer File /RANGE Start End Kind [/REPLACE]

    A range of image planes can be specified through /RANGE option. Kind  is
    the unit of the Start-End values: CHANNEL, VELOCITY, or FREQUENCY.

    Restrictions:
    - The option is still experimental
    - The Buffer name must be specified (* is not valid here)
    - This does not apply to UV data.

\end{verbatim}
\subsubsection{WRITE /APPEND}
\index{WRITE!/APPEND}
\begin{verbatim}
        [CLEAN\]WRITE Buffer File /APPEND [/RANGE Start End Kind]

    EXPERIMENTAL: The selected channels are appended to an existing file.

\end{verbatim}
\subsubsection{WRITE /REPLACE}
\index{WRITE!/REPLACE}
\begin{verbatim}
        [CLEAN\]WRITE Buffer File /REPLACE [/RANGE Start End Kind]

    EXPERIMENTAL: The selected channels are replaced in an existing file.

\end{verbatim}
\subsubsection{WRITE /TRIM}
\index{WRITE!/TRIM}
\begin{verbatim}
        [CLEAN\]WRITE UV File /TRIM

    Remove the flagged visibilities while writing.



\end{verbatim}


\newpage
\section{CALIBRATE Language Internal Help} \label{CLEANC}
\subsection{Language}
\index{Language}
\begin{verbatim}
    APPLY         : Apply gain solution to current UV Data
    SCALE_FLUX    : Adjust flux scale on a daily basis
    MODEL         : Compute a UV model from the Clean Components Table
    SOLVE         : Solve for complex gains using the UV model
    UV_SELF       : Build the Self Calibration UV Table and dirty image

\end{verbatim}
\subsection{APPLY}
\index{APPLY}
\begin{verbatim}
        [CALIBRATE\]APPLY [AMPLI|PHASE [gain]] [/FLAG]

      Apply  gain solution computed by MODEL and SOLVE (which are called im-
    plicitely by SELFCAL)  or obtained by READ CGAINS to the current UV  da-
    ta.  The optional arguments indicate whether this should be an AMPLITUDE
    or PHASE gain, and what gain value is used (in range 0 to 1).
      If no argument is given, the current SELF_MODE (see HELP  SELFCAL)  is
    used, and the gain is 1.0.


    The /FLAG option controls whether data without a valid gain solution are
    kept unchanged or flagged.

\end{verbatim}
\subsubsection{APPLY /FLAG}
\index{APPLY!/FLAG}
\begin{verbatim}

        [CALIBRATE\]APPLY [AMPLI|PHASE [gain]] /FLAG

      Apply gain solution (in AMPLITUDE or PHASE) and flag  data  without  a
    corresponding valid gain solution.


\end{verbatim}
\subsection{SCALE\_FLUX}
\index{SCALE\_FLUX}
\begin{verbatim}
        [CALIBRATE\]SCALE_FLUX Find|Apply|List|Calibrate [Args...]

    A  set  of commands to check flux calibration on a day to day basis.  It
    gives the ratio between the observed flux (in the current UV  data  set)
    and  the model flux for each separate period.  The Model flux can be de-
    rived from Clean Component Tables using the MODEL command, or read  from
    an outer file using READ MODEL.

    Error bars are approximate. The User-defined command SOLVE_FLUX performs
    a more accurate evaluation of the error, but is typically 50 times slow-
    er.

    SCALE_FLUX FIND [DateTolerance [UVmin UVmax]]
    determines,  by linear regression, the best scaling factor to match date
    by date the UV data set with the MODEL data set.  Separate  Periods  are
    defined  when Dates differ more than DateTolerance (default 1 day). Only
    data with baseline lengths in the range UVmin UVmax are  considered  for
    the regression (default all).

    SCALE_FLUX APPLY VarName
    apply  previously  determined  flux scale factors to the MODEL data set,
    previously read by READ MODEL. This is in general used only in an itera-
    tive  search  way,  e.g.  by  the user-defined command SOLVE_FLUX (which
    calls procedure solve_flux). The resulting UV data set  is  loaded  into
    the specified VarName SIC variable.

    SCALE_FLUX LIST
    print out dates, baselines and determined flux factors

    SCALE_FLUX CALIBRATE
    apply  previously  determined  flux scale factors to the current UV data
    set (i.e. divide the visibilities by the scaling factor  of  each  date,
    and  correct  the  weight  accordingly).  This may then be written using
    command WRITE UV .

    SCALE_FLUX SOLVE [DateTolerance [UVmin UVmax]]
    combines FIND and PRINT behaviours.

\end{verbatim}
\subsection{MODEL}
\index{MODEL}
\begin{verbatim}
        [CALIBRATE\]MODEL [MaxIter] [/MINVAL Value [Unit]]

    Compute visibilities on the current UV sampling  using  a  source  model
    made of the MaxIter first Clean Components, or of all pixel values above
    the given Value if /MINFLUX is present.

\end{verbatim}
\subsubsection{MODEL /MINVAL}
\index{MODEL!/MINVAL}
\begin{verbatim}
        [CALIBRATE\]MODEL [MaxIter] /MINVAL Value [Unit]

    Construct the source model using all Clean Components until MaxIter (all
    if  MaxIter  is  0  or not specified). These components are stacked on a
    grid, and then all pixels above the given Value are taken as source mod-
    el to derive visibilities.

    Unit can be Jy, mJy, K or sigma. The default value is Jy.

\end{verbatim}
\subsection{SOLVE}
\index{SOLVE}
\begin{verbatim}
        [CALIBRATE\]SOLVE Time SNR [Reference]
          /MODE [Phase|Amplitude] [Antenna|Baseline] [Flag|Keep]

    Solve  the baseline or antenna based gains using the current UV data and
    current MODEL.

    Time is the integration time for the solution.  SNR is the minimum  Sig-
    nal to Noise Ratio required to find a solution.

\end{verbatim}
\subsubsection{SOLVE /MODE}
\index{SOLVE!/MODE}
\begin{verbatim}
        [CALIBRATE\]SOLVE Time SNR [Reference]
          /MODE [Phase|Amplitude] [Antenna|Baseline] [Flag|Keep]

    Dependin on the /MODE arguments, the gains can be antenna-based or base-
    line-based, and include Phase or Amplitude, and data  without  solutions
    either KEEPed or FLAGged,


\end{verbatim}
\subsection{UV\_SELF}
\index{UV\_SELF}
\begin{verbatim}
        [CALIBRATE\]UV_SELF [CenterX CenterY UNIT [Angle]]
      [/RANGE [Min Max Type]] [/RESTORE]

    Use  (and  if  specified  and/or needed create) the "Self Calibrated" UV
    dataset to make a dirty image, instead of using the current UV table.

    UV_SELF utilizes UV_MAP for imaging. See HELP UV_MAP for parameters.

\end{verbatim}
\subsubsection{UV\_SELF /RANGE}
\index{UV\_SELF!/RANGE}
\begin{verbatim}
        [CALIBRATE\]UV_SELF [CenterX CenterY UNIT [Angle]] /RANGE  [Min  Max
    Type]

    Create and image the "Self Calibrated" UV data.

    The "Self Calibrated" UV dataset is created from the current UV data set
    by extracting the range of channels specified by  the  /RANGE  arguments
    Min   Max  Type.  Type can be CHANNEL, VELOCITY or FREQUENCY.  If /RANGE
    has no argument, all channels are averaged together.

    It is then updated by command SOLVE at each self-calibration loop.   See
    SOLVE and CLEAN\SELFCAL for details.

\end{verbatim}
\subsubsection{UV\_SELF /RESTORE}
\index{UV\_SELF!/RESTORE}
\begin{verbatim}
        [CALIBRATE\]UV_SELF /RESTORE

    As UV_RESTORE but for the self-calibrated UV table.

    Restores  the Clean image from the Clean Component Table by removing the
    components from the Self-calibrated UV data and  imaging  the  residuals
    before adding them to the convolved Clean components.

    See UV_RESTORE for details.


\end{verbatim}


\newpage
\section{ADVANCED Language Internal Help} \label{CLEANN}
\subsection{Language}
\index{Language}
\begin{verbatim}
    CATALOG       : Define the spectral line catalog(s)
    EXTRACT       : Extract a subset from a data cube
    FLUX          : Compute integrated flux from Support or Mask
    MAP_CONTINUUM : Determine the continuum image from a spectral cube
    MASK          : Define the MASK
    MFS           : Multi Frequency Synthesis (under development - not functiona
    MOMENTS       : Compute Moments of the SKY or CLEAN data cubes
    SELFCAL       : Perform a self calibration
    SLICE         : Extract a Slice of the specified data cube
    STOKES        : Extract one Stokes parameter from multi-polarization UV tabl
    UV_ADD        : Add some extra column to the current UV data
    UV_DEPROJECT  : Deproject UV data
    UV_FIT        : Fit UV data with simple functions
    UV_MERGE      : Merge (possibly many) UV tables
    UV_PREVIEW    : Quick look at the spectral aspects of the UV data
    UV_RADIAL     : Deproject and compute radial average of UV data
    UV_SHORT      : Compute and add short spacings to UV data


\end{verbatim}
\subsection{CATALOG}
\index{CATALOG}
\begin{verbatim}
        [ADVANCED\]CATALOG [FileName [.. [FileNameN]]

    Define  or list the current catalog(s) for spectral line identification.
    A catalog is a file in the LINEDB data format. See HELP LINEDB\ for  de-
    tails.  Without argument, or with argument ?, the command will just list
    the names of the current catalog(s).

    At initialization, IMAGER search for $HOME/imager.linedb as default cat-
    alog, then gag_data:imager.linedb if not found. If the specified catalog
    does not exist, no spectral line identification will be done.

    Direct use of the LINEDB\USE command is also possible to define the cat-
    alog(s).

\end{verbatim}
\subsubsection{CATALOG Default}
\index{CATALOG!Default}
\begin{verbatim}

        A default catalog can be created (in the local directory) by execut-
    ing once
      @ gag_data:imager-linedb.sic
    This may take a while, and even get stuck occasionally because  it  per-
    forms  network access to remote databases. You can later copy this cata-
    log to your $HOME. See HELP LINEDB\ to add or remove species  from  such
    catalogs.

\end{verbatim}
\subsection{EXTRACT}
\index{EXTRACT}
\begin{verbatim}
        [ADVANCED\]EXTRACT VarName blc1 blc2 blc3 trc1 trc2 trc3

      Extract the subset VarName[blc1:trc1,blc2:trc2,blc3:trc3] of the image
    variable VarName and put it in the EXTRACTED image variable.

\end{verbatim}
\subsection{FLUX}
\index{FLUX}
\begin{verbatim}
        [ADVANCED\]FLUX Cursor|Mask|Support

      Compute the integrated flux from the CLEAN image, using zones  defined
    either  by calling the Cursor, or by the current Support, or by the cur-
    rent Mask.

      If the CLEAN data is a 3-D cube, the integrated flux will be  a  spec-
    trum.

      For  FLUX  MASK,  if the Mask defines several separate zones, an inte-
    grated flux will be computed for each zone. Zones  are  ordered  by  de-
    creasing number of pixels.

      The  FLUX  results are available in the FLUX SIC structure, and can be
    displayed by SHOW FLUX  and also SHOW COMPOSITE.

\end{verbatim}
\subsubsection{FLUX Limitations}
\index{FLUX!Limitations}
\begin{verbatim}

        FLUX MASK does not yet recognize properly Frequency dependent Masks.

\end{verbatim}
\subsubsection{FLUX Results}
\index{FLUX!Results}
\begin{verbatim}

          The FLUX results are available in the FLUX SIC structure
    FLUX%NC                       Number of channels
    FLUX%FREQUENCIES[Flux%Nc]     Frequencies of the Channels
    FLUX%VELOCITIES[Flux%Nc]      Velocities of the Channels
    FLUX%NF                       Number of Fields (Zones)
    FLUX%VALUES[Flux%Nc,Flux%Nf]  Integrated Flux for each channel and Zone

    For Cursor or Support defined regions, the enclosing polygon is also ac-
    cessible:
    FLUX%NXY                      Number of polygon summits
    FLUX%X                        X coordinates of summits
    FLUX%Y                        Y coordinates of summits

\end{verbatim}
\subsection{MAP\_CONTINUUM}
\index{MAP\_CONTINUUM}
\begin{verbatim}
        [ADVANCED\]MAP_CONTINUUM [DIRTY|CLEAN [Threshold]]

      Compute  a  continuum  image  from  the  Dirty or Clean data set.  The
    derivation of the continuum follows the same algorithm than  in  UV_CON-
    TINUUM, using the integrated spectrum over the image.

\end{verbatim}
\subsection{MASK}
\index{MASK}
\begin{verbatim}
        [ADVANCED\]MASK [Key [Arguments ...]

      Handle the MASK buffer, which can be used to select regions for Clean-
    ing (see SUPPORT /MASK) or to mask any data cube.

      Without argument (or with argument INTERACTIVE), an  interactive  tool
    is used to manipulate the mask. Valid operations are
      ADD         Add a region to the mask
      APPLY       Apply the mask to a buffer
      CHECK       Check Clean and Mask consistency
      INITIALIZE  Initialize the Mask (2D or 3D)
      INTERACTIVE Interactively define mask planes Step by Step
      OVERLAY     Overlay the Mask to the Clean image
      READ        Read the Mask from a file
      REMOVE      Remove a region from the mask
      SHOW        Show the Mask (as SHOW MASK)
      THRESHOLD   Compute an automatic mask
      USE         Use the Mask in Clean (as SUPPORT /MASK)
      WRITE       Write the Mask on file (as WRITE MASK)

\end{verbatim}
\subsubsection{MASK Tricks}
\index{MASK!Tricks}
\begin{verbatim}
          The MASK variable is ReadOnly. Yet, the user may want to modify it
    directly by himself.  The following commands
      DEFINE ALIAS WMASK MASK /GLOBAL
      LET WMASK /STATUS WRITE
    will create an alias to MASK which can be written by the user  at  will.
    This method is actually used in the Interactive MASK tool.

\end{verbatim}
\subsubsection{MASK ADD}
\index{MASK!ADD}
\begin{verbatim}
        [ADVANCED\]MASK ADD Shape [Arguments ...]

      Add the specified Shape to the current mask.  Shape can be
      CIRCLE      Ox Oy Diameter
      ELLIPSE     Ox Oy Major Minor Angle
      RECTANGLE   Ox Oy Major Minor Angle
      POLYGON     File
    where  Ox  Oy  are  the  center of the shapes,  Major and Minor the axes
    length (= side lengths for the Rectangle...) and Angle the Position  An-
    gle of the Major axis.

    For the POLYGON, the cursor is called if no File argument is given; else
    the polygon is read from the specified File.

\end{verbatim}
\subsubsection{MASK APPLY}
\index{MASK!APPLY}
\begin{verbatim}
        [ADVANCED\]MASK APPLY SicVariable

    Apply the mask to the corresponding 3-D variable.

    The space dimensions must coincide (pixel per pixel), but  the  spectral
    axes  can  differ.  The command will select the appropriate Mask channel
    for each plane of the SicVariable.  If the Mask is 2-D only, it will  be
    applied to all planes.

\end{verbatim}
\subsubsection{MASK CHECK}
\index{MASK!CHECK}
\begin{verbatim}
        [ADVANCED\]MASK CHECK [SicVariable]

    Check  the  Mask consistency against the specified SicVariable.  The de-
    fault is against Clean.

\end{verbatim}
\subsubsection{MASK INITIALIZE}
\index{MASK!INITIALIZE}
\begin{verbatim}
        [ADVANCED\]MASK INITIALIZE 2D|3D

      Initialize an empty 2-D or 3-D mask. For a 2-D mask,  the  interactive
    tool will use the mean image as a background for the definition.

\end{verbatim}
\subsubsection{MASK INTERACTIVE}
\index{MASK!INTERACTIVE}
\begin{verbatim}
        [ADVANCED\]MASK INTERACTIVE [Nchan]

      Enter the interactive tool, moving Nchan channels at each
      Next or Previous button. Default is 1.

        [ADVANCED\]MASK
    is equivalent to
        [ADVANCED\]MASK INTERACTIVE 1

\end{verbatim}
\subsubsection{MASK OVERLAY}
\index{MASK!OVERLAY}
\begin{verbatim}
        [ADVANCED\]MASK OVERLAY

    Display a SHOW-like overlay of the Mask on top of the current image (de-
    fault is Clean, normally). All SHOW parameters apply.

\end{verbatim}
\subsubsection{MASK READ}
\index{MASK!READ}
\begin{verbatim}
        [ADVANCED\]MASK READ File.[msk]

      Read the Mask from a Gildas data-cube.

\end{verbatim}
\subsubsection{MASK REMOVE}
\index{MASK!REMOVE}
\begin{verbatim}
        [ADVANCED\]MASK REMOVE Shape [Arguments ...]

      Remove the specified Shape from the current Mask.  Shape can be
      CIRCLE      Ox Oy Diameter
      ELLIPSE     Ox Oy Major Minor Angle
      RECTANGLE   Ox Oy Major Minor Angle
      POLYGON     File
    where Ox Oy are the center of the shapes,   Major  and  Minor  the  axes
    length  (= side lengths for the Rectangle...) and Angle the Position An-
    gle of the Major axis.

    For the POLYGON, the cursor is called if no File argument is given; else
    the polygon is read from the specified File.

\end{verbatim}
\subsubsection{MASK SHOW}
\index{MASK!SHOW}
\begin{verbatim}
        [ADVANCED\]MASK SHOW

      Similar to SHOW MASK command, but useing a spacing equal
      to 0.5 to illustrate the boundaries.

\end{verbatim}
\subsubsection{MASK THRESHOLD}
\index{MASK!THRESHOLD}
\begin{verbatim}
        [ADVANCED\]MASK THRESHOLD [Raw Smooth [Length [Guard]]

    Define  a  Mask from thresholding the CLEAN image. If the CLEAN image is
    3-D, the Mask will be 3-D.   `

    The method involves a first thresholding, followed by a smoothing to ex-
    tend the support and a second thresholding.

    The algorithm to define the mask is controlled by 4 parameters.
        Raw
    Initial  threshold  (in units of CLEAN image noise) under which the mask
    is set to 0.  Default is 6 sigma. The noise is taken from  the  computed
    Clean  noise  (clean%gil%rms)  if defined, or from the theoretical noise
    (dirty%gil%noise) if not.
        Smooth
    Threshold under which the Mask is set to 0 after smoothing. The  default
    is 2 sigma.
        Length
    FWHM  of  the smoothing gaussian to derive the smooth mask from the ini-
    tial mask.  The default is the Clean beam major axis.
        Guard
    Size of the guard band at edges where the mask is set to zero, in  units
    of  image  size. The default is 0.18, i.e. the mask can extends a little
    more than the inner quarter.  This is to avoid the edges where sidelobes
    aliasing occurs.

\end{verbatim}
\subsubsection{MASK USE}
\index{MASK!USE}
\begin{verbatim}
        [ADVANCED\]MASK USE

      Use the MASK in Clean deconvolution.

      Equivalent to SUPPORT /MASK command.

\end{verbatim}
\subsubsection{MASK WRITE}
\index{MASK!WRITE}
\begin{verbatim}
        [ADVANCED\]MASK WRITE File

      Save the Mask on a File. Default extension is .msk.

      Equivalent to WRITE MASK command.


\end{verbatim}
\subsection{MFS}
\index{MFS}
\begin{verbatim}
        [ADVANCED\]MFS

      ** NOT OPERATIONAL -- UNDER DEVELOPMENT **

      Multi-frequency  synthesis,  allowing  to  account  for spectral index
    changes across the observed field of view for very wide bandwidths  con-
    tinuum imaging.

\end{verbatim}
\subsection{MOMENTS}
\index{MOMENTS}
\begin{verbatim}
        [ADVANCED\]MOMENTS  [/METHOD Mean|Peak|Parabolic]
                         [/RANGE Min Max TYPE] [/THRESHOLD Thre]


      Compute  the  4  main "moments" of a data cube, and return them as SIC
    Image variables.  The 4 moments are
    M_AREA   Integrated area over the velocity range
    M_PEAK   Peak brightness value
    M_VELO   Mean or peak velocity
    M_WIDTH  Weighted line width
    The options control the method, the selected part of the  cube  (default
    channels from FIRST to LAST), and the threshold for detection (default 3
    sigma).

    The resulting images can be displayed using  command  SHOW  MOMENTS  and
    saved using WRITE MOMENTS.
\end{verbatim}
\subsubsection{MOMENTS /METHOD}
\index{MOMENTS!/METHOD}
\begin{verbatim}
        [ADVANCED\]MOMENTS  /METHOD Mean|Peak|Parabola
                         [/RANGE Min Max TYPE] [/THRESHOLD Thre]

    Specify which method to be used to compute the velocity and peak values.
    The default method, MEAN, computes an intensity weighted  velocity,  and
    takes  the peak intensity at each pixel.  Method PEAK takes the velocity
    at the peak intensity. Method PARABOLA makes  a  parabolic  fit  over  3
    channels  around  the peak intensity channel to derive the velocity, and
    takes the peak value of this parabola for the peak brightness.

    MOMENTS works on either the CLEAN or the SKY brightness data cubes  (SKY
    has priority if both are defined).

\end{verbatim}
\subsubsection{MOMENTS /RANGE}
\index{MOMENTS!/RANGE}
\begin{verbatim}
        [ADVANCED\]MOMENTS /RANGE Min Max TYPE
          [/METHOD Mean|Peak|Parabola] [/THRESHOLD Thre]

    Specify the range of channels to be selected. TYPE can be VELOCITY, FRE-
    QUENCY or CHANNELS.  If not present, the range is defined by  FIRST  and
    LAST variables.

\end{verbatim}
\subsubsection{MOMENTS /THRESHOLDS}
\index{MOMENTS!/THRESHOLDS}
\begin{verbatim}
        [ADVANCED\]MOMENTS /THRESHOLD Thre
          [/METHOD Mean|Peak|Parabola] [/RANGE Min Max TYPE]

    Specify the minimum intensity of the input data cube to consider a given
    pixel,channel as valid. The default is 3 sigma. The unit is that of  in-
    put data cube.


\end{verbatim}
\subsection{SELFCAL}
\index{SELFCAL}
\begin{verbatim}

            SELFCAL  [?|AMPLI|APPLY|PHASE|SUMMARY|SHOW [Last [First]] [/WID-
    GET]

      Command to perform Self Calibration (even on spectral line data).  The
    solution is computed, saved and/or applied.

      The arguments control the action.

        SELFCAL ?        Lists the self calibration parameters
        SELFCAL AMPLI    Compute an Amplitude only self calibration
        SELFCAL APPLY    Apply the computed solution
        SELFCAL PHASE    Compute a Phase only self calibration
        SELFCAL SHOW []  Show the computed corrections
        SELFCAL SAVE     Save self calibration parameters in selfcal.last
        SELFCAL SUMMARY  Display the improvements in Noise & Dynamic range
                         and calibration status
\end{verbatim}
\subsubsection{SELFCAL /WIDGET}
\index{SELFCAL!/WIDGET}
\begin{verbatim}
            SELFCAL /WIDGET

    Activates  the  widget  interface  to  perform self calibration (even on
    spectral line data set). A continuum data  set  is  extracted  from  the
    specified  channel range, with all selected channels averaged to provide
    improved sensitivity to find a solution.

      The buttons control the action.
         AMPLI      Perform an Amplitude only self calibration
         PHASE      Perform a Phase only self calibration
         Continue   Continue execution when the script is in Pause
         APPLY      Apply the solution
         INPUT      List parameters (as in SELFCAL ?)
         SAVE       Save the parameters in selfcal.last
         SHOW       Show the computed corrections
         SUMMARY    Display the improvements in Noise & Dynamic range
\end{verbatim}
\subsubsection{SELFCAL Arguments:}
\index{SELFCAL!Arguments:}
\begin{verbatim}
          The arguments control the action.
        SELFCAL ?        Lists the self calibration parameters
        SELFCAL AMPLI    Perform an Amplitude only self calibration
        SELFCAL APPLY    Apply the solution
        SELFCAL PHASE    Perform a Phase only self calibration
        SELFCAL SAVE     Save the parameters in selfcal.last
        SELFCAL SHOW []  Show the computed corrections
        SELFCAL SUMMARY  Display the improvements in Noise & Dynamic range
\end{verbatim}
\subsubsection{AMPLITUDE}
\index{SELFCAL!AMPLITUDE}
\begin{verbatim}
            SELFCAL AMPLI

    Compute an  Amplitude  only  self-calibration.  The  integration  times,
    SELF_TIME,  should  in  general be significantly larger than for a Phase
    only self-calibration.

    SELFCAL automatically adjusts the gains so that their mean  is  1.0,  to
    avoid changing the flux scale.

\end{verbatim}
\subsubsection{APPLY}
\index{SELFCAL!APPLY}
\begin{verbatim}
            SELFCAL APPLY [Type [Gain]]

    Apply  the  self  calibration  solution. This is done only if the return
    status from the previous computation, SELF_STATUS, is  greater  than  0.
    SELFCAL  APPLY  automatically  saves  the  parameters and results in the
    "selfcal-AMPLI.last" or "selfcal-PHASE.last" file, depending on the Type
    of solution applied.

      Type  is  the  type  of solution to apply. The default is 'SELF_MODE',
    i.e.  PHASE or AMPLI depending on the last type of self-calibration com-
    puted.   Type  can  also take the DELAY value, where the "phase" correc-
    tions are interpreted as atmospheric path changes and scale as  Frequen-
    cy.

      Gain is an optional gain factor (default is 1.) on the correction.

\end{verbatim}
\subsubsection{INPUT}
\index{SELFCAL!INPUT}
\begin{verbatim}
            SELFCAL INPUT  or SELFCAL ?

    Display SELFCAL control variables

\end{verbatim}
\subsubsection{PHASE}
\index{SELFCAL!PHASE}
\begin{verbatim}
            SELFCAL PHASE

    Compute  a  Phase-only  self calibration. The integration time should be
    short enough to correct for atmospheric errors, but large enough to  ob-
    tain significant signal to noise for most antennas.

\end{verbatim}
\subsubsection{SAVE}
\index{SELFCAL!SAVE}
\begin{verbatim}
            SELFCAL SAVE

        Save the parameters and results in the "selfcal.last" file.

\end{verbatim}
\subsubsection{SHOW}
\index{SELFCAL!SHOW}
\begin{verbatim}
            SELFCAL SHOW [Last [First]]

    Shows  the correction computed by self calibration. By default, the dif-
    ference between the last two iterations is displayed:  phase  should  be
    around 0, and amplitude around 1 if the solution is converged.

    If Last and First are present, it shows the difference between these two
    specified iteration. If Last only is present, it shows the total correc-
    tion between the original data and that iteration.

    The  displayed ranges are controlled by SELF_ARANGE[2] (limits of Ampli-
    tude  gain),   SELF_PRANGE[2]   (limits   of   Phase   correction)   and
    SELF_TRANGE[2] (limits for time axis).

\end{verbatim}
\subsubsection{SUMMARY}
\index{SELFCAL!SUMMARY}
\begin{verbatim}
            SELFCAL SUMMARY

    Display  a summary of the Self-calibration process: type of calibration,
    rms and dynamic range at each  iteration,  as  well  as  the  number  of
    flagged or uncalibrated visibilities.

\end{verbatim}
\subsubsection{Results:}
\index{SELFCAL!Results:}
\begin{verbatim}

    SELFCAL  returns  results  in  several variables, creates a CGAINS array
    containing the Gain values, and one file to display the computed correc-
    tion. The file name is specified by SELF_SNAME.

    The result variables are:
        SELF_APPLIED
          Indicates whether the solution has been applied
        SELF_STATUS
          Indicates if a solution has been computed
        SELF_DYNAMIC[Self_Loop+1]
          The dynamic range at each iteration
        SELF_RMSCLEAN[Self_Loop+1]
          The rms noise at each iteration
\end{verbatim}
\subsubsection{SELF\_APPLIED}
\index{SELFCAL!SELF\_APPLIED}
\begin{verbatim}

    Variable  SELF_APLPLIED  indicates  whether  a solution has been applied
    (#0) or not (0). The value indicates the type and quality  of  solution,
    as for SELF_STATUS

\end{verbatim}
\subsubsection{SELF\_STATUS}
\index{SELFCAL!SELF\_STATUS}
\begin{verbatim}

    Variable SELF_STATUS indicates if a solution has been computed
          0 no solution
          +/- 1   Phase solution
          +/- 2   Gain solution
        >0 means a good solution, <0 a poor one.
\end{verbatim}
\subsubsection{SELF\_DYNAMIC}
\index{SELFCAL!SELF\_DYNAMIC}
\begin{verbatim}

    SELF_DYNAMIC  is a variable length array of size Self_Loop+1, containing
    the dynamic range at each iteration.

\end{verbatim}
\subsubsection{SELF\_RMSCLEAN}
\index{SELFCAL!SELF\_RMSCLEAN}
\begin{verbatim}

    SELF_RMSCLEAN is a variable length array of size Self_Loop+1, containing
    the Clean map rms noise at each iteration.

\end{verbatim}
\subsubsection{Variables:}
\index{SELFCAL!Variables:}
\begin{verbatim}

    The  SELFCAL  behaviour  can be adjusted through control variables named
    SELF_... (see EXA SELF_ for a list), in addition to  the  control  vari-
    ables of UV_MAP (MAP_...) and CLEAN (CLEAN_...)

    The most important variable is SELF_TIMES, a variable length array which
    controls the integration time at each loop. SELF_NITER and  SELF_MINFLUX
    also control the number of Clean components and minimum flux retained in
    each loop. The size of these arrays define the number of loops.

    See HELP SELFCAL SELF_LOOP to find out how  to  control  the  number  of
    loops.

\end{verbatim}
\subsubsection{SELF\_CHANNEL}
\index{SELFCAL!SELF\_CHANNEL}
\begin{verbatim}
            SELF_CHANNEL[2]

    First  and  last channel to define the range to compute the UV table for
    the self-calibration solution. For example, if you have used UV\PLUGC to
    plug  a  continuum  data into the last channel, this could be set to the
    number of channels. 0 0 means all channels are averaged to  compute  the
    "continuum" image.

\end{verbatim}
\subsubsection{SELF\_COLOR}
\index{SELFCAL!SELF\_COLOR}
\begin{verbatim}
            SELF_COLOR

    Controls the LUT color range at each self-calibration cycle if non Zero.
    Since the dynamic range evolves, this can be a useful way  to  highlight
    the  gain. If non Zero, SELF_COLOR is passed as argument to a COLOR com-
    mand executed at each cycle after display.

    Pratical values for SELF_COLOR are -8 ("bright" version, highlights  the
    noise level) or +8 ("dark" version, hides the noise), but lower absolute
    values may be needed for higher dynamic ranges.

    See HELP COLOR for details.

\end{verbatim}
\subsubsection{SELF\_LOOP}
\index{SELFCAL!SELF\_LOOP}
\begin{verbatim}

    Number of self-iteration loops. It is a ReadOnly variable that is  auto-
    matically  computed  from  the  size  of  the SELF_TIMES, SELF_NITER and
    SELF_MINFLUX arrays.

      These variable length arrays can be resized using the LET /RESIZE com-
    mand. For example
        LET SELF_TIMES 40 20 10 /RESIZE
    will lead to an array of 3 elements, SELF_TIMES[3].

    SELF_TIMES,  SELF_NITER  and SELF_MINFLUX must be of equal size. To sim-
    plify the process, constant arrays are assumed of  arbitrary  length  in
    this  determination.   If  all 3 arrays have constant values, SELF_TIMES
    determines the number of loops.

    Constant arrays are assumed of arbitrary length in  this  determination.
    If  all  3 arrays have constant values, SELF_TIMES determines the number
    of loops.

\end{verbatim}
\subsubsection{SELF\_NITER}
\index{SELFCAL!SELF\_NITER}
\begin{verbatim}

    Variable length array specifying the number of Clean components retained
    for  each  loop  of the self-calibration process.  Default is 0, meaning
    all Clean components found by CLEAN.

    For simple structures and Phase calibration, 10 may be enough. For  more
    complex  ones,  be sure to include enough Clean components in the model.
    More components can be taken at each step, although the better knowledge
    of phase errors often allows the source to be represented with a smaller
    number of components after self-calibration than before.  The default is
    a  reasonably  good  compromise,  although faster convergence may be ob-
    tained with smaller number of components.

    This variable length array can be resized using the LET /RESIZE command.
    For example
        LET SELF_NITER 10 0 0 /RESIZE
    will  lead  to  3 self-calibration loops (if SELF_TIMES and SELF_MINFLUX
    are also of size 3), the first one selecting only 10 components, the two
    others all components.

\end{verbatim}
\subsubsection{SELF\_TIMES}
\index{SELFCAL!SELF\_TIMES}
\begin{verbatim}

    Variable  length  array specifying the integration time (in seconds) for
    each loop of the self-calibration process.

    At NOEMA, 45 sec is the normal minimum integration  time.  Depending  on
    Signal  to  Noise,  you  may need to have this longer, e.g. 120 sec. For
    several loops, start with a longer value, and decrease only at the end.

    At ALMA, the minimum integration time is 6 sec. At VLA, this may  be  as
    small as 1 sec.

    It  is recommended to use the same integration time for the last two it-
    erations, to simplify the interpretation of the results and of the SELF-
    CAL SHOW display.

    This variable length array can be resized using the LET /RESIZE command.
    For example
        LET SELF_TIMES 180 90 45 /RESIZE
    will lead 3 self-calibration loops (if SELF_TIMES and  SELF_MINFLUX  are
    also of size 3) with decreasing integration times.

\end{verbatim}
\subsubsection{SELF\_MINFLUX}
\index{SELFCAL!SELF\_MINFLUX}
\begin{verbatim}

    Variable  length  array specifying the minimum flux density (in Jy/beam)
    to be considered in the Clean image for each loop of  the  self-calibra-
    tion  process. Note that this is the brightness of a pixel, not the flux
    of a Clean component.

    This variable length array can be resized using the LET /RESIZE command.
    For example
        LET SELF_MINFLUX 0.001 0 0  /RESIZE
    will lead 3 self-calibration loops (if SELF_TIMES and SELF_NITER are al-
    so of size 3) selecting on regions brighter than 1mJy/beam in the  first
    one, and all regions in the last 2 ones.

\end{verbatim}
\subsubsection{SELF\_REFANT}
\index{SELFCAL!SELF\_REFANT}
\begin{verbatim}

    The  reference  antenna number. If 0, the program will peak the one with
    the shortest average baselines.

\end{verbatim}
\subsubsection{SELF\_FLUX}
\index{SELFCAL!SELF\_FLUX}
\begin{verbatim}

    Maximum value in the FLUX window. If 0, the FlUX window of Clean is  not
    displayed

\end{verbatim}
\subsubsection{SELF\_PRECISION}
\index{SELFCAL!SELF\_PRECISION}
\begin{verbatim}

    Tolerance to test for self-calibration convergence. The default is 0.01.
    SELFCAL SUMMARY will write a message when no more selfcal iteration  im-
    proves the solution (noise and dynamic range) at this precision level.

\end{verbatim}
\subsubsection{SELF\_RESTORE}
\index{SELFCAL!SELF\_RESTORE}
\begin{verbatim}

    Use  UV_SELF /RESTORE after Cleaning. This avoids signal aliasing at im-
    age edges, and leads to a better estimate of the noise level.

    It also allows to use smaller images, in practice,

\end{verbatim}
\subsubsection{SELF\_DISPLAY}
\index{SELFCAL!SELF\_DISPLAY}
\begin{verbatim}

    If YES, display Clean image before each calibration loop, and prompt for
    user  input. If YES, Cleaning at each step will use the number of itera-
    tions specified by NITER, while if set to  NO,  Cleaning  will  stop  at
    SELF_NITER, saving time.

    The  dynamic  range  progress report is accurate only if SELF_DISPLAY is
    set.

\end{verbatim}
\subsubsection{SELF\_FLAG}
\index{SELFCAL!SELF\_FLAG}
\begin{verbatim}

        If SELF_FLAG is YES, SELFCAL will flag data with no solution. If NO,
    it will keep the data as it was before.

\end{verbatim}
\subsubsection{SELF\_SNR}
\index{SELFCAL!SELF\_SNR}
\begin{verbatim}

    Minimum Signal to Noise ratio on the antenna gain to consider a solution
    to be valid for an antenna. 6 is a good value, 3 a lower  limit.  Beware
    that  this  SNR  makes  sense only if the a priori estimate of the noise
    from the UV weights is correct: see SELF_SNOISE.

\end{verbatim}
\subsubsection{SELF\_SNOISE}
\index{SELFCAL!SELF\_SNOISE}
\begin{verbatim}

    Noise scaling factor. This should be 1, but some  noise  estimates  need
    corrections.  Continuum  data  from ALMA often requires sqrt(2) instead,
    for example.  Command UV_PREVIEW may help you checking the noise scale.

\end{verbatim}
\subsubsection{CLEAN\_ARES}
\index{SELFCAL!CLEAN\_ARES}
\begin{verbatim}

        Maximum absolute residual to stop Cleaning

\end{verbatim}
\subsubsection{CLEAN\_FRES}
\index{SELFCAL!CLEAN\_FRES}
\begin{verbatim}

        Maximum fractional residual to stop Cleaning

\end{verbatim}
\subsubsection{CLEAN\_NITER}
\index{SELFCAL!CLEAN\_NITER}
\begin{verbatim}

    Maximum number of Clean components. If all of CLEAN_ARES, CLEAN_FRES and
    CLEAN_NITER  are 0, Clean stops by checking the stability of the cleaned
    flux over CLEAN_NKEEP iterations. See HELP CLEAN for details.



\end{verbatim}
\subsection{SLICE}
\index{SLICE}
\begin{verbatim}
        [ADVANCED\]SLICE VarName Start1 Start2 End1 End2 UNIT

      Cut a slice through the 3-D image variable VarName with the  specified
    edges and put it in the SLICE image variable. UNIT is a keyword indicat-
    ing in which units the edges are specified. It can  be  PIXELS,  DEGREE,
    MINUTE,  SECOND, RADIAN, of ABSOLUTE.  For ABSOLUTE, Start1 and End1 are
    assumed to be in hours and Start2 End2 in degrees, with sexagesimal  no-
    tation allowed.

\end{verbatim}
\subsection{STOKES}
\index{STOKES}
\begin{verbatim}
        [CLEAN\]STOKES Key /FILE UVin UVout

    Derive  a  single  polarization  UV  table (UVout) with the polarization
    state specified by Key from a multi-polarization UV table  (Uvin).   Key
    is any of the following: NONE, I, Q, U, V, RR, LL, HH, VV, XX, YY.

    A typical use is after command FITS on CASA data:
      FITS Fits.uvfits TO UVin.uvt
      STOKES NONE /File UVin Uvout

\end{verbatim}
\subsection{UV\_ADD}
\index{UV\_ADD}
\begin{verbatim}
        [ADVANCED\]UV_ADD ITEM [Mode] [/FILE FileIn FileOut]

      Compute  and  add  some missing information in a UV Table, such as the
    Doppler correction and the Parallactic Angle

    The information is derived from the Observatory  coordinates,  from  the
    Telescope name in the input UV table. This can be supplied by the SPECI-
    FY TELESCOPE command if not available.

    ITEM can take values DOPPLER, PARALLACTIC or * for both.

    Mode is a debug control integer which indicates in which column the  in-
    formation  should be placed. The default is 0, i.e. the command will re-
    use the column of the appropriate type, or add it if  not  present.   It
    may  be  set  to 3, as often column 3 contains the Scan number, which is
    not a relevant information for imaging.  Other values are at the  user's
    peril...


\end{verbatim}
\subsubsection{UV\_ADD /FILE}
\index{UV\_ADD!/FILE}
\begin{verbatim}
        [ADVANCED\]UV_ADD ITEM [Mode] /FILE FileIn FileOut

    Use the UV data in the file FileIn, and write the completed visibilities
    in the file FileOut.

\end{verbatim}
\subsection{UV\_DEPROJECT}
\index{UV\_DEPROJECT}
\begin{verbatim}
        [ADVANCED\]UV_DEPROJECT x0 y0 Rota Incli

    Deproject a UV table for inclination and orientation  (Incli,  Rota,  in
    Degrees)  around  a  specified  center (x0,y0  in Radians).  This can be
    useful for almost planar or cylindrical objects (e.g.  galaxies, or pro-
    toplanetary disks)

    The result becomes the current UV table.

\end{verbatim}
\subsection{UV\_FIT}
\index{UV\_FIT}
\begin{verbatim}
        [ADVANCED\]UV_FIT   [Func1 .. FuncN]  [/QUIET] [/SAVE File] [/WIDGET
    N]

    Fit UV data with a few simple functions. Func1 to FuncN (currently  N  <
    5)  are  the  names of the functions to be fitted. If not specified, the
    last names (or those attributed by the /WIDGET options) are used.

    The models are either simple functions or linear combinations of  simple
    functions.   The results of the fitting process are the position offsets
    in R.A. and Dec (in arc second) of the model source from the phase  ref-
    erence  center,  and  its flux (Jansky).  Depending on the fitting func-
    tions additional fit results are possible. Currently supported distribu-
    tions and additional fit parameters are:
    POINT     Point source              : None
    E_GAUSS   Elliptic Gaussian source  : FWHM Axes (Major and Minor), Pos Ang
    C_GAUSS   Circular Gaussian source  : FWHM Axis
    C_DISK    Circular Disk             : Diameter
    E_DISK    Elliptical (inclined) Disk: Axis (Major and Minor), Pos Ang
    RING      Annulus                   : Diameter (Inner and Outer)
    EXPO      Exponential brightness    : FWHM Axis
    POWER-2   B = 1/r^2                 : FWHM Axis
    POWER-3   B = 1/r^3                 : FWHM Axis
    E_RING    Inclined Annulus          : Inner, Outer, Pos Ang, Ratio

    The  function  parameters  are found in a SIC structure named UVF% under
    names UVF%PARi%PAR[7]  (starting  values),  UVF%PARi%RANGE[7]  (starting
    ranges)  and UVF%PARi%START[7] (number of starts). The /WIDGET option is
    a convenient way to set these variables.

    UV_RESIDUAL will compute the fit residual when used after UV_FIT,  while
    it  computes  the  residual  from the Clean component list if used after
    CLEAN.

    WRITE UV_FIT file[.uvfit] will save the fit results in a  GILDAS  table,
    in the same format than that of the UV_FIT task.

    This  command is similar to the task UV_FIT, but works on the current UV
    buffer.

\end{verbatim}
\subsubsection{UV\_FIT /QUIET}
\index{UV\_FIT!/QUIET}
\begin{verbatim}
        [ADVANCED\]UV_FIT   /QUIET  [/WIDGET N]

    Activate the quiet mode. Only a progress report (25%, 50% and 75%  done)
    is  issued in this case, not a per-channel message.  This mode is recom-
    mended for many channels.

\end{verbatim}
\subsubsection{UV\_FIT /SAVE}
\index{UV\_FIT!/SAVE}
\begin{verbatim}
        [ADVANCED\]UV_FIT   /SAVE OutputFile

    Save the input parameters, into a text output file. This file is then  a
    script which can be re-executed to set the input parameters for UV_FIT.

    If the data has only 1 channel, the fit results (see HELP UV_FIT Result-
    Values) are also written in the same file.

\end{verbatim}
\subsubsection{UV\_FIT ResultValues}
\index{UV\_FIT!ResultValues}
\begin{verbatim}

        For data with only 1 channel, the fit UV_FIT results  are  available
    as  variables   UVF%PARi%RESULTS[7]  (for  the results) and UVF%PARi%ER-
    RORS[7] (for their formal errors) for every  function  number  i.  These
    variables  are written in the output file by the UV_FIT /SAVE OutputFile
    command.

    They are also available in the internal table named UV_FIT, as  for  any
    other number of channels (see HELP UV_FIT ResultTable)

\end{verbatim}
\subsubsection{UV\_FIT /WIDGET}
\index{UV\_FIT!/WIDGET}
\begin{verbatim}
        [ADVANCED\]UV_FIT   /WIDGET N [/QUIET]

    Create  a  Widget to specify the function names and input parameters for
    UV_FIT for N functions.  Once these are defined by  the  user,  clicking
    the GO button will launch the computation.

\end{verbatim}
\subsubsection{UV\_FIT ResultTable}
\index{UV\_FIT!ResultTable}
\begin{verbatim}

    The  UV_FIT  results  are stored in an internal table named UV_FIT. This
    table can later be saved using command WRITE UV_FIT FileName. The  table
    is  organized  as a MxN matrix, where M is the number of channels in the
    UV data and the organization in N is the following:
    N:  P1 P2 P3 Vel A1 A2 A3 Par1 Err1 Par2 Err2 ... A1 A2 A3 Par1 Err1 ...

    where P1 = RMS of the fitting process
          P2 = number of supplied functions  (NF$)
          P3 = total number of parameter
         Vel = velocity of the i-th channel (i lies between 1 and M)
          A1 = 1 (for the first function), = 2 (for the second function)
          A2 = function code (POINT = 1, E_GAUSS = 2, ... , POWER-3 = 8)
          A3 = number of parameters of the function
        Parx = result of the fit parameter (order of appearance in PARAMxx$)
        Errx = error of the fitting process for the parameter Parx

    For example, fitting models with the two  functions  POINT  and  C_GAUSS
    produce files with N=24.

    SHOW UV_FIT will display plots from this table.

\end{verbatim}
\subsection{UV\_MERGE}
\index{UV\_MERGE}
\begin{verbatim}

          [ADVANCED]\UV_MERGE OutFile /FILES In1 In2 ... Inn
        [/MODE [STACK|CONTINUUM [Index [Frequency]]]
        [/SCALES F1 ... Fn] [/WEIGHTS W1 ... Wn ]

      Merge  many  UV data files, with calibration factor and weight factors
    and (for Line data) spectral resampling as in the first  one. OutFile is
    the name of the output UV table.

      For  Line  data,  the  default is to merge lines of the same molecular
    transition (same Rest Frequency). However the STACK mode allows stacking
    UV data from different spectral lines, re-aligned in velocity.  This can
    allow detection of molecules with many transitions.

      For Continuum data (1 channel and/or option /MODE CONTINUUM), a  spec-
    tral  index  and  a reference Frequency can be specified to merge all UV
    tables.

\end{verbatim}
\subsubsection{UV\_MERGE /FILES}
\index{UV\_MERGE!/FILES}
\begin{verbatim}

          [ADVANCED]\UV_MERGE OutFile /FILES In1 In2 ... Inn
        [/MODE [STACK|CONTINUUM [Index [Frequency]]]
        [/SCALES F1 ... Fn] [/WEIGHTS W1 ... Wn ]

      Specify the names of the UV tables to be merged. The first one is used
    as a reference for resampling (line data) or Frequency (continuum data).

\end{verbatim}
\subsubsection{UV\_MERGE /MODE}
\index{UV\_MERGE!/MODE}
\begin{verbatim}

          [ADVANCED]\UV_MERGE OutFile /FILES In1 In2 ... Inn
        /MODE [STACK|CONTINUUM [Index [Frequency]]
        [/SCALES F1 ... Fn] [/WEIGHTS W1 ... Wn ]

      Specify the merging mode.

      For spectral line UV tables (more than 1 channel), the default is that
    the spectral lines must have the same Rest Frequency. Resampling (in ve-
    locity, which is then identical in frequency) is done on the grid of the
    first UV table In1.

      In mode STACK, the Rest Frequencies can  differ.  Resampling  is  then
    done in velocity, and the (u,v) coordinates scaled as the Rest Frequency
    ratios to conserve the angular resolution of the data.  The /SCALES  and
    /WEIGHT factors can be used to incorporate prio knowledge of the expect-
    ed line rations to optimize S/N.

      For continuum UV tables (1 channel or option /MODE CONTINUUM), a spec-
    tral  index  can  be specified, as well as a reference Frequency.  (u,v)
    coordinates are scaled appropriately, as well as  Flux  and  Weights  in
    this  case.  The /SCALES and /WEIGHTS factors are applied on top of this
    automatic spectral index scaling. Input Line UV Tables  are  treated  as
    multi-frequency Continuum ones (as in UV_CONTINUUM command).

\end{verbatim}
\subsubsection{UV\_MERGE /SCALES}
\index{UV\_MERGE!/SCALES}
\begin{verbatim}

          [ADVANCED]\UV_MERGE OutFile /FILES In1 In2 ... Inn
        [/MODE [STACK|CONTINUUM [Index [Frequency]]]
        /SCALES F1 ... Fn [/WEIGHTS W1 ... Wn ]

      Specify the flux scaling factors for each UV table.

      For  /MODE  CONTINUUM,  the  spectral index is applied separately from
    this flux scale factor (by further multiplication).

\end{verbatim}
\subsubsection{UV\_MERGE /WEIGHTS}
\index{UV\_MERGE!/WEIGHTS}
\begin{verbatim}

          [ADVANCED]\UV_MERGE OutFile /FILES In1 In2 ... Inn
        [/MODE [STACK|CONTINUUM [Index [Frequency]]]
        [/SCALES F1 ... Fn] /WEIGHTS W1 ... Wn

      Specify the weight scaling factors for each UV table. The weight scal-
    ing factor is independent of the flux scaling factor. This means that to
    conserve the Signa-to-Noise ration, one should normally use Wi = 1/Fi^2.

      For /MODE CONTINUUM, the spectral index  is  applied  separately  from
    this weight scaling factor (by further multiplication).

\end{verbatim}
\subsection{UV\_PREVIEW}
\index{UV\_PREVIEW}
\begin{verbatim}
        [ADVANCED\]UV_PREVIEW [Ntapers [Threshold [NHist]]] [/FILE FileIn]

    A  fast previewer to figure out if there is signal and what is its spec-
    tral shape.  The command attempts to find out the line free regions  and
    to estimate a continuum region.

    The  output  of  UV_PREVIEW  can be used for further processing commands
    (UV_BASELINE and UV_FILTER, UV_SPLIT, SPECIFY, etc...)

    If a catalog is defined (see HELP CATALOG), it will  also  display  line
    identification (red for detected ones, blue for the others).

      Optional arguments are
      Ntapers:      number of scale sizes  (default 4)
      Threshold:    Truncation level in Sigma (default 3.5)
      Nhist:        Histogram size  (default is variable)

\end{verbatim}
\subsubsection{UV\_PREVIEW /FILE}
\index{UV\_PREVIEW!/FILE}
\begin{verbatim}
        [ADVANCED\]UV_PREVIEW [Ntapers [Threshold [NHist]]] /FILE FileIn

    Without /FILE, UV_PREVIEW works from the current UV data set.

    With  the /FILE option, it will pre-view the UV data set from the speci-
    fied file. Edge channels are automatically dropped in this  process,  as
    many  telescopes  (NOEMA  or  ALMA) do not produce useable data in these
    ones. The default drop is 5% of bandwidth on each side.

\end{verbatim}
\subsubsection{UV\_PREVIEW Algorithm}
\index{UV\_PREVIEW!Algorithm}
\begin{verbatim}

        UV_PREVIEW computes for Ntapers different tapers  the  spectrum  to-
    wards  the phase center. The taper ranges are determined from the avail-
    able baseline lengths and telescope diameter.

    For each spectrum, UV_PREVIEW then attempts to figure out  if  there  is
    line  emission and the line-free channels to define the continuum level.
    This is based on the histogram of  the  intensity  distribution  of  all
    channels. The most likely value and the noise level is derived from this
    histogram. An iterative scheme, blanking out of range  (presumably  line
    emission) channels, is used for this to converge towards a Gaussian his-
    togram, which normally represents the noise distribution around the con-
    tinuum level.

    As  a last step, blanked channels are accumulated in a list of channels,
    which thus contain possible line emission at any of the Ntapers scales.

\end{verbatim}
\subsubsection{UV\_PREVIEW Limitations}
\index{UV\_PREVIEW!Limitations}
\begin{verbatim}

        UV_PREVIEW cannot identify lines if there are two few  channels.  It
    will  only display the spectra in this case. A minimum of 32 channels is
    required, but confused spectra may also prevent a proper  line  recogni-
    tion.

    The  line  detection is based on the current specified velocity. If this
    is incorrect, lines may appear shifted and considered has not  detected.
    PREVIEW%TOLERANCE indicates the matching precision in frequency (default
    is 2 MHz).

    Currently, no account for a Redshift is made.

\end{verbatim}
\subsubsection{UV\_PREVIEW Output}
\index{UV\_PREVIEW!Output}
\begin{verbatim}

          UV_PREVIEW returns the list  of  possible  line  channels  through
    variable  PREVIEW%CHANNELS.  If  no  line emission was identified at any
    scale, the list is empty and the variable does not exist.

    With this list, the user can compute the continuum image, using commands
    UV_FILTER /CHANNELS PREVIEW%CHANNELS (or simply UV_FILTER), then UV_CON-
    TINUUM and the usual UV_MAP and CLEAN. Alternatively, the user can  fil-
    ter out the continuum emission using UV_BASELINE.

    If  a  catalog  is  present UV_PREVIEW also creates two other variables,
    PREVIEW%EDGES and PREVIEW%FREQUENCIES which contains the start  and  end
    channels  (for  %EDGES, resp. frequencies for %FREQUENCIES) of each con-
    tiguous range of channels in PREVIOUS%CHANNELS. These variables are used
    for line identification and imaging in the
      @ image_lines
    script.

    Finally,  UV_PREVIEW  returns in PREVIEW%FMIN PREVIEW%FMAX the frequency
    coverage, and in PREVIEW%FREQ the rest  frequency  of  the  most  likely
    spectral line in the window, and in PREVIEW%LINES its name.

    If  no  spectral line has been identified, PREVIEW%FREQ is just the mean
    of PREVIEW%FMIN and PREVIEW%FMAX, and PREVIEW%LINES does not exist.

\end{verbatim}
\subsection{UV\_RADIAL}
\index{UV\_RADIAL}
\begin{verbatim}
        [ADVANCED\]UV_RADIAL x0 y0 Rota Incli  [/SAMPLING QSTEP [QMIN QMAX]]
    [/U_ONLY] [/ZERO [Flux]]

    Compute  a  UV  table containing the radial distribution of the azimutal
    average of the visibilities after deprojection for inclination and  ori-
    entation  (Incli, Rota, in Degrees) around a specified center (x0,y0  in
    Radians).

    The result becomes the current UV table.

    If not specified, x0 y0 Rota and Incli default to 0.

\end{verbatim}
\subsubsection{UV\_RADIAL /SAMPLING}
\index{UV\_RADIAL!/SAMPLING}
\begin{verbatim}
        [ADVANCED\]UV_RADIAL x0 y0 Rota Incli /SAMPLING  QSTEP  [QMIN  QMAX]
    [/U_ONLY] [/ZERO [Flux]]

    Specify  the  sampling  of the UV distances: Qstep is the step, Qmin and
    Qmax the min and max. Distances are in meter. If not present,  an  auto-
    matic  guess is made from the minimum and maximum baselines and the dish
    diameter.

\end{verbatim}
\subsubsection{UV\_RADIAL /U\_ONLY}
\index{UV\_RADIAL!/U\_ONLY}
\begin{verbatim}
        [ADVANCED\]UV_RADIAL x0 y0 Rota Incli /U_ONLY [/SAMPLING QSTEP [QMIN
    QMAX]] [/ZERO [Flux]]

    Indicate  that  the resulting UV table should have all V values equal to
    zero. This is convenient to display the azimutal average of the visibil-
    ities as a function of UV distance, but cannot be used for further imag-
    ing.

    If not present, the resulting UV table has a (u,v) coverage which is ex-
    tended  by  rotation,  so that it can be used to image the (rotationally
    symmetric) 2-D radial distribution using standard commands  like  UV_MAP
    and  CLEAN.   The radial profile of the brightness distribution can then
    be recovered by using any radial cut through this image.

\end{verbatim}
\subsubsection{UV\_RADIAL /ZERO}
\index{UV\_RADIAL!/ZERO}
\begin{verbatim}
        [ADVANCED\]UV_RADIAL x0 y0 Rota Incli /ZERO [Flux] [/SAMPLING  QSTEP
    [QMIN QMAX]] [/U_ONLY]

    Add  the zero spacing flux to the azimutal average.  If no value is giv-
    en, the zero spacing flux is extrapolated from  the  shortest  baselines
    using a parabolic interpolation centered on (u,v)=(0,0).


\end{verbatim}
\subsection{UV\_SHORT}
\index{UV\_SHORT}
\begin{verbatim}
        [ADVANCED\]UV_SHORT [Arg]


    Compute the Short Spacings from the current Single Dish dataset (read by
    READ SINGLE) and merge it to the current UV data.

      UV_SHORT takes sensible default guesses for most parameters.  UV_SHORT
    ?   lists  the  essential  parameter values, UV_SHORT ?? some additional
    ones, and UV_SHORT ??? even the debugging control variables.

        The current values can be overriden by the user,  who  need  to  set
    (and  if  needed  to  define  first)  the  corresponding  SHORT_whatever
    variable.  SHORT_SD_FACTOR  is the main one which may need to be  speci-
    fied  by  the user, as the Single Dish data is rarely in the appropriate
    unit.

      The resulting UV table becomes the current UV data, and can   be   im-
    aged, written, etc...

\end{verbatim}
\subsubsection{UV\_SHORT /REMOVE}
\index{UV\_SHORT!/REMOVE}
\begin{verbatim}
        UV_SHORT /REMOVE

      Removes any short spacing from the current UV data set.

\end{verbatim}
\subsubsection{UV\_SHORT Algorithm}
\index{UV\_SHORT!Algorithm}
\begin{verbatim}

    UV_SHORT  task  computes  pseudo-visibilities for short or zero spacings
    from a single dish table of spectra (Class  table)  or  LMV  data  cube.
    These pseudo visibilities are  appended to the current (presumably a Mo-
    saic) UV table.


    Short spacings are computed when the  Interferometer  dish  diameter  is
    smaller than the Single-dish diameter, Zero spacings otherwise (see HELP
    UV_SHORT Zero_Spacing for this case)

    For short spacings, the command performs two steps
      (1) Creation of a "well behaved" map from the spectra.
      (2) Extraction of UV visibilities from this map.

    See HELP UV_SHORT Step_i for detailed explanations of the method steps.

    With recent UV tables and Single Dish CLASS table, most  parameters  are
    automatically  determined.  The  only  parameter to be specified remains
    SHORT_SD_FACTOR (although that one may also be determined  automatically
    if  the  input  single dish data set is in main-beam brightness tempera-
    ture).

    A parameter set to 0 value indicates the appropriate default  should  be
    used.

\end{verbatim}
\subsubsection{UV\_SHORT Zero\_Spacing}
\index{UV\_SHORT!Zero\_Spacing}
\begin{verbatim}

          Zero  spacings  are  computed when the single dish diameter is the
    same as the interferometer dish diameter. Zero spacing  extraction  pro-
    ceeds differently for Class data tables and 3-D data cubes.

    In  the data cube case, the nearest pixel matching the direction of each
    field is taken as the Zero spacing for this field. If there is no  point
    close  enough, according to the specified position tolerance SHORT_TOLE,
    an error occurs.

    In the Class data table case, all spectra within the specified  position
    tolerance  of  a  field center are averaged together to produce the Zero
    spacing. If none is found, an error occcurs.


\end{verbatim}
\subsubsection{UV\_SHORT Step\_1}
\index{UV\_SHORT!Step\_1}
\begin{verbatim}

          Step (1) Creation of a "well behaved" map from the spectra.

    Step (1) only occurs if the input single-dish data  set  (read  by  READ
    SINGLE)  is  a  table of spectra. The table format  is described in  the
    CLASS\GRID command of CLASS.
      The identification of the input single-dish data set  as  a  table  of
    spectra is currently only based on the specified file extension: if that
    is .tab, it is assumed to be a table of spectra.

    It is recommended that this input  table is a collection of single-dish,
    Nyquist sampled  spectra  covering  twice  the interferometric field  of
    view  of interest. However, UV_SHORT does *NOT* make any  assumption. It
    thus  tries to compute  a "well behaved" map by linear operations  (con-
    volutions) from the  original spectra, in an  optimum way   from  signal
    to noise  point of view. The  map is extrapolated smoothly  towards zero
    at  the map  edge in order  to avoid  further  aliasing  in the  Fourier
    transform   operations required in  Step (2).  This extrapolation  has a
    scale  length of  twice the single-dish beam, in order to avoid spurious
    Fourier components.

    In detail, UV_SHORT performs the following operations:

      - Resampling  (in space) of  the original  spectra on  a regular  grid by
        convolution  with  a small  (typically  1/4  of  the single-dish  beam)
        gaussian convolving kernel. In  this process, the weights of individual
        spectra is carried on a weight map.
      - Extrapolation by  zero outside  the convex hull  of the  mapped region.
      - Convolution  of  the  result  by  a  gaussian  twice  as  wide  as  the
        single-dish beam.   Within the  convex hull of  the mapped  region, the
        smoothed map is replaced by the original map.

\end{verbatim}
\subsubsection{UV\_SHORT Step\_2}
\index{UV\_SHORT!Step\_2}
\begin{verbatim}
          Step (2) Extraction of UV visibilities from this map.

    From  the given input data cube, or the "well behaved" data cube created
    by Step (1), UV_SHORT  computes the visibilities in the following way:

      - Fourier transform of the single dish map;
      - Division by  the Fourier  transform of  the single dish  beam, up  to a
        maximum spacing (SHORT_SD_DIAM, in meters);
      - Inverse Fourier transform to the image plane and then for each pointing
        center;
      - Multiplication of the  image by the primary beam  of the interferometer
        elements;
      - Fourier transform back to the UV plane;
      - Creation  of the  visibilities,  with  a given  weight  SHORT_SD_WEIGHT
        and an appropriate calibration factor to Janskys SHORT_SD_FACTOR.

    Both the  single-dish and the  interferometer antennas are  assumed   to
    have gaussian beams (SHORT_SD_BEAM and SHORT_IP_BEAM, in radians).

\end{verbatim}
\subsubsection{UV\_SHORT Variables:}
\index{UV\_SHORT!Variables:}
\begin{verbatim}

      Control  variables  for  UV_SHORT are not predefined, except for the 3
    main ones:  SHORT_SD_FACTOR, SHORT_SD_WEIGHT and SHORT_UV_TRUNC.

      All others should be defined by the user in case the default value  is
    not appropriate, with their appropriate (Real, Char or Logical) types.

\end{verbatim}
\subsubsection{SHORT\_DO\_SINGLE}
\index{UV\_SHORT!SHORT\_DO\_SINGLE}
\begin{verbatim}

    Logical value, should be YES except for test purposes.

\end{verbatim}
\subsubsection{SHORT\_DO\_PRIMARY}
\index{UV\_SHORT!SHORT\_DO\_PRIMARY}
\begin{verbatim}

    Logical value, should be YES except for test purposes.

\end{verbatim}
\subsubsection{SHORT\_IP\_BEAM}
\index{UV\_SHORT!SHORT\_IP\_BEAM}
\begin{verbatim}

    Half-power  beam  width of the interferometer antennas, in  radians. The
    beam is assumed to be gaussian.

      Default value is 0, meaning that the beam is taken from the  Telescope
    section if present.

\end{verbatim}
\subsubsection{SHORT\_IP\_DIAM}
\index{UV\_SHORT!SHORT\_IP\_DIAM}
\begin{verbatim}

    Interferometer  diameter  for  which UV_SHORT will compute short spacing
    visibilities.

      Default value is 0, meaning that the diameter is taken from the  Tele-
    scope section if present.

\end{verbatim}
\subsubsection{SHORT\_MCOL}
\index{UV\_SHORT!SHORT\_MCOL}
\begin{verbatim}

      *** Obsolescent ***

    See  READ SINGLE ClassTable.tab /RANGE  command for an equivalent method
    of selecting the appropriate channel range.

    The first and last column to be mapped. For tables produced by GRID com-
    mand  of CLASS, SHORT_MCOL[1] should be 4 and SHORT_MCOL can be set to 0
    to process all channels.

      Default value: 4 0, appropriate for tables coming from CLASS\GRID com-
    mand.


\end{verbatim}
\subsubsection{SHORT\_MIN\_WEIGHT}
\index{UV\_SHORT!SHORT\_MIN\_WEIGHT}
\begin{verbatim}

    The  minimum  weights  under  which  a  given point in the map should be
    filled by the smooth map rather than by the gridded (original] map.

      Default value: 0.01

\end{verbatim}
\subsubsection{SHORT\_MODE}
\index{UV\_SHORT!SHORT\_MODE}
\begin{verbatim}

    This is an integer code used for backward compatibility  with  an  older
    version of the UV_SHORT task, and also for test purpose.

    Allowed values are :

    -1  indicates to create a single UV table with columns for the Phase
        center offsets only
    -2  indicates to create a UV table with columns for the Pointing
        center offsets
    -3  indicates to create a UV table with the additional columns
        type being Pointing or Phase, as in the original UV_TABLE$
    +1  indicates to append to the initial UV table the short spacings
        with Phase center offsets (which must thus match the initial UV
        table shape)
    +2  indicates to append to the initial UV table the short spacings
        with Pointing center offsets (which must thus match the initial
        UV table shape)
    +3  indicates to append to the initial UV table the short spacings
        (The extra column type being determined automatically).

    The  default  value is 3, i.e. automatic merging with the current UV ta-
    ble.

\end{verbatim}
\subsubsection{SHORT\_SD\_BEAM}
\index{UV\_SHORT!SHORT\_SD\_BEAM}
\begin{verbatim}

    Half-power beam width of the single dish antenna, in radians.  The  beam
    is assumed to be gaussian.

      Default  value is 0, meaning that the beam is taken from the Telescope
    section if present.

\end{verbatim}
\subsubsection{SHORT\_SD\_DIAM}
\index{UV\_SHORT!SHORT\_SD\_DIAM}
\begin{verbatim}

    Single dish diameter used to produce the input spectra, in meters.

      Default value is 0, meaning that the diameter is taken from the  Tele-
    scope section if present.

\end{verbatim}
\subsubsection{SHORT\_SD\_FACTOR}
\index{UV\_SHORT!SHORT\_SD\_FACTOR}
\begin{verbatim}

    Multiplicative calibration factor; it is used to convert from the single
    dish map units (e.g., main-beam brightness temperature) to janskys.

    A default value of 0 can be used if the original data file is in unit of
    Tmb, the main beam brightness temperature, because in this case the con-
    version factor can be derived from the beam size.

\end{verbatim}
\subsubsection{SHORT\_SD\_WEIGHT}
\index{UV\_SHORT!SHORT\_SD\_WEIGHT}
\begin{verbatim}

    Weight scaling factor for the generated visibilities.

    It is a relative scaling factor in the weights compared to a  supposedly
    optimal weighting to give the best combined synthesized beam. That opti-
    mal weighting essentially gives the same weight density par unit area in
    the  UV plane than the shortest baselines measured with the interferome-
    ter only.  However, if the single-dish data has not been  observed  long
    enough, or has baselines problems for example, this weight may add noise
    to the overall data set, so could be down-weighted.

      Default: 1.0

\end{verbatim}
\subsubsection{SHORT\_TOLE}
\index{UV\_SHORT!SHORT\_TOLE}
\begin{verbatim}

    The tolerance in position (in radians). The behaviour differ  for  Short
    and Zero spacings and  Table or 3-D cubes as Single-Dish data.

    If the Single-Dish data is a table of spectra, Spectra differing by less
    than this amount will be added together prior to gridding. A recommended
    value is below 1/10th of the Single Dish primary beam. This is valid for
    Short Spacings and Zero Spacing cases.

    If the Single-Dish data is 3-D data cube, SHORT_TOLE is  used  only  for
    Zero  Spacings.  If  no  pixel is within SHORT_TOLE of an Interferometer
    pointing center, no short spacing is added for this field and  an  error
    occur.

      Default  value  is 0, meaning using  1/16th of the Single Dish primary
    beam.

\end{verbatim}
\subsubsection{SHORT\_UV\_TRUNC}
\index{UV\_SHORT!SHORT\_UV\_TRUNC}
\begin{verbatim}

    No visibility at spacings higher than SHORT_UV_TRUNC is generated. Theo-
    retical  consideration  on  the  method  used  in this task implies that
    SHORT_UV_TRUNC should be at most (SHORT_SD_DIAM-SHORT_IP_DIAM).  Smaller
    values  may need to be applied if, for example, the pointing accuracy of
    the Single Dish is insufficient.

      Default value is 0, meaning to use SHORT_SD_DIAM-SHORT_IP_DIAM

\end{verbatim}
\subsubsection{SHORT\_WCOL}
\index{UV\_SHORT!SHORT\_WCOL}
\begin{verbatim}


    For tests only: The column of the spectra table containing the weights.

      Default value: 0=3, appropriate for tables coming from CLASS\GRID com-
    mand.

\end{verbatim}
\subsubsection{SHORT\_WEIGHT\_MODE}
\index{UV\_SHORT!SHORT\_WEIGHT\_MODE}
\begin{verbatim}

    The  weighting mode (NATURAL, UNIFORM or GRIDDED).  It is advised to use
    'NA' for Natural weighting.


\end{verbatim}
\subsubsection{SHORT\_XCOL}
\index{UV\_SHORT!SHORT\_XCOL}
\begin{verbatim}

        For tests only: The column of the spectra table  containing  X  off-
    sets.

      Default value: 0=1, appropriate for tables coming from CLASS\GRID com-
    mand.

\end{verbatim}
\subsubsection{SHORT\_YCOL}
\index{UV\_SHORT!SHORT\_YCOL}
\begin{verbatim}

        For tests only: The column of the spectra table  containing  Y  off-
    sets.

      Default value: 0=2, appropriate for tables coming from CLASS\GRID com-
    mand.

\end{verbatim}


%\section{Description of Associated Tasks}
%\input{mapping-task-summ}
%the file mapping-task-summ.tex was copied from gildas-src-dev/integ/doc/tasks
% see also mapping-tasks-list.tex

\newpage

\appendix{}
\section{IMAGER versus CASA}
\label{app:imager-casa}

\subsection{Imaging Philosophy and Data Architecture}
CASA intends to solve the \textit{Measurement Equation}, whatever the
complexity of this process.  It is a all-in-one package for this purpose,
where calibration and imaging are deeply intermixed and use a
unified data format.
As a result, a CASA Measurement Set is a complex architecture
encompassing relations between many components stored as Tables
in a directory-like tree. It can handle calibrated data, calibration
tables, multisource data sets, raw data and final images
in the same architecture,
allowing to retain all information to process complex images, such
as multi-frequency synthesis of polarized emission observed in
a mosaic of fields.

On the contrary, \gildas{} is designed to break the process in 
totally independent steps. For IRAM data, calibration is done
in one program (CLIC for interferometry with \NOEMA{} or 
CLASS for single-dish with the 30-m),
and imaging in another (here \imager{}), with a clear intermediate step
to change from the calibration data format (using the CLASSIC container)
to UV Tables or CLASS Tables. In particular, \imager{} does not
handle polarization transparently at the current time: 
polarization states must be imaged independently.
The Gildas Data Format stores a limited
number of informations in a single binary data file with a compact binary
header, and is only suited for calibrated data.  UV Tables
are little more elaborate than simple images, but still limited
in complexity. 

\subsection{Frequency and Velocity scales}
In the \ALMA{} use of CASA, all observations are kept in the Observatory
frequency frame, and only converted to  a celestial reference frame
(such as the Local Standard of Rest, \texttt{LSRK}) at later stages, during
imaging if requested.

\gildas{} is more analysis oriented, and contains a dual interpretation 
of the frequency axis.  This axis can either be interpreted with
a Velocity scale, usually in the \texttt{LSRK} frame relative to a spectral
line rest frequency, or as a Rest Frequency, with the astronomical
source velocity specified. The choice of representation depends
on the astronomer's science objective:  astronomical object study
(in which case the Velocity representation is more appropriate)
or chemical composition study (in which case the Rest Frequency
representation is preferred).

\subsection{Transfering UV data from \casa{}}

The basic idea of data transfer at this stage is to transfer
a whole spectral window to \gildas{}, and use the simpler and faster
tools available in \imager{} for extracting channels, resampling,
subtracting continuum, etc... rather than doing that in \casa{}.

\subsection{In CASA}
The steps in CASA involve
\begin{enumerate}
\item Separating spectral windows and different sources into 
independent (temporary) MS
\item Getting rid of flagged data 
\item Setting the velocity reference frame and correcting
for Doppler motions
\item Exporting to UVFITS 
\item Removing the intermediate MS
\end{enumerate}
The intermediate MS is created using \textit{mstransform},  with 
\texttt{keepflags=False, regridms=True, outframe='LSRK'}. Source and 
spectral window identification is done using keywords \texttt{spw} and 
\texttt{field}, and an appropriate rest frequency is specified (in MHz) 
using \texttt{restfreq}. Finally, \textit{exportuvfits}  is used on the 
simple intermediate MS. \imager{} provides to \casa{} a tool named
\textit{casagildas()} that scans the content of the Measurement Set
(using \textit{listobs()} to automatically do this on all combinations
of spectral windows and sources.

\subsection{In \imager{} }

The script \texttt{@ fits\_to\_uvt} handles the conversion from
UVFITS to \uv{} table format.
The steps in GILDAS involve
\begin{enumerate}
\item Converting UVFITS to UVT
\item Collapsing the polarization information
\item Adjusting the weights to properly estimate the noise
\item Identify and flag bad data
\item Setting the frequency and velocity references
\end{enumerate}
It also handles some nasty details, like the source coordinates
being hidden in different places depending on the \casa{} version.

\begin{itemize}
\item Step 1 \\
\texttt{fits 'name'.uvfits to 'name'.uvt}\\
will create a GILDAS UV table from the UVFITS file. The signal
is assumed to be unpolarized at this stage. With the /STOKES
option, polarization information would be carried on properly at this stage.
\item Step 2 
This step is only needed with the /STOKES option. It could be
done manually by \\
\texttt{run uv\_splitpolar} \\
that collapses the polarization information. The desired 
polarization state should be set to \texttt{NONE} to optimize signal to noise
ratio for unpolarized sources, to \texttt{I} if polarization
is a concern.
\item Step 3
This step is only carried on if the /WEIGHT option is present. 
Depending on CASA version, the weights are not handled in the same way. 
In Casa 3.4, they are approximately correct. In Casa 4.1, they are off 
by a large factor (perhaps the number of channels...).  Task 
\texttt{uv\_noise} utilizes the many channels available (3840 per 
spectral baseband) to compute a statistic per visibility, and adjust 
the weight through a median scaling factor. Task \texttt{uv\_noise} 
will also flag data with ``unusual'' weights (deviating from the median 
by more than some factor (user specified, default 3)).\\ \item Step 4 
\\ Task \texttt{uv\_trim} will remove the flagged data from the UV 
table, saving space.
\item In practice, the optional Steps 2 to 4are handled by a single
task called \texttt{uv\_casa}, which saves 2 intermediate files
(and thus 2 read and 2 writes of large files).
\item Step 5\\
At this stage, one could start usual imaging using the
standard Mapping commands \texttt{READ UV 'name'; UV\_MAP}. 
Commands \texttt{SPECIFY FREQUENCY Value} and \texttt{SPECIFY 
VELOCITY Value} will set the desired correspondence  between
Velocity and Frequency axis.
\end{itemize}



\subsection{Transfering Image data}

In the Quality Assessment step 2 (QA2), the ALMA ARC staff usually provides one
or more deconvolved images as FITS files.  These FITS files can readily
be converted into \gildas{} images with the \sic{} command \com{FITS}: \\
  \texttt{FITS 'name'.fits TO 'name'.gdf}\\
They will however have a frequency
axis labelled in FREQUENCY, while \gildas{} usually works with this
axis labelled in VELOCITY. 
They can also be accessed as \sic{} Image variables using command
\comm{DEFINE}{IMAGE}

In \imager{}, direct display of these FITS files is normally
possible with commands \com{SHOW} and \com{VIEW}.


\section{Properties of the Fourier Transform}
\label{app:fourier}
Let us name $f(x)$ a function, and $F(X)$ its Fourier transform. We
use here the simple, non-unitary convention
\begin{itemize}
\item Definition: $F(X) = \int_{-\infty}^{+\infty} f(x) e^{-2i\pi x X} dx$
\item Linearity:  $h(x) = a f(x) + b g(x)$, then $H(X) = a F5X) + b G(X)$
\item Translation:  $h(x) = f(x-x_0)$, then $H(X) = e^{-2i\pi x_0 X} F(X)$
\item Shifting: $h(x) = e^{2i\pi x X_0} f(x)$, then $H(X) = F(X-X_0)$
\item Scaling: $h(x) = f(ax)$, then $H(X) = \frac{1}{|a|}F(\frac{X}{a}$
(the so-called \textit{time reversal} property is obtained with $a=1$
\item Conjugation: if $h(x) = \bar{f(x)}$, then $H(X) = \bar{F(-X)}$
\item Integration: With $X=0$ in the definition\\
$F(0) = \int_{-\infty}^{+\infty} f(x) dx$
\item Convolution: $h(x)= f(x) * g(x)$ then $H(X) = F(X) G((X)$. \\
The Fourier Transform of a product of two functions is the convolution of
the Fourier Transforms of the functions.
\item Uncertainty principle:  the more concentrated $f(x)$ is, the 
more spread out its Fourier transform $F(X)$ must be. In particular, 
the scaling property of the Fourier transform may be seen as saying: if 
we squeeze a function in $x$, its Fourier transform stretches out in $X$. 
It is not possible to arbitrarily concentrate both a function and its 
Fourier transform. 
\end{itemize}
The definition, illustrated here with scalars, also holds for $x,X$ 
being 2-D vectors in Euclidean space, the product in the definition 
being a standard dot product of these vectors. 


% \backmatter{} % for BOOKS only

\bibliography{imager}
\bibliographystyle{aa}

\printindex{} %

\end{document}

%%%%%%%%%%%%%%%%%%%%%%%%%%%%%%%%%%%%%%%%%%%%%%%%%%%%%%%%%%%%%%%%%%%%%%%%%%%
