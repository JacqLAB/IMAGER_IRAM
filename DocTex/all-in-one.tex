\section{The "all in one" imager suite}
\label{sec:all}

With \ALMA{} or \NOEMA{}, you may end up with an observational data set
that contains several sources, each of them observed with
a number of spectral windows of different spectral resolutions,
covering many spectral lines. 

The bookkeeping of such data sets can be intricate. To help the
users to focus on the science, we have developped a suite of
scripts that automates the whole imaging process in this case.

These scripts have names starting with \texttt{all-}  and control
parameters are available in the \sicvar{ALL\%} global structure.

The principle is to gather all UV tables in a sub-directory (named
\texttt{./RAW/} by default), and to store the results of the
various processing steps (Self Calibration,  Spectral line
extraction, Imaging) in different sub-directories (respectively 
named \texttt{./SELF/, ./TABLES/, ./MAPS/} by default).

Naming conventions apply to identify which data set covers which 
spectral line. An automatic determination of the continuum
level is performed: this implies that no spectral line
confusion should occur.  The case of spectral confusion will be
handled later.

\subsection{Preparing the data}

For \NOEMA{}, the UV data can be created from the \texttt{.IPB,.hpb}
raw data files using the \clic{}  script \texttt{@ all-tables}. The 
resulting UV tables (\texttt{.uvt} files) should be placed in your
working directory.

For \ALMA{},  UVFITS files should be created from the original Measurement
Set (\texttt{.ms} directory) using the \texttt{casagildas()} Python tool 
in CASA, and placed in your working directory. 

Once your working directory is loaded with the \texttt{.uvt} or 
\texttt{.uvfits} files, you can start using the \texttt{all-widget}
script.


\subsection{The \texttt{all-} scripts}

\subsubsection{Getting started: \texttt{@ all-widget}}

\texttt{all-widget.ima} creates a Widget that is used to customize
the process and launch the various steps. It also creates
(through a call to \texttt{@ all-create}) the \sicvar{ALL\%} structure
and its components.
\begin{figure}[!h]
  \centering
  \includegraphics[width=14cm]{all-widget.png}
  \caption{The All-In-One control widget
\label{fig:allwidget}}
\end{figure}

\subsubsection{\texttt{ORGANIZE} step}
The \texttt{ORGANIZE} button will move the initial files
(\texttt{.uvfits} and \texttt{.uvt} files) into an appropriate sub-directory
structure. By default, \texttt{.uvfits} files will be in ./UVFITS/
(name controlled by \sicvar{all\%uvfits})
sub-directory, while \texttt{.uvt} files will be in ./RAW/ sub-directory
(name controlled by \sicvar{all\%raw}).

If absent, the \texttt{.uvt} files will be created from the \texttt{.uvfits} ones.

\subsubsection{\texttt{FIND} step}

The \texttt{FIND} button will explore the directory containing
the initial UV tables (\texttt{./RAW/} by default) to identify
\textit{Wide Band} UV tables, i.e. those that cover enough 
bandwidth to provide enough sensitivity
for self-calibration on continuum flux.

\textbf{CURRENT LIMITATIONS} \\
- No check is made on the continuum flux.\\
- No provision is made to use spectral line flux instead.\\

\subsubsection{\texttt{SELF} step}

The \texttt{SELF} button will use the list of \textit{Wide Band}
UV tables to compute a (phase and amplitude) Self calibration
solution, and apply it to all UV tables. It will identify which
UV tables correspond to a given Wide Band one, so that the proper
self calibration solution is applied.

The self calibrated UV tables are placed in another sub-directory
(\texttt{./SELF/} by default, controlled by \sicvar{all\%self}).

A prefix (controlled by \sicvar{all\%prefix\_self}) can be added
to the file names to remind the user that they have been self-calibrated.

\subsubsection{\texttt{TIME} step}

The \texttt{TIME} button will time average the self calibrated
UV tables to save space and speed up further processing.

\subsubsection{\texttt{IMAGING} step}

The \texttt{IMAGING} button will scan all spectral windows to
identify whether they can be used to produce a Continuum or
spectral line image. 

  Low resolution spectral windows, identified as those whose
resolution is coarser than \sicvar{ALL\%MINFRES} will be used to
produce continuum images, by filtering any detected spectral
signature in the data before.

  High resolution spectral windows, identified as those whose
resolution is better than \sicvar{ALL\%MINFRES}, will be scanned
for line identifications. For each spectral line in the current 
catalog(s) (defined by command \com{CATALOG}) 
that fall in a spectral window, a continuum-free UV table will be
created, covering the velocity range specified by the user (by
variable \sicvar{ALL\%RANGE}) around the line frequency. The naming
convention is the following:
\begin{verbatim}
  original-molecule-I-X
\end{verbatim}
where \\
- \texttt{original} is the name of the initial high resolution
spectral window\\
- \texttt{molecule} is the name of the spectral line in the catalog\\
- \texttt{I} is a sequence number, incremented each time a new
line is found from the same original UV table.\\
- \texttt{X} is a character code, equal to \texttt{D} if the line
is suspected to be detected, and to \texttt{U} if not. When several
lines are too close together, the \texttt{D} status may be incorrectly
affected, but this is just a naming convention, not a strong result.

  In addition an
\begin{verbatim}
  original-C
\end{verbatim}
file that contains line-free emission only is created for each
original UV table. 

  The imaging results are stored in a specific sub-directory 
(\texttt{./MAPS/} by default, controlled by \sicvar{all\%maps}). Only 
the Clean Component Table (\texttt{.cct}) and Clean image 
(\texttt{.lmv-clean}) files are written.

\subsubsection{\texttt{TABLES} step}

The \texttt{TABLES} button perform the same scanning and
identificatio process as the \texttt{IMAGING} button, but stores
the resulting UV Tables (instead of the .cct and .lmv-clean files for
the \texttt{IMAGING} button) in a sub-directory (\texttt{./TABLES/}
by default, controlled by \sicvar{all\%tables}).

  This steps is optional, and only needed if the user intends to 
perform some UV data analysis (like UV plane fitting, line stacking,
continuum spectral index determination, direct modeling, etc...).

\section{Really Huge Problems}

\imager{} assumes everything fits into memory. This is in general
quite fine for \NOEMA{}  data. However, for \ALMA{} data, if you are 
working with a too small computer (such as my laptop, which is 
otherwise fine), you may be lacking physical RAM, and \imager{} may 
become really inefficient by using Virtual Memory instead.

To avoid time losses, \imager{} prevents reading UV data whose size
exceeds the available RAM, and warns the user if it exceeds half
of the RAM.  To treat these cases, \imager{} provides instead
a number of tools that can work sequentially on the data set,
instead of loading it in memory all at once. 

\begin{enumerate}
\item Working by subsets:\\
  the \comm{READ}{/RANGE} command allows to select an ensemble of 
  channels from a UV data. If this ensemble is small enough, \imager{} 
  can work.  At the end the \comm{WRITE}{/APPEND} and 
  \comm{WRITE}{/REPLACE} command will allow to put these channels at 
  their proper places in a deconvolved data cube.
\item Working on UV data files:\\
  Some operations on UV data, like time averaging, separation of line 
  from continuum emission, or self-calibration, and of course, spectral 
  resampling, are best done using all valid channels to avoid loosing 
  sensitivity. 
  To allow \imager{} to do them even for large data
  files, most UV-related commands have a \texttt{/FILE} option which
  instructs the command to work from the corresponding data file,
  instead of the UV buffers. This includes \comm{UV\_PREVIEW}{/FILE}, 
  \comm{UV\_BASELINE}{/FILE}, \comm{UV\_FILTER}{/FILE} and especially 
  the \comm{UV\_SPLIT}{/FILE} commands. Time averaging can be performed 
  by \comm{UV\_TIME}{/FILE}, and prior sorting by time order can be
  done by \comm{UV\_SORT}{/FILE}. Spectral range extraction is possible by 
  \comm{UV\_EXTRACT}{/FILE}, spectral resampling by \comm{UV\_COMPRESS}{/FILE}
  \comm{UV\_RESAMPLE}{/FILE}  and \comm{UV\_HANNING}{/FILE}.
\end{enumerate}
By using the above commands, all operations can be done in a
quasi sequential way, avoiding to load in memory whole data sets.

Two commands have no equivalent using the \imager{} buffers, and work
only on files. Their \texttt{/FILE} option is used to provide an
homogeneous syntax, but is  mandatory.
\begin{enumerate}
\item the \comm{UV\_MERGE}{/FILE}\\
that allows to merge together an arbitrary number of UV tables,
in spectral line (with spectral resampling) or continuum (with 
flux scaling according to a spectral index) modes.
\item the \comm{UV\_SPLIT}{/FILE}\\
that combines the capabilities of \com{UV\_BASELINE} and 
\com{UV\_FILTER} in a single command, since both operations require 
the same parameters and provide complementary informations.
\item The \comm{UV\_SORT}{/FILE} command also has a different behaviour than
its memory only version \com{UV\_SORT}. While the later creates
a transposed version of the UV table, the former keeps the normal
organisation with the visibility axis first. 
\end{enumerate}

