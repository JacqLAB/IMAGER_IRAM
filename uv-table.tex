\section{UV Tables}

The main goal of \imager{} is to convert interferometric measurements 
stored in \uv{} tables into images suitable for astrophysical interpretation. 
\uv{} tables contain a set of visibilities. \uv{} tables being the
starting point, \imager{} contains a number of commands to handle
them.


\subsection{\uv{} table description}

\subsubsection{\uv{} table data format}

A \uv{} table is a specific 2-D Gildas table, with a few additional
informations in the header, and a special interpretation of the data
organisation.

In a standard \uv{} table, each \textit{line} describes a visibility. 
Here a \textit{line} designate either the first or second axis of the 
table, and a \textit{column} the other one. \uv{} tables may appear in 
both orders. The default one is \textit{line} on 1st axis (\texttt{.uvt} 
ordering, used by most application). The \texttt{.tuv} ordering 
obtained by a 21 transposition is used essentially for display, as in 
this case the \textit{column} as the same meaning as for the 
\com{COLUMN} of \greg\ .

The number of lines of a \uv{} table is thus the number of visibilities described
in the table. Each \textit{column} of the table stores a particular property of the
visibilities, namely:
\begin{description}\itemsep 0pt
\item[Column 1] U in meters;
\item[Column 2] V in meters;
\item[Column 3] Scan number;
\item[Column 4] Observation date (integer \class{}/\clic{} Day
  Number\footnote{The \class{}/\clic{} is a "radio Julian date" (or "Jansky
    Julian date"), which starts as $-2^{15}$ on the date of the first radio
    observation by Karl Jansky. It is thus the Modified Julian date minus
    60549.  That choice was made to maximize the time interval over which
    radio astronomical data could be usefully stored in an
    \texttt{integer*2}, back when 2 bytes of header space per spectrum were
    a significant consideration.  This date has little meaning outside the
    rather sparse community of souls gathered around the \class{} and
    \clic{} programs, however...});
\item[Column 5] Time in seconds since 0:00 UT of above date;
\item[Column 6] Number of the first antenna used to measure the visibility;
\item[Column 7] Number of the second antenna used to measure the visibility;
\item[Column 8] Real part for the first frequency channel;
\item[Column 9] Imaginary part for the first frequency channel;
\item[Column 10] Weight for the first frequency channel;
\item[Columns 11-13] Same as column 8-10 but for the second frequency
  channel, or for the second Stokes parameter of this channel.
\item[...] etc\ldots for all channels
\item[Columns N-ntrail+1 ... N] Trailing columns after the channel visibilities.
\end{description}
If a \uv{} table describes \texttt{nvis} visibility spectra composed of
\texttt{nchan} frequency channels, each with \texttt{nstokes} Stokes parameters,
the size of the table will thus be:
\texttt{nvis} lines of \texttt{7+3*nchan*nstokes+ntrail} columns, where \texttt{ntrail}
is the number of trailing column.

\subsubsection{\uv{} header}

A \uv{} table header contains all the informations of a GDF header but some
of these informations have a special meaning in this context. Command
\com{HEADER} is the standard way inside \gildas{} to display in a human
readable way the header of GDF file. For instance, the command \\
\texttt{IMAGER> header gag\_demo:demo-line.uvt}  \\
would display
  %I-GIO_RIH,  File is  [EEEI to IEEE] , Header Version 1 (32 bit)
\begin{verbatim}
 1  W-GDF,  UNKNOWN Velocity type defaulted to LSR
 2  File : /Users/guilloteau/gildas/gildas-exe-dev/demo/demo-line.uvt  REAL*4
 3  Size        Reference Pixel           Value                  Increment       
 4       103   16.0000000000000       220398.688000000     -0.183792725205421
 5      9146   0.00000000000000       0.00000000000000       1.00000000000000
 6  Blanking value and tolerance      1.23455997E+34   0.0000000
 7  Source name         GG_TAU
 8  Map unit            Jy
 9  Axis type           UV-DATA      RANDOM
10  Coordinate system   EQUATORIAL          Velocity    LSR
11  Right Ascension   04:32:30.34200        Declination       17:31:40.5230
12  Lii        0.000000000000000            Bii       0.000000000000000
13  Equinox            2000.0000
14  Projection type     AZIMUTHAL           Angle     0.000000000000000
15  Axis 0     A0     04:32:30.34200        Axis 0     D0     17:31:40.5230
16  Baselines               0.0       0.0
17  Axis 1 Line Name    13CO(21)            Rest Frequency   220398.6880000000
18  Resolution in Velocity   0.25000000     in Frequency        -0.18379273
19  Offset in Velocity        6.3000002     Doppler Velocity     -40.755900
20  Beam                   0.00                0.00                 0.00
21  NO Noise level
22  NO Proper motion
23  NO Telescope section
24  UV Data    Channels:     32, Stokes: 1 None        Visibilities:        9146
25  Column            1 (Size 1) contains U           
26  Column            2 (Size 1) contains V           
27  Column            4 (Size 1) contains DATE        
28  Column            5 (Size 1) contains TIME        
29  Column            6 (Size 1) contains IANT        
30  Column            7 (Size 1) contains JANT        
31  Column            3 (Size 1) contains SCAN        
\end{verbatim}
Comments:
\begin{description}\itemsep 0pt
\item[Line 1] Indicates the velocity frame. If not present in the table
(as here), it is assumed to be LSR.
\item[Line 2] Indicates the filename associated to the currently displayed
  header.
\item[Lines 3-5] Display the dimensions of the associated array. Here it is
  a rank 2 array of dimension 9146 lines times 9146 lines, \ie\ 9146 
  visibility spectra of 32 frequency channels.  Line
  4 describes the frequency axis of the visibility spectra stored in the
  \uv{} table. Be careful that this is a convention, \ie\ it must be
  decoded using the particular form of the table. In our case, each spectra
  has 32 frequency channels of width -183.8~kHz, the frequency of the
  reference pixel 16.0 corresponding to 220398.688~MHz. This last
  frequency is the frequency delivered by the correlator, \ie\ seen by the
  observatory. In particular, this is the frequency that must be used to
  compute the primary beam of the interferometer.
\item[Line 8] Indicates the unit of the real and imaginary parts of the
  visibilities, normally the Jansky (Jy).
\item[Line 9] Indicates that this is \uv{} table (\texttt{UV-DATA} and
  \texttt{RANDOM}).
\item[Lines 10-13] Describe the coordinate system.
\item[Lines 14-15] Describe the projection system. In the \uv{} table
  format, \texttt{A0} and \texttt{D0} indicate the phase center while
  \texttt{Right Ascension} and \texttt{Declination} indicate where the
  antenna pointed when acquiring the signal. These information are in
  general identical for single field imaging and different for mosaicing.
\item[Lines 16] Indicates the baseline range in meters (m). 
\item[Lines 17-19] Describe additional information about the frequency axis
  of the visibility spectra. In particular, the rest frequency (here
  220398.688~MHz, that of the $^{13}$CO J=2-1 line) corresponding to a 
  velocity of 6.3~km/s in the velocity frame indicated at line 1 (in general LSR).
\item[Line 20] Indicates the primary beam size of the interferometer in
  radian. This is an obsolescent way to pass the size of the interferometer
  antennas.
\item[Line 21] The noise section has no meaning for the UV table.
\item[Line 22] If present, proper motions are given in mas/yr. The epoch
is used as the time origin.
\item[Line 23] If the TELESCOPE section is present, this line would 
indicate telescope name, its geographic coordinates and the antenna 
diameter (in m). This is the new way to specify the primary beam.
\item[Line 24] UV data section: number of channels, number of Stokes
parameters and number of visibilities.
\item[Line 15 to end] Special columns description.
\end{description}


\subsection{UV Table handling}

Besides the \comm{READ}{UV} and \comm{WRITE}{UV} commands to read
or write \uv{} tables, \imager{} has a number of commands to manipulate
the current \uv{} table buffer. These commands have names starting
by \texttt{UV\_}. Most of them are in the \lang{CLEAN} language, some
in the \lang{NEWSTUFF} one. 

\imager{} works using UV buffers. Most commands only work on 
the current UV buffer, but some of them keep track of the previous
buffer to allow the user to revert the operation.

\begin{description}\itemsep 0pt

\item[Data inspection and editing]:
\begin{itemize}\itemsep 0pt
\item \comm{SHOW}{COVERAGE} display the \uv{} coverage
\item \comm{SHOW}{UV} display the \uv{} data
\item \com{UV\_FLAG} allow flagging visibilities
\item \com{UV\_PREVIEW} provides a quick view of the visibilities as a function
of frequencies, and attempts to automatically find the continuum level
and parts of the bandwidth with spectral line emissions.
\end{itemize}

\item[Data size reduction routines]:
\begin{itemize}\itemsep 0pt
\item \com{UV\_COMPRESS} is a simple spectral smoothing, providing only channel 
averaging by integer number of channels.
\item \com{UV\_RESAMPLE} provides a more flexible spectral smoothing and resampling
facility. 
\item \com{UV\_TIME}  can be used to time-average the UV data set, leading
to faster processing. However, using \com{UV\_TIME} too early may limit
your ability to perform accurate phase self-calibration.
\end{itemize}

\item[Continuum processing commands]:
\begin{itemize}\itemsep 0pt
\item \com{UV\_BASELINE} allows to remove the continuum baseline, by
0th or 1st order baseline fitting of each visibility. 
\item Conversely, \com{UV\_FILTER} will filter the spectral line
range to leave only the channels with continuum emission.
Both \com{UV\_BASELINE} and \com{UV\_FILTER} can use the results provided
by \com{UV\_PREVIEW} to specify where spectral lines may be found.
\item \com{UV\_CONTINUUM} converts a spectral line \uv{} table into a bandwidth 
synthesis continuum \uv{} table. \com{UV\_CONTINUUM} requires some knowledge of 
the image size to evaluate how many channels should be averaged 
together. This is done using the same parameters and subroutines as 
for commands \comm{UV\_STAT}{SETUP} and \com{UV\_MAP}.
\end{itemize}

\item[Image preparation]:
\begin{itemize}\itemsep 0pt
\item \com{UV\_CHECK} inspects the \uv{} data to figure out how many
different synthesized beams are needed.
\item \com{UV\_SHORT} adds the short (or zero) spacing information provided by
an additional single dish data, read by \comm{READ}{SINGLE}.
\item \com{UV\_STAT} evaluates the impact of robust weighting
and tapering on the synthesized beam. It provides recommandations for
the image and pixel sizes.
\item \com{UV\_TRUNCATE} restricts the \uv{} baseline length range.
\end{itemize}

\item[Miscellaneous]:
\begin{itemize}\itemsep 0pt
\item \com{UV\_DEPROJECT} de-project the $(u,v)$ coordinates given
a specified phase center, orientation and inclination. This can be
useful for inclined, flattened, nearly axi-symmetric structures such as
proto-planetary disks or galaxies. 
\item \com{UV\_RADIAL} computes the azimutal average of the visibilities.
It is useful for rotationally symmetric structures such as proto-planetary disks, 
for example.
\item \com{UV\_REWEIGHT} changes the visibility weights.
\item \com{UV\_SHIFT} changes the phase center
\end{itemize}

\end{description} 

The remaining \com{UV\_...} commands are related to imaging and
deconvolution: \com{UV\_MAP} computes the dirty image, \com{UV\_RESTORE}
computes the Clean image from a Clean component list by removal
of the Clean components in the \uv{} plane, and imaging of the residuals.
\com{UV\_RESIDUAL} just computes the residuals by subtraction of the
Clean components. 

Finally, \com{UV\_SELF}, in the \lang{CALIBRATE} language, is a specific variant
of \com{UV\_MAP} used to compute the intermediate images required for
self-calibration. It is not intended for direct use by normal users.

