\section{Self calibration}

\subsection{In a nutshell}

\begin{verbatim}
  1  read uv YourData
  2  selfcal phase
  3  selfcal summary
  4  selfcal show
  5  selfcal apply
  6  uv_map
  7  clean
  6  write * YourSelf 
\end{verbatim}
\begin{enumerate}\itemsep 0pt
\item Read your \uv{} data
\item Self-calibrate the phase
\item Get a summary of the result
\item Show the phase correction between the last 2 iterations
\item Apply the self-calibration
\item Image as usual
\item Clean as usual
\item Save the result if it is worthwhile...
\end{enumerate}
You are done. But, now, if the result is crazy, do not blame the
software. Rather read carefully the information below: \com{SELFCAL}
is simple to use, but there are some pitfalls... 


\subsection{Principle}

The self-calibration idea is based on the fact that the dominant error terms
are antenna-based, while source information is baseline-based. 
With $N$ antennas, one gets at any time $N (N-1) /2$ visibility measurements, 
but $N$ amplitude gains, and only $N-1$ error terms for the morphology of the source (phase gains). 
The $N-1$ number is because only relative phases count. The absolute flux
scale is a separate problem, and therefore also $N-1$ relative amplitude gains count.

The measured visibilities 
on baselines from antenna $i$ to antenna $j$ at time $t$ are, 
from the simplified measurement equation:
\begin{equation}
    V_{\rm obs}(i,j,t) = G(i,t) G^{{\star}}(j,t)  V_{\rm true}(i,j) + Noise
\end{equation}    
where $G(i,t)$ is the complex (phase and amplitude) gain for the 
antenna $i$ at time $t$. The true visibility $V_{\rm true}(i,j)$ only 
depends on the baseline $(i,j)$, not on the time.
  
Given a source model $V_{mod}(i,j)$, one can derive the antenna gain
products at time $t$, based on the system:
 \begin{equation}
  \frac{V_{\rm obs}(i,j,t)}{V_{\rm mod}(i,j)} = G(i,t) G^{{\star}}(j,t) 
  \end{equation}
which is an over-constrained process, since there are $N(N-1)/2$ 
constraints for $N-1$ unknowns. Solving for this over-constrained 
problem is similar to deriving the amplitude and phase solution from a 
calibrator observation. In the calibrator case (i.e., an unresolved 
source like a distant bright quasar), $V_{\rm mod(i,j)} = (1.0,0.0)$ 
(constant amplitude, zero phase), so there is no risk of noise 
amplification in the process. 
%The antenna gain corrections are determined 
%by minimizing the distance between the observed data and the ''model'':
%\begin{equation}
%    S_{k} = \Sigma_{i,j} w_{i,j} \| V_{\rm obs}(i,j,tk) - G(i,tk) G^{{\star}}(j,tk)  V_{\rm true}(i,j)\|^{2}
%\end{equation}

For any (not a point-like) source, $V_{\rm mod}$ must be guessed. 
Self-calibration will use your source to improve the calibration of the 
antenna-based (complex) gains as a function of time. The practice is to 
proceed iteratively, based on a preliminary deconvolution solution. Let 
$V_{\rm obs}(k)$ be the ``observed'' visibilities at iteration $k$, with 
$V_{\rm obs(k=0)} = V_{\rm obs} $ the raw calibrated visibilities. Some 
of the Clean components derived from $V_{obs}(k)$ are used to define 
''model' visibilities $V_{\rm mod}(k)$. Then, solving for the antenna 
gains, one obtains: 

\begin{equation}
V_{\rm obs}(k+1) = \frac {V_{obs}(k)}{(G_i G^{{\star}}_j)}
\end{equation}

The model is thus progressively refined, and in the end, satisfies 
better the initial constraints on the source shape and on the antenna 
gains as a function of time provided by the measurements. Note that the 
absolute phase (and hence the position) can be lost in the 
self-calibration process and it should not be used for absolute 
astrometry.

There are two types of self-calibration: phase and amplitude self-calibration. 
The amplitude gain is a more complex problem than the phase gain.
Amplitude gains can (and often do) vary with time, but from the
measurement equation, a scale factor in the amplitude gain can be
exchanged by a scale factor on the source flux. It is thus customary
to re-normalize the gains so that the source flux is conserved
in the process. An alternate (perhaps not strictly equivalent) solution 
is to ensure that the time averaged product of the amplitude gains is 1.
The two approaches differ by the averaging process.

For any typical source, $V_{\rm mod}$ is non zero and of magnitude 
smaller than 1 (using the total flux as a scale factor) since the 
source is partially resolved. So in computing $V_{obs}/V_{\rm mod}$, 
there is noise amplification. It may even be the case that $V_{mod}$ is 
zero (case of an extended, over-resolved emission), and thus some 
(long) baselines will yield no direct constraint on the antenna gains 
$G(i) G^{{\star}}(j)$. But this should not matter too much for 
self-calibration, for two reasons. First, other (i.e., shorter) 
baselines may provide contraints on the gains. Second, if all 
$V_{\rm obs}$ for an antenna are
close to zero, it implies $V_{\rm mod}$ must be close to zero too, 
so an error on the phase of those visibilities (as well as on
its magnitude) is not so important. 

Self-calibration is related to the ``closure'' relations. For any triplet 
of antennas, the phase of the triple product $V_{ij} V_{jk} V_{ki}$ is 
independent of the antenna errors, and thus is (within the noise) a 
bias free constraint on the source. Similarly, for any quadruplet of 
antennas, the amplitude of the ratio $(V_{ij} V_{kl}) / (V_{ik} 
V_{jl})$  is independent of the antenna errors. But here, the noise 
amplification can be large because of the likelihood to have two small 
visibilities. For this reason, amplitude self-calibration requires in 
practice higher signal to noise ratios than phase calibration in the 
initial deconvolved data set used as a model. 
%EDF: Typical numbers here ?, 

Among the advantages of self-calibration, one may emphasize that 
antenna gains are derived at the correct time of the science object 
observation, while they must be interpolated in the classical 
calibration approach. Both atmospheric and electronic noises are 
supposed to vary with time, although with different timescales. Gains 
are also computed in the correct direction on the celestial sphere, 
while the calibrator-based approach introduces differences in the 
pointing direction with respect to the science object. The robustness 
of the approach increases with the number of baselines. 

In order to implement self-calibration, it is however necessary that 
the signal to noise ratio be large enough (the process will require a 
sufficient bright source). Self-calibration can especially bring 
significant improvements to the calibration solution in the case of 
higher than expected background noise, or in the presence of 
convolutional artifacts around objects, especially point sources.



\subsection{Implementation}

The typical procedure for self-calibration consists in an iterative 
process based on the following steps:
\begin{enumerate}\itemsep 0pt
\item \comm{UV\_MAP}{/SELF} + \com{CLEAN} + \comm{UV\_RESTORE}{/SELF}: \\
from the classically calibrated (and 
preliminarily flagged) data, define an initial source model through a 
conservative (not too deep) first deconvolution : the model consists of 
a reasonably modest number of CLEAN components.
\item \com{SOLVE}: determine an estimate of the antenna gains (best fit 
to the observed visibilities)
\item \com{APPLY}: apply the derived gains to correct the observed data
\item \com{STATISTIC}: compare to the initial Image, and estimate the 
improvement through an adequate quality assessment (e.g., improvement 
of the dynamic range)
\item If necessary, re-build a new model from these corrected data, and 
iterate until the solution is satisfactory.
\end{enumerate}

Phase-only self-calibration is less stringent on the signal-to-noise 
ratio (SNR) threshold than amplitude self-calibration, and it should 
therefore be attempted first. For phase self-calibration to work, the 
SNR values in the initial data should be at least of \texttt{SNR>3} per 
antenna (in a solution interval shorter than the time for significant 
phase variations for all baselines to a single antenna). The SNR 
threshold in the initial image depends on the number of antennas and on 
the adopted time averaging. Depending on the complexity of the source 
(and the contribution from extended emission), all available baselines 
may not be considered in the process, and specific preliminary flagging 
could be necessary. Amplitude errors tend to be negligible for dynamic 
ranges below about 500. Amplitude self-calibration will thus 
be eventually attempted in a subsequent step.

Self Calibration is available in \imager{} through the command 
\com{SELFCAL}, which uses an iterative scheme driven by a script 
\texttt{(gag\_pro:p\_selfcal.ima)}.  From the above principle, the script 
controls\\
  a) the number of iterations\\
  b) the number of selected clean components at each iteration\\
  c) the time scale of the solution, i.e. the integration time
  over which the gains are assumed to be constant

The parameters of the command \com{SELFCAL} are available as 
\sicvar{SELF\_Names} \sic{} variables.  The script uses the commands 
\com{UV\_MAP}, \com{CLEAN}, and  \com{SOLVE} and \com{APPLY} from the 
\lang{CALIBRATE} language. By default, a solution is searched for the 
phase calibration only (\sicvar{SELF\_MODE}\texttt{=PHASE}), and 
the number of iterations is 3 (\sicvar{SELF\_LOOP}). For each 
iteration: 
\begin{itemize}\itemsep 0pt
\item All components found by \com{CLEAN} are kept by default 
(\sicvar{SELF\_NITER}\texttt{=0}), but for simple source structures, 10 components only 
may be enough (the maximum number \sicvar{NITER} of  CLEAN components to 
subtract is automatically guessed by the program in the default 
process, see \sicvar{CLEAN\_NITER} and other usual clean convergence control 
variables),
\item the minimum flux density per pixel to be considered by CLEAN can 
be defined (\sicvar{SELF\_MINFLUX}\texttt{=0}) 
\item The ''integration'' times (gain averaging) are fixed to a default 
value \sicvar{SELF\_TIMES}\texttt{=45}\,s (minimum value for NOEMA, while it is only 
6\,s for ALMA). This can be adapted for each loop (in general, one 
should start with larger solution times, depending on the SNR values 
and try to decrease it in order to better sample the atmospheric 
fluctuations). 
\end{itemize}
The number of iterations can be changed by resizing the 
\sicvar{SELF\_TIMES}, \sicvar{SELF\_NITER}, or \sicvar{SELF\_MINFLUX} 
arrays (the number of loops \sicvar{SELF\_LOOP} is then automatically 
recomputed). You should make sure that these 3 arrays have the same 
dimension. If any of the \sicvar{SELF\_NITER} and/or 
\sicvar{SELF\_MINFLUX} array are constant then their dimension is 
accordingly changed by \com{SELFCAL} each time one of these arrays is 
resized. For instance, the following command allows to define 5 loops 
with an integration time for solution of 45\,sec:

\texttt{let Self\_times 45 45 45 45 45 /resize}

\sicvar{SELF\_NITER} and \sicvar{SELF\_MINFLUX} are automatically enlarged
to the same size if, and only if, they were already constant.
By default, all channels are averaged to compute a 
''continuum'' image, but  the range of adequate channels can be 
specified through \sicvar{SELF\_CHANNEL}. 

\comm{SELFCAL}{PHASE} will compute a phase calibration, but will not 
apply it. One needs to call \com{SELFCAL} once more with the argument 
APPLY in order to apply the solution (the script does really apply the 
solution if and only if the previously found solution can be considered 
as a good one (see SELF\_{STATUS} argument value). 

\comm{SELFCAL}{APPLY} automatically saves the parameters and results in 
the \textit{selfcal.last} file. By default, data are flagged if no 
sufficiently good solution is found. \comm{SELFCAL}{APPLY} keep tracks 
of whether the solution has already been applied through 
\sicvar{SELF\_APPLIED}. \comm{SELFCAL}{APPLY} will refuse to apply 
``bad'' solutions: solutions are declared ``bad'' if the improvement in 
dynamic range and noise level is insufficient (i.e. below a precision 
level controlled by \sicvar{SELF\_PRECISION}). In this case, the 
\sicvar{SELF\_STATUS} variable is negative. In this case, the user can 
still decide to apply the solution directly using command \com{APPLY}, 
but the \sicvar{SELF\_APPLIED} variable will not be updated.

The merit criteria for the quality assessment of the computed solution 
are the final dynamic range, and the Clean map noise at each iteration. 
These quantities are stored in the \sicvar{SELF\_DYNAMIC} and 
\sicvar{SELF\_RMSCLEAN} variables (at each iteration). The dynamic is 
defined as the ratio of the peak flux density value to the noise in the 
clean map (automatically estimated with the command \com{STATISTIC}). 
The  minimum signal to noise ratio value (for an antenna) for a valid 
solution is \sicvar{SELF\_SNR}\texttt{=6} by default. 

\com{SELFCAL} can be controlled through a widget, using
command \comm{SELFCAL}{/WIDGET} (see Figure \ref{fig:selfcal}) 

\begin{figure}
  \centering
  \includegraphics[width=10cm]{selfcal-widget.png}
  \caption{The Self-Calibration widget
\label{fig:selfcal}}
\end{figure}



It is possible to visualize the computed corrections with the command 
\comm{SELFCAL}{SHOW}. The solution computed with the \com{SOLVE} 
command is written in a \textit{'self\_sname'.tab} file. By default, 
the difference between the last two iterations is displayed. For 
\texttt{PHASE}, the phase difference should be close to zero if the 
solution converged, and for \texttt{AMPLI} the amplitude values close 
to 1. It is also possible to show the difference between two specified 
iterations. In addition, the command \comm{SELFCAL}{SUMMARY} will 
display the results of the process in terms of resulting noise and 
improved dynamic range. If the solution is satisfactory, the command 
\comm{SELFCAL}{SAVE} can be used to save both the results and the 
parameters in the \textit{selfcal.last} file. (\comm{SELFCAL}{APPLY} 
performs an implicit \comm{SELFCAL}{SAVE}.)

\subsection{Basic use }
\subsubsection{Timescale for averaging solution interval}

The solution interval, or timescale used to average the solution for 
the gain variations, is the result of a tradeoff between the timescale 
of the true gain variations (e.g., changes in the atmospheric 
conditions or electronics) and the data averaging which is necessary to 
reach a minimum SNR value for the visibilities. In principle, this 
timescale should be smaller than the coherence time
of the atmospheric fluctuations for the phase solution, typically
one minute. It is in general longer for the amplitude gains, since here
these are changes in the atmospheric transparency or antenna gains
which matter. It is 
therefore recommended to start iterations with a not too short 
averaging time, and to decrease it in a second step if the 
self-calibration was successful. It is also recommended to use the same 
integration time for the last two iterations, in order to ease the 
interpretation of the results and of the \comm{SELFCAL}{SHOW} display.
 
The current self-calibration method computes baseline-based
gains (observed complex visibility divided by model prediction) 
for each visibility, and performs the time averaging on these
baseline-based gains. Antenna based gains are derived from
the time-averaged baseline-based gains.  This is in principle an 
optimal method, since the model is not noisy: the linearity of the
first step guarantees a noise decrease as square root of time.

%Boxcar smoothing was used in AIPS in the early VLA times. CASA uses the CLIC engine in a unknown way.

\subsubsection{Quality assessment and data flagging}

\begin{description}\itemsep 0pt
\item [Validation of the solution] 
One of the difficulties of self-calibration is to evaluate whether
it has improved the image or not. The self-calibration 
solution is biased towards the assumed model. If used with insufficient
signal to noise, it will tend to produce a point source at the
initial peak position, and the bias will be of order of the noise.
This may be inappropriate. 
Currently, the validity of the self-calibration solution is based on 
the estimated signal to noise ratio for the gains at each time step. 
If that SNR is below a user-controlled threshold (by default, \sicvar{SELF\_SNR}\texttt{=6}), 
the corresponding data is flagged (default value: \sicvar{SELF\_FLAG}\texttt{=YES}) 
or kept WITHOUT self calibration (if \sicvar{SELF\_FLAG}\texttt{=NO}). 
%
\item [Flagging or not flagging ?] 
The decision to flag or not results from a trade-off:
\begin{itemize}\itemsep 0pt
\item \sicvar{SELF\_FLAG} $=$ \texttt{Yes} : will result in no contamination by bad data, but
may lead to lower angular resolution since long baselines may be flagged.
\item \sicvar{SELF\_FLAG} $=$ \texttt{No}: will avoid loosing all long baselines (where the SNR is lower)
\end{itemize}
Both options may be explored, and it is recommended to check afterwards the final angular 
resolution with and without flagging.
\end{description}

The following scheme is proposed to check the validity of the self-calibration solution:
\begin{itemize}\itemsep 0pt
\item read the status using \comm{SELFCAL}{SUMMARY}
\item use \comm{SELFCAL}{SHOW} to verify if the solution is converged
\item if it looks good, but noise is still far from theoretical, try 
again to self-calibrate with a shorter integration time (SELF\_TIMES). 
\item if it is not good, try to increase \sicvar{SELF\_TIMES} and find an 
optimum value. For ALMA data, typical values may be in the range 
$6-60$\,s, and for NOEMA in the range $45-120$\,s. Alternatively, you 
can also try to decrease \sicvar{SELF\_SNR} to lower values, but never
less than 3.
\end{itemize}

From our experience, the number of loops \sicvar{SELF\_NLOOP} does not impact 
much the quality of the solution, and 2 to 3 iterations are usually 
sufficient.

%Propositions and discussion:
%Min/max values should be displayed:  Nyquist sampling time could be automatically calculated 
%how is the maximum expected dynamic range computed ? (smaller than the one obtained after selfcal ! ;)
%sometimes Expected rms noise is       NaN microJy/beam (SNR>=4)
%comment afficher les donnees flagguees (visualiser quelles bases on ete flagguees)

Caveat: the comparison with theoretical noise relies on a proper 
scaling of the weights of the UV data. This was fine for the previous
IRAM correlator, but data exported from CASA is not always correct in this respect.
It also remains to be checked with the PolyFIX NOEMA correlator. 
\com{UV\_PREVIEW} can warn you about potential issues in this respect. 
The appropriate scaling factor can be specified by variable \sicvar{SELF\_SNOISE}.

\subsection{Advanced use}

\paragraph{Amplitude self calibration}
The amplitude calibration (\comm{SELFCAL}{AMPLI}) is a secondary step in the self-calibration process.
In general, it should only be attempted if the phase calibration was 
already excellent, i.e., once you obtained the best solution by 
adjusting the \sicvar{SELF\_TIMES} parameter for the 
\comm{SELFCAL}{PHASE} command. If possible or needed, the amplitude 
self-calibration should use a \textit{longer timescale} than the phase 
calibration (typically, \sicvar{SELF\_TIMES}$=120$s). \com{SELFCAL} 
automatically adjusts the gains so that their mean is 1, in order to 
avoid changing the flux scale. In practice, it is useless if the expected noise
limited dynamic range is less than about 300.

\paragraph{Cases where Amplitude self calibration may be essential}
If \texttt{PHASE} self-calibration does not sufficiently improve the image
despite ample signal-to-noise, Amplitude self calibration may do it. 
This situation often occurs when the observations span different dates,
so that the relative flux calibration between the separate dates is 
inconsistent. In this case, a \texttt{AMPLI} self-calibration may
be of great help. The flux consistency scale across dates can be
also checked and cured using command \com{SCALE\_FLUX}.
 
Note however that the resulting improvement does not a higher fidelity
image. It removes the inconsistencies, but the selected (average) flux scale
has no guarantee to be good. Flux calibration should be independently
checked if this situation occurs.

\paragraph{Support restriction, flux threshold}
Support restriction in the \com{CLEAN} process may be needed to build a 
simpler model for very complex, extended sources only.  Command \comm{MASK}{THRESHOLD} can
be useful in this respect. Similarly, limiting
the flux per pixel in the model (see \sicvar{SELF\_MINFLUX}) may help.

\subsection{Transfering the solution to other \uv{} data sets.}
An important use of the Self-calibration is to compute a calibration solution
using wide bandwidth data (\textit{continuum} data, where signal-to-noise can 
often be maximized) and apply it to high spectra resolution (\textit{line}) data.

The technique is quite simple. Solve for Self-calibration using the wide bandwidth
data as usual, and save the results (gain tables) using the \comm{WRITE}{CGAINS}
command (eventually one for each of the \texttt{PHASE} or \texttt{AMPLI} steps of 
Self-calibration).

At any time, you can read back these gain tables using \comm{READ}{CGAINS}, read
an \uv{} data set with \comm{READ}{UV} and apply the gain solution using command
\com{APPLY}.  Since Self-calibration normally corrects for atmospheric errors,
the derived `phase'' correction is in general here to compensate for a pathlength
change, i.e. the corresponding phase correction should scale as Frequency.  \comm{APPLY}{DELAY}
instructs \imager{} to take this effect into account.

 
