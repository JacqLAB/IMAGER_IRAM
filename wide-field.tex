
\section{Wide-field imaging and deconvolution}

We are often asked why the wide-field imaging and deconvolution steps are
more difficult than their equivalent for single-field. The main answer is
that doing wide-field observations with an interferometer is kind of
paradoxical. Indeed, (sub)millimeter interferometers are before all tuned
to get the best possible spatial resolution. A natural consequence is the
lack of measurement of the low spatial frequencies which are extremely
important in wide-field observations. Hence the paradox.  

Progress in the design (ALMA was designed with wide-field imaging as a 
main goal) or in performances (NOEMA and the 30-m) has led to 
wide-field images being now customary.  The tools have become much 
simpler and user-friendly (see below !) but because of its paradoxical 
nature, wide-field imaging with an interferometer implies a 
knowledgeable use of those tools.

\subsection{In a nutshell}

\begin{verbatim}
  1  read uv gag_demo:demo-mosaic-v22.uvt
 (2)  read single gag_demo:demo-single-v22.tab
 (3)  uv_short
  4  uv_map
  5  clean
  6  write * MyDemo
\end{verbatim}
\begin{enumerate}\itemsep 0pt
\item Read your \uv{} data
\item Optionally, read your single-dish data
\item Optionally, merge it with with the \uv{} data set,
by using the single-dish data to provide short spacings
for the interferometer data.
\item Image as usual
\item deconvolve as usual
\item Save the result.
\end{enumerate}
You are done. If the \uv{} data set is a \textbf{Mosaic} data set,
it works as if it is a single-field. But, now, if the result is crazy, do not blame the
software. Rather read carefully the information below: \textbf{Mosaics} and
\com{UV\_SHORT} are simple to use, but can be tricky to use well !...

\subsection{General considerations about wide-field imaging}

The measurement equation for a millimeter interferometer is to a good
approximation (after calibration)
\begin{equation}
  V(u,v) = \mbox{FT}\cbrace{B_\emr{primary}.I_\emr{source}}(u,v)+N
\end{equation}
where $\mbox{FT}\cbrace{F}(f)$ is the bi-dimensional Fourier transform of
the function $F$ taken at the spatial frequency $f$, $I_\emr{source}$ the
sky intensity, $B_\emr{primary}$ the primary beam of the interferometer
(\ie\ a Gaussian of FWHM the natural resolution of the single-dish antenna
composing the interferometer), $N$ some thermal noise and $V(u,v)$ the
calibrated visibility at the spatial frequency $u,v$. The product of the
sky intensity by the primary beam, which ``quickly'' decreases to zero,
implies that an interferometer looking at a particular direction of the sky
will have its field-of-view limited by the size of the primary beam.
 
To image a field-of-view larger than the primary beam size, the antennae of
an interferometer will be successively pointed in different directions of
the sky typically separated by half the size of the primary beam. This
process is called mosaicing and the result requires specific imaging and
deconvolution steps. Another possibility is to acquire data as the
interferometer antenna continuously slew through a portion of the sky. This
second observing mode is called interferometric On-The-Fly (OTF). While
mosaicing is standard at NOEAM (see section~\ref{sec:mosaicing}), some
efforts are currently done to commission the OTF observing mode.

Mosaicing and OTF clearly belongs to wide-field imaging. However
considerations about wide-field imaging start as soon as the size of the
source is larger than about 1/3 to 1/2 of the interferometer primary beam.
Indeed, a multiplicative interferometer (\eg\ all interferometer in the
(sub)mm range) is a bandpass instrument, \ie\ it filters not only the large
spatial frequencies (this is the effect of the finite resolution of the
instrument) but also the small spatial frequencies (all the frequencies
smaller than typically the diameter of the interferometer antennas). An
important consequence is that a multiplicative interferometer do \emph{not}
measure the total flux of the observed source. This derives immediately
from the following property of the Fourier Transform: The Fourier transform
of a function evaluated at zero spacial frequency is equal to the integral
of your function. Adapting this to our notation, this gives
\begin{equation}
  V(u=0,v=0) \stackrel{\mbox{FT}}{\rightleftharpoons} \sum_{ij~\in~\emr{image}} \cbrace{B_\emr{primary}.I_\emr{source}}_{ij}.
\end{equation}
\ie\ the visibility at the center of the \uv{} plane is the total intensity
of the source. As a multiplicative interferometer filters out in particular
$V(u=0,v=0)$, the information about the total flux of the observed source
is lost. In summary, a multiplicative interferometer only gives information
about the way the flux of the source is distributed in the spatial
frequencies larger than the primary beam but no information about the total
flux.

Deconvolution algorithms use, in one way or another, the information of the
flux at the smallest \emph{measured} spatial frequencies to extrapolate the
total flux of the source. This works correctly when the size of the source
is small compared to the primary beam of the interferometer. The extreme
case is a point source at the phase center for which the amplitude of all
the visibilities is constant and equal to the total flux of the source:
Extrapolation is then exact. However, the larger the size of the source,
the worst the extrapolation, which then underestimates the total source
flux. This is the well-known problem of the missing flux that observers
sometimes note when comparing the flux of the source delivered by a mm
interferometer with the flux observed with a single-dish antenna. The
transition between right and wrong extrapolation is not well documented. It
depends on the repartition of the flux with spatial frequencies but also of
the signal-to-noise ratio of the measured spatial frequencies. It is often
agreed that the transition happens for sizes between 1/3 and 1/2 of the
interferometer primary beam. For larger source size, information from a
single-dish telescope is needed to fill in the missing information and to
thus obtain a correct result. This is the object of
section~\ref{sec:short-spacings}.

\subsection{Mosaicing}
\label{sec:mosaicing}

\subsubsection{Observations and processing}

In a single-field observation, an interferometer tracks a particular
direction of the sky, named the phase center. The portion of the sky which
can be image around this direction is directly linked to the size of the
primary beam. The easiest way to image field-of-view larger than the
primary beam size is to track one direction of the sky after another until
the desired field-of-view is filled with small images made around many
different tracking directions. This observing mode is called mosaicing and
the tracked observations which constitute the mosaic are called fields.

There are many constraints to optimize mosaicing.
\begin{description}
\item[Nyquist sampling of the mosaic field-of-view and mosaic pattern] The
  mosaic field-of-view must at least be Nyquist-sampled to obtain a
  reliable image. Each observed field can produce a reliable image of the
  same shape than the primary beam, \ie\ a circular Gaussian (This assumes
  that the short-spacing problem has been solved). Nyquist sampling thus
  implies that the mosaic fields follow an hexagonal compact pattern as
  this ensures a distance between all neighboring fields of half the
  primary beam size. When the total observing time is fixed, Nyquist
  sampling is the best compromise between sensitivity and total
  field-of-view. Indeed, the distance between neighboring fields could be
  less (in which case the mosaic would be oversampled) than half the
  primary beam size. In this case, the sensitivity on each pixel of the
  final image would increase with the share of the time spent to observe
  this direction.
\item[Uniform imaging properties and quick loop around the fields] Getting
  uniform imaging properties is a desirable feature in the final result.
  This implies that a \uv{} coverage and a noise level as uniform as
  possible among the different fields. Quickly looping around the different
  fields is the easiest way to reach this goal. However, dead time to
  travel from one field to another must almost be minimized. At NOEMA,
  the compromise is to pause at least 1 minute on each field and to try to
  loop over all the fields between two calibrations every 20 minutes.
  Hence, mosaic done in a single observing run is made of at most 20
  fields.  Larger mosaic must be observed by group of fields in different
  observing runs.
\end{description}

\subsubsection{Imaging}

When combining together (dirty or clean) images, it is important to correct
the primary beam attenuation to avoid modulation of the signal in the
combined image. If we forget for the moment the dirty beam convolution, the
images associated to each fields are noisy measurement of the same quantity
(the sky brightness distribution) weighted by the primary beam. The best
estimation of the measured quantity is thus given by the least mean square
formula
\begin{equation}
  \label{eq:mosaic:signal}
  \displaystyle %
    M(\alpha,\delta) = \frac{\displaystyle\sum\nolimits_i \frac{B_i(\alpha,\delta)}{\sigma_i^2}\,F_i(\alpha,\delta)}
      {\displaystyle\sum\nolimits_i \frac{B_i(\alpha,\delta)^2}{\sigma_i^2}},
\end{equation}
where $M(\alpha,\delta)$ is the brightness of the dirty/cleaned mosaic
image in the direction $(\alpha,\delta)$, $B_i$ are the response functions
of the primary antenna beams in the tracking direction of field $i$, $F_i$
are the brightness distributions of the individual dirty/cleaned maps, and
$\sigma_i$ are the corresponding noise values. As may be seen on this
equation, the intensity distribution of the mosaic is corrected for primary
beam attenuation. This implies that noise is inhomogeneous. Indeed, if
$N(\alpha,\delta)$ is the noise distribution and $\sigma(\alpha,\beta)$ is
its standard deviation in the direction $(\alpha,\beta)$, we have
\begin{equation}
  \label{eq:mosaic:noise}
    N(\alpha,\delta) = \frac{\sum\nolimits_i \frac{B_i(\alpha,\delta)}{\sigma_i^2}\,N_i(\alpha,\delta)}
      {\sum\nolimits_i \frac{B_i(\alpha,\delta)^2}{\sigma_i^2}},
\end{equation}
and
\begin{equation}
  \label{eq:mosaic:sigma}
    \sigma(\alpha,\delta) = %
    \frac{\sqrt{\sum\nolimits_i \frac{B_i(\alpha,\delta)}{\sigma_i^2}}}
    {\sum\nolimits_i \frac{B_i(\alpha,\delta)^2}{\sigma_i^2}} = %
    \frac{1}{\sqrt{\sum\nolimits_i \frac{B_i(\alpha,\delta)^2}{\sigma_i^2}}}
\end{equation}
Not only, the noise strongly increases near the edges of the mosaic
field-of-view. But also, the center of each field is contaminated by
increased noise level coming from the external regions of the neighboring
fields. Indeed, the noise corrected for the primary beam attenuation is
largely increasing where the primary beam is going to zero. To limit these
effects, both the primary beams used in the above formula and the resulting
mosaic are truncated.

\subsubsection{Deconvolution}

Standard \clean{} algorithms must be slightly modified to work on a dirty
mosaics. Indeed, the use of truncated primary beam in the above equations
is only a first order measure to avoid noise artifacts. However, the noise
level still increases at the edges of the mosaic, implying that at some
point the \clean{} algorithms will confuse noise peaks at the mosaic edges
with true signal. To avoid this, the iterative search is made on a
signal-to-noise image $M(\alpha,\beta)/\sigma(\alpha,\beta)$ instead of the
residual image. At the restoration step, the clean component list is used
to produce the residual map and clean map. Nothing particular is done with
the remaining signal-to-noise image.

\subsubsection{Typical use}

The processing of mosaics for NOEMA is essentially similar to that
of single fields. There are only two small changes
\begin{description}\itemsep 0pt
\item[Creation of \uv{} table]  A mosaic UV table should be created
using the /MOSAIC option of command TABLE in CLIC.
\item[Imaging] is done through \com{UV\_MAP} as for a single field.
However, the process is different. The command takes into account
the various fields, the primary beams, and select an optimum projection
center (phase center). The later may need to be specified by the user
using the \sicvar{MAP\_CENTER} string, or as argument to \com{UV\_MAP}
\item[Deconvolution] is also similar, but not all algorithms are
available. Only  HOGBOM and CLARK are possible so far.
The change of behavior of the \clean{} algorithms is visualized
through the change of prompt from \texttt{IMAGER>} to \texttt{MOSAIC}.
Two additional variables are used for mosaic deconvolution
\begin{description}\itemsep 0pt
\item[\sicvar{CLEAN\_SEARCH}] The minimum fraction of a primary beam
below which no Clean component is searched for.
\item[\sicvar{CLEAN\_RESTORE}] The minimum fraction of a primary beam
below which no image restoration is performed. Below this threshold,
the Clean image is blanked.
\end{description}
Finally, note that the mosaic deconvolution produces sky brightness
images while single-field deconvolution produces images attenuated by the
primary beam.
\end{description}
Although \imager{} makes no specific assumption about the \uv{} coverage
of individual fields, mosaicing deconvolution will work better
if all fields are equivalent in \uv{} coverage and noise level
\footnote{IS THAT TRUE ? : An additional subtlety of the current \imager{} implementation of
mosaicing is that \mapping{} assumes that the individual fields of the
mosaic have similar noise level.}

A mosaicing session would thus just be like a single-field
imaging:
\begin{verbatim}
       1 read uv gag_demo:demo-mosaic 
       2 uv_map
       3 hogbom /flux 0 10
       4 show residual
       5 show clean
       6 write * demo
\end{verbatim}
Comments:
\begin{description}\itemsep 0pt
\item[Step 1] Read the UV table
\item[Step 2] Image the mosaic
\item[Step 3] Deconvolve
\item[Steps 4-5] Look at the result
\item[Steps 6] Save the result
\end{description}

\section{Short and Zero spacings} 
\label{sec:short-spacings}

\subsection{In a nutshell}

\begin{verbatim}
  1  read uv gag_demo:demo-mosaic-v22.uvt
  2  read single gag_demo:demo-single-v22.tab
  3  uv_short
  4  uv_map; clean
  5  view clean
  6  write * MyDemo
\end{verbatim}
\begin{enumerate}\itemsep 0pt
\item Read your \uv{} data
\item read your single-dish data
\item Merge it with with the \uv{} data set,
\item Image and deconvolve as usual
\item check the result
\item Save it. 
\end{enumerate}
You are done. This works for mosaics as well as single fields.
But, again, if the result are crazy, do not blame the software. 
Rather read carefully the information below: 
\com{UV\_SHORT} is simple to use, but can be tricky to use well !...

\subsection{Principle}
Let's note $D$ the diameter of the single-dish antenna ($D = 30$m for the
IRAM-30m telescope) used to produce the short-spacing information and $d$
the diameter of the interferometer antennas ($d = 15$m for \PdBI{}). We
already mentioned that a multiplicative interferometer filters out all the
spatial frequencies smaller than $\sim d$ meters. When this information is
needed to get reliable results, the source should also be observed with a
single-dish antenna to produce the missing information.  The single-dish
antenna furnishes information about all spatial frequencies up to $\sim D$
meters (but this information is weighted by the single-dish beam shape,
\ie\ high frequencies are measured with a worse signal-to-noise ratio than
low frequencies). To recover all the information at spatial frequencies
smaller than $\sim d$ meters, the diameter of single-dish antenna must be
larger or equal to the diameter of the interferometer antennae: $D \ge d$.

\subsection{Algorithms to merge single-dish and interferometer information}

The measurement equations of a single-dish and an interferometer are quite
different from each other. Indeed, the measurement equation of a
single-dish antenna is
\begin{equation}
  I_\emr{meas}^\emr{sd} = B_\emr{sd} \star I_\emr{source} + N,
\end{equation}
\ie\ the measured intensity ($I_\emr{meas}^\emr{sd}$) is the convolution of
the source intensity distribution ($I_\emr{source}$) by the single-dish
beam ($B_\emr{sd}$) plus some thermal noise, while the measurement equation
of an interferometer can be rewritten as
\begin{equation}
  I_\emr{meas}^\emr{int} = B_\emr{dirty} \star \cbrace{B_\emr{primary}.I_\emr{source}} + N,
\end{equation}
\ie\ the measured intensity ($I_\emr{meas}^\emr{int}$) is the convolution
of the source intensity distribution times the primary beam
($B_\emr{primary}.I_\emr{source}$) by the dirty beam ($B_\emr{dirty}$) plus
some thermal noise. $B_\emr{sd}$ has very similar properties than
$B_\emr{primary}$ and very different properties than $B_\emr{dirty}$.  In
radioastronomy, $B_\emr{sd}$ and $B_\emr{primary}$ both have Gaussian
shapes. Moreover, the fact that we will use the single-dish information to
produce the short-spacing information filtered out by the interferometer
implies that $B_\emr{sd}$ and $B_\emr{primary}$ have similar full width at
half maximum. Now, $B_\emr{dirty}$ is quite far from a Gaussian shape with
the current generation of interferometer (in particular, it has large
sidelobes) and the primary side lobe of $B_\emr{dirty}$ has a full width at
half maximum close to the interferometer resolution, \ie\ much smaller than
the FWHM of $B_\emr{sd}$.

Merging both kinds of information obtained from such different measurement
equations thus asks for a dedicated processing. There are mainly two
families of short-spacing processing: The hybridization and the
pseudo-visibility techniques.

\subsubsection{Hybridization technique}

In this family, most of the processing is done on the interferometric data
alone. Indeed, the interferometric data is deconvolved and corrected for
the primary beam contribution to obtain
\begin{equation}
  I_\emr{clean}^\emr{int} = B_\emr{clean} \star I_\emr{source} + N',
  \label{eq:clean}
\end{equation}
where $B_\emr{clean}$ is a Gaussian of FWHM equal to the interferometer
resolution and $N'$ is some thermal noise corrected for the primary beam
contribution.  Two main facts are hidden in this formulation: 1) the
field-of-view of the observation is obviously limited to the observed
portion of the sky and 2) more importantly, the lack of short-spacings has
not yet been overcome and a better formulation would be
\begin{equation}
  I_\emr{clean}^\emr{int} = \mbox{Highpass-filter}\cbrace{B_\emr{clean} \star I_\emr{source}} + N'.
\end{equation}
The simplicity of equation~\ref{eq:clean} is thus slightly misleading but
we will keep it for the sake of simplicity. The hybridization method
consists in combining two images ($I_\emr{meas}^\emr{sd}$ and
$I_\emr{clean}^\emr{int}$) in the \uv{} plane.
\begin{enumerate}
\item Both images are first spatially regridded on the same fine grid.
\item The FFT of those two images are computed, and linearly combined by
  selecting the low spatial frequencies from FFT($I_\emr{meas}^\emr{sd}$)
  and the high spatial frequencies from FFT($I_\emr{clean}^\emr{int}$).
  The transition between low and high spatial frequency
\item The result is FFTed back to the image plane to produce a final,unique
  image, which takes into account both single-dish and interferometric
  information.
\end{enumerate}
The method has the following free parameters: the transition radius and the
detailed shape of that transition.  To avoid discontinuity, the transition
shape is chosen to be reasonably smooth. The spatial frequency of
transition is generally chosen to the smallest spatial frequency reliably
measured by the interferometer (\eg\ about 20~m for NOEMA).

\subsubsection{Pseudo-visibility technique}

\paragraph{General description}

In this family, the single-dish information is heavily processed before
merging with the interferometric information. The basic idea is to produce
from the single-dish observations pseudo-visibilities similar to the ones
that would be produced by the interferometer if they were not filtered out.
\begin{enumerate}\itemsep 0pt
\item The Single-Dish measurements are re-gridded and then FFTed into the
  \uv{} plane.
\item The data are deconvolved of the single-dish beam ($B_\emr{sd}$)
  convolution by division by its Fourier Transform (truncated to the
  antenna diameter).
\item The data are FFTed back to the image plane and multiplied by the
  interferometer primary beam, $B_\emr{primary}$.
\item The result is FFTed again in the \uv{} plane where the visibilities
  are sampled on a regular grid.
\item In the case of a mosaic, the two last operations are performed for
  each pointing center.
\end{enumerate}
Using the properties of the Fourier transform, we can rewrite the
measurement equation of an interferometer as
\begin{equation}
  V(u,v) = \cbrace{\mbox{FT}(B_\emr{primary}) \star \mbox{FT}(I_\emr{source})}(u,v)+N.
\end{equation}
This equation means that the visibility measured by an interferometer at
the spatial frequency $(u,v)$ is the convolution of the Fourier transform
of the source intensity distribution by the Fourier transform of the
primary beam. Hence, to get pseudo-visibilities truly consistent with
interferometric visibilities, we must be able to reliably compute the
convolution by the Fourier transform of the primary beam. This implies that
we can compute pseudo-visibilities only for spatial frequencies lower than
$D-d$. The use of the IRAM-30m to produce the short-spacing information of
the \PdBI{} is thus ideal as it enables to recover pseudo-visibilities up
to 15~m ($=30\,\mbox{m}-15\,\mbox{m}$). Once the pseudo-visibilities have
been computed, they are merged with the interferometric visibilities and
standard imaging and deconvolution are then applied to the merged data set.


\paragraph{Single-dish vs interferometer weight}

In all cases involving short spacings, the relative weight of the single
dish data to interferometer data is critical. Within the restrictions
imposed by the noise level, this relative weight is a free parameter. It is
all the more important that the Fourier transform of the \uv{} plane
density of weights is the dirty beam, a key parameter of the deconvolution.
The general goal is to have a dirty beam as close as possible to a
Gaussian. As the Fourier transform of a Gaussian is a Gaussian, we search
for obtaining a \uv{} plane density of weights as close as possible to a
Gaussian. In general, the short spacing frequencies are small compared to
the largest spatial frequency measured by an interferometer. This implies
we can use the linear approximation of a Gaussian, \ie\ a constant, in the
range of frequencies used for the short spacing processing. We thus end up
with the need to match (as far as possible) the Single-Dish and
interferometric densities of weights in the \uv{} plane. In practice, we
compute the density of weights from the single-dish in a \uv{} circle of
radius $1.25 d$ and we match it to the averaged density of weights from the
interferometer in a \uv{} ring between 1.25 and $2.5 d$. Experience shows
that this gives the right order of magnitude for the relative weight and
that a large range of relative weight around this value gives very similar
final results.

When processing IRAM-30m data to combine to NOEMA data, this criterion
implies a large down-weighting of the IRAM-30m data which may make think
that too much observing time was used at the IRAM-30m data. However, just
using the above criterion to determine the observing time needed at the
IRAM-30m would in general lead to very low signal-to-noise ratio of the
single-dish map. In such a case, it is very difficult to detect problems
which would translate in strong artifacts in the IRAM-30m + NOEMA
combination. We recommend to ask for enough IRAM-30m time to get a
\emph{median} signal-to-noise ratio of 5 on the single-dish map. This ratio
should be achieved for all velocity channels of interest (which may include
the line wings).

\subsubsection{Comparison}

The simplicity of the hybridization technique is its main advantage. It is
simple to understand and simple to implement. However, this method works
badly in practice because it is truly difficult to obtain a
reliable deconvolution of interferometric data alone when short-spacing
information is important. An interferometer is a spatial pass-band
filter, filtering in particular the zero spacing. This implies that the total flux in the
dirty image is zero (\ie\ as much negative as positive flux in the dirty
image) but that the dirty beam integral is also zero (\ie\ as much negative
as positive sidelobes). Adding the short-spacing information (and in
particular the zero spacing) through the pseudo-visibility method, we
enforces the positivity of the dirty image total flux and of the dirty beam
integral. It is well-known that trying to deconvolve a mosaic built only
with interferometric data is quite difficult. It almost always requires the
definition of support where the \clean{} algorithms can search for clean
components with the clear risk to bias the final result. In contrast,
adding the short-spacing information through pseudo-visibilities enables an
almost straightforward \clean{} deconvolution \emph{without} the need of
any support.

For the sake of illustration, let us assume an intensity distribution 
made of a large scale structure (\eg{} a smoothly varying intensity) 
superimposed with a small scale distribution both in emission and 
absorption. An interferometer will filter out the smooth 
distribution. If there is no additional zero spacing 
information, the smooth distribution is completely lost with 
the important consequence that the final deconvoled image will have 
positive (emission) and negative (absorption) structures. Trying to 
reproduce both negative and positive structures is one of the most 
difficult task for deconvolution algorithms. In addition, the 
presence of large negative structures create instabilities in the 
algorithms of the \clean{} family (because it is difficult to 
distinguish between negative absorption structures and negative 
sidelobes of emission structures). Only the definition of support 
around positive emission peaks may succeed to stabilize the \clean{} 
algorithms with the drawback of biasing the result.

Both kind of algorithms (hybridization and pseudo-visibility) are
implemented in \gildas{}. However, we strongly recommend to use the
pseudo-visibility algorithm. That is why only the pseudo-visibility method
is packaged in a user-friendly way, in the \com{UV\_SHORT} command.
\citet{pety01b,pety01a,pety01c} showed through
simulations that 1) the pseudo-visibility algorithm implemented in
\gildas{} enable extremely reliable results (fidelities of a few thousands)
on ideal observations and 2) the accuracy of the wide-field imaging is
limited by pointing errors, amplitude calibration errors and atmospheric
phase noise (and not by the used algorithms), even for ALMA.

\subsection{Hybridization technique and ALMA}
A special case where Hybridization can be extremely useful is that
of ALMA observations involving the main 12-m array, the ACA and single-dish
data with the 12-m antennas.  In some circumstances, an optimal
result is obtained by hybridizing Cleaned images produced
from the mosaics obtained by combining (using the pseudo-visibility method)
ACA and Single-dish data
as short spacings, with another mosaic obtained with the 12-m
data and the Single-dish data together.

The deconvolution of each mosaic is stabilized by the addition
of the 12-m single-dish zero or short spacings, and these good
mosaics are then merged in an optimal way by using the best
region of the \uv{} plane that they sample.

\subsection{The Zero spacing: an important subset}

An important subset of the pseudo-visibility method is the production of
only the zero spacing. Indeed, the zero spacing is just the total flux of
the observed field-of-view. Hence, if the observed field-of-view is small
enough to fit in the single-dish beam (this is in particular always the
case if $D = d$), a single spectrum observed with the single-dish telescope
in the direction of the interferometer phase center may be used as zero
spacing, only a scaling from Kelvin to Jansky is needed. This is the poor
man solution as only part of the short spacing information is recovered
by this technique.

\subsection{Short Spacings in practice: command \texttt{UV\_SHORT}}

Our algorithm to produce the short-spacing information is coded in the
\com{UV\_SHORT} command.  \com{UV\_SHORT} will \textbf{add} the 
short spacing information to the current \uv{} table (read by
command \comm{READ}{UV} and optionally transformed by further 
\texttt{\color{magenta} UV\_...} processing commands). 

\com{UV\_SHORT} has a substantial number (17)
of control variables, but with experience, they have been reduced
to 5 significant ones, among which only 3 really matter in most cases
but often can be used with their default values:
\begin{description}\itemsep 0pt
\item{\sicvar{SHORT\_SD\_FACTOR}} The single-dish brightness unit to 
flux conversion factor. If set to zero, \com{UV\_SHORT} will attempt
to derive it from the information available in the single-dish data   
\item{\sicvar{SHORT\_UV\_TRUNC}} The longest baseline retained in
the pseudo-visibilities. It defaults to the maximum theoretically possible,
the single-dish diameter minus the interferometer diameter. Smaller
values are allowed, and even recommended if the pointing quality
of the single-dish data is moderate.
\item{\sicvar{SHORT\_SD\_WEIGHT}} The relative weight scaling factor
between the pseudo-visibilities and the interferometer visibilities.
\end{description}
The relative weight of these visibilities is derived by \com{UV\_SHORT}
in order to optimize the shape of the overall synthesized beam. 
\sicvar{SHORT\_SD\_WEIGHT} is a scale factor to this optimum weight,
which may need to differ from 1 in case of poor \uv{} coverage in
the interferometer data or 
noisy single-dish data (it should be lower than 1 in this case).


\texttt{UV\_SHORT ?} will list these 3 major ones, and \texttt{UV\_SHORT ??} the
2 remaining main ones:
\begin{description}\itemsep 0pt
\item{\sicvar{SHORT\_TOLE}} The position tolerance in the single-dish map
\item{\sicvar{SHORT\_MIN\_WEIGHT}}  The minimum (relative) weight for a spectrum
in the single-dish map to be included.    
\end{description}
as well as four optional ones needed only if the original single-dish
and \uv{} data lacks the proper information (antenna diameter and beam sizes)

The \com{UV\_SHORT} command starts from data in a the format produced by the \class{} command
\texttt{TABLE}
command, and read in \imager{} through command \comm{READ}{SINGLE}. 
Basically, this is a GDF table containing one line per spectrum,
the columns representing the lambda offset, beta offset, weight, and the
spectrum intensities. \footnote{This format is subject to change: Please, refer to
the \com{TABLE} documentation for up-to-date information}. This data
must match spectrally the velocity sampling of the interferometric
data. This can be obtained using the \texttt{/RESAMPLING} option of
command \texttt{TABLE} in \class{}.

The \comm{READ}{SINGLE} and \com{UV\_SHORT} commands also support
a 3-D data cube (as produced by e.g. command \texttt{XY\_MAP} in \class{})
as input instead of a \class{} table. Again, the
velocity axis must match that of the interferometric data.

\emph{\color{red} temporary results as SIC variables and associated WRITE commands
to save them ?}

\com{UV\_SHORT} will automatically produce the Zero spacing from
the single-dish data when the data does not allow other short spacings
to be evaluated.

Finally, as \com{UV\_SHORT} \textbf{adds} the 
short spacing information, \comm{UV\_SHORT}{/REMOVE} allows to remove
it (there is no direct ``replace'' possibility).

\subsection{Practical considerations}

\subsubsection{When are short-spacing information needed?}

\begin{itemize}
\item If the source size is smaller than 1/3 the primary beam size,
  short-spacing information is superfluous.
\item If the source size is between 1/3 and 1/2 the primary beam size of
  NOEMA antennas, a single spectra obtained at the IRAM-30m telescope in
  the direction of the source can be used to produce the zero spacing
  information with the \com{UV\_ZERO} task. Indeed, the IRAM-30m diameter
  being twice the diameter of the \PdBI{} antenna, all the flux of the
  source will be measured by a single IRAM-30m spectrum only if the size of
  the source is smaller than 1/2 the primary beam size of \PdBI{} antennae.
\item If the source size is larger than 1/2 the primary beam size of
  NOEMA antennas, short-spacing information under the form of an IRAM-30m
  map is almost always mandatory. The only exception could be wide-field
  imaging of a region made of unresolved or small (compared to the primary
  beam size) sources as it may happen when mapping close-by external
  galaxies for instance. However, adding short-spacing will anyway help the
  deconvolution. In case of doubt, always add short-spacing information.
\end{itemize}

\subsubsection{How to optimize single-dish observations?}

One of the main difficulty of the short-spacing problematic is the need of
observations from a single-dish telescope at least as big as the
interferometer antennas\footnote{If there were no pointing errors,
a single-dish of the same size as the interferometer antennas
would be strictly sufficient.}. In this respect, the IRAM-30m and NOEMA are very
complementary. Nevertheless, when observing with the single-dish telescope,
a few precautions are needed to avoid contaminating the interferometric
data with possible artifacts of single-dish data.
\begin{itemize}\itemsep 0pt
\item The field-of-view of the single-dish map must be twice the
  field-of-view covered by the mosaic. The only exception to this rule
  happens when the source intensity decreases to zero in a smaller
  field-of-view. Indeed, there is no point in observing an empty sky.
\item The observing strategy must enforce Nyquist sampling (or better) of
  the source at the resolution of the single-dish telescope.
\item A particular care should be taken of the pointing, tracking and
  amplitude calibration and baseline removal as those are critical issues
  in obtaining a high quality single-dish map to produce short-spacing
  information. For instance, data with too large tracking errors should be
  discarded.
\item We advise to make many On-The-Fly coverages of the observed
  field-of-view to get homogeneous observing conditions. Scanning in
  perpendicular directions is needed to decrease stripping.
\end{itemize}
Sometimes, single-dish telescope time is scarce and some of the above
criteria can not be fulfilled. In those cases, you can still try to use
your single-dish observations and our algorithm will try to make its best
to get a sensible result. However, any artifact in the combination may
directly come from wrong single-dish observations. In other words, do
\emph{not} blame the software unless you are sure of the quality of the
quality of your single-dish (and interferometric) observations...

